\section{Introduction}
The purpose of this paper is to study the relationships between ordered spaces and their induced topology. In particular, we want to study the p
\subsection{Topological spaces}
Using standard notation as in \cite{Bour04}, a topological space is a set $X$ and a subset $\tau \subset \mathcal{P}(X) = 2^X$ fulfulling the following axioms:
\begin{enumerate}
\item $X$ itself and the empty set $\emptyset$ are elements in $\tau$, and
\item $\tau$ is closed under finite union and intersection.
\end{enumerate}
Speaking in the manner of abstract algebra, $\tau$ is a monoid wrt. intersection $\cap$ and union $\cup$ with neutral elements $X$ and $\emptyset$, respectively. Associativity is rather obvious and uniquness of the neutral elements follows immediately from uniqueness of $X$ itself and the empty set.
\subsubsection{Open and closed topological spaces}
Given a family of subsets $\mathcal{U} \subset \tau$, we call $\mathcal{U}$ a family of open subsets (and its elements the open subsets of $X$/in $\tau$)if
$$\bigcup_{U \in \mathcal{U}'} U \in \tau,\ \forall \mathcal{U}' \subset \mathcal{U}.$$
In other words, the topology $\tau(\mathcal{U})$, generated by a family of open sets $\mathcal{U}$, is closed under arbitrary union. $\tau(\mathcal{U})$ is sometimes referred to as an open topology.\\
\indent Given a family of subsets $\mathcal{F} \subset \tau$, we call $\mathcal{F}$ a family of closed subsets
if
$$\bigcap_{F \in \mathcal{F}'} F \in \tau,\ \forall \mathcal{F}' \subset \mathcal{F}.$$
In other words, the topology $\tau(\mathcal{F})$, generated by a family of closed sets $\mathcal{F}$, is closed under arbitrary intersection. $\tau(\mathcal{F})$ is sometimes referred to as a closed topology.\\
\paragraph{Exercise} Let us discuss some examples:
\begin{enumerate}
\item Given $X = \{0, 1, 2, 3\}$ which of the following families of subsets is a topology
\begin{enumerate}
\item $\tau = \left\{\emptyset, \{0, 1\}, \{0, 1, 2\}, \{1, 2, 3\}, X\right\}$.
\item $\tau = \left\{\emptyset, \{0, 1\}, \{0, 1, 2\}, \{0, 1, 3\}, X\right\}$.
\item $\tau = \left\{\emptyset, \{2\}, \{3\}, \{2, 3\}, X\right\}$.
\item $\tau = \left\{\emptyset, \{0, 1\}, \{2, 3\}, X\right\}$.
\item $\tau = \left\{\emptyset, \{1, 2\}, \{2, 3\}, \{1, 2, 3\}, X\right\}$.
\item $\tau = \left\{\emptyset, \{1, 3\}, \{2, 3\}, \{1, 2, 3\}, X\right\}$.
\item $\tau = \left\{\emptyset, \{1, 2\}, \{0, 3\}, \{1, 2, 3\}, X\right\}$.
\end{enumerate}
\item Compute the topology $\tau(\beta)$ generated by the following families $\beta$ of subsets of $X$ as above:
\begin{enumerate}
\item $\beta = \left\{\{0\}, \{2\}, \{1, 3\}\right\}$.
\item $\beta = \left\{\{1\}, \{2\}, \{1, 3\}\right\}$.
\end{enumerate}
\end{enumerate}
\paragraph{Complement}
The complement of any given subset $X' \subset X$ for some set $X$ - denoted by $X'^{c}$ - is simply:
$$X'^c := \{x \in X : x \notin X'\}.$$
The complement of an open set $U \in \tau$ is closed in $\tau$ and, conversely, the complement of a closed set is, indeed, open. Therefore, the topology of the complement of an open topology is closed and, conversely, the topology of the complement of a closed topology is open.\\
\indent More interestingly, the subset $\emptyset, X \in \tau$ are closed and open at the same time (as $X$ can be seen as the union of all its open sets, whatever they are, and $\emptyset$ is the intersection of all closed sets). Thus, $X^c = \emptyset$ is open and $\emptyset^c = X$ is closed. We call these sets clopen subsets in $\tau$. A clopen topology $\tau$ is generated by clopen sets. Furthermore, if all elements in $\tau$ are clopen, we can show that $\tau$ is a subtopology of $\mathcal{P}(X)$.
\subsection{Ordered sets}
Given a non empty set $X$, a relation is a subset $R \subset X \times X$. We say a pair $(x,y) \in X^2$ holds $R$ if $(x,y) \in R$ or in a manner of notation: $x R y$. An equivalence relation $\Delta \subset X^2$ is a relation, fulfilling the following axioms
\begin{description}
\item[Reflexivity] $x \Delta x$, for all $x \in X$,
\item[Symmetry] $x \Delta y \Leftrightarrow y \Delta x$, for all $x, y \in X$ and
\item[Transitivity] $x \Delta y$ and $y \Delta z$ implies $x \Delta z$, for all $x, y, z \in X$.
\end{description}

