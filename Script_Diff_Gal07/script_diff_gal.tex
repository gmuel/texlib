\documentclass[pdftex,11pt]{article}
\usepackage[T1]{fontenc}
\usepackage[latin1]{inputenc}
\usepackage{a4wide}
%\usepackage{ngerman}%
\usepackage[english]{babel}%ngerman}
\usepackage{graphicx}
\usepackage{amsmath}
%\usepackage{makeidx}
\usepackage{multind}
\usepackage{amssymb}
\usepackage{amsfonts}
\usepackage{MnSymbol}
\usepackage[all]{xy}
\usepackage{longtable}
\usepackage{float,array}
\usepackage{color}
\usepackage{fancyhdr}
\usepackage{hyperref}
\usepackage{cite}
\makeindex{Index}
\makeindex{Symbol}
\cfoot{\thepage}
\renewcommand{\headheight}{40pt}
\renewcommand{\textheight}{595pt}
\renewcommand{\footskip}{20pt}
\renewcommand{\baselinestretch}{1.5}
\newcommand{\cz}{\mathbb{C}}
\newcommand{\rz}{\mathbb{R}}
\newcommand{\qz}{\mathbb{Q}}
\newcommand{\zz}{\mathbb{Z}}
\newcommand{\nz}{\mathbb{N}}
\newcommand{\mat}[3]{M(#1,#2:#3)}
\newcommand{\LK}{Linearkombination}
\newcommand{\ov}[1]{\overline{#1}}
\newcommand{\tsfdet}{\textsf{det}}
\newcommand{\bn}{\begin{enumerate}}
\newcommand{\en}{\end{enumerate}}
\newcommand{\bi}{\begin{itemize}}
\newcommand{\ei}{\end{itemize}}
\newcommand{\bd}{\begin{description}}
\newcommand{\ed}{\end{description}}
\newcommand{\bad}[3]{\begin{array}{#1#2#3}}
\newcommand{\bao}[1]{\begin{array}{#1}}
\newcommand{\basi}[6]{\begin{array}{#1#2#3c#4#5#6}}
\newcommand{\ea}{\end{array}}
\newcommand{\im}{\textrm{im }}
\newcommand{\Sp}{\textrm{Spur}}
\newcommand{\rank}{\textrm{rank}}
\newcommand{\MIT}{\textsf{, mit }}
\newcommand{\und}{\textsf{ und }}
\newcommand{\fur}{\textsf{ f\"ur }}
\newcommand{\END}{\textrm{End}}
\newcommand{\ggT}{\textrm{ggT}}
\newcommand{\kgV}{\textrm{kgV}}
\newcommand{\id}{\textit{id}}
\newcommand{\Gl}[2]{\textrm{Gl}_{#1}(#2)}
\newcommand{\SL}[2]{\textrm{SL}_{#1}(#2)}
\newcommand{\LRA}{\Leftrightarrow}
\newcommand{\RA}{\Rightarrow}
\newcommand{\LA}{\Leftarrow}
\newcommand{\RE}{\textsf{Re}}
\newcommand{\IM}{\textsf{Im}}
\newcommand{\End}{Endomorphism}
\newcommand{\VR}{Vektorraum}
\newcommand{\KE}{K\"orpererweiterung}
\newcommand{\lati}{\widetilde{\lambda}}
\newcommand{\muti}{\widetilde{\mu}}
\newcommand{\wt}{\widetilde}
\newcommand{\uv}{\underline}
\newcommand{\half}{\frac{1}{2}}
\newcommand{\hasq}{\frac{1}{\sqrt{2}}}
\newcommand{\A}{\mathcal{A}}
\newcommand{\factorg}[3]{#1/#2#3}
\newcommand{\ifacg}[1]{\zz/#1\zz}
\newcommand{\facg}[2]{#1/#2}
\newcommand{\homo}{\textrm{Hom}}
\newcommand{\homox}[2]{\textrm{Hom}(#1,#2)}
\newcommand{\Homo}{\textsf{Homomorphism}}
%\newcommand{\grn}[1]{\cz_{p^{#1}}}
\newcommand{\sgp}{Untergruppe}
\newcommand{\syl}{$p$-Sylow \sgp}
\newcommand{\psyl}[1]{$#1$-Sylow \sgp}
\newcommand{\ord}[1]{\textrm{ord}(#1)}
\newcommand{\sgn}{\textrm{sgn}}
\newcommand{\arga}{(a_1,a_2,a_3)}
\newcommand{\argb}{(b_1,b_2,b_3)}
\newcommand{\primefield}[1]{\mathbb{F}_{#1}}
\newcommand{\polyfield}{K[X]}
\newcommand{\polyring}{R[X]}
\newcommand{\uniring}{R^\times}
\newcommand{\unifield}{K^\times}
\newcommand{\ann}{\textrm{Ann}}
\newcommand{\primeideal}{\mathfrak{p}}
\newcommand{\maxideal}{\mathfrak{m}}
\newcommand{\chak}{\textsf{char}}
\newcommand{\grad}{\textrm{grad}}
\newcommand{\eps}{\varepsilon}
\newtheorem{defi}{Definition}[section]
\newtheorem{bmkg}{Bemerkung}[section]
\newtheorem{koro}{Corollary}[section]
\newtheorem{lemm}{Lemma}[section]
\newtheorem{prop}{Proposition}[section]
\newtheorem{satz}{Theorem}[section]
\newcommand{\Int}[1]{\textsf{Int}\left(#1\right)}
\newcommand{\topo}{\mathcal{O}}
\newcommand{\emp}{\emptyset}
\newcommand{\base}{\mathcal{B}}
\newcommand{\basex}[1]{\base\left(#1\right)}
\newcommand{\bsp}{\paragraph{Example}}
\newcommand{\bws}{\paragraph{Proof}}
\newcommand{\bmk}{\paragraph{Remark}}
\newcommand{\bsl}{\backslash}
\newcommand{\cov}{\mathcal{U}}
\newcommand{\trm}[1]{\mathrm{#1}}
\newcommand{\ideal}[1]{\mathfrak{{#1}}}
\newcommand{\idealp}{\ideal{p}}
\newcommand{\aff}[1]{\mathbb{A}^{#1}}
\newcommand{\affn}{\aff{n}}
\newcommand{\tb}[1]{\textbf{#1}}%\index{#1}}
\newcommand{\prj}[1]{\mathbb{P}^{#1}}
\newcommand{\prjn}{\prj{n}}
\newcommand{\polygen}[2]{[x_{#1},\hdots,x_{#2}]}
\newcommand{\polygpr}{\polygen{0}{n}}
\newcommand{\polygaf}{\polygen{1}{n}}
\newcommand{\proj}{\trm{Proj}}
\newcommand{\spec}{\trm{Spec}}
\newcommand{\currfield}{\ov{\qz}}
\newcommand{\currfieldx}{\currfield(x_1)}
\newcommand{\filteredA}{\mathcal{A}}
\newcommand{\commt}[1]{{\color{red} #1}}
\title{Exemplifying the Differential Galois Theory of Simple Artinian Module Algebras}

\author{Gabriel M\"uller}
\begin{document}
\include{title}
\pagestyle{plain}
\pagenumbering{roman}
\tableofcontents
\newpage
%\section*{Abbreviations}
We are going to use standard notation $\qz, \zz, \nz = \zz_{\geq0}, \nz_0 = \zz_{>0}$ for the set of rational numbers, integer numbers, non-negative integers and natural numbers, respectively. Additionally, we define
\begin{longtable}{rcl}
$\trm{Set}$ &:& category of sets\\
$\trm{Top}$ &:& category of topological spaces\\
$\trm{Grp}$ &:& category of groups\\
$\trm{Abel}$ &:& category of abelian groups\\
$\trm{Rng}$ &:& category of rings\\
$\trm{CRng}$ &:& category of commutative rings\\
$\trm{Mod}_R$ &:& category of $R$-modules\\
$\trm{Alg}_R$ &:& category of $R$-algebras\\
$\trm{CAlg}_R$ &:& category of commutative $R$-algebras\\
$\trm{Sch}$ &:& category of schemes\\
$\trm{GrpSch}$ &:& category of group schemes.\\
\end{longtable}
The algebraic closure of a field $C$ is denoted by $\ov{C}$. The completion of a topological space $(X, \tau)$ is denoted by $(\hat{X}, \hat{\tau})$.
\pagestyle{fancy}
\pagenumbering{arabic}
%\section{Questions}
Some general and more specific questions:
\bn
\item Is intro of internal module algebra important, as this construct is never used (here, it is used in \cite{Heid13}).
\item Why is completion of $K[[t]] \otimes A[[w]]$, wrt $\left<w\right>$-adic topology, important?
\item What is $\trm{Ume}(K/k)(A)$ if $N(A) = \{0\}$ (should be trivial group)? Not an answer, but nilpontency is important such that composition is well-def:

\item What is a PV-ring $R$ in the context of for (implicit/explicit) non-linear diff. equation $F \subset k\{u\}$ (with $\trm{char} k = 0$)?\\
Should be something like:
$$R \simeq S^{-1}_x k\{u\}/I(F) := S^{-1}_x k\{u\}/\left<\partial^i f : i \in \nz_0, f \in F\right>,$$
where $x \in \left<u_i\right> \bsl \{0\}$.
\item What is $\partial (\det X)$ for $L = \partial^2 - a \in k[\partial]$ if $a \in k\bsl k^\partial$ and $X \in \trm{Gl}_2(R)$ fundamental matrix?\\
Should be zero, however, direct computation shows $\partial \det X = \partial(a) x_1^2$.
\item For a differential field $(k,\partial)$ and its ring of differential operators $k[\partial] \subset \trm{End}_{C_k}(k)$, the module structure map $\Psi \in \trm{Hom}_{C_k}(k[\partial] \otimes k,k)$ is only $C_k$-linear. Hence, all tensors must be understood as $\otimes_{C_k}$.
%\item Invarance in PV-theory is not clear, is $\F
\en
%intro

\section{Prerequisits}

Following the standard notation from Bourbaki, we will denote for a given set $X$ the power set $\calp(X) = 2^X = \{A : A \subset X\}$.
\begin{defi}
A universe $\calu$ is a set satisfying the following properties:
\begin{enumerate}
\item $\emptyset \in \calu$,
\item $U \in \calu$ implies $U \subset \calu$,
\item $U \in \calu$ implies $\{U\} \in \calu$,
\item $U \in \calu$ implies $\calp(U) \in \calu$,
\item if $I \in \calu$ and $U_i \in \calu$ for $i \in I$ then
$$\bigcup_{i \in I} U_i \in \calu,$$
\item $\nz \in \calu$.
\end{enumerate}
We call a set $U$ a $\calu$-set if $U \in \calu$. We call a set $U$ $\calu$-small if it is isomorphic to a set in $\calu$.
\end{defi}
Following Grothendieck we shall add to the axiom system of Zermelo-Fraenkel the axiom demanding for any set $X$ there is a universe $\calu$ containing $X$.
\begin{defi}
An order on a set $I$ is a relation $\leq$ which is reflexive, antisymmetric and transitive. We call an order directed or filtrant if $I$ is non empty and if for any pair $i, j \in I$ there is $k \in I$ such that $i \leq k$ and $j \leq k$. We call an order total if for all $i, j \in I$ we have either $i \leq j$ or $j \leq i$ (or both). An ordered set $I$ is inductively ordered if any totally ordered subset $I' \subset I$ has an upper bound. 
\end{defi}
We remark that following the axiom of choice any inductively ordered set as an maximal element (Zorn's lemma).

\subsection{Categories and functors}
\begin{defi}
A category $\calc$ consists of the following data:
\begin{enumerate}
\item the class of objects $\objc$,
\item the class of morphisms $\morc$ which consists for all $X, Y \in \mrm{Ob}(\calc)$ of the classes of morphisms $\homc{X,Y}$ and 
\item the composition law of morphisms: for any triple $X, Y, Z$ of objects in $\calc$ and morphisms $f \in \homc{X,Y}$ and $g \in \homc{Y,Z}$ we get a morphism $h = g \circ f \in \homc{X,Z}$.
\end{enumerate}
For each object $X \in \objc$ there is a morphism $e \in \homc{X,X}$ such that $f \circ e = f$ and $e \circ g = g$ for all $f \in \homc{X,Y}$ and $g \in \homc{Y,X}$.
\end{defi}
We remark that the composition is associative and the latter morphism is simply the identity map. We call a category $\calc$ $\calu$-category if for all $X, Y \in \objc$ the class $\homc{X,Y}$ is $\calu$-small. A $\calu$-small category is a $\calu$-category $\calc$ such that $\objc$ is $\calu$-small.
\begin{defi}
Given a category $\calc$ we define the opposite category $\calc^{\mrm{op}}$ to be:
$\obj{\calc^{\mrm{op}}} = \objc$ and $\homo{\calc^{\mrm{op}}}{X,Y} = \homc{Y,X}$ for all $X, Y \in \objc$.
The composition is simply inverted: for $f \in \homo{\calc^{\mrm{op}}}{X,Y}$ and $g \in \homo{\calc^{\mrm{op}}}{Y,Z}$ we get:
$$g \circ^{\mrm{op}} f = f \circ g.$$
\end{defi}
A morphism $f : X \longrightarrow Y$ in $\calc$ is called an isomorphism if there exists a morphism $g : Y \longrightarrow X$ in $\calc$ such that
$$ f \circ g = id_Y  \ \mrm{and}\ g \circ f = id_X.$$
An endomorphism is a morphism with target object $Y = X$ - an automorphism is and endomorphism and isomorphism. Two morphism $f, g$ are parallel if they have the same source and targets:
$$f, g : X\  \substack{\longrightarrow\\\longrightarrow}\ Y.$$
A morphism $f : X \longrightarrow Y$ is a monomorphism if for any parallel morphisms $g_1 , g_2 : Z\  \substack{\longrightarrow\\\longrightarrow}\ X$ we have
$$g_1 \circ f = g_2 \circ f\ \Rightarrow\ g_1 = g_2.$$
A morphism $f : X \longrightarrow Y$ is an epimorphism if $f^{\mrm{op}}$ is a monomorphism.
We call a category $\calc'$ a subcategory of $\calc$ if $\obj{\calc'} \subset \objc$ and $\homo{\calc'}{X,Y} \subset \homc{X,Y}$ for all $X, Y \in \obj{\calc'}$. We call a subcategory $\calc'$ of $\calc$ full if $\homo{\calc'}{X,Y} = \homc{X,Y}$. A full subcategory $\calc'$ of $\calc$ is saturated if $X \in \calc$ belongs to $\calc'$ whenever $X$ is isomorphic to an object in $\calc'$. A category is discrete if all morphism are identity morphisms. A category $\calc$ is non empty if $\objc$ is non empty. A category is a groupoid if all morphisms are isomorphisms. A category $\calc$ is finite if $\morc$ is finite as a set. A category is connected if its non empty and for any pair objects $X, Y \in \calc$ there is a finite sequence of objects $X = X_0, \ldots, X_i = Y$ such that at least one of the sets $\homc{X_j,X_{j+1}}$ or $\homc{X_{j+1},X_j}$ is non empty for all $0 \leq j \leq i - 1$.\\
\indent A diagram in a category $\calc$ is a family of symbols representing objects in $\calc$ and arrows betweens these representing morphisms of these objects. The definition of a commutative diagrams follows in an obvious fashion.
\bsp 
\begin{enumerate}
\item $\mrm{Set}$ is the category of $\calu$-sets and maps, $\mrm{Set}^f$ the full subcategory of finite $\calu$-sets.
\item The category $\mrm{Rel}$ of binary relations is defined to be:
$\obj{\mrm{Rel}} = \obj{\mrm{Set}}$ and $\homo{\mrm{Rel}}{X,Y} = \power{X \times Y}$, the set of subsets of $X \times Y$. The composition law is defined as follows: if $f : X \longrightarrow Y$ and $g : Y \longrightarrow Z$ then $g \circ f$ is
$$\{(x,z) \in X \times Z : \exists y \in Y,\ (x,y) \in f \wedge (y,z) \in g\}.$$
The identity morphism is the diagonal map $\Delta : X \longrightarrow X \times X$.
\item  Let $R$ be a unital ring (not necessarily commutative, $R \in \calu$). The category of $R$ left modules belonging to $\calu$  is denoted $\mrm{Mod}(R)$. The category of $R$ right modules is simply the $R^{\mrm{op}}$ left modules where $R^{\mrm{op}}$ is the opposite ring (with multiplication flipped). Its class of morphisms is
$$\homo{\mrm{Mod}(R)}{\;\cdot,\;\cdot} = \homo{R}{\;\cdot,\;\cdot}.$$
We denote with $\mrm{End}_R(M)$ the ring of $R$ endomorphisms on $M$ and $\mrm{Aut}_R(M)$ the group of automorphisms on $M$. We denote by $\mrm{Mod}^{\mrm{f}}$ the category of finitely generated modules over $R$ (recall: finitely generated iff there is a surjective $R$ linear map $u : R^\oplus \longrightarrow M$ for some $n \geq 1$). They are also called modules of finite type.\\
\indent We denote by $\mrm{Mod}^{\mrm{fp}}$ the category of finitely presented $R$ modules. Recall a module is finitely presented if it is of finite type and $\ker u$, as defined above, is also of finite type.
\item Let $(I,\leq)$ be an ordered set. We associate to it a category $\call{I}$ as follows:
$$\begin{array}{rcl}
\obj{\call{I}} &=& I\\
&&\\
\homo{\call{I}}{i,j} &=& \begin{cases}
\ast,& i \leq j\\
\emptyset, & \mrm{else}\\
\end{cases}\\
\end{array}$$
where $\ast$ stands for some pointed space. Thus, the set of morphisms is either single-pointed or empty.
\item We call a category of boolean type or a boolean category if there three maps
$$\begin{array}{rrcl}
a : & B \times B & \longrightarrow& B\\
&&&\\
o : & B \times B & \longrightarrow& B\\
&&&\\
n : & B &\longrightarrow& B\\
\end{array}$$
such that the following diagrams commute:
$$\begin{array}{cc}
\xymatrix{
B \ar[rd]_{id_B}\ar[r]^n&B\ar[d]^n\\
&B\\
} &
\xymatrix{
B \ar[d]_n & B \times B \ar[l]_a \ar[r]^o \ar[d]_{n \times n} & B\ar[d]^n\\
B & B \times B \ar[l]^o \ar[r]_a &B\\
}\\
\end{array}$$
Its morphisms are simply the maps preserving the each of the three maps.
\end{enumerate}
\subsubsection{Topological spaces as category}

\subsection{Functors}

\subsection{Yoneda functors and Yoneda lemma}
Given a universe $\calu$ and a $\calu$-category $\calc_\calu$ we define two functors:
$$\begin{array}{rrcl}
\hat{h} : &\calc_\calu &\longrightarrow& \hat{\calc}_\calu\\
&&&\\
&C& \longmapsto& \homo{\calc_{\calu}}{\;\cdot\;,C}\\
&&&\\
\hat{k} :&\calc &\longrightarrow& \check{\calc}_\calu\\
&&&\\
&C&\longmapsto&\homo{\calc_{\calu}}{C,\;\cdot\;}\\
\end{array}$$



\subsection{Group objects}
Given a category $\calc$ with initial object $\ast$ and finite products we call an object $G \in \calc$ a group object (in $\calc$) %there are morphisms $e \in \homo{\calc}{\ast,G}$, $m \in \homo{\calc}{G \times G, G}$ and $S : G \longrightarrow G$ if and only if $\mrm{im} m \supset G$ and the following diagrams commute:
if there is a functor $\overline{G} : \calc^{\mrm{op}} \longrightarrow \mrm{Grp}$ such that $G$ represents the composition functor $\mrm{For} \circ \overline{G}$ given in the following diagram:
$$\xymatrix{ \calc^{\mrm{op}} \ar[r]^{\overline{G}} \ar[rd]_{\mrm{For}\;\circ\;\overline{G}}& \mrm{Grp}\ar[d]^{\mrm{For}}\\
&\mrm{Set}.\\
}$$
Here, $\mrm{For}$ is simply the forgetful functor:
$$\mrm{Grp} \longrightarrow \mrm{Set}.$$
Furthermore, representation means that $\overline{G} \simeq h_{\calc}(G) = \homo{\calc}{\;\cdot\;,G}$. Therefore, we may identify $G$ and $\overline{G}$ ($G$ thought of as a functor). Furthermore, there is a functorial isomorphism:
$$G(X) \times G(X) \simeq (G \times G)(X)$$
making $m : G \times G \longrightarrow G$ a morphism in $\hat{\calc}$.
we get functors% $m : \hat{\calc} \times \hat{\calc} \longrightarrow \hat{\calc}$, $e : \hat{h}
\begin{description}
\item[Unity:]
$$\xymatrix{
\ast \times G \ar[rd]_\simeq \ar[r]^{e \times id_G}& G\times G \ar[d]^m&G\times \ast\ar[l]_{id_G \times e}\ar[ld]^\simeq\\
&G&\\
}$$
\item[Associativity]
$$\xymatrix{
G \times G \times G \ar[rr]^{m \times id_G}\ar[d]_{id_G\times m} && G\times G \ar[d]^m\\
G \times G \ar[rr]_m &&G\\
}$$
\end{description}
\subsubsection{Functors and schemes}
Two prominent examples of functors are the following:
\paragraph{Additive group scheme}

\paragraph{Multiplicative group scheme}
Given a commutative ring $R$ with unit, we define the following functor:
$$\mathbb{G}_m : \mrm{CURng} \longrightarrow \mrm{Grp},\ R \longmapsto \mrm{Gl}_1(R).$$
In this case the base scheme is simply the spectrum of $R$ and its affine scheme is:
$$\mrm{Spec} \left(R\left[x,x^{-1}\right]\right)$$
as follows: the ring of Laurent polynomials $R[x,x^{-1}]$ over $R$ has the comultiplication
$$\Delta : R[x,x^{-1}] \longrightarrow R[x,x^{-1}]^{\otimes 2},\ x \longmapsto x \otimes x,$$
and counit
$$\varepsilon : R[x,x^{-1}] \longrightarrow R,\ x \longmapsto 1.$$
Now, we take the dualisation via the spectrum functor as follows:
\begin{enumerate}
\item we pick a minimial prime ideal $\mathfrak{p}$ over zero in $R$ and form a singleton $(\{\mathfrak{p}\},\mathfrak{p})$ and the map:
$$\mrm{Spec}(\varepsilon) : \{\mathfrak{p}\} \longrightarrow \mrm{Spec} R[x,x^ {-1}],\ \mathfrak{p} \longmapsto \left<x - 1\right>$$
Note that if $R$ is some integral domain this chosen prime ideal is naturally the zero ideal. This map becomes our inclusion $e : \ast \longrightarrow R^\times$ of the trivial subgroup.
\item Next, we note that the tensor product $R[x,x^{-1}] \otimes R[x,x^{-1}]$ is isomorphic to $R[x,x^{-1},y,y^{-1}] = S^{-1}_{x,y} R[x,y]$ and we therefore only need to discuss the prime ideals in the latter ring (actually only those in $R[x,x^{-1}]$ as only those will be in the preimage of $\Delta$):
$$\bao{rrcl}
\mrm{Spec}(\Delta) : &\mrm{Spec} (R[x,x^{-1}] \otimes R[x,x^{-1}]) &\longrightarrow &\mrm{Spec}(R[x,x^ {-1}])\\
& \left<a x - 1\right> \otimes \left<b x - 1\right> &\longmapsto & \left<a b x - 1\right>\\
\ea$$
\end{enumerate}

\section{Mathematical Prerequists}
Let $R$ be a unital commutative ring (in literature one finds also unitary with the same meaning, i.e. a ring with one). Let us recall some important definitions from commutative and non-commutative algebra. For simplicity, let $\trm{Hom}$, $\otimes$ denote $\trm{Hom}_R$, $\otimes_R$, resp.
\index{Index}{ring!unital}
\index{Index}{ring!unitary}
\index{Index}{ring!with one}
\subsection{\texorpdfstring{Basics in commutative and\\non-commutative algebra}{Basics in commutative and non-commutative algebra}}
\begin{defi}\label{defi01}
An \tb{$R$-algebra} $A$ is an $R$-module, with an $R$-linear map $\mu : A \otimes A \longrightarrow A$. \bn
\item If $\mu$ is associative, i.e.
$$\xymatrix{
A \otimes A \otimes A \ar[rr]^{id_A \otimes \mu} \ar[d]_{\mu \otimes id_A} & & A \otimes A \ar[d]^{\mu}\\
A \otimes A \ar[rr]_{\mu} &&A\\
}$$
commutes we call $A$ associative.
\index{Index}{algebra!associative}
\item\label{alg_unital} %If there is an element $1_A \in A$, with left- and right action as identity (i.e. $\mu(1_A \otimes\_)= [x \longmapsto 1_A \cdot x] = id_A = \mu(\_ \otimes 1_A) = [x \longmapsto x \cdot 1_A]$) $A$ is said to be \tb{unital}. Alternatively, 
If there is an $R$-linear map $\eta : R \longrightarrow A$ called the unit such that
$$\xymatrix{
R \otimes A  \ar[rr]^{\eta \otimes id_A} \ar[rrd]_\sim& & A \otimes A\ar[d]_\mu& & A \otimes R\ar[ll]_{id_A \otimes \eta} \ar[lld]^\sim\\
&&A&&\\
}$$
commutes, we call $A$ unital.
\index{Index}{algebra!unital}
\item If $\mu(x \otimes y) = \mu(y \otimes x)$ holds for all $x, y \in A$ then $A$ is commutative.
\index{Index}{algebra!commutative}
\item An $R$-submodule $B \subset A$ is an $R$-subalgebra, if $\trm{im} \mu\mid_{B\otimes B} \subset B$.
\item An $R$-subalgebra $I \subset A$ is an left-, right- or two-sided ideal if $\mu(A \otimes I) \subset I$, $\mu(I \otimes A) \subset I$ or both, respectively.
\index{Index}{algebra!ideal}
\item $A$ is (left-, right-,two-sided-) noetherian, if for every chain of ascending (left-, right-, two-sided-) ideals $\ldots \subset I_{n} \subset I_{n+1} \subset \ldots$ in $A$ there is some $m \in \nz$ such that $I_m = I_{m+1}$.
\item $A$ is called artinian if for every chain of descending (left-, right-, two-sided) ideals $\ldots \supset I_n \supset I_{n+1} \supset \ldots$ in $A$ there is some $m \in \nz$ sucht that $I_{m} = I_{m+1}$.
\index{Index}{algebra!noetherian}
\index{Index}{algebra!artinian}
\index{Symbol}{$\mu$}
\index{Symbol}{$\eta$}
\en
\end{defi}
\bmk \label{alg_general} Some alternative remarks:\\
\bn
\item The reader may find an alternative definition of multiplication in the literature, as an $R$-bilinear map $A \times A \longrightarrow A$. However, due to the universal properaty of the tensor product we will use them interchangeably. Furthermore, being a unital algebra is equivalent in demanding a unique element $1_A \in A$ such that its left action $\mu(1_A \otimes \_) = [a \longmapsto \mu(1_A \otimes a)]$ is identity on $A$.%We may define an algebra $A$ as follows: $A$ is an $R$-module with an $R$-bilinear map $\cdot : A \times A \longrightarrow A$. However, due to the universal property of the $R$-tensor product, we get the following commuting diagram:
%$$\xymatrix{A \times A \ar[rd]_\cdot\ar[r]^\otimes & A \otimes A \ar[d]^\mu\\&A\\}$$
%meaning, we may either start with an $R$-bilinear map $\cdot : A \times A \longrightarrow A$ and define $\mu(a \otimes b) := \cdot(a,b)$ or start with an $R$-linear map $\mu$ and the tensor product and may define $\cdot(a,b) := \mu(\otimes(a,b))$. Hence, both definition are equivalent. Therefore, we call both maps multiplication and use them interchangeably.\\
\item Commutativity can be rephrase as a commutative diagram wrt. to the flip isomorphism:
$$\tau : A \otimes A \longrightarrow A\otimes A,\ a \otimes b \longmapsto b \otimes a:$$
$A$ is commutative if and only if
$$\xymatrix{
A \otimes A \ar[r]^\tau \ar[rd]_{\mu} & A \otimes A \ar[d]^\mu\\
&A\\
}$$
commutes.
\index{Index}{isomorphis!flip}
\index{Symbol}{$\tau$}
%\item \label{alg_unital} For any unital $R$-algebra $A$ we define an additional $R$-linear map:
%$$\eta : R \longrightarrow A,\ 1_R \longmapsto 1_A,$$
%with commuting diagram:
%$$\xymatrix{
%R \otimes A \ar[r]^{\eta \otimes id_A}\ar[rd]_\sim & A \otimes A\ar[d]^\mu & A \otimes R\ar[l]_{\id_A \otimes \eta}\ar[ld]^\sim\\
%&A.&\\
%}$$
%\en
\item The triplet $(A,\mu,\eta)$ characterizes uniquely any unital, associative algebra over $R$.
\en
From our definitions we get immediately
\begin{prop}\label{prop01}
Let $(A,\mu,\eta)$ be an unital associative $R$-algebra and $T A$ denote the tensor algebra over $A$. There is a unique two-sided ideal $I A \subset T A$ wrt. $\mu$, s.t. $T A /I A \simeq A$.% up to isomorphism.
\end{prop}
%\commt{
\bws We offer two alternative proofs.
\bn
\item Note, that $T A = \sum_{n \geq 0} A^{\otimes n}$ - in particular, we have $A^{\otimes 0} = R \simeq R.1_A$, $A^{\otimes 1} = A$. Thus, $T A = \bigoplus_{n \geq 1} A^{\otimes n}$. Then, put $J A := \left<a \otimes b - a b : a, b \in A\right>$. Now, it is enough to show that $J A = I A$. First, let us define $\varphi : T A \longrightarrow T A/I A$ and the degree map
$$\deg : \bigcup_{i \geq 0} A^{\otimes i} \longrightarrow \zz \cup \{-\infty\},\ x \in A^{\otimes i} \longmapsto \begin{cases}
i & x \neq 0\\
-\infty & \trm{else}\\
\end{cases},$$ extended on $T A\bsl\{0\}$ by $x = \sum_i x_i \longmapsto \max\{\deg x_i : x_i \in A^{\otimes i}\bsl\{0\}\}$,
 being sub additive (additive in the case $A$ is a domain). Then by definition, all elements in our quotient algebra $B: = T A/J A$ are represented by elements of at most degree 1, i.e. $\ov{x} = x + J A \in B \RA \deg x \leq 1$. Take the inclusion $f : J A \longrightarrow T A$ and the projection $\pi : T A \longrightarrow B$ defining a short exact sequence of (ass) $R$-algebras:
$$0 \longrightarrow J A \stackrel{f}{\longrightarrow} T A \stackrel{\pi}{\longrightarrow} B \longrightarrow 0.$$
For the inclusion map $\iota : A \longrightarrow T A$ we clearly have $\pi \circ \iota \equiv id_A$ which shows the claim.\\
\item We recall the universal property of the $R$-tensor product: i.e.
$$\xymatrix{
A \times A \ar[rd]_{\cdot}\ar[r]^{\otimes} & A \otimes_R A\ar[d]^{\mu}\\
& A\\
}$$
commutes. Hence, we may define:
$$A \otimes_A A := A \otimes_R A/\sim,\ \trm{where}\ \sim := \left\{(a \gamma \otimes b, a \otimes \gamma b) \in \left(A^{\otimes_R 2}\right)^2 : \gamma \in A\right\}$$
and by universal prop.: $A \simeq A\otimes_A A$ and therefore
$$\xymatrix{
A \times A \ar[r]\ar[d] & A \otimes_R A \ar[d]\ar[ld]_{\mu}\\
A & A \otimes_A A.\ar[l]_{\sim}\\
}$$
Hence, the two-sided $A$-submodule $I A := \sum_{a \otimes b \in A \otimes A}T A.(a \otimes b - a b). T A$ is also a two-sided $T A$-submodule and has the required property:
$$T A/I A \simeq A,\ a \otimes 1_A \equiv 1_A \otimes a \equiv a, a \otimes b \equiv ab \mod I A,\ \forall a, b \in A.$$
Summarizing, we get the following commuting diagram:
$$\xymatrix{
A \ar@{^{(}->}[r]^\iota \ar[rd]_\sim& T A\ar[d]^\pi\\
& T A/I A\\
}$$
%It suffice to show that $I A$ is stable under $A$-automorphisms, i.e. a $G := \trm{Aut}_{R-\trm{alg}}(A)$-equivariant space/module.
%\indent By definition, $I A$ is stable under conjugation, i.e. $G' = \left\{\varphi_h \in G : \varphi = \left[g \longmapsto h^{-1} g h\right], h \in A^\times\right\}$, since:
%$$a \otimes b - a b \equiv 0 \mod I A\LRA \gamma^{-1} a \gamma \otimes \gamma^{-1} b \gamma - \gamma^{-1} a b \gamma \equiv 0 \mod I A.$$
%Now we assume there is a $\varphi \in \trm{Out}(A)$, where $\trm{Out}(A) := G/G'$ s.t. $\varphi(a) \otimes \varphi(b) - \varphi(a) \varphi(b) \nequiv 0 \mod I A$. First, we extend $G$ on $T A$ by simply defining
%$$\bao{crrcl}
%\hat{\varphi} = \sum_{n \geq 1} \hat{\varphi}_n,&\hat{\varphi}_n : &T^n A &\longrightarrow& T^n A\\
%&&&&\\
%&&a_{i_1} \otimes \ldots \otimes a_{i_n} &\longmapsto& \varphi(a_{i_1}) \otimes \ldots \otimes \varphi(a_{i_n}),\\
%&&&&\\
%&&\hat{\varphi}_n\mid_{T A \bsl T^n A} &=& 0 \in \trm{End}(T A)\\
%\ea$$
%The quotient $I A/G' := \sum T A.(a \otimes b - a b). T A/G' = \{[c a c' \otimes d b d' - c a c' d b d'] : c, c', d, d' \in T A, \varphi_h(c a c') \otimes \varphi_h(d b d') - \varphi_h(c a c') \varphi_h(d b d') \in [c a c'\otimes d b d' - c a c' d b d'],\ \forall \varphi_h \in G'\}$.
\en
\begin{defi}\label{defi04}
Let $A$ be an algebra.
\bn
\item $A$ is called graded if there exist submodules $A_n$, such that
$$A = \bigoplus_{n\geq 0} A_n\ \trm{and}\ \mu_A(A_n \otimes A_m) \subset A_{n+m}.$$
\item $A$ is called filtered if there exist submodules $A^{\leq n}$, such that
$$A = \bigcup_{n \geq 0} A^{\leq n}\ \trm{and}\ \mu_A(A^{\leq n} \otimes A^{\leq m}) \subset A^{\leq n + m}.$$
\item For a filtered algebra $\mathcal{A} = \bigcup \mathcal{A}^{\leq n}$ we call
$$\trm{gr} \ \mathcal{A} := \bigoplus_{n \geq 1} A^n,\ \trm{where}\ A^n := \mathcal{A}^{\leq n}/\mathcal{A}^{\leq n - 1}$$
the associated graded algebra. Multiplication $\mu_{\trm{gr} \mathcal{A}}$ is defined by $[x y] \in A^{n + m}$ for all $x \in \mathcal{A}^{\leq n}, y \in \mathcal{A}^{\leq m}$.
\item An algebra $A$ is called a deformation of a filtered algebra $\mathcal{A}$ if $\trm{gr} \mathcal{A} \simeq A$.
\en
\index{Index}{algebra!graded}
\index{Index}{algebra!filtered}
\index{Index}{algebra!associated graded}
\end{defi}
\bmk Deformation theory is fundamental in the study of singularity theory. For instance, deformation theory of Kleinian singularities $\cz^2/\Gamma$ leads to the classification scheme also found in simple Lie algebras (the ADE-system), where $\Gamma$ is a finite subgroup of $\trm{Sl}_2(\cz)$.\\
\indent The submodules $A_n$ for a given grading are sometimes called the homogeneous submodules and its elements homogeneous.
\subsubsection{Algebras and their modules}
Let $(A,\mu,\eta)$ be an $R$-algebra and $M$ an $R$-module. $A$ is assumed to be associative and unital.
\begin{defi}
$M$ is called a left A-module, if there is a homomorphism $\rho : A \longrightarrow \trm{End}_R(M)$ of $R$-algebras - i.e. the following diagrams commute:
$$\bao{cc}
\xymatrix{
A \otimes A \otimes M \ar[rr]^{\id_A \otimes \Psi}\ar[d]_{\mu_A \otimes id_M} && A \otimes M \ar[d]^\Psi\\
A \otimes M\ar[rr]_\Psi && M\\
} & 
\xymatrix{
R \otimes M \simeq M \ar[rrd]_\sim \ar[rr]^{\eta \otimes id_M} && A \otimes M \ar[d]^\Psi \\
&&M\\
}
\ea$$
where $\Psi : A \otimes M \longrightarrow M, a \otimes m \longmapsto \rho(a)(m)$.
\end{defi}
\bmk A right $A$-module is constructed similarly (just reversing sides). The pair $(M,\rho)$ is called an $A$-representation. The associativity is not needed for its definition, hence modules over non-associative algebras are permitted.
\subsection{Ore Extensions}
Unless mentioned otherwise, $A$ is always an unital associative algebra over $R$. Let $X$ be a non-empty set. The $R$-algebra $T X$ is defined as the tensor algebra
$$T X := \bigoplus_{n \geq 0} M(X)^{\otimes n},\ \trm{where}\ M(X) := \bigoplus_{x \in X} R.x \ \trm{and}\ M(X)^{\otimes 0} := R.$$
\begin{defi}\label{defi02}
Let $A[X]$ be the left $A$-module $A \otimes T X$ with left $A$ action $a \cdot 1_A \otimes x = a \otimes x$, for all $x \in T X$, $a \in A$.
\bn
\item For an injective algebra homomorphism $\alpha : A \longrightarrow A$ an $\alpha$-derivation $\delta : A \longrightarrow A$ is an $R$-linear map such that
$$\delta(a b) = \alpha(a) \delta(b) + \delta(a) b\ \forall a, b \in A.$$
\item If in addition, $\alpha$ is surjective, the triple $A[X, \alpha, \delta]$ is called the Ore extension of $A$ with
$$x a = \alpha(a) x + \delta(a), \forall x \in X, a \in A.$$
In other words, $A[X]$ has a unique right $A$-module structure, as well, defining an $A$-algebra.% with structure maps:
%$$\bao{rrclcl}
%\eta :& A & \longrightarrow&A[X]\\
%& 1_A&\longmapsto&1_A x^0 =: 1_{A[X]}\\
%&&&\\
%\mu_{A[X]} :&A[X] \otimes A[X] & \longrightarrow& A[X]\\
%& a x \otimes b y &\longmapsto& a \alpha(b) x y + a \delta(b) y,\ \forall x, y \in X,\ a, b \in A\\
%&a x \otimes b y &\longmapsto& %a x_1 \ldots x_n b y\\
%%& &=& a x_1 \ldots x_{n-1} (\alpha(b) x_{n} + \delta(b)) y\\
 %%&&=& 
%a \alpha^n(b) x_1 \ldots x_n y + a \delta(\alpha^{n-1}(b)) x_2 \ldots x_n y +\ldots\\
%&&&a \alpha^{n-1}(\delta(b)) x_1 \ldots x_{n-1} y + a \delta^2 (\alpha^{n-2}(b) x_3 \ldots x_n y + \\
%&&&a \delta(\alpha(\delta(\alpha^{n-3}(b)))) x_2 x_4 \ldots x_n
%\ea$$
\en
\index{Index}{extension!Ore}
\index{Symbol}{$A[X,\alpha,\delta]$}
\end{defi}
Clearly, the definition of an Ore extension is analogous to the situation of adjoining a transcendental element to a field/ring as in polynomial rings. However, as $A$ or the action of $\alpha$ does not need to be commutative there are obstacles with regard to right and left-sidedness. The aim is to define a right $A$-module structure on $A[X]$. Firstly, we identify $A \otimes X$ with $A \otimes M(X)$ and set $B^n := (A \otimes X)^{\otimes n} \otimes A$ for all $n \geq 1$ and $B^0 = A$.
%\begin{lemm}
%The two-sided $A$-module $B$, where $B := \bigoplus_{i \geq 0} B^i$, has a natural $A$ algebra structure given by the $A$ linear maps:
%\scriptsize{
%$$\mu^{i,j} : B^i \otimes B^j \longrightarrow B^{i+j},\ (a_{i_0} \otimes x_{i_1} \otimes a_{i_1} \otimes \ldots \otimes a_{i_{n-1}} \otimes x_{i_n} \otimes a_{i_n} $$
%}
%\end{lemm}
\begin{prop}\label{prop02}
Let $B$ be the two sided $A$-module, as above:%generated by $A$ and $X$, i.e. $B^n = (A \otimes X)^{\otimes n} \otimes A$ and
$$B = \bigoplus_{n \geq 0} B^n.$$
Then $B$ has a graded $A$-algebra structure via multiplication
$$\bao{rrcl}
\mu^{n,m} :& B^n \otimes B^m &\longrightarrow& B^{n+m}\\
&&&\\
&a_{i_0} \otimes x_{i_1} \otimes \ldots \otimes  x_{i_n} \otimes a_{i_{n}} & & a_{i_0} \otimes x_{i_1} \otimes \ldots \otimes  x_{i_n} \otimes a_{i_{n}}\\
&\otimes &\longmapsto&\\
& b_{j_0} \otimes x_{j_1} \otimes \ldots \otimes  x_{j_m} \otimes b_{j_{m}} & &  b_{j_0} \otimes x_{j_1} \otimes \ldots \otimes  x_{j_m} \otimes b_{j_{m}}\\
\ea$$
for all $a_{i_k}, b_{j_l} \in A$, $x_{i_k}, x_{j_l} \in X$ and $1 \leq k \leq n, 1 \leq l \leq m$.
\end{prop}
\bws The proof is straightforward if we put $\mu^{n,m}\mid_{B \otimes B \bsl B^n \otimes B^m} = 0$ and $\mu = \sum_{n,m \in \nz_0} \mu^{n,m}$.
\begin{prop}\label{prop03}
Let $B$ be the algebra as above. The Ore extension $A[X,\alpha,\delta]$ is isomorphic to
$B/I(\alpha,\delta)$ where
$$I(\alpha,\delta) := \left<1_A \otimes x \otimes a - \alpha(a) \otimes x \otimes 1_A - \delta(a): x \in X, a \in A\right>$$
is a two-sided ideal in $B$. Moreover, if $A$ is zero divisor free then so is $A[X,\alpha,\delta]$.
\end{prop}
\bws Comes in several steps:
\bn
\item First, let us compute $\delta(1_A) = \delta(1_A \cdot 1_A) = \alpha(1_A) \delta(1_A) + \delta(1_A) 1_A$. Equivalently, $0 = \alpha(1_A) \delta(1_A)$. But being a monomorphism $\alpha(1_A) = 1_A$, hence $\delta(1_A) = 0$.
\item Put $\iota_l : A[X] \longrightarrow B$, $a \otimes x_i := a \otimes x_{i_1} \ldots x_{i_n}  \longmapsto a \otimes (x_{i_1} \otimes 1_A) \otimes \ldots \otimes (x_{i_n} \otimes 1_A)$. This clearly defines a left $A$-module homomorphism. Similarly, we get $\iota_r : [X]A \longrightarrow B$, $x_i \otimes a = x_{i_1} \ldots x_{i_n} \otimes a \longmapsto (1_A \otimes x_{i_1}) \otimes \ldots \otimes (1_A \otimes x_{i_n}) \otimes a$ a right $A$-module homomorphism. Both morphisms are injective as we readily see by our definition. Thus, we see that
$$A[X], [X]A \subset B/I(\alpha,\delta).$$
Moreover, $A[X]$ is a $T X$-right module and $[X]A$ is a $T X$-left module.
\item Set $C := B/I(\alpha,\delta)$ and $\varphi : B \longrightarrow C$.
\bd
\item[Claim:] $\varphi$ is an $A$-algebra morphism. We are going to proof this claim sequentially.
\bn
\item $\varphi$ is a two-sided $A$-module homomorphism and in particular, we have $\varphi\mid_{A[X]} \circ \iota_l \equiv id_{A[X]}$. As we just proved both single sided $A$-modules, $A[X], [X]A$, have isomorphic images in $B$ and its quotient algebra. By our definition, $\varphi$ respects the left $A$-module structure. So let us proof the right $A$-module morphism. As we just saw there is a right $A$ submodule $\iota_r(X \otimes A) = 1_A \otimes X \otimes A$. Its image under $\varphi$ is the subset $\varphi(X \otimes A)$. But, no matter if we first apply the right action of $A$ on our monomials $x_i \in X$ or apply the action on $\varphi(x_i \otimes a)$ we still get the same value:
$$\alpha(a) x_i + \delta(a) = \mu_C(\varphi(x_i) \otimes \varphi(a)) = \varphi(x_i \otimes a) \in C.$$
In particular we have
$$\varphi(X \otimes A) \simeq \varphi(1_A \otimes X \otimes A) \subset \left\{\alpha(a) \otimes x \otimes 1_A + \delta(a) : a \in A,\ x \in X\right\} \subset C^{\leq 1}.$$
This extends to all elements in $[X]A$ as follows: pick $x_{i_1} \ldots x_{i_n} a \in X^{\otimes n} \otimes A$ and identify with $1_A \otimes x_{i_1} \otimes 1_A \otimes \ldots \otimes 1_A \otimes x_{i_n} \otimes a \in B^n$. Then, iteratively apply the following rule - if $I = \{i_j : 1 \leq j \leq n\}$ is the index set of our monomial $x_i \in T X$ and $M := \trm{map}(I,\{0,1\}) = \{0,1\}^I$, then:
$$x_{i_1} \ldots x_{i_n} a = \sum_{\substack{m \in M}} w_{m}(a) x_{i_1}^{m(i_1)} \ldots x_{i_n}^{m(i_n)} \in C,\ \trm{where}$$
$$w_{m} := w_{m(i_1)} w_{m(i_2)} \ldots w_{m(i_n)},\ w_{m(i_j)} = \delta^{1-m(i_j)} \alpha^{m(i_j)},$$
where multiplication simply means composition. 
%and $w_{k \in \nz^0} = \delta^n$ or, equivalently, having deleted all monomials from $x_1 \ldots x_n$.
Easily proved by induction: assuming we have the above formular in case $X^{\otimes n} \otimes A$, then for $x_{i_0} x_{i_1} \ldots x_{i_n} a \in X^{\otimes n + 1} \otimes A$ we get:
\begin{align*}
x_{i_0} x_{i_1} \ldots x_{i_n} a &= x_{i_0} \sum_{m \in M} w_k (a) x_{i_1}^{m(i_1)} \ldots x_{i_n}^{m(i_n)}\\
 &= \sum_m \alpha( w_m(a) ) x_{i_0}^1 x_{i_1}^{m(i_1)} \ldots x_{i_n}^{m(i_n)}\\ & + \sum_m \delta(w_m(a)) x_{i_0}^0 x_{i_1}^{m(i_1)} \ldots x_{i_n}^{m(i_n)}&&\\
\end{align*}
Defining two functions $\wt{m}_{0,1} : I \cup \{i_0\} \longrightarrow \{0,1\}$, with $\wt{m}_{0,1}\mid_I = m$ and $\wt{m}_0(i_0) = 0, \wt{m}_1(i_0) = 1$ we get two extensions for each $m \in M$ on $I \cup \{i_0\}$:
\begin{align} 
x_{i_0} \ldots x_{i_n} a &= \sum_{\wt{m} \in \wt{M}} w_{\wt{m}} (a) x_{i_0}^{\wt{m}(i_0)} \ldots x_{i_n}^{\wt{m}(i_n)},\\
\end{align}
where $\wt{M} = \{0,1\}^{I \cup \{i_0\}}$.
\item By definition $B$ is a graded $A$-algebra with a filtration induced by the grading. This filtration is obviously kept under $\varphi$ ($\trm{im} \mu\mid_{B^{\leq n}} \subset C^{\leq n}$). Thus, we will show that the following diagram commutes:
$$\xymatrix{
B \otimes B \ar[d]_{\mu_B}\ar[r]^{\varphi \otimes \varphi} & C \otimes C\ar[d]^{\mu_C} \\
B \ar[r]_{\varphi} & C.\\
}$$
\newcommand{\mmap}{\mathfrak{m}}
\newcommand{\nmap}{\mathfrak{n}}
Omitting the tensor symbol and ones, the image of any element $x_i a \otimes x_j b := x_{i_1} \ldots x_{i_m} a \otimes x_{j_1} \ldots x_{j_n} b \in B^m \otimes B^n$ is:
$$\bao{rrcl}
&\varphi \otimes \varphi(x_i a \otimes x_j b) &=& \sum_{\mmap \in M, \nmap \in N} w_\mmap(a) x_{i_1}^{\mmap(i_1)} \ldots x_{i_m}^{\mmap(i_n)}\\
&&& \otimes w_\nmap(b) x_{j_1}^{\nmap(j_1)} \ldots x_{j_n}^{\nmap(j_n)}\\
&&&\\
\RA&\mu_C(\varphi \otimes \varphi)(x_i a \otimes x_j b) &=& \sum_{\mmap, \mmap' \in M, \nmap\in N} w_\mmap(a) w_{\mmap'}(w_\nmap(b)) x_{i_1}^{\mmap(i_1) \cdot \mmap'(i_1)} \ldots x_{i_m}^{\mmap(i_m) \cdot\mmap'(i_m)}\\
&&&\cdot x_{j_1}^{\nmap(j_1)} \ldots x_{j_n}^{\nmap(j_n)},\\
\ea$$
where $M = \{0,1\}^{\{i_1,\ldots,i_m\}}, N = \{0,1\}^{\{j_1,\ldots,j_n\}}$ and $w_q$ as above. The lower part of the diagram yields:
$$\bao{rrcl}
&\mu_B(x_i a \otimes x_j b) &=& x_i a x_j b\\
&&&\\
\RA&\varphi(x_i a \otimes x_j b) &=& \sum_{\nmap \in N} w_\mmap(a) x_i^{\mmap(i)} w_\nmap(b) x_j^{\nmap(j)}\\
&&&\\
&&=& \sum_{\mmap, \mmap' \in M, \nmap \in N} w_\mmap(a) w_{\mmap'}(w_\nmap(b)) x_i^{\mmap(i) \cdot \mmap'(i)} x_j^{\nmap(j)},\\
\ea$$
where $x_i^{\mmap(i)} = x_{i_1}^{\mmap(i_1)} \ldots x_{i_m}^{\mmap(i_m)}$, etc. Obviously, the commutativity holds for all algebra generators. Hence, $\varphi$ is an $A$-algebra homomorphism.
%$$\bao{rcl}
%\varphi \otimes \varphi(a\otimes b) &=& \sum_{n_k \in X_k,m_l \in Y_l} a_{i_0} w_{n_1}(a_{i_1}) \ldots w_{n_n}(a_{i_n}) x_{i_1}^{\prod_k n_k(i_1)} \ldots x_{i_n}^{\prod_k n_k(i_n)}\\
%&&\\
%&& \otimes b_{j_0} w_{m_1}(b_{j_1}) \ldots w_{m_m}(b_{j_m}) x_{j_1}^{\prod_l m_l(j_1)} \ldots x_{j_m}^{\prod_l m_l(j_m)},
%\ea$$
%where $X_k = \{0,1\}^{\{i_1,\ldots,i_k\}}, Y_l = \{0,1\}^{\{j_1,\ldots,j_l\}}$ and each $w_q$ is defined as above. The product is
%$$\bao{rcl}
%\mu_C(\varphi \otimes \varphi(a\otimes b)) &=&  \sum_{n_k, n'_k \in X_k,m_l \in Y_l} a_{i_0} w_{n_1}(a_{i_1}) \ldots w_{n_n}(a_{i_n}) w_{n'_0}(b_{j_0})\\&&\\
%&& w_{n'_1}(w_{m_1}(b_{j_1})) \ldots w_{n'_n}(w_{m_m}(b_{j_m})) \\
%&&\\
%&& x_{i_1}^{\prod_k n_k(i_1) \cdot n'_k(i_1)} \ldots x_{i_n}^{\prod_k n_k(i_n)\cdot n'_k(i_1)} x_{j_1}^{\prod_l m_l(j_1)} \ldots x_{j_m}^{\prod_l m_l(j_m)}.
%\ea$$
%Commuting the lower part of the diagram:
%$$\mu_B (a \otimes b) = a_{i_0} x_{i_1} \ldots x_{i_n} a_{i_n} b_{j_0} x_{j_1} \ldots x_{j_m} b_{j_m}.$$
%The image is then:
%$$\varphi(\mu_B(a \otimes b)) = 
%We have $\varphi(a_{i_0} \otimes x_{i_1} \otimes 1_A) \varphi(a_{i_1} \otimes x_{i_2} \otimes a_{i_2}) = \varphi(a_{i_0} \otimes x_{i_1} \otimes a_{i_1}) \varphi(1_A \otimes x_{i_2} \otimes a_{i_2})$ for all $a_{i_k} \in A, x_{i_k} \in X$. First we observe that $\varphi\mid_{B^0} = id_{B^0}$ and $\varphi(B^1) \subset A \otimes X \otimes 1_A \oplus A$. Also note that the equivalence equals the statement $\varphi = \mu_C(\varphi \otimes \varphi) = \varphi \mu$. Then we compute
%$$\bao{rcl}
%\varphi(a_{i_0} \otimes x_{i_1} \otimes a_{i_1} \otimes x_{i_2} \otimes a_{i_2}) &=& a_{i_0} \alpha(a_{i_1}) \alpha^2(a_{i_2}) x_{i_1} x_{i_2} + \alpha(a_{i_1}) \alpha \delta(a_{i_2}) x_{i_1}\\
%&&\\
%&& + \delta(a_{i_1} \alpha(a_{i_2})) x_{i_2} + \delta(a_{i_1} \delta(a_{i_2}))\\
%&&\\
%\varphi(a_{i_0} \otimes x_{i_1} \otimes 1_A) \varphi(a_{i_1} \otimes x_{i_2} \otimes a_{i_2}) &=& a_{i_0} x_{i_1} a_{i_1} (\alpha(a_{i_2}) x_{i_2} + \delta(a_{i_2}))\\
%&&\\
%&=& a_{i_0} x_{i_1} a_{i_1} \alpha(a_{i_2}) x_{i_2} + a_{i_0} x_{i_1} a_{i_1} \delta(a_{i_2})\\
%&&\\
%&=& a_{i_0} \left[\alpha(a_{i_1} \alpha(a_{i_2})) x_{i_1} + \delta(a_{i_1} \alpha(a_{i_2})) \right] x_{i_2}\\
%&&\\
%&& + a_{i_0} \left[\alpha(a_{i_1} \delta(a_{i_2}))x_{i_1} + \delta(a_{i_1} \delta(a_{i_2}))\right]\\
%&&\\
%\ea$$
%$$\bao{rcl}
%\varphi(a_{i_0} \otimes x_{i_1} \otimes a_{i_1}) \varphi(1 \otimes x_{i_2} \otimes a_{i_2}) &=& a_{i_0} (\alpha(a_{i_1}) x_{i_1} + \delta(a_{i_1})) (\alpha(a_{i_2}) x_{i_2} + \delta(a_{i_2}))\\
%&&\\
%&=& a_{i_0} \left(\alpha(a_{i_1}) x_{i_1} \alpha(a_{i_2}) x_{i_2} + \alpha(a_{i_1}) x_{i_1} \delta(a_{i_2})\right)\\
%&&\\
%&& + a_{i_0} \left(\delta(a_{i_1}) \alpha(a_{i_2}) x_{i_2} + \delta(a_{i_1}) \delta(a_{i_2})\right)\\
%&&\\
%&=& a_{i_0} \alpha(a_{i_1}) [\alpha^2(a_{i_2}) x_{i_1} + \delta(\alpha(a_{i_2}))] x_{i_2}\\ &&\\
%&& + a_{i_0} \alpha(a_{i_1}) [\alpha(\delta(a_{i_2})) x_{i_1} + \delta(\delta(a_{i_2}))]\\
%&&\\
%&& + a_{i_0} \left(\delta(a_{i_1}) \alpha(a_{i_2}) x_{i_2} + \delta(a_{i_1}) \delta(a_{i_2})\right)\\
%&&\\
%&=& a_{i_0} \left(\alpha(a_{i_1} \alpha(a_{i_2})) x_{i_1} x_{i_2} + \alpha(a_{i_1} \delta(a_{i_2})) x_{i_1}\right)\\
%&&\\
%&& + a_{i_0} \underbrace{\left[\alpha(a_{i_1}) \delta(\alpha(a_{i_2})) + \delta(a_{i_1}) \alpha(a_{i_2})\right]}_{\delta(a_{i_1} \alpha(a_{i_2}))} x_{i_2}\\
%&&\\
%&& + a_{i_0} \underbrace{\left[\alpha(a_{i_1}) \delta(\delta(a_{i_2})) + \delta(a_{i_1}) \delta(a_{i_2})\right]}_{\delta(a_{i_1} \delta(a_{i_2}))}\\&&\\
%\ea$$
%$$\bao{rcl}
%\varphi(a_{i_0} \otimes x_{i_1} \otimes a_{i_1}) \varphi(1 \otimes x_{i_2} \otimes a_{i_2}) &=& \varphi(a_{i_0} \otimes x_{i_1} \otimes 1_A) \varphi(a_{i_1} \otimes x_{i_2} \otimes a_{i_2})\\
%\ea$$
%In addition, $\varphi(\varphi(a_{i_0} \otimes x_{i_1} \otimes a_{i_1}) \otimes x_{i_2} \otimes a_{i_2}) = \varphi(a_{i_0} \otimes x_{i_1} \otimes \varphi(a_{i_1} \otimes x_{i_2} \otimes a_{i_2}))$. Hence, our claim above is just shown for $B^{\leq 2}$. The claim can thus be iteratively extended to all $A$ submodules $B^i$ showing that $\varphi$ is indeed an algebra homomorphism. Moreover, the two-sided $A$ submodule $I(\alpha,\delta)$ is a two-sided ideal in $B$.
%\item The ideal $\ker \varphi \subset B$ equals $I(\alpha,\delta)$. Set $m(a,x) = 1_A \otimes x \otimes a - \alpha(a) \otimes x \otimes 1_A - \delta(a)$. By definition all elements $f \in I(\alpha,\delta)$ are of the form $f = \sum_{i \in \nz_0} f_i m(x,a) g_i$. Multiplicativity we have $\varphi(f) = \sum \varphi(f_i) \varphi(m(x,a)) \varphi(g_i) = \sum \varphi(f_i) (\varphi(1_A \otimes x \otimes a) - \alpha(a) \otimes x \otimes 1_A - \delta(a)) \varphi(g_i) = \sum \varphi(f_i) \cdot 0 \cdot \varphi(g_i) = 0$. Therefore, $f \in \ker \varphi$. Conversely, let $f \in \ker \varphi \bsl I(\alpha,\delta)$
\en
\ed
\item By our proposition \ref{prop01} that every algebra $A$ has an ideal $IA$ in its tensor algebra $TA$, such that $TA/IA \simeq A$ it is enough to show that the following diagram commutes:
$$\xymatrix{
B \ar[d]\ar[r]_\varphi & C\ar[ld]^{\stackrel{?}{\sim}}\ar[d]^\iota\\
A[X,\alpha,\delta]& T C \ar[l]^\pi.\\
}$$
But clearly, the ideal $I C = \left<a \otimes b - a b : a, b \in C\right>$ being the kernel of $\pi : TC \longrightarrow C$, is equally represented as
$$I C = \left<\underbrace{a \otimes x - a x}_{A-\trm{left}\ \trm{module}}, \underbrace{x \otimes a - \alpha(a) x - \delta(a)}_{A-\trm{right}\ \trm{module}} : a \in A, x \in X\right>,$$
where we explicitly give the relations on the tensor elements in $T C$.
%$B \simeq T B^1$, for
%$$IA := \left<1 \otimes x \otimes a - \alpha(a) \otimes x \otimes 1 - \delta(a) : a \in A, x \in X\right>.$$
%On the other hand, $I C := \left<a \otimes b - a b : a, b \in C\right>$.
\item If $\trm{Ann}_R(A) = \{0\}$ then: $a b = 0$ if and only if $a = 0$ or $b = 0$, i.e. $A$ is a domain. Then by definition we have for $f, g \in A[X,\alpha,\delta]$ $lt(f g) \in C^{\deg f + \deg g}$, where $lt(f)$ is the leading term of $f$ - i.e. the monomial of highest degree. If $f g = 0$ then $\deg (f g) = \max \underbrace{\{i \in \nz_0 : \sum_{|k| = i} f_j \wt{g}_{i-j} \neq 0\}}_{= \emptyset} = -\infty$ implies either $f_{i_1} = 0$ or $g_{i_2} = 0$ for all $0 \leq i_1 \leq \deg f, 0 \leq i_2 \leq \deg g$ concluding our proof.
%we conclude the $I(\alpha,\delta) = IA[X,\alpha,\delta]$.
%First, let us reassure that $A[X] \subset B$. Consider the element $a \otimes (x \otimes 1_A)^{\otimes n}$. Then by iteratively applying the definition we have 
%$$a \otimes (x \otimes 1_A)^{\otimes n - 2} \otimes x \otimes (1_A\cdot(\alpha(1_A) + \delta(1_A)) \otimes x \otimes 1_A = \ldots = a \sum_{\gamma_{i_j} \in \{\alpha,\delta\}} \prod_{j=1}^n \gamma_{i_j}(1_A) \otimes (x \otimes 1_A)^{\otimes n}$$
%Obviously, we have $1_A = \alpha(1_A) + \delta(1_A)$. Being an algebra monomorphism $\alpha(1_A) = 1_A$ thus $\delta(1_A) = 0_A$. Therefore, we may identify $a \otimes(x\otimes 1_A)^{\otimes n}$ with $a \otimes x^n$.
%a \otimes (x \otimes 1_A)^{\otimes n - 3} \otimes x \otimes (\alpha(\alpha(1_A) + \delta(1_A)) + \delta(\alpha(1_A) + \delta(1_A))) \otimes x \otimes 1_A \otimes x \otimes 1_A$
\en
\bsp We want to show two prominent examples for some field $R$ - the polynomial ring $R[X_1,\ldots,X_n]$ and the tensor algebra $T(R^n)$. In both cases, $\alpha = id_R$ and $\delta = 0_R$. Thus, the verification both being Ore extensions of $R$ is straightforward.\\
\indent A more complicated situation we find in the so called quantum algebras: let $q \in R^\times \bsl\{1\}$ and $A = R[X]$. Then the Ore extension $A[Y, \alpha, \delta]$, with
$$\alpha = [X^i \longmapsto q^i X^i] \in \trm{Aut}_{R-\trm{alg}}(A)\ \forall i \geq 0,\ \delta \equiv 0,$$
defines a non-commutative algebra - as $Y X = q X Y$ implies.
\begin{prop}\label{prop04}
If $A$ is noetherian and $|X| < \infty$ then $A[X,\alpha,\delta]$ is noetherian.
\end{prop}
\bws If $|X| < \infty$ the module $M(X)$ is noetherian over $R$. In particular, it is an $R$-module free of rank $|X|$. Thus, it suffices to show that $A \otimes X$ is noetherian as a left $A$-module. If $A$ is noetherian, then each $a \in A$ is a finite $R$-linear combination of only finitely many generators $B_A := \{a_i\} \subset A$. Hence if $\left<a_i : i \in I\right> = A$, $|I| < \infty$, then
$$A \otimes X = \bigoplus_{\substack{i \in I\\x \in X}} A.a_i.A \otimes x$$
is also finitely generated. In particular for $\rank A < \infty$, we have $\rank A \otimes X = \rank A \cdot |X|$, since $M(X)$ was a free $R$-module of rank $|X|$. Therefore, we get that each finite $A$ linear combination of elements in $B$ is also finitely generated over $R$. But being noetherian is kept under epimorphisms, hence we get our claim.
\subsection{Derivations and Lie Algebras}
For the sake of clarity, we repeat:
\begin{defi}\label{defi05}
For an algebra $A$ an $R$-derivation is a map $D$ with
\bn
\item $D \in \trm{End}_R(A)$,
\item $D(x y) = D(x) y + x D(y)$ (Leibniz rule).
\en
Furthermore, we have
\bn
\item the set of $R$-derivations is denoted by $\trm{Der}_R(A)$,
\item $A$ is called a (non-trivial) differential algebra over $R$ if $\trm{Der}_R(A) \neq \{0\}$,
\en
\index{Index}{derivation}
\index{Index}{algebra!differential}
\index{Symbol}{$\trm{Der}_R(A)$}
%\index{Symbol}{$\trm{Der}_R(A)$}
\end{defi}
Any ring $R$ defines a trivial differential algebra over itself, taking the zero-homomorphism as $R$-derivation.
%\bmk Note, an $R$-derivation is an $id_A$ derivation in the sense of Ore extensions. That to say, is equivalent to saying $A$ is an $R[X,id_A,D]$ algebra over $R$, where
%$$X = \{x \in A : \exists D \in \trm{Der}_R(A),\ D(x) \neq 0\}$$
%and $D$ are generators of $\trm{Der}_R(A)$.
\begin{defi}\label{defi06}
An Lie algebra $\mathfrak{g}$ is an $R$-algebra with the $R$-linear map $\mu : \mathfrak{g} \otimes \mathfrak{g} \longmapsto \mathfrak{g}$, s.t.
\bn
\item $\mu(x \otimes y) = -\mu(y \otimes x)$ (anti symmetric) and
\item for any $x_i \in \mathfrak{g}$, $i = 1, 2, 3$, we have
$$\mu \circ(\mu \otimes id_\mathfrak{g})(x_1 \otimes x_2 \otimes x_3 + x_2 \otimes x_3 \otimes x_1 + x_3 \otimes x_1 \otimes x_2) = 0,$$ 
the Jacobian identity.
\item \label{LieAlgFromAlg} For any algebra $A$ the $R$-module $\mathfrak{g}(A)$ with $\mu = [x \otimes y \longmapsto x y - y x]$ is the Lie algebra associated with $A$, coinciding with $A$ as a set. The $R$-bilinear map (multiplication) is denoted by $[.,.] : \mathfrak{g} \times \mathfrak{g}\longrightarrow \mathfrak{g}$, the so called commutator of $A$.
\en
\index{Index}{Lie algebra}
\index{Index}{commutator}
\index{Symbol}{$\mathfrak{g}$}
\end{defi}
\begin{koro}\label{koro02}
The set of derivations $\trm{Der}_R(A)$ is a Lie algebra via the above defined commutator on $\trm{End}_R(A)$.
\end{koro}
\bmk All definitions of morphisms, sub Lie algebras, modules and ideals translate from algebras (although, ideals in Lie algebras are always two-sided). We repeat, if $(A, \mu, \eta)$ is some $R$-algebra its Lie algebra $\mathfrak{g}(A)$ is the same set, only with a different multiplication map.
\bsp Some prominent examples:
\bn
\item for any $R$-module $M$, the endomorphism algebra $\trm{End}_R(M)$ is a Lie algebra with the commutator inducing multiplication $\mu$, denoted by
$$\mathfrak{gl}_R(M).$$
If $M \simeq R^n$, then $\mathfrak{gl}_R(M)$ is also denoted by $\mathfrak{gl}_n(R)$,
%\item for any associative $R$-algebra $A$, the algebra $(A, \mu_{\trm{Lie}})$, where $\mu_{\trm{Lie}} := [.,.]$, is an $R$-Lie algebra - denoted by
%$$\mathfrak{g}(A),$$
\item the set of upper triangular matrices, $\mathfrak{b} \subset \mathfrak{gl}_n(R)$, and strictly upper triangular matrices $\mathfrak{n} \subset \mathfrak{b}$ are Lie algebras,
\item the set of all square matrices with trace zero is a sub Lie algebra of $\mathfrak{gl}_n(R)$, denoted by $\mathfrak{sl}_n(R)$, called the special Lie algebra.
\en
\begin{defi}\label{PartialDiff}
A differential algebra $(A,\mu,D)$ is called a partial differential algebra, if the Lie algebra $\trm{Der}_R(A) = \left<D\right>_{\trm{Lie}}$ is commutative:
$$[\partial,\delta] = 0,\ \forall \partial, \delta \in \trm{Der}_R(A).$$
\index{Index}{algebra!partial differential}
\end{defi}
Firstly, we remark that $A$ itself may be an Lie algebra (hence, we omit the unit map). This definition is due to Ritt (\cite{Ritt}, pg. 163), who, however, simply demanded commutativity of the derivations.
\bsp \label{partial_diff_alg_examp}We give an example and a counter-example.
\bn
\item\label{partial_diff_exp01} For all $n \geq 1$ is $(k[x_1,\ldots,x_n], \Delta = \{\partial_1,\ldots,\partial_n\})$ a partial differential algebra:
$$\bao{rclcl}
\partial_i &=& [x_{j}^{k} &\longmapsto& \sum_{k} \delta_{ij} x_j^{k-1} = \delta_{ij} k x_j^{k-1}]\\
&&&&\\
\partial_i \partial_j &=& [x^k := x_1^{k_1} \ldots x_n^{k_n} &\longmapsto & k_i k_j x_1^{k_1}\ldots x_i^{k_i - 1} \ldots x_j^{k_j-1} \ldots x_n^{k_n}],\ \forall 1 \leq i < j \leq n\\
&&&&\\
\partial_j \partial_i &=& [x^k := x_1^{k_1} \ldots x_n^{k_n} &\longmapsto & k_i k_j x_1^{k_1}\ldots x_i^{k_i - 1} \ldots x_j^{k_j-1} \ldots x_n^{k_n}],\ \forall 1 \leq i < j \leq n,\\
\ea$$
where $\delta_{ij}$ is the Kronecker-Delta. The special case $n = 1$ is simply the differential algebra over a polynomial ring in one indeterminate over $k$.
\item\label{partial_diff_exp02} Let $k$ be a field with $\trm{char} k \neq 2$ and $k[x,x^{-1}]$ denote the localization of $k[x]$, i.e. the ring of Laurent polynomials with $n = 1$ in our last example. The differential ring $(k[x,x^{-1}], \Delta = \{\partial_1, \partial_{-1}\})$, with:
$$\bao{rcl}
\partial_1 &=& \left[
\bao{rcl}
x^i &\longmapsto& i x^{i-1}\\
x^{-i}&\longmapsto& -i x^{-i-1}\\
\ea\right]\\
&&\\
\partial_{-1} &=& \left[
\bao{rcl}
x^i &\longmapsto& -i x^{i+1}\\
x^{-i} &\longmapsto& i x^{-i+1},\\
\ea\right]\\
\ea$$
is not a partial differential algebra over $k$, as
$$\bao{rcl}
\partial_{-1} \partial_1 &=& \left[
\bao{rcl}
x^i &\longmapsto& -i (i + 1) x^i\\
x^{-i}&\longmapsto& -i (i - 1) x^{-i}\\
\ea\right]\\
&&\\
\partial_1 \partial_{-1} &=& \left[
\bao{rcl}
x^i &\longmapsto& -i (i - 1) x^i\\
x^{-i}&\longmapsto& -i (i + 1) x^{-i}\\
\ea\right]\\
\ea$$
do not agree and therefore its commutator is:
$$\bao{rcl}
[\partial_1,\partial_{-1}] &=& \left[\bao{rcl}
x^i &\longmapsto& 2 i x^i\\
x^{-i} &\longmapsto& -2 i x^{-i}.\\
\ea\right]\\\ea$$
We claim that our derivation Lie algebra $\mathfrak{g}$, generated by $\Delta$, is isomorphic to $\mathfrak{sl}_2(k)$. In particular, $k[x,x^{-1}]$ is an $\mathfrak{sl}_2(k)$-module. Identifying $H = [\partial_1,\partial_{-1}]$, $X = \partial_1$ and $Y = \partial_{-1}$, we get an isomorphism of $k$-vector spaces (using $\mathfrak{sl}_2(k) = k.X \oplus k.Y \oplus k.H$). Now, it is enough to show this is also an isomorphism of Lie algebras. But by our computation above, we see already: $[X,H] = 2 X, [Y,H] = -2 Y$ and, by definition, $[X,Y] = H$, completing the proof.
\en
\subsubsection{The Universal Enveloping Algebra}
A consequence of the universal property of algebras we have the following definition:
\begin{defi}\label{defi07}
For every Lie algebra $\mathfrak{g}$ there exists a unique associative algebra $U(\mathfrak{g})$ and a homomorphism of Lie algebras $\iota : \mathfrak{g} \longrightarrow U(\mathfrak{g})$, universal in the following sense. For each associative algebra $A$ and a homomorphism of Lie algebras $f : \mathfrak{g} \longrightarrow \mathfrak{g}(A)$, there exists an unique morphisms of algebras $g : U(\mathfrak{g}) \longrightarrow A$ such that
$$\xymatrix{
\mathfrak{g} \ar[r]^{\iota}\ar[dr]_f & U(\mathfrak{g})\ar[d]^g\\
&A\\
}$$
commutes.
\index{Index}{algebra!univeral enveloping}
\index{Symbol}{$U(\mathfrak{g})$}
\end{defi}
We recall that $\mathfrak{g}(A)$ and $A$ coincide as sets (see def. \ref{defi06}\ref{LieAlgFromAlg}) and the Lie algebra structure on $U(\mathfrak{g})$ is given by the commutator. The above definition gives us the so called universal property, hence the name. In later examples we will see the application of both concepts.
\begin{prop}\label{prop05}
Let $T\mathfrak{g}$ be the tensor algebra of $\mathfrak{g}$, then
$$T \mathfrak{g}/\left<x \otimes y - y \otimes x - [x,y]: x, y \in \mathfrak{g}\right> \simeq U(\mathfrak{g}).$$
\end{prop}
\bws We recall proposition \ref{prop01} that $U(\mathfrak{g}) \simeq T U(\mathfrak{g})/I U(\mathfrak{g})$ for some two-sided ideal $IU(\mathfrak{g})$ in the universal enveloping algebra. The ideal is generated by all elements of the form $a \otimes b - a b \in \mathfrak{g}^{\otimes 2} \oplus \mathfrak{g}$. Clearly, $b \otimes a - ba$ is also a generator. Therefore, the element $a \otimes b - a b - b \otimes a + b a$ is indeed another generator. Recalling the definition of the commutator $[a,b] = ab - ba$ we get that
the generators
$$a \otimes b - b \otimes a - \underbrace{a b - b a}_{[a,b]} = a \otimes b - b \otimes a - [a,b]$$
uniquely define an associative algebra containing $\mathfrak{g}$ as Lie algebra.\\
\bmk The universal enveloping algebra of a one-dimensional $\qz$ Lie algebra will be the centerpiece of the examples discussed in this paper.
\subsection{Coalgebras}
The concept of a coalgebra is dual to that of an (associative unital) algebras. To recall, each associative unital algebra $A$ is described by the two $R$-linear maps $\mu : A \otimes A \longrightarrow A$ and $\eta : R \longrightarrow A$, i.e. the triple $(A,\mu,\eta)$ contains already all information of $A$ given the commutative diagrams in definition \ref{defi01}, \ref{alg_unital} and its subsequent remark.
\begin{defi}\label{defi08}
A coalgebra is an $R$-module $C$ with two $R$-linear maps $\Delta : C \longrightarrow C \otimes C$, $\eps : C \longrightarrow R$, called the comultiplication or coproduct and counit, and the following properties:
\bn
\item $(id_C \otimes \Delta) \circ \Delta = (\Delta \otimes id_C) \circ \Delta$ (coassociativity) and
\item $(id_C \otimes \eps) \circ \Delta = (\eps \otimes id_C) \circ \Delta = id_C$ (counitarity).
\en
In addition, if $\tau : C \otimes C \longrightarrow C\otimes C$, $c \otimes c' \longmapsto c' \otimes c$ is the flip isomorphism, we call $(C, \Delta, \eps)$ cocommutative, if $\tau \Delta = \Delta$.
\index{Index}{coalgebra}
\index{Index}{coassociativity}
\index{Index}{counitality}
\index{Index}{cocommutative}
\index{Symbol}{$\Delta$}
\index{Symbol}{$\eps$}
%\index{Symbol}{counitality}
%\index{Symbol}{cocommutative}
\end{defi}
\bmk The properties of coalgebras can be reformulated in terms of commuting diagrams:
$$\bao{cc}
\xymatrix{
C \otimes C \otimes C && C \otimes C\ar[ll]_{id \otimes \Delta}\\
C\otimes C \ar[u]^{\Delta \otimes id}&& C\ar[ll]^{\Delta}\ar[u]_{\Delta}\\
} & \xymatrix{R \otimes C & \ar[l]_{\eps \otimes id_C} C \otimes C \ar[r]^{id_C \otimes \eps}& C \otimes R\\
&C \ar[lu]^\simeq \ar[u]_\Delta \ar[ru]_\simeq&\\
}\ea$$
where we identify $R \otimes C \simeq C \simeq C \otimes R$ and $C \otimes (C \otimes C) \simeq C^{\otimes 3} \simeq (C \otimes C) \otimes C$. The left and right hand diagrams represent the coassociativity and the counitality, respectively. Lastly, cocommutativity can expressed as
$$\xymatrix{
C \otimes C & C \otimes C\ar[l]_\tau\\
&C \ar[lu]^\Delta\ar[u]_\Delta.\\
}$$
%One important theorem in the theory of coalgebras, which we are not going to prove, reads as
%\begin{satz}
%Any element in a coalgebra is contained in a finitely generated sub-coalgebra.
%\end{satz}
Clearly, reversion of arrows in the commutative diagrams defining algebras (commutative, associative, unital) results in the above diagrams (cocommutative, coassociative and counital, resp.).
\subsubsection{Duals of coalgebras}
\begin{prop}\label{prop06}
If $C$ is a coalgebra, then $C^* = \trm{Hom}(C,R)$ is an associative unital algebra. The multiplication is called convolution.\index{Index}{convolution}
\end{prop}
\bws First, let us define the multiplication and unit via the duals of the comultiplication and counit:
$$\bao{rrcl}
\eta_{C^*} := \eps^* :& R \simeq R^* &\longrightarrow& C^*\\
&&&\\
&r &\longmapsto& r \eps = [x \mapsto r \eps(x)]\\
&&&\\
\mu_{C^*} := \Delta^* :& (C \otimes C)^* \simeq C^* \otimes C^* &\longrightarrow & C^*\\
&&&\\
&\alpha \otimes \beta &\longmapsto& \mu_R \circ (\alpha \otimes \beta) \circ \Delta\\
\ea$$
Now, it is easy to check that the above defined diagrams commute. For simplicity we omit the subscripts. Let $\alpha, \beta, \gamma \in C^*$, first we compute:
$$\mu_{C^*} = \mu_R \circ (ev \otimes ev) \circ (id_{C^*} \otimes \tau_{C^*\otimes C} \otimes id_C) \circ (id_{C^{* \otimes 2}} \otimes \Delta),$$
where $ev : C^* \otimes C \longrightarrow R, \alpha \otimes c \longmapsto \alpha(c)$. Now we see that
$$\bao{rcl}
\mu_{C^*} \circ (id_{C^*} \otimes \mu_{C^*})(\alpha \otimes \beta \otimes \gamma) &=& \mu_R(\alpha \otimes \mu_R(\beta \otimes \gamma)(id_C \otimes \Delta)) \circ \Delta\\
&&\\
\mu_{C^*}(\mu_{C^*} \otimes id_{C^*})(\alpha\otimes\beta\otimes\gamma) &=& \mu_R \circ (\mu_R \circ (\alpha \otimes \beta) \circ \Delta \otimes \gamma)\circ \Delta.\\
\ea$$
So for any given $x \in C$ we have by linearity and coassociativity ($(\Delta \otimes id)\Delta(x) = (id \otimes \Delta)\Delta(x)$):
$$\bao{rcl}
\mu_R(\alpha \otimes \mu_R(\beta \otimes \gamma)(id_C \otimes \Delta)) \circ \Delta(x) &=& \sum_{(x)} \mu_R(\alpha(x_{(1)}) \otimes \mu_R(\beta \otimes \gamma)\Delta(x_{(2)}))\\
&&\\
 &=& \sum_{(x),(x_2)} \mu_R(\alpha(x_{(1)}) \otimes \beta(x_{(21)}) \gamma(x_{(22)}))\\
&&\\
 &=& \sum_{(x),(x_2)} \alpha(x_{(1)}) \beta(x_{(2)}) \gamma(x_{(3)})\\
&&\\
\mu_R \circ (\mu_R \circ (\alpha \otimes \beta) \circ \Delta \otimes \gamma)\circ \Delta(x) &=& \sum_{(x)} \mu_R(\mu_R(\alpha \otimes \beta)\Delta(x_{(1)}) \otimes \gamma(x_{(2)}))\\
&&\\
 &=& \sum_{(x_1),(x)} \mu_R(\alpha(x_{(11)}) \beta(x_{(12)}) \otimes \gamma(x_{(2)}))\\
 &&\\
 &=& \sum_{(x_1),(x)} \alpha(x_{(1)}) \beta(x_{(2)}) \gamma(x_{(3)})\\
\ea$$
Hence, we have an associative linear map $\mu$. On the other hand - identifying $R \otimes C^* \simeq C^* \simeq C^* \otimes R$, we get from counitality:
$$\bao{rclcl}
((id \otimes \eta)\alpha\otimes r)\Delta(x) &=& \sum_{(1),(2)} r \alpha(x_{(1)}) \otimes \eps(x_{(2)}) &=& r \sum_{(1),(2)} \alpha(x_{(1)}) \otimes \eps(x_{(2)})\\
&&&&\\
&=& \alpha\underbrace{\left(\sum_{(1),(2)} x_{(1)} \eps(x_{(2)})\right)}_{x} \otimes r &=& r \alpha(x)\\
&&&&\\
((\eta \otimes id) r\otimes \alpha)\Delta(x) &=& \sum_{(1),(2)} r \eps(x_{(1)}) \otimes \alpha(x_{(2)}) &=& \sum_{(1),(2)} r \eps(x_{(1)}) \otimes \alpha(x_{(2)})\\
&&&&\\
&=& r \otimes \alpha(\underbrace{\sum_{(1),(2)} \eps(x_{(1)}) x_{(2)}}_{x}) &=& r \alpha(x)\\
\ea$$
which defines our two structure maps in the triple $(C^*, \mu, \eta)$.\\
\indent Note however, in general the converse does not hold.
\begin{defi}\label{defi09} Let $R$ be a ring.
 \bn
 \item An $R$-module $M$ is called projective if there is a lifting property:
 Let $N, P$ be two $R$-modules and $\varphi : N \rightarrow P$ an $R$-epimorphism, for every $f \in \trm{Hom}(M,N)$ there is at least one $g \in \trm{Hom}(M,P)$ such that $f \circ \varphi = g$.
 \item An $R$-module is of finite type if every proper $R$-submodule $M'$ is noetherian.
 \en
\index{Index}{module!finite type}
\index{Index}{module!projective}
\end{defi}
Note, the first definition does not provide a universal condition. We may have more than one such map. However, we get
\begin{prop}\label{prop07}
If the algebra $A$ is a projective $R$-module of finite type then, $A^*$ is a coalgebra.
\end{prop}
We omit the proof and refer the reader to \cite{OmSho}. An important consequence is that, for $R$ a field, all finite-dimensional $R$-algebras have coalgebras as duals. An other immediate consequence of \ref{prop06} we get
\begin{koro}\label{koro03}
Let $C$ be a (coass, counital) coalgebra and $A$ an associative, unital algebra, both over $R$. The set $\mathcal{A}_C := \trm{Hom}(C^*,A)$ is an associative unital algebra, with multiplication
$$\mu_* : \mathcal{A}_C \otimes \mathcal{A}_C \longrightarrow \mathcal{A}_C,\ f \otimes g \longmapsto \mu_A \circ(f\otimes g) \circ \Delta_C$$
called the convolution (denoted by $*$) and unit $\eta_A \circ \eps_C$.
\end{koro}
\subsubsection{Comodules, coideals and homomorphisms of coalgebras}
\begin{defi}\label{defi10}
For each coalgebra $C$ we define a left $C$-comodule $M$ as a module over $R$ with structure map $\rho_M = \rho : M \longrightarrow C \otimes M$ such that
\bn
\item $(\Delta \otimes id_M) \rho = (id_C \otimes \rho)\rho$,
\item $(\eps \otimes id_M) \rho = id_M$.
\en
Additionally, we define a sub-comodule $N \subset M$ via restriction of $\rho\mid_N$. A sub coalgebra is a sub-comodule $C' \subset C$ with structure map $\Delta_{C'} = \Delta\mid_{C'}$. A coideal $I \subset C$ is a two-sided sub $C$-comodule of $\ker \eps$ such that
$$I \subset \ker \eps \ \wedge \ \Delta(I) \subset I \otimes C + C \otimes I.$$
For two left $C$-comodules $M, N$, a morphism of left $C$-comodules is a module homomorphism $f : M \longrightarrow N$ such that
$$\rho_N f = (id_C \otimes f) \rho_M.$$
Moreover, if $C$, $C'$ are coalgebras, then a homomorphism of coalgebras is morphism $f \in \trm{Hom}(C, C')$, such that:
$$\Delta_{C'} f = (f \otimes f) \circ \Delta_C.$$
\index{Index}{comodule}
\index{Index}{coideal}
\index{Index}{coalgebra!homomorphisms of}
\index{Symbol}{$\rho_M$}
\end{defi}
The definitions can be readily extended for right and two-sided $C$ comodules and homomorphism.
\bsp \label{coalg_example} We give two examples via the duals of finite-dimensional/free of finite rank $R$-algebras:
\bn
\item \label{coalg01} Let $R$ be a unique factorization domain and $A := R[x]/\left<x^2\right>$ the ring of dual numbers over $R$. Its dual is
$$C := A^* = R.\delta_{\ov{1}} \oplus R.\delta_{\ov{x}},\ \trm{where}\ \delta_y(z) = \begin{cases}
1 & y = z\\
0 &\trm{else}\\
\end{cases},\ \forall y, z \in \{\ov{1},\ov{x}\}.$$
The comultiplication is the dual map of the multiplication on $A$:
$$\Delta := \mu^* = \left[\alpha \longmapsto \alpha \circ \mu\right].$$
Hence, we get:
$$\bao{rcl}
\delta_{\ov{1}} &\longmapsto& \delta_{\ov{1}} \otimes \delta_{\ov{1}}\\
\delta_{\ov{x}} &\longmapsto& \delta_{\ov{1}} \otimes \delta_{\ov{x}} + \delta_{\ov{x}} \otimes \delta_{\ov{1}}\\
\ea$$
as images for the comultiplication. The images of the counit are consequently:
$$\bao{rcl}
\delta_{\ov{1}} &\longmapsto& 1\\
\delta_{\ov{x}} &\longmapsto& 0.\\
\ea$$
The coassociativity is easily checked for both generators. Moreover, cocommutativity is evident.
\item \label{coalg02} Let $R$ be a field such that $x^2 + 1$ is irreducible over $R$ and $A := R[x]/\left<x^2 + 1\right>$. Following our last example, we get for $C := A^*$:
$$\bao{rrcl}
\Delta:& C &\longrightarrow& C \otimes C\\
& \delta_{\ov{1}} &\longmapsto & \delta_{\ov{1}} \otimes \delta_{\ov{1}} - \delta_{\ov{x}} \otimes \delta_{\ov{x}}\\
& \delta_{\ov{x}} &\longmapsto & \delta_{\ov{1}} \otimes \delta_{\ov{x}} + \delta_{\ov{x}} \otimes \delta_{\ov{1}}\\
\ea$$
and a counit similar to the one in our last example. As in the last example, $C$ is cocommutative. However, coassociativity is not that easily seen:
$$\bao{rcl}
(\Delta \otimes id_C)\Delta(\delta_{\ov{1}}) &=& \Delta \otimes id_C\left(\delta_{\ov{1}} \otimes \delta_{\ov{1}} - \delta_{\ov{x}} \otimes \delta_{\ov{x}}\right)\\
&=& \left(\delta_{\ov{1}} \otimes \delta_{\ov{1}} - \delta_{\ov{x}} \otimes \delta_{\ov{x}}\right) \otimes \delta_{\ov{1}} - \left(\delta_{\ov{1}} \otimes \delta_{\ov{x}} + \delta_{\ov{x}} \otimes \delta_{\ov{1}}\right) \otimes \delta_{\ov{x}}\\
&=& \delta_{\ov{1}} \otimes \delta_{\ov{1}} \otimes \delta_{\ov{1}} - \delta_{\ov{x}} \otimes \delta_{\ov{x}}
\otimes \delta_{\ov{1}} - \delta_{\ov{1}} \otimes \delta_{\ov{x}} \otimes \delta_{\ov{x}} - \delta_{\ov{x}} \otimes \delta_{\ov{1}} \otimes \delta_{\ov{x}}\\
&&\\
(id_C \otimes \Delta)\Delta(\delta_{\ov{1}}) &=& id_C \otimes \Delta\left(\delta_{\ov{1}} \otimes \delta_{\ov{1}} - \delta_{\ov{x}} \otimes \delta_{\ov{x}}\right)\\
&=& \delta_{\ov{1}} \otimes \left(\delta_{\ov{1}} \otimes \delta_{\ov{1}} - \delta_{\ov{x}} \otimes \delta_{\ov{x}}\right) - \delta_{\ov{x}} \otimes \left(\delta_{\ov{1}} \otimes \delta_{\ov{x}} + \delta_{\ov{x}} \otimes \delta_{\ov{1}}\right)\\
&=& \delta_{\ov{1}} \otimes \delta_{\ov{1}} \otimes \delta_{\ov{1}} - \delta_{\ov{1}} \otimes \delta_{\ov{x}} \otimes \delta_{\ov{x}} - \delta_{\ov{x}} \otimes \delta_{\ov{1}} \otimes \delta_{\ov{x}} - \delta_{\ov{x}} \otimes \delta_{\ov{x}}
\otimes \delta_{\ov{1}}\\
\ea$$
$$\bao{rcl}
(\Delta \otimes id_C)\Delta(\delta_{\ov{x}}) &=& \Delta \otimes id_C\left(\delta_{\ov{1}} \otimes \delta_{\ov{x}} + \delta_{\ov{x}} \otimes \delta_{\ov{1}}\right)\\
&=& \left(\delta_{\ov{1}} \otimes \delta_{\ov{1}} - \delta_{\ov{x}} \otimes \delta_{\ov{x}}\right) \otimes \delta_{\ov{x}} + \left(\delta_{\ov{1}} \otimes \delta_{\ov{x}} + \delta_{\ov{x}} \otimes \delta_{\ov{1}}\right) \otimes \delta_{\ov{1}}\\
&=& \delta_{\ov{1}} \otimes \delta_{\ov{1}} \otimes \delta_{\ov{x}} - \delta_{\ov{x}} \otimes \delta_{\ov{x}}
\otimes \delta_{\ov{x}} + \delta_{\ov{1}} \otimes \delta_{\ov{x}} \otimes \delta_{\ov{1}} + \delta_{\ov{x}} \otimes \delta_{\ov{1}} \otimes \delta_{\ov{1}}\\
&&\\
(id_C \otimes \Delta)\Delta(\delta_{\ov{x}}) &=& id_C \otimes \Delta\left(\delta_{\ov{1}} \otimes \delta_{\ov{x}} + \delta_{\ov{x}} \otimes \delta_{\ov{1}}\right)\\
&=& \delta_{\ov{1}} \otimes \left(\delta_{\ov{1}} \otimes \delta_{\ov{x}} + \delta_{\ov{x}} \otimes \delta_{\ov{1}}\right) + \delta_{\ov{x}} \otimes \left(\delta_{\ov{1}} \otimes \delta_{\ov{1}} - \delta_{\ov{x}} \otimes \delta_{\ov{x}}\right)\\
&=& \delta_{\ov{1}} \otimes \delta_{\ov{1}} \otimes \delta_{\ov{x}} + \delta_{\ov{1}} \otimes \delta_{\ov{x}} \otimes \delta_{\ov{1}} + \delta_{\ov{x}} \otimes \delta_{\ov{1}} \otimes \delta_{\ov{1}} - \delta_{\ov{x}} \otimes \delta_{\ov{x}} \otimes \delta_{\ov{x}}.\\
\ea$$
\en
Both examples are cocommutative, coassociative and counital coalgebras. However, they are not isomorphic over the same field/ring. This is clear if we recall that the dual of a coalgebra is an algebra. If both would be isomorphic, so would be their respective duals. Nevertheless, the first example $(A_1 = R[x]/\left<x^2\right>$, see \ref{coalg01}) contained a nilpotent element, the second example $(A_2 = R[x]/\left<x^2 + 1\right>$, see \ref{coalg02}) is either reduced (trivial nilradical) or is an integral ring extension (if $x^2 + 1$ has no roots in $R$, as we demanded). On the other hand, both coalgebras contain a coideal generated by the element $\delta_{\ov{x}}$. Hence, we get a coalgebra homomorphism
$$A_i^* \longrightarrow R,\ a_1 \delta_{\ov{1}} + a_x \delta_{\ov{x}} \longmapsto a_1,\ \trm{for}\ i = 1,2$$
\begin{defi}\label{coalg_type}
Let $C$ be a non-trivial coalgebra over some field $k$.
\bn
\item\label{coalg_irred} We call $C$ irreducible if any two subcoalgebras $C', C'' \subset C$ have non-zero intersection.
\item\label{coalg_simp} We call $C$ simple if $C$ has no non-trivial proper subcoalgebra $C' \subsetneq C$.
\item\label{coalg_point} We call $C$ pointed if all its simple subcoalgebras are of dimension one.
\item \label{coalg_group}We call an element $c \in C$ group-like if $\Delta(c) = c \otimes c$.
\item\label{coalg_skew}We call $c \in C$ $(g,h)$-skew primitive if $\Delta(c) = g \otimes c + c \otimes h$ for some group-like elements $g, h \in C$.
\en
\index{Index}{coalgebra!irreducible}
\index{Index}{coalgebra!simple}
\index{Index}{coalgebra!pointed}
\index{Index}{element!group-like}
\index{Index}{element!skew-primitive}
\end{defi}
\bmk Clearly, all group-like elements of a coalgebra generate simple subcoalgebras. The first of examples \ref{coalg_example} has only one proper subcoalgebra generated by $\delta_{\ov{1}}$ which is simple. Therefore, it is reducible and pointed. The latter example is irreducible as $\Delta(\delta_{\ov{z}})$ is in $C^{\otimes 2}$ for $\ov{z} = \ov{1}, \ov{x}$, but not simple.
\begin{defi}
Let $(C, \Delta_C, \eps_C)$ and $(D, \Delta_D, \eps_D)$ be two coalgebras over $R$. The tensor product $C \otimes D$ has a coalgebra structure via
$$\Delta_\otimes = (id_C \otimes \tau \otimes id_D) (\Delta_C \otimes \Delta_D),\ \eps_\otimes = \eps_C \otimes \eps_D.$$
\end{defi}
\begin{lemm}
If $(C, \Delta, \eps)$ is a coalgebra over $R$ and $A \in \trm{CAlg}_R$ then
$$\iota : C \longrightarrow A \otimes C, c \longmapsto 1_A \otimes c$$
has a $A$-coalgebra structure via
$$\Delta_A := (id_{A \otimes C} \otimes \iota) (id_A \otimes \Delta),\ \eps_A = id_A \otimes \eps$$
\end{lemm}
\bws Let $c \in C$, $C' = A \otimes C$ and $C$ being coassociative then
$$\bao{rclcl}
\Delta_A(1_A \otimes c) &=& (id_{C'} \otimes \iota)(1_A \otimes \Delta(c)) &=& (id_{C'} \otimes \iota)\left(1_A \otimes \left(\sum_{(c)} c_{(1)} \otimes c_{(2)}\right)\right)\\
&&&&\\
&=& \sum_{(c)} 1_A \otimes c_{(1)} \otimes \iota(c_{(2)}) &=& \sum_{(c)} 1_A \otimes c_{(1)} \otimes 1_A \otimes c_{(2)}.\\
\ea$$
Coassociativity follows already from the coassociativity of $C$ via equality
$$\sum_{(c),(c_{(1)})} 1_A \otimes c_{(11)} \otimes 1_A \otimes c_{(12)} \otimes 1_A \otimes c_{(2)} = \sum_{(c),(c_{(2)})} 1_A \otimes c_{(1)} \otimes 1_A \otimes c_{(21)} \otimes 1_A \otimes c_{(22)}.$$
Counitality follows as
$$(\eps_A \otimes id_{C'})(1_A \otimes c) = \sum_{(c)} \eps_A(1_A \otimes c_{(1)}) \otimes 1_A \otimes c_{(2)} = \sum_{(c)} 1_A \otimes\underbrace{ \eps(c_{(1)}) c_{(2)}}_{c} = 1_A \otimes c\ \trm{and}$$
$$(id_{C'} \otimes \eps_A)(1_A \otimes c) = \sum_{(c)} 1_A \otimes c_{(1)} \otimes \eps_A(1_A \otimes c_{(2)}) = \sum_{(c)} 1_A \otimes\underbrace{c_{(1)} \eps(c_{(2)})}_{c} = 1_A \otimes c$$
implies $(\Delta_A \otimes id) \Delta_A = id_{C'} = (id \otimes \Delta_A)\Delta_A$.
\subsubsection{Cofree and cofree cocommutative coalgebras}
Returning to algebras and coalgebra, the question of the coalgebra structure for a general algebra $A$ hasn't been fully answered. This section is taken from \cite{Sweed}. Although, \cite{barr} gives a definition of cofree coalgebras for general rings we are following \cite{Sweed}: let $R$ be a field. Firstly, we need
\begin{defi}
Let $R$ be as above and $A$ an $R$-algebra. We define the sub $A^0$ to be
$$A^0 = \left<g \in A^\ast : \exists I \subseteq \ker g, A/I \simeq \bigcup_{i\leq n} B_i,\ B_i \simeq R^{n_i}, n_i \in \nz\right>.$$
\index{Symbol}{$A^o$}
\end{defi}
In words, $A^0$ is a subspace generated by all dual elements containing a cofinite ideal in $A$.
\begin{lemm}
Let $A, B$ be $R$-algebras and $f \in \mathrm{Hom}_{R-\mathrm{alg}}(A,B) =: \mathrm{Alg}_R(A,B)$ - we have:
\bn
\item $f^\ast: B^\ast \longrightarrow  A^\ast$ with $f^\ast = [\beta \longmapsto \beta \circ f]$ has $f^\ast(B^\ast) \subset A^\ast$,
\item Regarding $A^\ast \otimes B^\ast \subset (A \otimes B)^\ast$, we have $A^o \otimes B^o = (A \otimes B)^o$,
\item $\mu^\ast : A^\ast \longrightarrow (A \otimes A)^\ast$ has $\mathrm{im} \mu^\ast\mid_{A^o} \subset A^o \otimes A^o$.
\en
\end{lemm}
\bws See \cite{Sweed}, pg. 110 - 113.
\begin{prop}
For $A$ as above, $(A^o, \Delta, \varepsilon)$ is a $R$-coalgebra for 
$$\varepsilon : A^o \longrightarrow R,\ \alpha \longmapsto \alpha(1_A)\ \mathrm{and}\ \Delta := \mu^\ast\mid_{A^o}.$$
\end{prop}
\bws See \cite{Sweed}, pg. 113 - 114.

\bmk This vector space has some interesting properties which are
\bn
\item If $A$ and $B$ are $R$-algebras and $g \in \mathrm{Alg}_R(A,B)$ then last prop shows that $g^\ast \mid_{B^o} =: g^o$ is an coalgebra homomorphism.
\item $A^o$ is the maximal coalgebra in $A^\ast$ being induced by $\mu^{-1}(A^\ast \otimes A^\ast)$.
\item It may happen that $A^o = \{0\}$ for example for infinite degree field extension.
\item $A^\ast$ has a left $A$-module structure defined via:
$$\rho_l : A \otimes A^\ast \longrightarrow A^\ast,\ a \otimes \alpha \longmapsto \alpha \_ \cdot a := [b \longmapsto \alpha(b a)].$$
Furthermore, via
$$\rho_r : A^\ast \otimes A \longrightarrow A^\ast,\ \alpha \otimes a \longmapsto \alpha a \cdot \_ := [b \longmapsto \alpha(a b)],$$
making $A^\ast$ into an two-sided $A$-module.
\item for a coalgebra $C$ its bidual $C^{\ast \ast}$ has its image in $C^{\ast o}$.
\en
\begin{satz}
The functors $^o : \mathrm{Alg}_R \longrightarrow \mathrm{CoAlg}_R$ and $^* : \mathrm{CoAlg}_R \longrightarrow \mathrm{Alg}_R$ are adjoint to one another. There is a one-to-one correspondence between the sets
$$\mathrm{Alg}_R(A,C^\ast) \ \mathrm{and}\ \mathrm{CoAlg}_R(C, A^o)$$
\end{satz}
Given two spaces $V$ and $W$ and $\iota : V^\ast \otimes W^\ast \longrightarrow (V \otimes W)^\ast$ as identification. Then $V^\ast \otimes W^\ast$ is dense in $(V \otimes W)^\ast$ if $u_1 \in V \otimes W\backslash\{0\}$ there is an $u_2 \in V^\ast \otimes W^\ast$ such that
$$u_2(u_1) \neq 0.$$
\begin{defi}
An algebra $A$ is called proper if and only if $A^0$ is dense in $A^\ast$.
\end{defi}
\begin{lemm}
$A^o$ dense if and only if for each non-zero $a \in A$ there is a cofinite ideal excluding $a$.
\end{lemm}
\begin{satz}
If $A$ is a commutative finitely generated algebra then $A^o$ is dense in $A^\ast$.
\end{satz}
\bsp We consider $A = R[x]$ and denote with $ev_a : A \longrightarrow R, x^i \longmapsto a^i$, for all $a \in R$ then
$$A^o = \left<\chi_i : \chi_i(x^j) = \delta_{i,j}\right> + \left<ev_a : a \in R \right>$$
as each $I_i = \left<x^{i+1}\right> \subset \ker\chi_i$ has a quotient algebra free of rank $i + 1$ and by definition, $\ker ev_a \supseteq \left<x - a\right>$. We get that $\chi_i(1_A) = \delta_{0,i}$ and $ev_a(1_A) = 1_R$ and for $p = \sum p_i x^i, q = \sum q_i x^i$:
$$\bao{rclcl}
\Delta(\chi_i) &=& \chi_i \circ \mu_A &=& [p \otimes q \longmapsto \sum_{k+l=i} p_k q_l]\\
&=& \sum_{k+l=i} \chi_k \otimes \chi_l&&\\
&&&&\\
\Delta(ev_a) &=& ev_a \circ \mu_A &=& \left[p \otimes q \longmapsto ev_a(p q)\right]\\
&=& ev_a \otimes ev_a\\
\ea$$
Thus, $\chi_0$ and $ev_a$ are group-like and $\chi_1$ is primitive.
%Clearly, for any arbitrary bialgebra $H$, its dual $H^\ast$ is usually not an bialgebra. However, $H^o$ is indeed.
\begin{defi}[Cofree]
If $V$ is a vector space, a pair $(C, \pi)$ with $C$ a coalgebra and $\pi \in \mathrm{Hom}(C,V)$ is called cofree coalgebra on $V$ if for any coalgebra $D$ and $f \in \mathrm{Hom}(D,V)$ there is a unique $F \in \mathrm{CoAlg}(D,C)$, such that 
$$\xymatrix{
D \ar[dr]_f\ar[r]^F & C \ar[d]^\pi\\
& V\\
}$$
\index{Index}{coalgebra!cofree}
\end{defi}
\begin{satz}
For any vector space $V$ the cofree coalgebra always exists.
\end{satz}
\bmk $T(V^\ast)^o$ is the cofree coalgebra for $V^{\ast\ast}$ for any vector space $V$.
\begin{lemm}
Let $(C,\pi)$ be a cofree coalgebra on some space $X$ and $Y \subset X$ be a subspace. Let $D = \sum E$ with
$E$ subcoalgebras of $C$ such that $\pi(E) \subset Y$. Then $\rho := \pi\mid_D$ maps $D$ to $Y$ and $(D, \rho)$ is the cofree coalgebra on $Y$.
\end{lemm}
Now, we see that the cofree coalgebra for any vectors space $V$ can be recovered from the cofree coalgebra $(T(V^\ast)^o, \pi)$ on $V^{\ast\ast}$ where $\pi$ is defined via the composition map:
$$\pi : T(V^\ast)^o \longrightarrow T(V^\ast)^\ast \longrightarrow V^{\ast\ast}.$$
The first arrow is simply the inclusion whereas the second is the dual of the embedding of $V^\ast$ in $T(V^ \ast)$.
\begin{defi}[Cofree cocommutative coalgebra]
Let $V$ be a vector space and $C$ a cocommutative coalgebra. For $\pi \in \trm{Hom}(C,V)$ we call $(C, \pi)$ a cofree cocommutative algebra if for all cocommutative coalgebras $D$ and $f \in \trm{Hom}(D,V)$ there is a 	unique $F \in \trm{CoAlg}(C,D)$ such that
$$\xymatrix{
C \ar[r]^F \ar[rd]_\pi & D\ar[d]^f\\
&V\\
}$$
commutes.
\index{Symbol}{$T(V^\ast)^o$}
\index{Index}{coalgebra!cofree!cocommutative}
\end{defi}
\subsection{Bialgebras}
Both, the definition of algebras and coalgebras, with certain compatibility conditions, define a bialgebra.
\begin{defi}\label{defi11}
A bialgebra is a module $B$, with the structure of an algebra $(B,\mu,\eta)$ and of a coalgebra $(B,\Delta,\eps)$ with the following compatibility conditions expressed in commuting diagrams:
$$\bao{cc}
\xymatrix{
 B \otimes B \ar[r]^\mu \ar[d]_{\Delta \otimes \Delta} & B \ar[r]^\Delta & B \otimes B\\
 B \otimes B \otimes B \otimes B \ar[rr]_{id_B \otimes \tau \otimes id_B}&& B \otimes B \otimes B \otimes B \ar[u]_{\mu \otimes \mu}\\
 }
 %\item for multiplication $\nabla$ and counit $\eps$
 &\xymatrix{
 B \otimes B \ar[rd]_{\eps \otimes \eps} \ar[rr]^\mu & & B\ar[ld]^\eps\\
 &R&
 }\\
 \trm{co/multiplication} & \trm{multiplication~and~counit}\\
 &\\
 %\item for comultiplication $\Delta$ and unit $\eta$:
 \xymatrix{
 &R \ar[ld]_\eta \ar[rd]^{\eta \otimes \eta}&\\
  B \ar[rr]_\Delta&&B \otimes B \\ 
 }
 %\item and for co-/unit:
 &\xymatrix{
 R \ar[rr]^{id}\ar[rd]^\eta && R \\
 &B\ar[ru]^\eps&
 }\\
 \trm{comultiplication~and~unit} & \trm{co/unit}\\
 \ea$$
 where $\tau : B \otimes B \rightarrow B \otimes B$, $x \otimes y \mapsto y\otimes x$ is the flip map.
\index{Index}{bialgebra}
\end{defi}
\bmk The commuting diagrams can be rephrased as
\bn
\item $\eps$ and $\Delta$ are homomorphisms of algebras,
\item $\eta$ and $\mu$ are homomorphisms of coalgebras.
\en
\begin{defi}\label{defi12}
Let $B$ be a bialgebra. Group-like and $(g,h)$-skew primitive elements are defined via their coproducts as in def. \ref{coalg_type}, pt. \ref{coalg_skew} or \ref{coalg_group}. If $B$ has a unit $1_B$ we call an element primitive if it is a $(1_B,1_B)$-skew primitive element, i.e. 
$$\Delta(x) = x \otimes 1_B + 1_B \otimes x.$$
\index{Index}{element!primitive}
\end{defi}
\bmk Primitive elements and Lie algebras are intimitely connected what we are going to show in a short instance.
%\bmk Firstly, we remark that the definition only depends on the coalgebra structure. In deed, \cite{Sweed} only uses the skew-primitive and group-like definition in the context of coalgebras.\\
%\indent Secondly, the different types of primitive elements are interconnected. A primitive element is simply a $(1_B,1_B)$-primitive and $h$-skew primitives are $(1_B,h)$ skew-primitive. Recalling our definition of Ore-extensions $A[X,\alpha,\delta]$, we see that an $\alpha$-derivation $\delta$ is simply an $(id,\alpha)$-skew primitive element in $\trm{End}_R(A)$. To be precise, $\delta$ and $\alpha$ generate a bialgebra with $\alpha$ group-like and $\delta$ $(id,\alpha)$ skew-primitive.
\bsp As before we introduce two famous examples.
\bn
\item Coming back to our tensor algebra over some module free of rank $n$, $T R^n$, it has the the structure of a primitive bialgebra given by $x \in R^n$ being primitive and $\eps(x) = 0$ for all $x \in T R^n\bsl R$ and $1_{TR^n} \longmapsto 1_R$. This gives us a coalgebra, with algebra structure maps $\eta : R \longrightarrow TR^n$, $1_R \longmapsto 1_{TR^n}$ and the given multiplication.
\item The second example is the group algebra $R.G = R[G]$, for some group $G$. We have a comultiplication $\Delta(g) = g \otimes g$ and a counit $\eps(g) = 1$ for all $g \in G$. The multiplication is obvious and the unit is $1_R \longmapsto 1_{R.G}$.
\en
%\subsubsection{Primitives form a Lie algebra}
\begin{prop}\label{prop08}
Let $(B,\mu,\eta,\Delta,\eps)$ be a coassociative bialgebra. The module of all primitive elements of $B$ defines a Lie algebra, denoted $\mathcal{P}(B)$. The module of all group-like elements generates a sub bialgebra of $B$, denoted $\mathcal{G}(B)$.
\end{prop}
\bws Firstly, recall $\Delta$ is an algebra homomorphism
$$\bao{rcl}
[x,y] &=& x y - y x\\
 &\RA&\\
 \Delta([x,y]) &=& \Delta(x y) - \Delta(y x)\\ &=& \Delta(x) \Delta(y) - \Delta(y) \Delta(x)\\
&=&(1 \otimes x + x \otimes 1)(1 \otimes y + y \otimes 1) \\
&& - (1 \otimes y + y \otimes 1)(1 \otimes x + x \otimes 1)\\
 &=& 1 \otimes x y + y \otimes x + x \otimes y + x y \otimes 1\\
&& - 1 \otimes y x - x \otimes y - y \otimes x - y x \otimes 1\\
&=& 1 \otimes [x,y] + [x,y] \otimes 1\\
\ea$$
Secondly, clearly:
$$\sum_i \lambda_i g_i \in \mathcal{G}(B)\ \stackrel{\Delta}{\longmapsto} \ \sum_i \lambda_i \underbrace{g_i \otimes g_i}_{\in \Delta(\mathcal{G}(B))},\ \forall \lambda_i \in R,$$
i.e. the coproducts of linear combination of group-likes are simply its linear combination of its coproducts. Hence, we only need to show that the product of group-like elements is again group-like. But this is clear from coassociativity and the fact that both coproduct and counit are algebra homomorphisms.
\begin{defi}
A bialgebra $B$ is called pointed irreducible, if its coalgebra $(B, \Delta, \eps)$ is pointed and irreducible. A cocommutative pointed irreducible bialgebra $B$ is called of Birkhoff-Witt type if it its coalgebra $(B,\Delta_B\eps_B)$ is isomorphic to a cocommutative cofree pointed irreducible coalgebra.
\index{Index}{bialgebra!pointed irreducible}
\index{Index}{bialgebra!of Birkhoff-Witt type}
\end{defi}
\bmk In \cite{Take} it is said that any irreducible cocommutative bialgebra over a field of characteristic zero is Birkhoff-Witt. In positive characteristic $p$, there is a canonical $R^{1/p}$-linear map:
$$\mathcal{Y} : B \longrightarrow R^{1/p} \otimes_R B,$$
then $B$ is BW if and only if $\mathcal{Y}$ is surjective. This is the case if $C$ is spanned by divided by power sequences or alternatively, if for
$$T(C_+) = \bigoplus_{n \geq 0} C_+^{\otimes n},\ C_+ = \ker \eps$$
$\trm{Hom}(T(C_+), A)$ is a divided power algebra for all algebras $A$. We define:
\begin{defi}
Let $A$ be an algebra and $I$ a (two-sided) ideal in $A$ with a family of maps $\gamma_i : I \longrightarrow A$ indexed by $\nz_0$ such that
\bn
\item $\gamma_1(x) = x$ and $\gamma_0(x) = 1$ for all $x \in I$,
\item $\gamma_n(x) \gamma_m(x) = \left(\begin{array}{c}m + n\\m\\\end{array}\right) \gamma_{m + n} (x)$ for all $x \in I$ and $m, n \geq 0$,
\item $\gamma_n(a x) = a^n \gamma_n(x)$ for all $x \in I$ and $a \in A$,
\item $\gamma_n(x + y) = \sum_{i = 0}^n \gamma_i(x) \gamma_{n - i}(y)$ for all $x, y \in I$ and $n \geq 0$,
\item $\gamma_n(\gamma_m)) = \frac{(m n)!}{n! (m!)^n} \gamma_{n m}(x)$ for all $x \in I$ and $n, m \geq 0$,
\en
then we call the triple $(A, I, \gamma)$ a divided power algebra. Here, $\gamma = (\gamma_i)_{i \in \nz_0}$.
\index{Index}{algebra!pointed power sequences, of}
\end{defi}
\bsp \label{exp_bialg} To illustrate our last definitions, we give some examples.
\bn
\item \label{exp_bialg01} Let $\mathfrak{g}$ be a Lie algebra and $U(\mathfrak{g})$ be its universal enveloping algebra. Since the only group-like elements are all in $R.1$ we have that the universal enveloping algebra is our first example of a pointed irreducible bialgebra (with multiplication and unit as given and
$$\Delta = [x \longmapsto 1 \otimes x + x \otimes 1],\ \eps(x) = 0\ \forall x \in \mathfrak{g}).$$
In conjunction with prop. \ref{prop08}, we have a complete picture concerning Lie algebras or more precisely their universal envelopping algebras and primitive elements: if $A$ is a unital associative algebra and $A \simeq_{R-\trm{algs}} U(\mathfrak{g})$ for some Lie algebra $\mathfrak{g}$ then $A$ has a coalgebra structure via primitive generators $x \in \mathfrak{g}$. On the other hand, if $B$ is some bialgebra then its bialgebra of primitive elements is isomorphic to some universal envelopping algebra for a Lie algebra.
%In \cite{Heid13} bialgebras of this type, are refered to as pointed irreducible of Birkhoff-Witt type.
\item \label{exp_bialg02} On the other hand, for some group $G$, the group algebra $R.G$ with its (co-) multiplication and (co-) unit is a group-like bialgebra.\\
\item \label{exp_bialg03} Examples of skew-primitive bialgebras are for example $U_q(\mathfrak{sl}_2(\cz))$ for $q \in \cz^\times \bsl \{1\}$, where some of the relations among its generators can be described as actions of skew-primitive elements of $\trm{End}_\cz(U(\mathfrak{sl}_2(\cz)))$.\\
%\indent Two examples, where we have that the dual of an $R$-algebra are $R$-coalgebras:
\en
\subsubsection{Morphisms of bialgebras and bialgebra ideals}
\begin{defi}
Let $(B,\mu,\eta,\Delta,\eps)$ be a $R$-bialgebra.
\bn
\item An $R$-bimodule is an $R$-module $M$ which is a $(B,\mu,\eta)$-module and a $(B,\Delta,\eps)$-comodule.
\item A sub- bialgebra $B'$ is an $R$-submodule such that the restrictions of the structure maps yields a bialgebra.
\item A bialgebra ideal is an ideal of the associative algebra $(B,\mu,\eta)$ and a coideal of the coassociative coalgebra $(B,\Delta,\eps)$.
\en
\index{Index}{bimodule}
\index{Index}{bialgebra!sub-bialgebra}
\index{Index}{bialgebra!ideal of}
\end{defi}
\bmk Let $g, h \in B$ be group-like. First, we want to show that a $(g,h)$ skew-primitive element $x \in B$ is in a proper coideal. We define $S := \{x \in B : \exists! (g, h) \in B^2, \Delta(x) = g \otimes x + x \otimes h\}$ and let
$$I := B.S.B,$$
i.e. the two-sided ideal generated by $S$. We recall that $(\eps\otimes id)\Delta = id_B = (id\otimes \eps)\Delta$ and $\eps(g) = 1 = \eps(h)$. Therefore,
$$(id\otimes \eps)\Delta(x) = g \otimes \eps(x) + x \otimes \eps(h) = x = \eps(g) \otimes x + \eps(x) \otimes h = (\eps \otimes id) \circ \Delta(x) \LRA x \in \ker \eps\ \forall x \in S.$$
Thus, $S \subset \ker \eps$. Furthermore, for any product $a x b$, with $a, b \in B$, $x \in S$ we get:
$$\Delta(a x b) = \Delta(a) (g \otimes x + x \otimes h) \Delta(b) = \sum_{(a),(b)} (\underbrace{a_{(1)} g b_{(1)} \otimes a_{(2)} x b_{(2)}}_{\in B \otimes I} + \underbrace{a_{(1)} x b_{(1)} \otimes a_{(2)} h b_{(2)}}_{\in I \otimes B})$$
%if $x_{i_1} \ldots x_{i_n} \in B, x_{i_j} \in \mathcal{P}(B)$ then
%$$\Delta (x_{i_1} \ldots x_{i_n}) = \Delta(x_{i_1}) \ldots \Delta(x_{i_n}) = (1 \otimes x_{i_1} + x_{i_1} + x_{i_1} \otimes 1) \ldots (1 \otimes x_{i_n} + x_{i_n} \otimes 1)$$
%and 
%$$\bao{rcl}
% \Delta(x_{i_1}) \ldots \Delta(x_{i_n}) &=& \Delta(x_{i_1} \ldots x_{i_{n-1}}) (1 \otimes x_{i_n}) + (x_{i_n} \otimes 1) \Delta(x_{i_1} \ldots x_{i_{n-1}})\\
%\ea$$
In addition, $\eps(x y) = \eps(x) \eps(y)$ which shows $I$ is a coideal and, by definition, an ideal in $B$. Therefore, the canonical projection
$$\pi : B \longrightarrow B/I,\ x \longmapsto x + I$$
is a bialgebra morphism.\\
\indent Subsequently, all primitive elements, i.e. $(1,1)$ skew-primitives, also define a bialgebra ideal in $B$.
\begin{lemm}
Let $(B,\mu,\eta,\Delta,\eps)$ be an $R$-bialgebra und $I \subset B$ be an $R$-submodule. The following statements are equivalent:
\bn
\item\label{biideal} $I$ is a two-sided bialgebra ideal.
\item\label{bimodule} $I$ is two-sided $B$-sub bimodule of $\ker \eps$.
\en
\end{lemm}
\bws If $I$ is a two-sided biideal, then $I$ is a two-sided $(B,\mu,\eta)$-submodule of $B$. On the other hand, $I \subset \ker \eps$ and a two-sided $(B,\Delta,\eps)$-sub-comodule of $B$. Thus, we have \ref{biideal} $\RA$ \ref{bimodule}. The converse implication follows immediately.
\begin{lemm}\label{GroupLikeHopfIdeal}
Let $2 \nmid \trm{char}(R)$ as well as $\frac{1}{2} \in R$ and $B$ an $R$-bialgebra. For two group-like elements $g,h \in \mathcal{G}(B)$ the difference $g - h$ generates a biideal, the set $I := B.(g - h).B$.
\end{lemm}
\bws Simple computation shows:
$$\bao{rcl}
\Delta(g - h) &=& \underbrace{\frac{1}{2}(g + h) \otimes (g - h) + \frac{1}{2}(g - h) \otimes (g + h)}_{B \otimes I + I \otimes B}\\
&&\\
\eps(g - h) &=& 0\\
\ea$$
The biideal property is a consequence of the fact that $\eps$ and $\Delta$ are algebra homomorphisms, as well as $\mu$ and $\eta$ being coalgebra homomorphisms. Furthermore, image of the canonical projection $\pi : B \longrightarrow B/I$ is a pointed-irreducible bialgebra (note, all group-likes are equivalent to $1_{B/I}$).
\paragraph{Morphisms}
Let $(B,\mu_B,\eta_B,\Delta_B,\eps_B)$, $(C,\mu_C,\eta_C,\Delta_C,\eps_C)$ be to two $R$-bialgebras.
\begin{defi}
A morphism of $R$-modules $f: B \longrightarrow C$ is a bialgebra morphism if and only if it is morphism of $R$-algebras and $R$-coalgebras.
\index{Index}{bialgebra!morphism of}
\end{defi}
Equivalently, we could have demanded the any $R$-module morphism commuting with the structure maps defining each bialgebra would also yield the above definition.
\subsubsection{Module algebras}
The definition of an Ore extension can be easily extended as follows. Let $(A,\mu_A,\eta_A)$ be an $R$-algebra and $(B,\mu_B,\eta_B,\Delta_B,\eps_B)$ be an $R$-bialgebra and in addition let $A$ be a left $B$-module (equivalently, there is an algebra homomorphism $\rho : B \longrightarrow \trm{End}(A)$, i.e. a $B$-representation on $A$)
\begin{defi}\label{defi09a}
We call $A$ a $B$-module algebra, if there is a $\Psi \in \trm{Hom}(B \otimes A, A)$ with
$$\Psi : B \otimes A \longrightarrow A,\ b \otimes a \longmapsto \rho(b)(a)$$
such that
\bn
\item $\Psi(b \otimes a a') = \sum_{(b)} \mu_A\left(\rho(b_{(1)})(a) \otimes \rho(b_{(2)})(a')\right)$, for all $a, a' \in A$ and $b \in B$,
where $\Delta_B(b) = \sum_{(b)} b_{(1)} \otimes b_{(2)}$,
\item $\Psi(b \otimes 1_A) = \eps_B(b) 1_A$ for all $b \in B$.
\en
\index{Index}{module algebra}
\end{defi}
\bmk Firstly, in Heiderich 2010 and Heiderich 2011 a module algebra is defined for some coalgebra $(C,\Delta,\eps)$. Indeed, the module algebra structure solely depends on the coalgebra structure maps. But, most of the examples we will encounter are bialgebras. Secondly, the above definition can be rephrased in the context of commuting diagrams, the first is
$$\xymatrix{
B \otimes A \otimes A \ar[d]_{\Delta_B\otimes id_{A\otimes A}}\ar[rrr]^{id_B\otimes \mu_A}&&& B\otimes A\ar[dd]^{\Psi_A}\\
B \otimes B \otimes A \otimes A \ar[d]_{id_B \otimes \tau_{B\otimes A} \otimes id_A}&&&\\
B\otimes A \otimes B \otimes A\ar[rr]_{\Psi_A\otimes \Psi_A}&&A \otimes A\ar[r]_{\mu_A}&A,\\
}$$
where $\tau_{B\otimes A} : B \otimes A \longrightarrow A \otimes B$ is the flip isomorphism. The second is simply
$$\xymatrix{
B \simeq B \otimes R \ar[rr]^{id_B \otimes \eta_A}\ar[d]_{id_B\otimes \eta_A}&&B \otimes A\ar[d]^{\eps_B \otimes id_A}\\
B \otimes A \ar[rr]_{\Psi_A}&& A \simeq R \otimes A\\
}$$
\bsp Recalling example \ref{partial_diff_exp02} on pg. \pageref{partial_diff_alg_examp}, $k[x,x^{-1}]$ and $\mathfrak{sl}_2(k)$ as derivation Lie algebra. We define $B = U(\mathfrak{sl}_2(k))$ with multiplication and unit as given, and comultiplication and counit given via the primitive generators $x \in \mathfrak{sl}_2(k)$ (see also example \ref{exp_bialg}. \ref{exp_bialg01} on pg. \pageref{exp_bialg01}). This makes $A := k[x,x^{-1}]$ into a $U(\mathfrak{sl}_2(k))$-module algebra, if we choose
$$\bao{rrcl}
\Psi : & U(\mathfrak{sl}_2(k)) \otimes_k k[x,x^{-1}] &\longrightarrow& k[x,x^{-1}]\\
&&&\\
& \partial_{i_1} \ldots \partial_{i_n} \otimes x^j & \longmapsto & \partial_{i_1} \circ \ldots \circ \partial_{i_n}(x^j)\\
\ea$$
as structure map, where $\partial_{i_j} \in \{\partial_{\pm 1}, [\partial_1,\partial_{-1}]\}$. Clearly, $\Psi(b \otimes 1) = \eps(b)\cdot 1 = \begin{cases}0 & \deg b \geq 1\\b_0.1 & \trm{else}\\\end{cases}$ for $b = b_0 + b_1 \partial_1 + b_{-1} \partial_{-1} + b_{1,-1} [\partial_1,\partial_{-1}] + \ldots$, as demanded. Moreover, a product $y z \in k[x,x^{-1}]$ gets mapped to:
$$\partial_{i} \otimes y z \longmapsto \mu \circ (id_A \otimes \partial_i + \partial_i \otimes id_A)(y \otimes z) = \mu(\Psi \otimes \Psi)(id_B \otimes \tau_{B \otimes A} \otimes id_A)(\Delta \otimes id_{A^{\otimes 2}})(\partial_i \otimes y \otimes z),$$
with $\partial_i$ as above. Recalling that $\Delta$ is an $k$-algebra homomorphism, we see that this applies to all weight spaces $k.\partial_{i_1} \ldots \partial_{i_n}$ of degree $n$.
\begin{defi}
Let $(A,\Psi_A)$ be a $B$-module algebra. 
\bn
\item The subset
$$A^\Psi := \{a \in A : \Psi_A(b \otimes a) = \eps_B(b) a,\ \forall b \in B\}$$
is called the constant $B$-module (sub)algebra.
\item if $A' \subset A$ is a subalgebra, and $\trm{im}\Psi\mid_{B\otimes A'} \subset A'$, then $(A', \Psi_{A'} := \Psi\mid_{B\otimes A'})$ is also a $B$-module algebra.
\item An ideal $I \subset A$ is called $B$-stable, if $\Psi_A(b \otimes a) \in I$ for all $a \in I$ and $b \in B$.
\item Let $(A',\Psi_{A'})$ be an other $B$-module algebra. A morphism of algebras $\varphi : A \longrightarrow A'$ is called a morphism of $B$-module algebras if
$$\xymatrix{
B \otimes A \ar[r]^{id_B\otimes \varphi}\ar[d]_{\Psi_A} & B \otimes A'\ar[d]^{\Psi_{A'}}\\
A \ar[r]_{\varphi} & A'\\
}$$
commutes.
\en
\index{Index}{module algebra!constant}
\index{Index}{module algebra!subalgebra}
\index{Index}{module algebra!$B$-stable ideals}
\index{Index}{module algebra!homomorphisms of}
\end{defi}
Recall that for two unital algebras $(A,\mu_A,\eta_A)$, $(B,\mu_B,\eta_B)$ the tensor product $A \otimes B$ has an unital algebra structure via:
$$\mu_{A\otimes B} := (\mu_A \otimes \mu_{B}) \circ (id_A \otimes \tau_{B \otimes A} id_B),\ \eta_{A \otimes B} = \eta_A \otimes \eta_B.$$
\begin{lemm}\label{d_mod_tens_prod}
Let $D$ be a cocommutative bialgebra, $(A,\Psi_A)$ and $(B,\Psi_B)$ be two $D$-module algebras. Then the algebra $A \otimes B$ has a $D$-module algebra structure via:
$$\Psi_{A\otimes B} := \left(\Psi_A \otimes \Psi_B\right) \circ (id_D \otimes \tau_{D \otimes A} \otimes id_B) \circ (\Delta_D \otimes id_{A \otimes B}).$$
\end{lemm}
\bws Let $d \in D$ and $a \otimes b, a'\otimes b' \in A\otimes B$. We need to show that the two commutative diagrams in the last remark hold.
\bn
\item Firstly, let $f$ denote the morphism of the lower path of the first diagram and let $(\Delta \otimes \Delta) \circ \Delta(d) := \sum_{(d)} \Delta(d_{(1)}) \otimes \Delta(d_{(2)}) = \sum_{(d_{(1)}),(d_{(2)})} d_{(11)} \otimes d_{(12)} \otimes d_{(21)} \otimes d_{(22)}$, then
{\scriptsize
$$\bao{rcl}
\Psi_{A\otimes B}(d \otimes (a a' \otimes b b')) &=& \sum_{(d)} \Psi_A(d_{(1)} \otimes a a') \otimes \Psi_B(d_{(2)} \otimes b b')\\
&&\\
&=& \sum_{(d_{(1)}),(d_{(2)})} (\Psi_A(d_{(11)} \otimes a)\Psi_A(d_{(12)}\otimes a')) \otimes (\Psi_AB(d_{(21)} \otimes b)\Psi_B(d_{(22)} \otimes b'))\\
&&\\
f(d \otimes (a \otimes b) \otimes (a'\otimes b')) &=& \sum_{(d)}\mu_{A\otimes B} \circ (\Psi_{A\otimes B} \otimes \Psi_{A\otimes B})(d_{(1)} \otimes (a \otimes b) \otimes d_{(2)} \otimes (a' \otimes b'))\\
&&\\
&=& \sum_{(d)} \mu_{A\otimes B} (\Psi_{A\otimes B}(d_{(1)} \otimes (a \otimes b)) \otimes \Psi_{A\otimes B}(d_{(2)} \otimes (a' \otimes b')))\\
&&\\
&=& \sum_{(d_{(1)}),(d_{(2)})} \mu_{A\otimes B}\left(\Psi_A(d_{(11)} \otimes a) \otimes \Psi_B(d_{(12)} \otimes b) \otimes \Psi_A(d_{(21)} \otimes a') \otimes \Psi_B(d_{(22)} \otimes b')\right)\\
&&\\
&=& \sum_{(d_{(1)}),(d_{(2)})} (\Psi_A(d_{(11)} \otimes a)\Psi_A(d_{(21)}\otimes a')) \otimes (\Psi_A(d_{(12)} \otimes b)\Psi_B(d_{(22)} \otimes b'))\\
\ea$$}
Recall from our definition of coassociative cocommutative (counital) coalgebras:
$$
\bao{cc}
\xymatrix{
D \ar[r]^{\Delta}\ar[d]_{\Delta}&D^{\otimes2}\ar[d]_{id\otimes \Delta}\\
D^{\otimes2} \ar[r]_{\Delta\otimes id}&D^{\otimes3}\\
} &
%\xymatrix{
%D \ar[r]^{\Delta} \ar[d]_{\Delta}&D^{\otimes2}\ar[d]^{\eps \otimes id}\\
%D^{\otimes2} \ar[r]_{id \otimes \eps} & D\\ 
%} &
\xymatrix{
D \ar[r]^\Delta\ar[rd]_{\Delta}&D^{\otimes2}\ar[d]_\tau\\
&D^{\otimes2}\\}
\ea$$
as well as each tensor module $D^{\otimes n}$ has a natural (right) comodule structure via $\rho_n := id^{\otimes n - 1} \otimes \Delta : D^{\otimes n} \longrightarrow D^{\otimes n} \otimes D$ and similarily a (left) comodule structure $\wt{\rho}_n := \Delta \otimes id^{\otimes n - 1} : D^{\otimes n} \longrightarrow D \otimes D^{\otimes n}$. In particular, we have $\wt{\rho}_3 \rho_2 \Delta = \rho_3 \wt{\rho}_2 \Delta = (\Delta \otimes \Delta)\Delta$ and we may always apply cocommutativity where ever $\Delta$ appears. Hence, $(id_D \otimes \tau \otimes id) \circ (\Delta \otimes \Delta) \circ \Delta = (\Delta \otimes \Delta) \circ \Delta$ (a proper proof in Heiderich 2010, Lem 2.15). Note, cocommutativity is essential in this step (i.e. in general the tensor product of $D$-module algebras is not a $D$-module algebra).
\item Computing $\Psi_{A\otimes B}(d \otimes (1_A\otimes 1_B)) = \sum_{(d)}\Psi_A(d_{(1)} \otimes 1_A) \otimes \Psi_B(d_{(2)} \otimes 1_B) = \sum_{(d)}\eps(d_{(1)}) 1_A \otimes \eps(d_{(2)}) 1_B = \eps(d) (1_A \otimes 1_B)$.
\en
\subsubsection{Smashed product}
\begin{defi}\label{defi03}
Let $A$ be an algebra, $G$ some group and $R.G$ the group algebra over $R$. Let $\rho : G \longrightarrow \trm{Aut} A$ define a representation, then
$$\bao{rrcl}
A \# G := A \otimes R.G,\ \mu_{A\#G} :& A\#G \otimes A\#G &\longrightarrow& A\#G\\
& (m_1 \otimes g_1) \otimes (m_2 \otimes g_2) &\longmapsto& m_1 \rho(g_1)(m_2) \otimes g_1 g_2\\
\ea$$
defines an associative unital algebra, the so called smashed product.
\index{Index}{smashed product}
\end{defi}
Note, that $A\#G$ can be interpreted as the semi-direct product $A \rtimes_\rho R.G$ of the two monoids $A$ and $R.G$. We also note, that $A$ has the structure of $R.G$-module algebra, via:
$$\Psi : R.G \otimes A \longrightarrow A,\ r g \otimes a \longmapsto r g(a),$$
extending to the $R.G$-module algebra $A\#G$:
$$\Psi' : R.G \otimes A\#G \longrightarrow A\#G,\ r g \otimes a \otimes 1_G \longmapsto r g(a) \otimes g,$$
sumarized in the following
\begin{koro}\label{koro01}
Let $A\#G$ be the smashed product for some algebra $A$ and some group $G$ with $G$-representation $\rho : G \longrightarrow \trm{Aut}_R(A)$.
\bn
\item $A\#G$ is isomorphic to 
$$T(A\otimes R.G \otimes A)/\left<1_A \otimes g \otimes a - \rho(g)(a) \otimes g \otimes 1_A: a \in A, g \in G\right>,$$
where $T(A \otimes R.G \otimes A)$ is the tensor algebra generated by the two-sided $A$-module $A \otimes R.G \otimes A$.
\item $A\# G$ is a $R.G$-module algebra given by
$$\bao{rrcl}
\Psi_{A\#G} : &R.G \otimes A\# G &\longrightarrow& A\#G\\
&&&\\
&(r g, a\# h) &\longmapsto&r \rho(g)(a)\# g h.\\
\ea$$
\en
\end{koro}
\bws The first statement is an immediate consequence of Prop. \ref{prop01} - the second statement is an immediate consequence of our definition of module algebras. %More general, we get an extended Ore extension $A[B]$ for any pair of algebras with semi-direct monoidal product $A \rtimes_\rho B$, where $\rho : B \longrightarrow \trm{Aut}A$ defines some representation.
\bmk The smashed product and its quotient algebras play an important role in deformation of singularities of algebraic varieties, a subfield of algebraic geometry. In addition, we will encounter smashed products in the context of general differential Galois theory.
\subsubsection{The internal module algebra}
Given a $B$-module algebra $(A,\Psi)$ we define for $\trm{Hom}_R(B,A)$ the $B$-module algebra structure
as follows
\begin{lemm}
The map $\Psi_{\trm{int}} : B \otimes \trm{Hom}_R(B,A) \longrightarrow \trm{Hom}_R(B,A)$ given
by
$$\Psi_{\trm{int}} = \left[b \otimes f\longmapsto f \circ \mu_B(b \otimes\_) := [b' \longmapsto f \circ \mu_B(b \otimes b')]\right]$$
is the proposed structure map.
\end{lemm}
\bws Again, we want to show that the two commuting diagrams following def. \ref{defi09a} hold.
\bn
\item let $b \in B$ and $f, g \in \trm{Hom}_R(B,A)$, then we have for $\Delta_B(b) = \sum_{(b)} b_{(1)} \otimes b_{(2)}$:
$$\bao{rcl}
\Psi_{\trm{int}}(b \otimes \mu_{\trm{Hom}_R(B,A)}(f \otimes g)) &=& \sum_{(b)} \mu_A\circ\left(f \circ \mu_B(b_{(1)} \otimes\_) \otimes g \circ \mu_B(b_{(2)} \otimes\_)\right)\circ\Delta\\
&&\\
&=& \left[b' \longmapsto \sum_{(b),(b')} f(b_{(1)} b'_{(1)}) g(b_{(2)} b'_{(2)})\right]\\
&&\\
\phi(b \otimes f \otimes g) &=& \sum_{(b)}\mu_{_C\mathcal{M}(B,A)} \circ (\Psi_{\trm{int}}(b_{(1)} \otimes f) \otimes \Psi_{\trm{int}}(b_{(2)} \otimes g))\\
&&\\
&=& \left[b' \longmapsto \sum_{(b),(b')} f(b_{(1)} b'_{(1)}) g(b_{(2)} b'_{(2)})\right]\\
\ea$$
As multiplication $\mu_{\trm{Hom}_R(B,A)}$ is given via convolution $\mu_A (f \otimes g) \Delta$ we showed the first diagram. Here, we use $\phi = \mu_A \circ (\Psi_{\trm{int}} \otimes \Psi_{\trm{int}}) \circ (id_B \otimes \tau_{B \otimes \trm{Hom}_R(B,A)} \otimes id_{\trm{Hom}_R(B,A)}) \circ (\Delta \otimes id_{\trm{Hom}_R(B,A)^{\otimes2}})$.
\item $1_{\trm{Hom}_R(B,A)} = \eta_A \eps_B$ then for all $b \in B$:
$$\bao{rcl}
\Psi_{\trm{int}}(b \otimes 1_{\trm{Hom}_R(B,A)}) &=& \eta_A\eps_B(\mu_B(b \otimes\_))\\
&&\\
&=& \left[b' \longmapsto \eta_A(\eps_B(b b')) = \eps_B(b b') \eta_A(1_R) = \eps_B(b) \eta_A(\eps_B(b'))\right]\\
&&\\
&=& \eps_B(b) \eta_A \eps_B\\ 
\ea$$
showing the second diagram.
\en
%\bmk We previously claimed that the two examples \ref{coalg_example} on pg. \pageref{coalg_example} are not isomorphic over the same ring/field.
\subsection{Hopf Algebras}
The concept of bialgebras has another specialization in the so called Hopf algebras, with one additional condition. For its definition we need to expand some concepts.
\begin{defi}\label{defi13}
Let $(B,\mu,\eta,\Delta,\eps)$ be a bialgebra.
\bn
\item For each algebra $(B,\mu,\eta)$, its opposite algebra $B^{\trm{op}}$ is defined as $(B,\mu^{\trm{op}},\eta^{\trm{op}})$, where $\mu^{\trm{op}} := \mu \tau_{B\otimes B}$, $\eta^{\trm{op}} = \eta$.
\item For each coalgebra $(B,\Delta,\eps)$, its opposite coalgebra $B^{\trm{cop}}$ is $(B,\Delta^{\trm{op}},\eps^{\trm{op}})$, where $\Delta^{\trm{op}} := \tau_{B \otimes B} \Delta$ and $\eps^{\trm{op}} = \eps$.
\item The opposite bialgebra $B^{\trm{copop}}$ is simply $(B,\mu^{\trm{op}},\eta,\Delta^{\trm{op}},\eps)$.
\item An antipode $S : B \longrightarrow B^{\trm{op}}$ is a homomorphism of algebras such that $$S * id_B := [x \longmapsto \sum_{(x)} S(x_{(1)}) x_{(2)}] = [x \longmapsto \sum_{(x)} x_{(1)} S(x_{(2)})] =: id * S = \eta \eps$$
(i.e. $S$ the two-sided inverse of identity with respect to convolution on $\trm{Hom}((B, \Delta, \eps),(B, \mu, \eta))$).
\item\label{defi131} An Hopf algebra $B$ is a bialgebra with antipode $S \in \trm{Hom}(B,B)$.
\en
\index{Index}{Hopf algebra}
\index{Index}{antipode}
\index{Index}{algebra!opposite}
\index{Index}{coalgebra!opposite}
\index{Index}{bialgebra!opposite}
\index{Symbol}{$\Delta^{\trm{op}}$}
\index{Symbol}{$\mu^{\trm{op}}$}
\index{Symbol}{$B^{\trm{op}}$}
\index{Symbol}{$B^{\trm{cop}}$}
\index{Symbol}{$B^{\trm{copop}}$}
\index{Symbol}{$S$}
\end{defi}
The concept of opposite algebra can be extended to objects in the category of groups: let $G$ be an object in $\trm{Grp}$, its opposite group $G^{\trm{op}}$ is the same set $G$, with the same unit map $e : \ast \longrightarrow G$, but with multiplication:
$$m : G \times G \longrightarrow G, (g,h) \longmapsto h g.$$
The notation used here is introduced in the appendix, in the category theory section. An anti-homomorphism in the category of groups is a group homomorphism $i : G \longrightarrow G^{\trm{op}}$. Thus, we can characterize the antipode as a bialgebra homomorphism:
$$S : B \longrightarrow B^{\trm{copop}},$$
i.e. an antihomomorphism (in the category of bialgebras). The following composed commutative diagram describes definition \ref{defi13}.\ref{defi131}:
$$\xymatrix{
B \ar[d]_\Delta\ar[r]^\eps & R\ar[r]^\eta & B\\
B\otimes B\ar[rr]^{id_B \otimes S}_{S \otimes id_B} &&B\otimes B\ar[u]_\mu\\
}$$
%\begin{defi}
%Let $(B,\mu,\eta,\Delta,\eps,S)$ be a Hopf algebra.
%\bn
%\item 
%\en
%\end{defi}
\subsubsection{Skew symmetric polynomials}
From now on, we assume $R$ to be some algebraically closed field. Let $q \in R^\times$ then the polynomial ring $A := R[X]$ has a Hopf-algebra structure as mentioned before. Revisiting the notion of Ore-extionsions, let $\alpha := [X^i \longmapsto (q X)^i]$ be an $R$-algebra homomorphism on $A$. Clearly, the zero-homomorphism is an $\alpha$-derivation.
\begin{defi}[Skew symmetric polynomials]
Let $\delta = 0_A$. The $A$-algebra $B := A[Y,\alpha,\delta]$ is called the ring of skew-symmetric polynomials.
\end{defi}
\bmk This can be constructed via the tensor algebra over some free $R$-module free of rank 2:
$$B \simeq R\left<X,Y\right>/\left<Y X - q X Y\right>.$$
This can be extended in the following sense: let $\mathfrak{g}$ denote an $R$-Lie algebra and $U(\mathfrak{g})$ denoted its universal enveloping algebra. If $\left<x_i : x_i \in \mathfrak{g}\right> = \mathfrak{g}$ we construct the quantized universal enveloping algebra $U_{\trm{quant}}(\mathfrak{g})$ via Ore-extensions as follows:
$A_1 := R[x_1]$ and $A_i := A_{i-1}[x_i,\alpha_i,\delta_i]$, with $\alpha_i \in \trm{Aut}_{\trm{alg}}(A_{i-1})$ and $\delta_i \in \trm{Der}_{\alpha_i}(A_{i-1})$.
 \bsp The best understood examples are the $q$-quantized universal enveloping algebras of finite dimensionals simple Lie algebras as $\mathfrak{sl}_n(R)$ (where $R$ is a field and $q \in R^\times\bsl\{1\}$). See for instance Klimyk \cite{Klim}, for $U_q(\mathfrak{so}_n)$ or Saito  \cite{Sait} for the general case.
\bmk Lastly, we like to note that Saito and Umemura gave a brief introduction to quantized Galois theory over $\qz$ \cite{SaitoUmemura,SaitoUmemura01}. There, the differential Galois group is a quantum group. However, this beyond the scope of this essay.
\section{Differential Rings and Modules}
Recalling our definition of differential rings (an associative algebra with derivation Lie algebra) we wish to extend our set of notations.
\subsection{Differential modules and ideals}
\begin{defi}
Let $(R,D)$ be a differential ring.
\bn
\item A differential module $(M,D_M)$ is an $R$-module with an additive map:
$$D : M \longrightarrow M,\ r m \longmapsto \partial_R(r) m + r D(m)\ \forall D \in D_M,\ r \in R,\ m \in M,$$
where $\partial_R \in \trm{Der}(R) = \left<D\right>$.
\item A differential ideal $I \subset R$ is a differential $R$-submodule of $R$, i.e. $\partial(I) \subset I$ for all $\partial \in D$.
\item Let $(R,D_R)$ and $(S,D_S)$ be two differential rings. A ring homomorphism $f : R \longrightarrow S$ is called differential if
$$\partial_S \circ f = f \circ \partial_R$$
or equivalently the following diagram commutes:
$$\xymatrix{
R \ar[d]_{\partial_R} \ar[r]^f & S\ar[d]^{\partial_S}\\
R \ar[r]_{f\mid_{\partial_R(R)}} & S.\\
}$$
%where $\partial_X(X) := \trm{im} \partial_X$ for $X = R, S$.
\item A differential ring $(R,D)$ is called simple differential if it has no proper differential ideals.
\item A differential field is a differential ring with no proper ideals (differential or non-differential).
\item Let $(R,D)$ be a differential ring. The subset
$$R^\partial := \{x \in R : \partial(x) = 0\ \forall \partial \in D\}$$
defines a subring and is called the ring of constants.
\en
\index{Index}{module!differential}
\index{Index}{ring!differential}
\index{Index}{ring!differential!homomorphism of}
\index{Index}{ring!differential!simple}
\index{Index}{ring!differential! of constants}
\index{Index}{field!of constants}
\index{Symbol}{$\trm{Der}_R(A)$}
\index{Symbol}{$R^\partial$}
\end{defi}
\bmk As mentioned, the derivation maps $D$ on differential modules $(M,D_M)$ are only additive (are in general not in $\trm{End}_R(M)$, but in $\trm{End}_{R^\partial}(M)$ - i.e. its ring of constants). This is due to the Leibniz-rule as defined above.\\
\indent The ring of constants $R^\partial$ for a differential field $(R, D)$ is also a field, since:
$$\partial(a a^{-1}) = 0 = \underbrace{\partial(a)}_{=0} a^{-1} + a \partial(a^{-1}) \LRA a \partial(a^{-1}) = 0$$
holds for all $a \in R^\times$.
\bsp Some prominent examples:
\bn
\item any ring $R$ is a (trivial) differential ring, via $0 : R \longrightarrow 0$. Thus, all $R$-modules are also differential modules via the zero-homomorphism.
\item The polynomial ring in one indeterminate:
$$\left(k[X], \partial = \frac{d}{dX}\right),$$
$k$ a field with characteristic zero, is a simple differential ring since all ideals $I$ generated by some polynomial of degree greater zero eventually fulfill $\partial^i(I) = (1)$. On the other hand, $(k[X], \partial = X \frac{d}{dX})$ has non-trivial differential ideals:
$$I_i = \left<X^i\right>,\ \forall i \geq 1$$
as each $k$-sub vector space $k.X^i$ is $\partial$-stable.
\item Let $p \in \nz$ be a prime number, the polynomial ring $\mathbb{F}_p[X]$ with $\mathbb{F}_p$-derivation $\partial = \frac{d}{d X}$, as in the last example, has an interesting property. Its ring of constants is $\mathbb{F}_p[X^p]$ as $\partial(X^p) = p X^{p-1} = 0$ and this ring contains indeed non-trivial differential ideals (in contrast to characteristic zero fields):
$$I_k := \left<X^{p^k}\right>,\ k \geq 1$$% : k \geq 1\right>,$$
since $\partial(X^{p^k} f + X^{p^k} g) = \partial(X^{p^k} f) + \partial(X^{p^k} g) = X^{p^k} \partial(f + g)$ for all $f, g \in \mathbb{F}_p[X]$.
\item The field of rational functions $k(X)$ is a differential field with derivation Lie algebra generated by
$$D = \left\{\partial'_x = \left[\frac{f}{g} \mapsto \frac{\partial_x(f) g - f \partial_x(g)}{g^2}\right] : x \in X\right\}.$$
\item Let $U \subset \rz^n$ be open and connected then $(C^\infty(U),D=\{\partial_i : 1 \leq i \leq n\})$ is a partial differential ring with a non-trivial differential ideal $\mathfrak{m} := \{f \in C^\infty(U) : \trm{supp} (f) \subsetneq U\}$, as we may define a proper differential ideal for all $f \in C^\infty(U) \bsl \rz[\{x\}]$ by simply putting
$$\left<\{f\}\right> = \left<\partial_{i_1} \circ \hdots \circ \partial_{i_k}(f) : 1\leq i_j \leq n,\ k \in \nz_0\right>,$$
where $\rz[\{x\}]$ denotes the ring of real convergent power series in $U$. These ideals are, in general, not finitely generated.
\en
\subsection{General Differential Algebra}
Let $(k,D)$ be a differential ring with $D = \{\partial\}$ and $k^\partial$ its ring of constants.
\subsubsection{Ring of differential operators}
In terms of Ore extension, the subring $D := k[\partial] \subset \trm{End}_{k^\partial}(k)$ is isomorphic to the Ore-extension $k[X,id_k,\partial]$ over $k$. To show this we depend on its intrinsic module algebra structure:\\
$$\Psi := \Psi_{\trm{int}} \mid_{D \otimes k.id_D} = \left[d' \longmapsto \left[d \longmapsto \mu_D(d' \otimes d) = d'(d)\right]\right],$$
where $\Psi_{\trm{int}} : D \otimes \trm{End}_{k^\partial}(D) \longrightarrow \trm{End}_{k^\partial}(D)$, restricted to the subalgebra $k.id_D$.
\indent $k[\partial]$ has the structure of a pointed-irreducible bialgebra, of Birkhoff-Witt type (only $1_{k[\partial]}$ as group-like, generator $X \in \mathfrak{g}_{k^\partial}(k)$ has primitive coproduct and $(k[\partial],\mu,\eta)$ is isomorphic to the enveloping algebra of some Lie-algebra). Its structure map
$$\Psi : D \otimes D \longrightarrow D,\ \partial \otimes a \partial^i \longmapsto \partial(a) \partial^i + a \partial^{i+1}.$$
To see the isomorphism wrt. Ore-extensions, we consider the $k$-left module $k.\partial^i$ with its above described $k$-right module structure:
$$\partial^i \otimes a \sim \sum_{j=0}^i \left(\bao{c}i\\j\\\ea\right) \partial^j(a) \otimes \partial^{i-j}$$
Hence, the ideal generators $1 \otimes X \otimes a - a \otimes X \otimes 1 - \partial(a) \in \bigoplus_{n \geq 0} (k \otimes \partial)^{\otimes n} \otimes k$, as in \ref{prop03} on pg. \pageref{prop03}, yield - evaluated via $id_k \otimes \Psi : k \otimes D \otimes k \longrightarrow k \otimes k \simeq k$:
$$id \otimes \Psi(1 \otimes X \otimes a - a \otimes X \otimes 1 - \partial(a) \otimes 1_D \otimes 1_k) = X(a) - a X(1) - \partial(a).$$
Identifying $X = \partial$, we get indeed zero as desired. Hence, the quotient yields our desired Ore-extension $k[X,id_k,\partial] \simeq k[\partial]$. There are more interesting properties for $k[\partial]$ that are discussed in \cite{vdPS01}, chapter 2, in greater detail.\\
\indent Concluding this subsection, we note that $k[\partial]$ is a unital, associative, coassociative, cocommutative bialgebra (indeed, has an antipode, which will be discussed below) acting on $k$ via evaluation.
\subsubsection{Ring of differential polynomials}\label{RingOfDiffPolys}
As above we are using a differential field $(k,\partial)$, $k^\partial$ its ring of constants and its ring of differential operators $k[\partial]$ being generated by one element $\partial$.%, pick some differential field $(k,\partial)$ and
 We consider the ring of polynomials $R := k[u_1,\ldots,u_n]$, i.e. a noetherian ring over $k$. In general, we have no unique extension of $\partial$ to $R$ except for that of a trivial differential ring: $u_i \longmapsto 0$ (enlarging the ring of constants to $R^\partial[u_1,\ldots,u_n]$). However, we may use a non-noetherian transcendental extension $S$ over $R$
$$S := k[u_{i,j} : 1 \leq i \leq n, j \in \nz_0],$$ where $u_{i} \longmapsto u_{i,0}$ defines an embedding of $R$ in $S$. Indeed, this ring still has no canonical $D = k[\partial]$-module algebra structure. However, the module:
$$D \otimes_k S := D \otimes_{k^\partial} S/\left<d \otimes x s - \sum_{(d)} d_{(1)}(x)d_{(2)} \otimes s : x \in k, d \in D, s \in S\right>,$$
where $\Delta(d) = \sum_{(d)} d_{(1)} \otimes d_{(2)}$, has:
$$\partial^i \otimes f_\alpha u^\alpha \sim_\partial \sum_{j=0}^i\left(\bao{c}i\\j\\\ea\right) \partial^j(f_\alpha) \partial^{i-j}\otimes u^\alpha.$$
Its module algebra structure map $\Psi : D \otimes_k (D \otimes_k S) \longrightarrow D \otimes_k S$ is given by the above equivalence relation. The ideal
$$I := \left<\partial^{\alpha_3} \otimes u_{\alpha_1,\alpha_2} - \partial^\beta \otimes u_{\alpha_1,\alpha_2 + \alpha_3 - \beta} : 1 \leq \alpha_1 \leq n, \alpha_2, \alpha_3 \in \nz_0, 0 \leq \beta \leq \alpha_3 - 1\right> \subset D \otimes_k S$$ gives us our desired:
\begin{defi}[Ring of differential polynomials]
Let $S$ and $I$ be defined as above. The quotient ring $D \otimes_k S/I$ is called the ring of differential polynomials and is denoted by:
$$k\left\{u_1,\ldots,u_n\right\} := D \otimes_k S/I.$$
\end{defi}
\bmk %Firstly, we remark that if $\partial u_{i,j} = u_{i,j+1} \in K\{u\}$ then also 
%$$\partial^l(u_{i,j}) = \partial^{l-1}(u_{i,j+1}) = \ldots = u_{i,j+l} \in k\{u\}\ \forall 1 \leq i \leq n,\ j \geq 0,\ l \geq 1.$$
Sometimes we may use $u_i^{(j)}$ instead of $u_{i,j}$. Secondly, note that this, indeed, defines a (non-noetherian) differential ring, with derivations $\{\partial, \partial_{u_{i,j}}\}$. Here, $\partial_{u_{i,j}}$ are $k$-derivations, while $\partial$ is a $k^\partial$-derivation. Additionally, we may still recover $R$:
$$R \simeq k\{u_1,\ldots,u_n\}/J,\ \trm{where}\ J := \left<u_{i,1} - 1, u_{i,j} : 1 \leq i \leq n, j \geq 2\right> \subset k\{u_{i,j}\}.$$
\indent The definition naturally translates to differential rings - if $(R,\partial)$ is our differential ring, with ring of constants $R^\partial$ and $D$-stable ideal $I$ as above, then
$$R\{u_1,\ldots,u_n\} := D \otimes_k R[u_{i,j} : 1 \leq i \leq n, j \geq 0]/I.$$
Although being non-noetherian (any non-trivial differential ideal does not fulfill the ascending chain condition), the factor rings we will consider are in fact noetherian.
\begin{defi}
Let $(k,\partial)$ be a differential ring (field). An associative unital $k$-algebra $K$ is called a differential extension over $k$, short $K/k$, if there is an $n \in \nz$ and some differential ideal $I \subset k\{u_1\ldots u_n\}$ such that
$$K \simeq k\{u_1\ldots u_n\}/I.$$
\index{Index}{extension!differential}
\end{defi}
\bmk Given a differential ring $(k,\partial)$ and a finite family of differential polynomials $\mathcal{F} \subset k\{u_1\ldots u_{|\mathcal{F}|}\}$ the associated differential ideal $I$ is simply the differential saturation:
$$I := \left<\partial^i(f) : f \in \mathcal{F},\ i \geq 0\right>.$$
Such a family is called a differential equation - linear if the degree of all monomials is at most one, otherwise non-linear. We call it an explicit differential equation if each element in $\mathcal{F}$ is linear with respect to $\partial u_i$ and all coefficients of monomials of the form $\prod_{i,j} \partial^j u_i$ are zero, for $j \geq 2$. Otherwise, it is called implicit.
\index{Index}{differential saturation}
\index{Index}{differential equation}
\index{Index}{differential equation!linear}
\index{Index}{differential equation!non-linear}
\index{Index}{differential equation!explicit}
\index{Index}{differential equation!implicit}
\begin{defi}
Let $k\{u_1,\hdots,u_n\} =: k\{u\}$ be the ring of differential polynomials over some differential field $k$, $k[\partial]$ the left $k$-module of differential operators on $k\{u\}$. The map
$$\bao{rrcl}
ev &: k[\partial] \otimes_{k^\partial} k\{u\} & \longrightarrow &k\{u\}\\
&&&\\
& \sum_{\substack{0 \leq i \leq n\\\alpha \in \nz_0^k\\k\geq 0}} (a_i \partial^i \otimes b_\alpha u_\alpha) &\longmapsto& \sum_{\substack{i,\alpha\\k \leq i}} \left(\bao{c}i\\k\\\ea\right) a_i \partial^k(b_\alpha) \partial^{i-k}(u_\alpha)\\
\ea$$
is called the evaluation homomorphism.
\end{defi}
\bmk %The evaluation homomorphism gives us a $k[\partial]$-module algebra structure on $k\{u\}$, where $(k[\partial],\mu,\eta,\Delta,\eps)$ is the bialgebra structure on $k[\partial]$ (infact, it has a Hopf-algebra structure via $S : k[\partial] \longrightarrow k[\partial]$, $\partial \longmapsto -\partial$.
Obviously, the definition $ev$ and $\Psi$ are equal. Using this setting we get
\begin{koro}
The two-sided $k$-module $k[\partial]$ defines (in general non-commutative if $k^\partial \neq k$) a unital $k$-algebra.
\end{koro}
\bws As $k$ is a unital commutative simple algebra over $k^\partial$ and $k[\partial]$ is a Ore-extension as shown above, there is nothing more to show.%Note that the map $\mu : k[\partial] \otimes k[\partial] \longrightarrow k[\partial], a \partial^i \otimes b \partial^j \longmapsto \sum \left(\bao{c}i\\k\\\ea\right) a \partial^k(b) \partial^{i+j-k}$ defines a multiplication on the monomial terms of $k[\partial]$. The unit is simply $\eta : K \longrightarrow K[\partial], 1_k \longmapsto 1_{k[\partial]}$.
\bmk Moreover, $k[\partial]$ has a $k^\partial$-coalgebra structure:
$$\Delta =\left[ \partial^i \longmapsto \sum \left(\bao{c}i\\j\\\ea\right) \partial^{i-j} \otimes \partial^j\right],\ \eps = [\partial^i \longmapsto \delta_{i,0}]$$
making it to a $k^\partial$-bialgebra as clearly: $\eps \eta = id_k$, $(\eta \otimes \eta) \circ \eta = \Delta \eta = id_{k.1_{k[\partial]} \otimes k.1_{k[\partial]}}$.% With $S := [\partial^i \longmapsto (-1) \partial^i]$ we get an antipode since $\mu((S\otimes id)(1_k)) = \mu((id_k \otimes S)(1_k)) = \eta(\eps(1_k))$.
\begin{prop}
The following statements are equivalent:
\bn
\item\label{item01} $k\{u\}$ is a $k[\partial]$-module algebra (or $\Psi := ev$ defines a module algebra structure on $k\{u\}$).
\item \label{item02} Given the evaluation homomorphism and multiplication on $k[\partial]$ then the following diagram commutes
$$\xymatrix{
k[\partial] \otimes k[\partial] \otimes k\{u\} \ar[r]^{\mu \otimes id_{k\{u\}}}\ar[d]_{id_{k[\partial]} \otimes ev} & k[\partial] \otimes k\{u\}\ar[d]^{ev}\\
k[\partial] \otimes k\{u\} \ar[r]_{ev} & k\{u\}\\
}$$
\en
\end{prop}
\bws We show first, that the second statement is indeed true.
\bn
\item By simple computation on the monomial terms $a \partial^i, b \partial^j \in k[\partial]$ and $c u_\alpha \in k\{u\}$ we get
{\scriptsize
$$\bao{rcl}
ev(\mu \otimes id_{k\{u\}})(a \partial^i \otimes b \partial^j \otimes c u_\alpha) &=& \sum_{k' \leq i} \sum_{l' \leq i + j - k'} \left(\bao{c}i\\k'\ea\right) \left(\bao{c}j + i - k'\\l'\ea\right) a \partial^{k'}(b) \partial^{l'}(c) \partial^{j+i-k'-l'}(u_\alpha)\\
&&\\
ev(id_{k[\partial]} \otimes ev)(a \partial^i \otimes b \partial^j \otimes c u_\alpha) &=& \sum_{k \leq i} \sum_{l \leq j} \sum_{m \leq i - k} \left(\bao{c}i\\k\ea\right) \left(\bao{c}j\\l\ea\right) \left(\bao{c}i - k\\m\ea\right) a \partial^k(b) \partial^{l+m}(c) \partial^{j+i-k-l-m}(u_\alpha)\\
\ea$$}
Fixing $l' \leq j$ and putting $k = k'$ we see that our equivalence implies $$\sum_{l + m = l'} \left(\bao{c}j\\l\ea\right) \left(\bao{c}i - k\\m\ea\right) a \partial^k(b) \partial^{l+m}(c) \partial^{j+i-k-l-m}(u_\alpha) = \left(\bao{c}j + i - k\\l + m\\\ea\right) a \partial^k(b) \partial^{l + m}(c) \partial^{j + i - l - m - k}(u_\alpha).$$
Since the degree of the differential operators on each factor do agree, we get
$$\sum_{l + m = l'} \left(\bao{c}j\\l\ea\right) \left(\bao{c}i - k\\m\ea\right) = \left(\bao{c}j + i - k\\l + m\\\ea\right).$$
But this is just a rewriting of the Vandermonde identity for all $k \leq i$ and proves our claim.
\item If $k\{u\}$ is a $k[\partial]$ module algebra we have that $(x y) v = x(y v)$ for all $x, y \in k[\partial]$, $v \in k\{u\}$. Expanding with our standard notation this translates into $ev(\mu\otimes id) = ev(id\otimes ev)$ proving \ref{item01} $\RA$ \ref{item02}.\\
The opposite direction follows immediately from the definition.
\en
\begin{koro}\label{HopfModAlgDiffPoly}
With bialgebra structure maps $\mu, \eta, % = [1_k \longmapsto 1_{k[\partial]}], 
\Delta$ and $ % = \left[a \partial^i \longmapsto \sum_{0 \leq j \leq i} \left(\bao{c}i\\j\\\ea\right) a \partial^j \otimes \partial^{i - j}\right], 
\eps$ as defined above and the $k$-homomorphism
% = [1_{k[\partial]} \longmapsto 1_k, a \partial^i \longmapsto 0],
$$S : k[\partial] \longrightarrow k[\partial],\ \partial^i \longmapsto (-1)^i \partial^i,$$
we have that $k\{u\}$ is a $k[\partial]$-Hopf-module algebra.
\end{koro}
\bws Note that the generator $\partial$ is a primitive cocommutative Hopf algebra element, i.e. $\Delta(\partial) = 1 \otimes \partial + \partial \otimes 1$ implying $S(\partial) = - \partial$. We simply have to show that the commutative diagram for Hopf-algebras does commute.
$$\partial^i \stackrel{\Delta}{\longmapsto} \sum_{j=0}^i\left(\bao{c}
i\\
j\\
\ea\right) \partial^j \otimes \partial^{i-j} \stackrel{S \otimes id}{\longmapsto} \sum_{j} \left(\bao{c}
i\\
j\\
\ea\right) (-1)^j \partial^j \otimes \partial^{i-j} \stackrel{\mu}{\longmapsto} \sum_j \left(\bao{c}
i\\
j\\
\ea\right) (-1)^j \partial^i.$$
According to Pascals rule we see via induction that except for $i = 0$ all sums are zero. As $\eta(\eps(\partial^i)) = \delta_{i,0}$, we have just shown the required commutativity.
%Hence, all elements $\sum c_\alpha u_\alpha$ can be associated with some element of the form $\sum \left(\bao{c} i\\j\\\ea\right) \partial^j(b_\alpha) \partial^{i-j}(u_\alpha)$ we get an (not necessarily unique) element such that for $\rho = \left[\sum c_\alpha u_\alpha \longmapsto \sum \left(\bao{c}i\\j\\\ea\right) \partial^j(b_\alpha) \otimes \partial^{i-j}(u_\alpha)\right]$
\subsection{Linear Differential Equations}
Let $(k,\partial)$ be a (not necessarily non-trivial with characteristic zero) differential field with field of constants denoted by $k^\partial$ and $M$ a noetherian differential module over $k$ (i.e. finite dimensional vector space, with derivation).
\begin{defi}
A (scalar) linear differential equation is $k^\partial$-linear map $L : M \longrightarrow M$, with $L = \sum a_i \partial^i$, i.e. $L \in k[\partial] \subset \trm{End}_{k^\partial}(k \otimes_{k^\partial} M)$. The solution space is the $k^\partial$-subspace of $M$:
$$S(L) := \{x \in M : L(x) = 0\}.$$
\index{Index}{space!solution}
\index{Symbol}{$S(L)$}
\end{defi}
Alternatively, one can define linear differential equations simply via linear algebra:
$$\partial(x) = A x,\ A \in \trm{End}(M),\ x \in M.$$
Its solution space is simply generated by the kernel elements of $\partial - A$ in $M$. Hence, all differential extensions are generated by solutions over $k$. Now, an important definition:
\begin{defi}
Let $K/k$ be a differential extension to the differential equation $\partial(x) = Ax$.
\bn
\item we call a matrix $X \in \trm{Mat}_n(K)$ a solution matrix, if
$$\partial(X) = A X \in \trm{Mat}_n(K).$$
If $X = (x_{ij}) \in \trm{Mat}_n(K)$ is a solution matrix for the above differential equation we call the $n^2 \times n^2$ matrix
$$Wr(X) = \left(\partial^l(x_{ij})\right)_{\substack{0 \leq l \leq n^2-1\\1 \leq i,j \leq n}}$$
the Wronskian matrix. Its determinant is simply called Wronskian.
\item $K/k$ is called a differential extension (over $k$), if $K$ contains the solution space for some linear differential equation $\partial(x) = A x$.
\en
%A solution matrix in the general linear group $\trm{Gl}_n(R)$ is called a fundamental matrix.
\index{Index}{matrix!solution}
\index{Index}{matrix!Wronskian}
\index{Index}{matrix!fundamental}
\index{Index}{Wronskian}
\end{defi}
\bmk As we already defined differential extensions via the ring of differential polynomials $k\{u\}$ (to be precise via differential quotient rings) we shall show both definitions are equivalent. But clearly, the family of differential polynomials is simply $\mathcal{F} := \{\partial u_i - \sum_j a_{i,j} u_j \in k\{u\} : 1 \leq i \leq n\}$ for a given differential equation $\partial u = A u$.\\
\indent Secondly, we get the Wronskian matrix by constructing column vectors $$y_{ij} = \left(x_{ij},\partial(x_{ij}),\ldots,\partial^{n^2-1}(x_{ij})\right)^t$$running over all indices $1 \leq i,j \leq n$. Furthermore, the definition of the Wronskian is broader - for some differential extension $R/k$ and elements
$y_1, \ldots, y_m \in R$ the Wronskian matrix is simply $Wr(y_1,\ldots,y_m) := (\partial^l(y_i))_{1 \leq i,l + 1 \leq m}$.
\begin{defi}
Let $K/k$ be a differential field extension for a given differential equation $\partial y = A y$.
\bn
\item A Picard-Vessiot ring $R$ is a sub-ring of $K$, such that
$R^\partial = k^\partial$, $R$ contains no non-trivial differential ideals and there exists a solution matrix $X \in \trm{Gl}_n(R)$.
\item A solution matrix in a PV-ring $R$ is called a fundamental matrix.
\item A Picard-Vessiot field is the localization of a Picard-Vessiot ring.
\en
\end{defi}
\index{Index}{extension!differential}
\index{Index}{extension!Picard-Vessiot}
\index{Index}{ring!Picard-Vessiot}
\index{Index}{field!Picard-Vessiot}
The last definition requires a little
\begin{lemm}
A simple differential ring is zero-divisor free.
\end{lemm}
\bws Let $a \in \trm{Ann}(R)$, then there is a $b \in \trm{Ann}(R)\bsl\{0\}$, s.t. $a b = 0$. We get $\partial(a b) = 0 = \partial(a) b + a \partial(b) \LRA \partial(a) b = -a \partial(b)$. Multiplying both sides with $b$ we have:
$$\partial(a) b^2 = -a b \partial(b) = 0,$$
i.e. $\partial(a) \in \trm{Ann}(R)$. As the only proper differential ideal is zero, we see that $\trm{Ann}(R)$ is trivial.\\
Now, we have that indeed the localization of a Picard-Vessiot ring is well-defined in that sense that the localization is not the zero ring. Alternatively, we could define the Picard-Vessiot field extension $K/k$ as a field containing the solution space and having the same field of constants, i.e. $K^{\partial} = k^{\partial}$.
\newcommand{\minpoly}[1]{\trm{Min}(\alpha,#1)}
\newcommand{\minpolyC}{\minpoly{k^\partial}}
\newcommand{\minpolyR}{\minpoly{R}}
\begin{lemm}
Let $(k,\partial)$ be a differential field of characteristic zero and let $R$ be a differential subring of $k$
with the same field of constants, i.e. $R^\partial = k^\partial \subset R \subset k$.
\bn
\item If $\alpha \in R$ is algebraic over $k^\partial$, i.e. $\minpolyC \in k^\partial[X]$, with $k^\partial(\alpha) \simeq k^\partial[X]/\left<\minpolyC\right>$, then $\partial(\alpha) = 0$.
\item If $\alpha \in k$ is algebraic over $R$ and $\partial(\alpha) = 0$, then $\alpha$ is algebraic over $k^\partial$.
\en
\end{lemm}
\bws Let $\minpoly{S} = p = \sum_{i=0}^n p_i X^i \in S[X]$ for $S = R, k^\partial$.
\bn
\item By definition, we have $p \in k^\partial[X]$. Hence, evaluating $p$ at $\alpha$ in $R$ gives
$$p(\alpha) = \sum_{i=0}^n p_i \alpha^i = 0$$
Differentiating:
$$\partial(p(\alpha)) = \sum_{i=1}^n p_i \partial(\alpha^i) = \sum_{i=1}^n i p_i \partial(\alpha) \alpha^{i-1} = \left(\sum_{i=0}^{n-1} (i + 1) p_{i+1} \alpha^i\right) \partial(\alpha) = 0$$
Since $p_n = 1$ we see that the left hand factor cannot be zero. On the other hand, $k^\partial(\alpha)$ is an integral domain. Hence, $\partial(\alpha) = 0$.
\item We interpret 'algebraic over $k$ as 'integral over $R$. Thus, there is a monic polynomial as defined above over $R$. Proceeding as in the last part (i.e. evaluating $p$ at $\alpha$ in $R[\alpha]$ and differentiating), we get
$$\partial(p(\alpha)) = \sum_{i=0}^{n-1} \left(\partial(p_{i}) + (i + 1) p_{i+1} \partial(\alpha)\right) \alpha^i + \partial(p_n) \alpha^n = 0.$$
Since $p_n = 1$ we get
$$\partial(p_i) + \underbrace{(i + 1) p_{i+1} \partial(\alpha)}_{=0} = 0$$
for all $i = 0,\cdots,n - 1$. As $\trm{char} R = 0$ implies $\partial(p_i) = 0$ showing $p \in k^\partial[X]$.
\en

\begin{prop}\label{PicardVessiotRing}
Let $\partial(x) = A x$, as above.
\bn
\item A Picard-Vessiot ring $R$ is isomorphic to
$$k[x_{ij},1/\det X],$$
where $X = (x_{ij}) \in \trm{Gl}_k(M)$ is a fundamental matrix.
\item A matrix $X \in \trm{Mat}_n(R)$ is a fundamental matrix, if and only if its Wronskian is non-zero over $k^\partial$.
%\item The entries of the largest sub matrix of the Wronskian matrix  $Wr(L) = \left(x^{(k)}_{ij}\right)_{\substack{0 \leq k \leq n^2 - 1\\1 \leq i, j \leq n}}$ is a $C$-basis of the solution space, if and only if its Wronskian matrix has non-zero determinant. The associated fundamental matrix $X$ is spanned by this basis.
\item \label{PVLemma3}Two fundamental matrices $X_1, X_2$ are right-associated wrt. $\trm{Gl}_n(k^\partial)$.
\item \label{PVLemma4}Two PV-rings $R_1, R_2$ of the same equation are isomorphic as differential rings.
\en
\end{prop}
\bws A proof can be found in \cite{vdPS01}, pg. 15. However, for the last two statements we are going to present a sketch of proof:
\bd
\item[ad \ref{PVLemma3}] We have: $\partial(X_{1,2}) = A X_{1,2}$ and assume: $X_2 = X_1 M$ for some $M \in \trm{Gl}_n(R)$. Then:
$$\partial(X_2) = \partial(X_1 M) = \partial(X_1) M + X_1\partial(M) = A X_1 M + X_1 \partial(M) \stackrel{!}{=} A X_2 \LRA M \in \trm{Gl}_n(k^\partial).$$
We recall that $X_i \in \trm{Gl}_n(R)$ and the $X_1, X_2$ are left-associated if and only if $[M, A] = 0$.
\item[ad \ref{PVLemma4}] Let $S(L)_i$ denote the two solution spaces. As we just saw, we may define an isomorphism of differential modules $\phi : S(L)_1 \longrightarrow S(L)_2, x_{ik} \longmapsto \sum_{j} x_{ij} m_{jk}$. As the PV-ring is commutative, we deem $\trm{Sym}(S(L)_i)$ as an appropriate choice for construction (not necessarily the PV-ring, rather a subring). Now, we may extend $\phi$ to both algebras:
$$s = \sum s_i x^i = \sum s_i x_1^{i_1} \ldots x_{n}^{i_n} \longmapsto \sum s_i \phi(x_1)^{i_1} \ldots \phi(x_n)^{i_n}.$$
Since $1_{S(L)_1} \longmapsto 1_{S(L)_2}$ and $\phi$ is an isomorphism on $\trm{Sym}^1(S(L)_i)$ we get the desired isomorphism of differential rings. Localizing both algebras wrt. $\det X_i$ we have that $R_i \simeq S_{\det X_i} (L)_i$ for $i = 1,2$.
\ed
\bmk $\trm{Sym}(S(L)) \simeq k[x_1,\ldots,x_n]$ is not a simple differential ring (the maximal ideal $I = \left<x_i : 1 \leq i \leq n\right>$ is closed under $\partial$-action). Nevertheless, its localization wrt. $\det X = \sum_{\sigma \in S_n} sign(\sigma) \prod_{i=1}^n x_{i,\sigma(i)}$ is a simple differential ring, as $S_{\det X}^{-1}I$ contains units ($S_{\det X}^{-1}I = R$).\\
\indent In \cite{vdPS01}, the following theorem is given (prop. 2.9, pg. 40 and lem 2.10, pg. 41):
\begin{satz}
Let $\partial(x) = A x$ be a linear differential equation. There are $L_i \in k[\partial]$, such that
$$V_A := \{y \in M : \partial(y) = A y\} \simeq \bigoplus_{i=1}^n k[\partial]/k[\partial].L_i^*.$$
To specify, each matrix equation has a solution space $V_A$ isomorphic to the solution space of a scalar equation $L = \prod L_i$.
\end{satz}
The decomposition is not unique wrt. left- and right-sidedness, as $k[\partial]$ is in general not commutative. We are not going to prove this theoremo - only loosely scatching a proof.\\
First we note, for any field $k$ and a matrix $A \in \trm{Mat}_n(k)$, there exists vectors $c_i \in \bigoplus_{i=0}^n k.e_i$ such that $B_i = \left\{c_i,...,A^{j_i} c_i : i\right\}$ is a $k$-basis. The elements $c_i$ are called cyclic vectors, each subspace $M_i$ generated by such an element is called a cyclic vector space and the transformed matrix wrt. to the cyclic basis is called the rational normal form of $A$. If $\chi_A \in k[X]$ is the charactistic polynomial of $A$ and $\prod_{j = 1} p_j^{s_j}$ a prime decomposition of $\chi_A$ than its minimal polynomial, which is the monic generator of $\{f \in k[X] : f(A) = 0\}$, gives us all distinct cyclic vectors (each cyclic vectors space has different dimension $t_j \deg p_j$, where $p_j^{t_j} \mid m_A$ and $p_j^{t_j+1} \nmid m_A$). A similar reasoning is provided in \cite{vdPS01}, chapter 2. Here, a cyclic decomposition is given for any linear homogeneous differential equation.
% due to the maximality of $I$: any differentially closed ideal containing $I$ is already $R = (1)$. To see that its ring of constants $R^\partial = C$, consider $f = \sum_{i} f_i x_1^{i_1} \ldots x_n^{i_n}/\det X^{i_{n+1}} \in \ker \partial_R$:
%$$\partial(f) := \sum_i \partial(f_i) x^i/\det X^{i_{n+1}} + \sum_i i_k f_i x_1^{i_1} \ldots \partial(x_k) x_k^{i_k-1} \ldots x_n^{i_n}/\det X^{i_{n+1}}$$
\subsubsection{Differential Galois group}
In the theory of field extensions, the Galois group is the set of $k$-vector space automorphisms permuting the root elements of a given polynomial over $k$ leaving $k$ fixed. For a given linear differential equation, the definition is slightly different:
\begin{defi}
Let $\partial(x) = A x$ for some $A \in \trm{Gl}_k(M)$ and $x \in M$. The differential Galois group is the zentralizer of $\left<\partial\right> \subset \trm{Gl}_R(M)$, i.e.
$$\trm{DGal}(R/k) = \{\varphi \in \trm{Gl}_{k^\partial}(M) : \partial \varphi = \varphi \partial\}.$$
\index{Symbol}{$\trm{DGal}(R/k)$}
\end{defi}
This property is also called equivariance (e.g. in algebraic geometry).
%\paragraph{Examples for the linear case}
\begin{satz}[Galois correspondence]
Let $K/k$ be a PV extension for $\partial(x) = A x$. Let $\mathfrak{G}$ be the set of all closed subgroups of $G := \trm{DGal}(K/k)$ and $\mathcal{M}$ the set of all differential subfields $k \subset M \subset K$. In analogy to algebraic fields extension theory, we define 
$$\bao{rrclrcl}
\trm{Fix} : &\mathfrak{G} &\longrightarrow& \mathcal{M},& H &\longmapsto& K^H := \{x \in K : \sigma(x) = x\ \forall \sigma \in H\}\\
&&&&&\\
\trm{DGal} : & \mathcal{M} &\longrightarrow& \mathfrak{G},& M &\longmapsto& \trm{DGal}(M/k),\\
\ea$$
then we have:
\bn
\item both functors are inverse to one another.
\item $H \in \mathfrak{G}$ is normal in $G$ if and only if for $M = K^H$:
$$G(M) \subset M,$$
i.e. is $G$-invariant as a set ($g x \in M$ for all $g \in G$, $x \in M$).
\item If $H \in \mathfrak{G}$ is normal then the canonical projection $G \longrightarrow \trm{DGal}(K^H/k)$ is surjective and has $H$ as kernel. Furthermore, $K^H$ is a PV extension for some linear differential equation over $k$.
\item Let $G^0 \leq G$ be the connected component of identity, then $K^{G^0} = k$.
\en
\end{satz}
\bmk A proof is given in \cite{vdPS01}. The connected component of identity is to be understood as follows. If $(G,m,e,\tau)$ is a topological group, with $\tau$ its topology, an open subset $U \in \tau$ is called connected if the only disjoint union of open subsets $U' \cup U'' = U$ is trivial (i.e. $U' \in \{\emptyset, U\}$, $U'' = U\bsl U'$). In this case, $U$ is also called a connected component. If furthermore, $1 \in U$ is a subgroup, $U$ is called the connected component of identity which is denoted by $G^0$. These definitions were taken from \cite{Milne}.
\bsp Some examples:
\bn
\item $(\zz,+,0,\mathcal{P}(\zz))$ is a topological group wrt. discrete topology. Hence, all pointed subspaces are open subsets. Therefore, $\{0\} = \zz^0$.
\item $(\rz^\times,\cdot,1,\tau_{\trm{standard}})$ is a topological group with connected components: $\rz_{>0}, \rz_{<0}$. Hence, $\rz_{>0} = \left(\rz^{\times}\right)^0$.
\en
\subsection{Example}
Now, let us consider two simple examples.
\subsubsection{1-dim case}
Let $k = \currfield(z)$ with $\currfield$-derivation $\partial = \frac{d}{d z}$, as well as $a \neq 0$ and $\partial(x) = ax$. Here, we have a 1-dimensional differential equation, i.e. the solution space is a 1-dim $\currfield$-vector space. Let us compute the
\bd
\item[Fundamental matrix] Put $X = x \in \trm{Gl}_1(\currfield(z))$. Clearly, $Wr(X) = (x)$ and $\det Wr(X) = x \neq 0$. Hence, $x$ belongs to the $\currfield$-basis of the solution space.
\item[Picard-Vessiot ring] By prop. \ref{PicardVessiotRing} we know that $R \simeq \currfield(z)[x,1/x]$.
\item[Differential Galois group] Pick some $f \in \trm{Gl}_1(R)$, by definition $f \in \trm{DGal}(R/k)$ if $f$ commutes with $\partial$:
$$\partial(f x) = \partial(f) x + f a x = f a x = f \partial(x) \LRA \partial(f) = 0,$$
i.e. $f$ is in $\currfield.id$. Thus, we have that $\trm{DGal}(R/k) = \left<\partial\right> \simeq \qz^\times$.
\ed
\subsubsection{2-dim case} \label{twoD}
Let $k \subseteq \currfield$ with trivial derivation $\partial\mid_{\currfield} = 0_{\currfield}$, as well as $a \neq 0$ and $\partial^2(x) = a x$. Clearly, the solution space in question is contained in a two-dimensional space with companion matrix $A = \left(\bao{cc}
0 & 1\\
a & 0\\
\ea\right)$.
\paragraph{Fundamental matrix} Put $X = \left(\bao{cc}
x_{11} & x_{12}\\
x_{21} & x_{22}\\
\ea\right)$ as fundamental matrix with $\partial(x_{11}) = x_{21}, \partial(x_{21}) = a x_{11}$ as well as $\partial(x_{12}) = x_{22}, \partial(x_{22}) = a x_{12}$. The Wronskian matrix is
{\scriptsize
$$\left(\bao{cccc}
x_{11} & x_{21} & x_{12} & x_{22}\\
x_{21} & a x_{11} & x_{22} & a x_{12}\\
a x_{11} &  a x_{21} & a x_{12} & a x_{22}\\
a x_{21} & a^2 x_{11} & a x_{22} & a^2 x_{12}\\
\ea\right)
$$}
Obviously, $\det Wr(X) = 0$ as the fourth and third row are linear combinations of the first two rows. For instance, $x_1 = x_{11}$ and $x_2 = x_{21}$ forms a $\currfield$-basis of the solutions space. Since the remaining indeterminates are linearly dependent over $\currfield$ we compute for $x_{12} = \lambda_{11} x_1 + \lambda_{12} x_2, x_{22} = \lambda_{21} x_1 + \lambda_{22} x_2$ and $\lambda_{ij} \in \currfield$:
$$\bao{rclcl}
\partial(x_{12}) &=& \lambda_{11} \partial(x_1) + \lambda_{12} \partial(x_2) &=& \lambda_{12} a x_1 + \lambda_{11} x_1\\ 
&&&&\\
&\stackrel{!}{=}& \lambda_{21} x_1 + \lambda_{22} x_2 &\LRA& \lambda_{21} = \lambda_{12}, \lambda_{22} = \lambda_{11} a\\
&&&&\\
\partial(x_{22}) &=& \lambda_{21} \partial(x_1) + \lambda_{22} \partial(x_2) &=& a \lambda_{11} x_1 + a \lambda_{11} x_1\\
&&&&\\
&\stackrel{!}{=}& a \lambda_{11} x_1 + a\lambda_{12} x_2\\
\ea$$
We see that we may chose any $\lambda_{11} =: \lambda_1, \lambda_{12} = \lambda_2 \in \currfield$ such that $\det X$ is a unit. Continuing:
$$\det X = \det \left(
\bao{cc}
x_1 & \lambda_1 x_1 + \lambda_2 x_2\\
x_2 & a \lambda_2 x_1 + \lambda_1 x_2\\
\ea\right) = %a \lambda_2 x_1^2 + \lambda_1 x_1 x_2 - \lambda_1 x_1 x_2 - \lambda_2 x_2^2 
= a \lambda_2 x_1^2 - \lambda_2 x_2^2 = \lambda_2 (a x_1^2 - x_2^2).$$
We can set $\lambda_1$ arbitrarily but chose zero and $\lambda_2 = 1$,
%Identifying the remaining variables with $x_{12} = a^{-1} x_2$, $x_{22} = a x_1$
 we get the fundamental matrix $X = \left(\bao{cc}
x_1 & x_2\\
x_2 & a x_1\\
\ea\right)$. Its inverse is
$$X^{-1} = \frac{1}{a x_1^2 - x_2^2} \left(
\bao{cc}
a x_1 & -x_2\\
-x_2 & x_1\\
\ea\right)$$
In Heidereich 2010, he provides the condition $\sum_{i=0}^n a_i \partial^i(X) X^{-1} \in M_2(k)$ for the fundamental matrix. But clearly, if $\partial(X) = A X$ then $\partial^i(X) = A^i X$ and the condition holds.% In our case:
%$$\partial(X)X^{-1} = A X X^{-1} = \frac{1}{a x_1^2 - a^{-1} x_2^2}\left(\bao{cc}
%x_2 & a x_1\\
%a x_1 & a x_2\\
%\ea\right)\left(
%\bao{cc}
%x_1 & a^{-1} x_1\\
%x_2 & a x_2\\
%\ea\right) = A.$$
\paragraph{Differential Galois group} we have $g = (g_{ij}) \in \trm{DGal}(\currfield[x_1,x_2,1/\det X]/\currfield)$ iff $g A - A g = 0$ and $\det g \in \currfield^\times$. We get:
$$\bao{cc}
a g_{12} - g_{21} = 0 & -a g_{12} + g_{21} = 0\\
a g_{11} - a g_{22} = 0 & -a g_{11} + a g_{22} = 0.\\
\ea$$
Therefore we have $g = \left(\bao{cc}
g_{11} & g_{12}\\
a g_{12} & g_{11}\\
\ea\right)$. The determinant is $\det g = g_{11}^2 - a g_{12}^2$. If $g_{ij} \in \currfield$ we get that $\det g = 0 \LRA g_{11}^2 = a g_{12}^2$. Since $\currfield$ is algebraically closed, there is a $b \in \currfield$ such that $b^2 = a$. Hence, $g_{11} = \pm b g_{12}$. Thus we get:
$$\left\{\left(
\bao{cc}
g_{11} & g_{12}\\
a g_{12} & g_{11}\\
\ea\right) \in \trm{Gl}_2(\currfield) : g_{11} \neq \pm b g_{12}\right\}$$
as a subgroup of the differential Galois group. To conclude, the Galois group is
$$\trm{DGal}(\currfield[x_1,x_2,1/\det X]/\currfield) = \left<g\right> \supset \left<\partial\right>.$$
Now, we can express $\trm{DGal}(\currfield[x_1,x_2,1/\det X]/\currfield)$ as an open algebraic set $U$ in $\aff{2}_{\currfield}$:
$$\bao{rcl}
\trm{DGal}(\currfield[x_1,x_2,1/\det X]/\currfield) &\simeq_{\trm{claim}}& U\\
 &:=& \{(g_{11},g_{12}) \in \aff{2}_{\currfield} : g_{11}^2 - a g_{12}^2 \neq 0\}\\
&=& \aff{2}_{\currfield} \bsl \left\{(g_{11},g_{12}) \in \aff{2}_{\currfield} : g_{11}^2 - a g_{12}^2 = 0\right\}\\
&&\\
&=& \aff{2}_{\currfield}\bsl Z(g_{11}^2 - a g_{12}^2)\\
\ea$$
As is shown in \cite{CohCuySte}, the differential Galois group is a subgroup of $\trm{Sl}(S(L))$. Therefore we may substitute:
$$U' = \{(g_1',g_2') \in \aff{2}_{\currfield} : g_1'^2 - a g_2'^2 = 1\}.$$
The multiplication is simply:
$$m_U := \left[\left((g_{11},g_{12}) , (g'_{11},g'_{12})\right) \longmapsto (g_{11} g'_{11} + a g_{12} g'_{12}, g_{11} g'_{12} + g'_{11} g_{12})\right].$$
Checking:
$$\bao{rcl}
(g_{11} g'_{11} + a g_{12} g'_{12}, g_{11} g'_{12} + g'_{11} g_{12}) &\stackrel{?}{\in}& U:\\
&&\\
(g_{11} g'_{11} + a g_{12} g'_{12})^2 - a (g_{11} g'_{12} + g'_{11} g_{12})^2 &=& g_{11}^2 {g'}_{11}^2 + a^2 g_{12}^2 {g'}_{12}^2 + 2 a g_{11} g_{12} g'_{11} g'_{12}\\
&&- a\left(g_{11}^2 {g'}_{12}^2 + g_{12}^2 {g'}_{11}^2 + 2 g_{11} g_{12} {g'}_{11} {g'}_{12}\right)\\
&&\\
&=& (g_{11}^2 - a g_{12}) {g'}_{11} - (g_{11}^2 - a g_{12}) a {g'}_{12}\\
&&\\
&=& \underbrace{(g_{11}^2 - a g_{12})}_{= 1} \underbrace{({g'}_{11}^2 - a {g'}_{12})}_{= 1} = 1,\\
\ea$$
implying $m_U\left((g_{11},g_{12}),({g'}_{11},g'_{12})\right) \in U$. Moreover, the map $\iota: U \ni (g_{11},g_{12}) \longmapsto g \in  \trm{DGal}(\currfield[x_1,x_2,1/\det X]/\currfield)$ is a morphism of groups (by definition), injective as only $(1,0) \in \ker \iota$ and surjective as $(g_{11},g_{12}) \in \iota^{-1}(g)$. This proves the claim the above defined open subset $U$ is isomorphic to $\trm{DGal}(\currfield[x_1,x_2,1/\det X]/\currfield)$.
%This group has three classes of subgroups:
%$$H_1 := \left\{\left(\bao{cc}
%g_{11} & 0\\
%0 & g_{11}\\
%\ea\right) : g_{11} \in \currfield^\times\right\},\ H_2 := \left\{\left(\bao{cc}
%0 &g_{12}\\
%a g_{12} & 0\\
%\ea\right) : g_{12} \in \currfield^\times\right\},$$
%$$\ H_3 := \left\{\left(
%\bao{cc}
%g_{11} & g_{12}\\
%a g_{12} & g_{11}\\
%\ea\right) : (g_{11},g_{12}) \in G_{<\infty} \times \currfield^\times \cap U\right\}$$
On the other hand, $G$ can be viewed as an closed subset of $\aff{3}_{\currfield}$:
$$I := \left<(X^2 - a Y^2) Z - 1\right> \subset \ov{\qz}[X,Y,Z]\ \RA\ Z(I) = \left\{\left(g_{11}, g_{12}, 1/(g_{11}^2 - a g_{12}^2)\right) : g_{11}, g_{12} \in \currfield\right\}$$
Its multiplication is given by
$$\bao{rrcl}
m_{Z(I)} :& Z(I) \times Z(I) & \longrightarrow & Z(I)\\
&&&\\
&((g_{11}, g_{12}, (g_{11}^2 - a g_{12}^2)^{-1}), & \longmapsto &(g_{11} {g'}_{11} + a g_{12} {g'}_{12}, g_{11} {g'}_{12} + g_{12} {g'}_{11},\\
& ({g'}_{11}, {g'}_{12}, ({g'}_{11}^2 - a {g'}_{12}^2)^{-1}))&&(g_{11}^2 - a g_{12}^2)^{-1}({g'}_{11}^2 - a {g'}_{12}^2)^{-1})\\
\ea$$
The proof of the above claim follows analogously to the open case. %The main difference is that $Z(I) = U \times \currfield^\times \cup \left\{0_{\aff{3}_{\currfield}}\right\}$, i.e. Zariski-dense and $Z(I)$ is a monoid (has $0$ as non-invertible and idempotent).
\paragraph{Characterization of field extension}
\bn
\item As we noted above, $\partial(\det X) = 0$. Since $R$ is a simple differential ring, we conclude $\det X \in k^\partial = k = \currfield$. Thus there is some $b = \det X \in \currfield^\times$. Therefore, $K = S^{-1}R \simeq \currfieldx[X]/\left<X^2 - a x_1^2 + b\right>$. In other words: the Picard-Vessiot field $K$ is an algebraic extension of (some appropriate) transcendental extension of $\currfield$. Rescaling our differential equation, we may assume that $b = 1$.
\item An immediate consequence of the above is the following: We consider $R' := \currfield[x_1,x_2] \simeq \trm{Sym}(S(L))$ wrt. the $\currfield$-basis $\{x_i : i = 1,2\}$. Now, take $I = \left<x_2^2 - a x_1^2 + 1\right> \subset \trm{Sym}(S(L))$, we get:
$$\trm{Sym}(S(L))/I \simeq R'/\left<x_2^2 - a x_1^2 + 1\right>.$$
Since the $\currfield$-derivation does not reduce the degree of any polynomial of positive degree we can use the degree induced filtration on $\trm{Sym}$:
$$\filteredA^{\leq l} = \left\{x \in \trm{Sym}(S(L)) : \deg x \leq l\right\} \ \RA \partial_{\trm{Sym}}\mid_{\filteredA^{\leq l}} (\filteredA^{\leq l}) \subset \filteredA^{\leq l},\ \filteredA^{\leq 0} := \currfield,\ \filteredA^{\leq -n} := \{0\}, n, l \in \nz,$$
i.e. $\filteredA^{\leq l}$ is $\partial_{\trm{Sym}}$ invariant $\currfield$-subspace. Here we set $\deg(0) := -\infty$ and $\deg$ is the sum degree. On the other hand:
$$\filteredA^{\leq 2} = \bigoplus_{|\alpha| \leq 2} \currfield.x_1^{\alpha_1} x_2^{\alpha_2}\ \RA\ \filteredA^{\leq2} / \left(\filteredA^{\leq2} \cap I\right) = \currfield \oplus \currfield.x_1 \oplus \currfield.x_2 \oplus \currfield.x_1^2 \oplus \currfield.x_1 x_2,$$
in particular $\lambda_{0,2} x_2^2 = \lambda_{0,2} - a \lambda_{0,2} x_1^2 \in \currfield \oplus \currfield.x_1^2$ for any $\lambda_{0,2} \in \currfield$. In general, for $\filteredA^{\leq n}$ we may substitute all terms $x_2^{2 l}$ with $(a x_1^2 - 1)^l$ and all $x_2^{2l + 1}$ with $(a x_1^2 - 1)^l x_2$, for all $l \in \{d \in \nz : 2 \mid d\}$. Therefore the quotient spaces are
$$A_n := \filteredA^{\leq n}/ \left(\filteredA^{\leq n} \cap I\right) /\left(\filteredA^{\leq n - 1} / \left(\filteredA^{\leq n - 1} \cap I\right) \right)\simeq \currfield.x_1^n \oplus \currfield.x_1^{n-1} x_2,$$
and therefore its associated graded $\currfield$-algebra is obviously:
$$A = \bigoplus_{n \geq 0} A_n \simeq \currfield[x_1] \oplus \currfield[x_1].x_2.$$
Summarizing we get
$$\trm{Sym}(S(L))/I = \filteredA/I = \bigcup_{n \geq 0} \filteredA^{\leq n}/  \filteredA^{\leq n} \cap I \simeq A \simeq \currfield[x_1] \oplus \currfield[x_1].x_2,$$
with multiplication
$$\mu = \left[x_1^{i_1} x_2^{j_1} \otimes x_1^{i_2} x_2^{j_2} \longmapsto x_1^{i_1 + i_2} (a x_1^2 - 1)^{(j_1 + j_2 - (j_1 + j_2 \mod 2))/2} x_2^{j_1 + j_2 \mod 2}\right],\ \forall i_l \in \nz, j_l = 0, 1.$$
We note that the subalgebra $\currfield[x_1]$ is not a differential algebra as 
$$\partial = \left[x_1^i x_2^j \longmapsto \begin{cases}
(1 + i) a x_1^{i+1} - i x_1^{i-1}& j = 1\\
i x_1^{i-1} x_2 & j = 0\\
\end{cases}\right].$$
Localizing wrt. $S := \currfield[x_1]\bsl\{0\}$, i.e. non-zero polynomials, and identifying with all corresponding symmetric tensors we have that
$$K^+ := S^{-1} \left(\currfield[x_1] \oplus \currfield[x_1].x_2\right) = S^{-1} \currfield[x_1] \oplus S^{-1} \currfield[x_1].x_2 = \currfieldx \oplus \currfieldx.x_2.$$
However, we have not shown yet whether $K^+$ is an integral domain. Now let us consider $p = Y^2 - a x_1^2 + 1 \in \currfieldx[Y]$. 
$$Y^2 - \underbrace{a x_1^2 + 1}_{= \alpha^2} = (Y - \alpha)(Y + \alpha).$$
If $\alpha \in \currfieldx$, then $\ov{Y \pm \alpha} = \ov{Y} \pm \alpha$ are zero-divisors in the quotient ring and with $\trm{gcd}(Y - \alpha, Y + \alpha) = 1$ we get
$$\currfieldx[Y]/\left<Y^2 - a x_1^2 + 1\right> \simeq \currfieldx^2$$
as a $\currfieldx$-vector space, with multiplication:
$$\mu(a \otimes \_) := [b \longmapsto a b] \equiv \left(\bao{cc}
a_0 & (a x_1^2 - 1) a_1\\
a_1 & a_0\\
\ea\right) \in M_2(\currfieldx),\ \forall a \equiv (a_0, a_1) \in \currfieldx^2$$
and with derivation:
$$\partial = \left[
\bao{ccc}
(1_{\currfield} x_1^i, 0) &\longmapsto& (0 , i x_1^{i-1})\\
(0, 1_{\currfield} x_1^i) &\longmapsto& ((1 + i) a x_1^i - i x_1^{i-1},0)\\ 
\ea
\right].$$ Hence, the Picard-Vessiot field $K$ would be simply $\currfieldx$. This implies:
$$\exists \alpha \in \currfield.x_1 \bsl \{0\}:\ \partial_{\currfieldx}(\alpha) \in \currfield.x_1 \LRA \exists v \in \ker (\alpha id_{\currfield.x_1} - \partial) \bsl\{0\},$$
i.e. an eigenvector of $\partial_K$. On the other hand, is $[\currfield(x_1,\alpha) : \currfieldx] := \deg \trm{Min}(\alpha,\currfieldx) = 2$:
$$\RA\ (K, \partial_K) \simeq \left(\currfield\left(x_1,\sqrt{a x_1^2 - 1}\right), \partial_{K_1}\right)$$% \simeq \left(\currfield\left(x_2,\sqrt{(1 + x_2^2)/(a)}\right), \partial_{K_2}\right),$$
is a field, with $\currfieldx/\currfield$ transcendental and with $\currfield$-derivation:%s:
$$\bao{rrcl}
\partial_{K}%_1}
 : &\currfield\left(x_1,\sqrt{a x_1^2 - 1}\right)&\longrightarrow&\currfield\left(x_1,\sqrt{a x_1^2 - 1}\right)\\
&&&\\
&x_1 &\longmapsto& \sqrt{a x_1^2 - 1}\\
&\sqrt{a x_1^2 - 1} &\longmapsto& a x_1\\
&&&\\
%\partial_{K_2} : &\currfield\left(x_2,\sqrt{(1 + x_2^2)/a}\right)&\longrightarrow&\currfield\left(x_2,\sqrt{(1 + x_2^2)/a}\right)\\
%&&&\\
%&x_2 &\longmapsto& \sqrt{\frac{1 + x_2^2}{a}}\\
%&\sqrt{\frac{1 + x_2^2}{a}} &\longmapsto& a x_2\\
\ea$$
However, we claim that since $x_1$ is transcendental over $\currfield$ and the polynomial $a x_1^2 +1$ is no square in $\currfield[x_1]$, the polynomial $X^2 - a x_1^2 + 1$ is irreducible over $\currfieldx$. First, let us assume there is a root $\alpha = \frac{\beta}{\gamma} \in \currfieldx$, where $\trm{gcd}(\beta,\gamma) = 1$ and $\gamma \neq 0$. Then we compute:
$$\alpha^2 = a x_1^2 - 1 = \frac{\beta^2}{\gamma^2} \LRA \beta^2 = \gamma^2 (a x_1^2 - 1),$$
i.e. $\beta \in \currfield[x_1]$ is a root of the polynomial $X^2 - \gamma^2(a x_1^2 - 1)$. Having $\beta = \sum_{j=0}^{n_1} \beta_j x_1^i, \gamma = \sum_{i=0}^{n_2} \gamma_i x_1^i$, with $\gamma_i, \beta_j \in \currfield$, we get that $n_1 = n_2 + 1$ due to the factor $a x_1^2$. On the other hand, both decompose to linear factors due to algebraic closedness of $\currfield$, e.g.:
$$\beta = \wt{\beta}_0 \prod_{i=1}^{m_1} (x_1 - \wt{\beta}_i)^{s_i},\ \gamma = \wt{\gamma}_0 \prod_{j=1}^{m_2} (x_1 - \wt{\gamma}_j)^{t_j},\ \trm{and}\ \sum_{i=0}^{m_1} s_1 = \sum_{j=0}^{m_2} t_j + 1.$$
Clearly, all but at most two factors on both sides of our polynomial equation cancel. So we get $n_2 \leq 0$ and $n_1 \leq 1$. In particular, $\alpha$ itself is already in $\currfield[x_1]$ for $n_2 = 0$ (i.e. $\gamma \neq 0$). This is not possible, as
$$\wt{\beta}_0^2 (x_1 - \wt{\beta}_1)^2 = \wt{\beta}_0^2(x_1^2 - 2 \wt{\beta}_1 x_1 + \wt{\beta}_1^2) = \wt{\gamma}_0^2 (a x_1^2 - 1) \LRA 2 \wt{\beta}_1 = 0 \wedge \wt{\beta}_0^2 \wt{\beta}_1^2 = -\wt{\gamma}_0^2$$
leads to the contradiction $\gamma = 0$. Hence, $\alpha$ is algebraic and $[\currfieldx(\alpha) : \currfieldx] = 2$.\\
\indent Nevertheless, one aspect we have not considered yet which we will discuss now.
%However, since $\currfield$ is algebraically closed, we get the abovementioned decomposition which will get discussed right now.
\item $\currfield(x_1,x_2) = K$ is contained in a $\currfield$-algebra generated by one distinct transcendental element $y$ being a unit in $\currfield[x_1,x_2]$: The ring $R = \currfield[x_1,x_2,\det X^{-1}]$ contains four non-trivial units (i.e. elements in $R^\times \bsl \currfield^\times$). Namely:
$$\pm \sqrt{a} x_1 \pm x_2,$$
to specify: four non-trivial divisors of $\det X$ in $\currfield[x_1,x_2]$. Computing their derivative wrt. $\partial$, we get:
$$\bao{rclcl}
\partial(\sqrt{a} x_1 \pm x_2) &=& \sqrt{a} x_2 \pm a x_1 &=& \pm \sqrt{a} (\sqrt{a} x_1 \pm x_2)\\
&&&&\\
\partial(-\sqrt{a} x_1 \pm x_2) &=& -\sqrt{a} x_2 \pm a x_1 &=& \mp\sqrt{a} (-\sqrt{a} x_1 \pm x_2).\\
\ea$$
Each of the four elements is an eigenvector of $\partial$ in $R$ with eigenvalue $\pm \sqrt{a}$ - or equivalently, solves the following differential equations:
$$L_{\pm} := \partial(x) \mp \sqrt{a} x = 0,$$
where the subscript sign defines the sign of the eigenvalue. Defining $y_1 \in S(L_+), y_{-1} \in S(L_-)$ we get that $K = \currfield(x_1,x_2)$ contains two subfields depicted in the following diagram:
$$\xymatrix{
& \currfield(y_1) \ar@{^{(}->}[rd]&\\
\currfield\ar@{^{(}->}[rd]\ar@{^{(}->}[ru]&&\currfield(x_1,x_2) \simeq \currfield(y_1)\\
& \currfield(y_{-1}) \ar@{^{(}->}[ru]&\\
}$$
By direct computation, we see that:
$$L = \partial^2 - a \cdot id = (\partial - \sqrt{a}) (\partial + \sqrt{a}).$$
And our differential module $S(L)$ decomposes:
$$S(L)^* \simeq \currfield[\partial]/\currfield[\partial].L \simeq \currfield[\partial]/\currfield[\partial].L_+ \oplus \currfield[\partial]/\currfield[\partial].L_- \simeq S(L_+)^* \oplus S(L_-)^*.$$
\en
\paragraph{The subfields}
Firstly, we like to compute the Picard-Vessiot ring for $M_{i} = \currfield(y_i)$:
$$R_i = \currfield[y_{i}, y_i^{-1}],\ \trm{for}\ i = \pm1$$
%As the image of the ideal $\left<y_i\right> \subset \currfield[y_i]$ is the whole ring under inclusion or equivalently $1_R \in R_i.y_i$, we have indeed that $R_i$ is simple differential. 
%Let $I \subset R_i$ be a proper ideal and we assume differential closedness - i.e. $\partial(I) \subset I$.
%As a noetherian $R$-submodule of a noetherian module $R$ (generated by $y_{\pm 1}$ over $\currfield$), $I$ is finitely generated. Hence, let $S:= \{s\} \subset I$ be one generating set. By differential closedness, we get for any $s \in S$:
%$$\partial(s) = \partial\left(\sum_{i=-m}^n s_i y_1^i\right) = \sum_{i=-m}^n s_i \partial(y_1^i) = \sum_{i=-m}^n i \sqrt{a} s_i y_1^i \in I$$
%$$\LRA \partial(s) - s = \sum_{i=-m}^n (i \sqrt{a} - 1) s_i y_1^i \in I$$
%But both, $s, \partial(s) - s$ are of degree $n$, or $m$ wrt. $y_{\pm 1}$ and $y_1^m (\partial(s) - s) \in \currfield[y_1]$.
% There is an ideal $I' \subset \currfield[y_1]$, such that $S_{y_1}^{-1}(I') \supset I$. By definition of $I$, we get
%$$y_1^m t \in I' \RA \partial(y_1^m t) = \underbrace{m \sqrt{a} y_1^m t}_{\in I'} + \underbrace{y_1^m \partial(t)}_{\in \partial(I')},$$
%but identifying $I' := I \cap \frac{\currfield[y_1]}{1}$ we get $\partial(I') \subset I'$. Being a PID, all ideals $I' \subset \currfield[y_1]$ are of the form $\left<s\right>$. On the other hand, $\partial$ operates on all weight spaces $\currfield.y_1^i$, $i \geq 1$, invariantly:
%$$\bao{rrcl}
%\partial_i := \partial\mid_{\currfield.y_1^i} : &\currfield.y_1^i &\longrightarrow& \currfield.y_1^i\\
%&&&\\
%&y_1^i &\longmapsto&i \sqrt{a} y_1^i\\
%\ea$$
%Thus we have:
%$$\partial s = \sum_{i=0}^n s_i \partial(y_1^i) = \sum_{i=0}^n i \sqrt{a} s_i y_1^i \in \left<s\right> \LRA \partial(s) \equiv 0 \mod s$$
%$$\LRA \sum_{i=0}^{n-1} (i - n) \sqrt{a} s_i y_1^i = 0 \LRA s_i = 0 \vee n - i = 0\ \forall 0 \leq i \leq n - 1,$$
%Hence, each derivative of the generators $s$ agree in degree but also reduce to zero modulo $\left<s\right>$ contradicting our claim $\partial(s) \in \left<s\right>$. Thus, all $D$-stable ideals in $R$ are indeed trivial.
Firstly, we get:
$$\partial(y_i^{-1}) = -\frac{\partial(y_i)}{y_i^2} = -i \sqrt{a} y_i^{-1}, i =\pm1$$
implying the multiplicative inverse solves the opposite differential equation. Hence, $R_i$ already contains all solutions of $L_\pm$ by simply putting $y_{-1} := y_1^{-1}$.\\
\indent Let $R = R_1$. Furthermore, we want to show that $R$ is indeed a simple differential ring over $\currfield$. Clearly, $\partial$ operates invariantly on $\currfield.y_{\pm 1}^i$ for all $i \neq 0$. In addition, $R$ is isomorphic to $S_{X}^{-1}\currfield[X]$, the localization of $\currfield[X]$ where
$$S_X = \currfield[X] \bsl \bigcup_{\substack{\mathfrak{p} \in \trm{Spec}(\currfield[X])\\X \notin \mathfrak{p}}} \mathfrak{p} = \{f \in \currfield[X] : f \notin \idealp\ \forall \idealp \in \trm{Spec} \currfield[X] \bsl \{\left<X\right>\}\}.$$ Furthermore, every prime ideal in $\currfield[X]$ is generated by $X - \alpha$, for some $\alpha \in \currfield$ as $\currfield$ is algebraically closed. Hence, on the one hand we get:
$$S_X = \{f \in \currfield[X]\bsl \currfield : f \notin \left<X - \alpha\right> \forall \alpha \in \currfield^\times\} = \{X^i : i \geq 0\}$$
and on the other hand: if $I' \subset R$ is an ideal, there is an ideal  $I \subset \currfield[X]$ such that $I' = S_{X}^{-1} I$. This implies every ideal is principal in $R$. In $\currfield[X]$ each element $\mathfrak{p} \in \trm{Spec}(\currfield[X])$ gets mapped to $\left<X\right>$ via $\partial$, as every element $p (X - \alpha)$ in a prime ideal $\left<X - \alpha\right>$ has an image in $\left<X\right>$:
$$p (X - \alpha) \longmapsto \partial(p) (X - \alpha) + \sqrt{a} p X,\ \forall \alpha \in \currfield,\ p \in \currfield[X]$$
where the constant coefficient $\left(\partial(p)\right)_0 = 0 \LRA p_0 \in \ker \partial \LRA \partial(p) \in \left<X\right>$. This shows the differential $\currfield$-algebra $\left(\currfield[X], \partial = \sqrt{a} X \cdot \frac{d}{d X}\right)$ has only one differential prime ideal, $\left<X\right>$ (- in our case even maximal). However, the localization cancels this as its generator $X$ is a unit in $R$ (using variable notation $X, y_1$ interchangeably). Therefore, our rings $R_1, R_{-1}$ are indeed simple differential rings - or PV. In words of $D$-module algebra, $R_1, R_{-1}$ are simple $D$-rings, for $D = \currfield[\partial]$.\\
\indent To conclude this paragraph, we want to show that $\currfield(x_1,x_2) \simeq \currfield(y_1)$ as differential field extensions over $\currfield$. To achieve this we need to show that the maps:
$$\bao{rrcl}
\Phi : & \currfield(y_1) &\longrightarrow& \currfield(x_1,x_2)\\
& y_1 &\longmapsto & \sqrt{a} x_1 + x_2\\
& y_1^{-1}&\longmapsto & \sqrt{a} x_1 - x_2\\
&&&\\
\Psi : &\currfield(x_1,x_2) &\longrightarrow& \currfield(y_1)\\
& x_1 &\longmapsto& \frac{1}{2 \sqrt{a}} (y_1 + y_1^{-1})\\
& x_2 &\longmapsto& \frac{1}{2} (y_1 - y_1^{-1})\\
\ea$$\label{PVisomorph}
are bijective and inverse to one another as differential $\currfield$-algebra homomorphisms (or $D$-module algebra homomorphisms). Clearly, the two $\currfield$-vector spaces $V_1 := \currfield.y_1 \oplus \currfield.y_1^{-1}$ and $V_2 := \currfield.x_1 \oplus \currfield.x_2$ are isomorphic by simple basis change induced by the restrictions $\mid_{V_i}$ of the above $\currfield$-vector space homomorphisms. Furthermore, $\Psi \mid_{V_2} \Phi \mid_{V_1} = id_{V_1}$ and $\Phi \mid_{V_1} \Psi \mid_{V_2} = id_{V_2}$ and both are $D$-modules. Hence, the basis change respects this property as shown above. This property is kept for the recursively defined family of $D$-modules:
$$V_{1,i} := V_{1,i-1} \oplus \currfield.y_1^i \oplus \currfield.y_1^{-i},\ V_{2,i} := V_{2,i-1} \oplus \currfield.x_1^i \oplus \currfield.x_1^{i-1} x_2, \forall i \geq 1$$
and $V_{j,0} = \currfield$ for $j = 1, 2$. We get:
$$\bao{rclclcl}
\bigcup_{i \geq 0} V_{1, i} &\simeq& \currfield[y_1,X]/\left<y_1 X - 1\right> &\simeq& \currfield[y_1,y_1^{-1}] &=:& R_1\\
&&&&\\
\bigcup_{i \geq 0} V_{2,i} &\simeq& \currfield[x_1,x_2]/\left<x_2^2 - a x_1^2 + 1\right> &\simeq& \currfield[x_1,\sqrt{a x_1^2 - 1}] &=:& R_2,\\
\ea$$
i.e. a $D$-stable filtration for each subalgebras $R_1, R_2$. Having already shown their simplicity we only need surjectivity as $\Psi\mid_{V_{2,0}} = id_{V_{2,0}}, \Phi\mid_{V_{1,0}} = id_{V_{1,0}}$ already implies injectivity. Let $f$ and $g$ equal $\sum_i \left(f_{1,i} y_1^i + f_{-1,i} y_1^{-i}\right) \in R_1$ and $\sum_i \left(g_{1,i} x_1^i + g_{2,i-1} x_1^i x_2\right) \in R_2$, respectively. Obviously, $\sum \left(f_{1,i} (\sqrt{a} x_1 + x_2)^i + f_{-1,i} (\sqrt{a} x_1 - x_2)^i\right)$ and
$\sum (y_1 + y_1^{-1})^i \left(\frac{g_{1,i}}{2 \sqrt{a}} + \frac{g_{2,i}}{2} (y_1 - y_1^{-1})\right)$ are elements in their respective preimages. This extends naturally to their localizations.
%Hence, any ideal stable under $\partial$ is, on the one hand, generated by a single element, on the other hand, $\deg f = \deg \partial(f)$, where $\deg = \left[f = \sum_{i=-m}^n f_i X^i \longmapsto \max(m,n)\right]$. In particular, $\left<f, \partial f\right>$ is also principal and a subideal of $I$.
\paragraph{Galois group of the subfields}
As above $\trm{DGal}(\currfield[y_1,y_{-1}]/\currfield) = \{a \in \trm{Gl}_1(\currfield) : \partial a = a \partial\}$. Obviously, the unit group of $\currfield$ is our differential Galois group.
% However, both generators are eigenvectors. Hence, the diagonal matrix
%$$A_{L_++L_-} = \left(\bao{cc}
%\sqrt{a} & 0\\
%0 & -\sqrt{a}\\
%\ea\right)$$
%represents $\partial$ on $S(L_+) \oplus S(L_-)$ with canonical basis vectors identified with $y_1, y_{-1}$, respectively. Clearly, all matrices with diagonal entries are the only ones fulfilling our definition. We get
%$$\trm{DGal}(\currfield[y_1,y_{-1}]/\currfield) \simeq \currfield^\times \times \currfield^\times.$$
\paragraph{Conclusion}
Given our above example, the PV-ring and field are
$$R = \currfield[y_1,y_{-1} := y_1^{-1}],\ K = S^{-1}R = \currfield(y_1),$$
respectively, with differential Galois group
$$\trm{DGal}(\currfield[y_1,y_{-1}]/\currfield) \simeq \currfield^\times.$$
The differential ring $\currfield[x_1,x_2] = \trm{Sym}(L)/\left<x_2^2 - a x_1^2 + 1\right> \simeq \trm{Sym}(\currfield^2)/\left<e_2^2 - a e_1^2 + 1\right>$ is isomorphic to
$$\currfield[y_1] \oplus \currfield[y_{-1}] \simeq \currfield[y_1]^2.$$
\bmk %The results are valid in case $a, \sqrt{a} \in \currfield[z]\bsl\currfield$, if our matrix equation gets slightly modified:
%$$A_L = \left(\bao{cc}
%0 & 1\\
%a \pm \partial(\sqrt{a}) & 0\\
%\ea\right) \in \trm{Gl}_2(\currfield(z))\LRA L(y) = \partial^2(y) - \left(a \mp \partial(\sqrt{a})\right)y = 0,\ \forall y \in S(L).$$
%The different signs arise from two isomorphic solution spaces $S_1 = S(L_+) \oplus S(L_-)$, $S_2 = S(L_-) \oplus S(L_+)$, or equivalently, by the order of operator evaluation:
%$$S_1 \simeq D/D.(\partial - \sqrt{a})(\partial + \sqrt{a}),\ S_2 \simeq D/D.(\partial + \sqrt{a})(\partial - \sqrt{a}),\ \trm{where}\ D := \currfield(z)[\partial], \partial(z) = 1.$$
A more exhaustive approach to $\partial^2 - a \cdot id$, where $a$ is not a constant, is given in \cite{vdPS01} and \cite{CohCuySte}.\\
\indent Unfortunately, our given examples are all linear homogeneous ODEs. The theory, we are going to present in the next chapter, is more general. It includes theory for general ODEs (linear, non-linear) in characteristic zero, (multivariate) iterative derivations in positive characteristic and difference equations in arbitrary characteristic.\\
\indent Nevertheless, as the general theory does not require the ring/field of constants to be algebraically closed we may examine a broader setting.
%though each factor is not differentially closed (i.e. $(1,0) \stackrel{\partial}{\longmapsto} (0,1) \stackrel{\partial}{\longmapsto} (a,0)$).
%We want to see if the following $\currfield$-linear maps are differentially invariant:
%$$\sigma_1 = [\pm y_1 \longmapsto \mp y_1,\ \pm y_1^{-1} \longmapsto \mp y_1^{-1}] \in \trm{Gl}(R_1)$$
%$$\sigma_2 = [y_1 \longmapsto y_1^{-1}] \in \trm{Gl}(R_1)$$
%Since $\pm y_1$ and $\pm y_1^{-1}$ have the same eigenvalue $\sqrt{a}$ and $-\sqrt{a}$ respectively, we have that $\sigma_1$ commutes with $\partial$. Hence, $\sigma_1 \in \trm{DGal}(M_1/\currfield)$. On the other hand, $\sigma_2(\partial(y_1)) = \sigma_2(\sqrt{a} y_1) = \sqrt{a} y_1^{-1} \neq \partial(\sigma_2(y_1)) = \partial(y_1^{-1}) = -\sqrt{a} y_1^{-1}$. %Now, we may express $v = \sum_{i \in \{1,2\}} v_i x_i \in \trm{Sol}(\partial^2 - a \cdot id)$ as linear combinations of $v = \sum_{i \in \{\pm 1\}} v'_i y_i$:
%$$\bao{rrcl}
%M_{B_x}^{B_y}(id):& v &=& v'_1 y_1 + v'_{-1} y_{-1}\\
%& &=& v'_1 (\sqrt{a} x_1 + x_2) + v'_{-1} (\sqrt{a} x_1 - x_2)\\
%&&&\\
%&&=& \sqrt{a} (v'_1 + v'_{-1}) x_1 + (v'_1 - v'_{-1}) x_2\\
%&&&\\
%M_{B_y}^{B_x}(id)&v &=& v_1 x_1 + v_{2} x_{2}\\
%&&=& \frac{v_1}{2 \sqrt{a}}(y_1 + y_{-1}) + \frac{v_{2}}{2} (y_1 - y_{-1})\\
%&&&\\
%&&=& \frac{1}{2 \sqrt{a}} \left[(v_1 + \sqrt{a} v_2) y_1 + (v_1 - \sqrt{a} v_2) y_{-1}\right]\\
%\ea$$
%Hence, we have the following basis transformation matrices:
%$$M_{B_x}^{B_y} = \left(\bao{cc}
%\sqrt{a} & \sqrt{a}\\
%1 & -1\\
%\ea\right),\ \ M_{B_y}^{B_x} = \frac{1}{2 \sqrt{a}}\left(\bao{cc}
%1 & \sqrt{a}\\
%1 & -\sqrt{a}\\
%\ea\right)$$
%The conjugate $M_{B_y}^{B_x} G M_{B_x}^{B_y}$ of $G:= \trm{DGal}(\currfield(x_1,x_2)/\currfield)$ defines a group action on $S(L_+) \oplus S(L_-)$ or equivalently, makes $S(L_+) \oplus S(L_-)$ a $G$-module:
%$$\bao{rrcl}
%\alpha : &G \times \left(S(L_+) \oplus S(L_-)\right)& \longrightarrow &S(L_+) \oplus S(L_-)\\
%&&&\\
%&\left(g := \left(\bao{cc}
%g_{11} & g_{12}\\
%a g_{12} & g_{11}\\
%\ea\right), v'_1 y_1 + v'_{-1} y_{-1}\right) &\longmapsto&M_{B_y}^{B_x} g %\left(\bao{cc}
%g_{11} & g_{12}\\
%a g_{12} & g_{11}\\
%\ea\right)
%M_{B_x}^{B_y}\left(\bao{c}
%v'_1\\
%v'_{-1}\\
%\ea\right)\\
%\ea$${\footnotesize
%$$M_{B_y}^{B_x}g %\left(\bao{cc}
%g_{11} & g_{12}\\
%a g_{12} & g_{11}\\
%\ea\right)
%M_{B_x}^{B_y}\left(\bao{c}
%v'_1\\
%v'_{-1}\\
%\ea\right) = \frac{1}{2 \sqrt{a}}\left(\bao{cc}
%2 \sqrt{a} g_{11} + (1 + a^2) g_{12}& -(1 - a^2) g_{12}\\
%(1 - a^2) g_{12} & 2 \sqrt{a} g_{11} - (1 + a^2) g_{12}\\
%\ea\right) \left(\bao{c}
%v'_1\\
%v'_{-1}\\
%\ea\right)\\
%$$}
%The fixed field $\currfield(y_1)^G$ is 
%Multiplying $T$ with the generator of $\trm{DGal}(\currfield(x_1,x_2)/\currfield)$ we get:
%$$T g = \left(\bao{cc}
%\sqrt{a} & 1\\
%\sqrt{a} & -1\\
%\ea\right) \left(\bao{cc}
%g_{11} & g_{12}\\
%a g_{12} & g_{11}\\
%\ea\right) = \left(\bao{cc}
%\sqrt{a} g_{11} + a g_{12} & \sqrt{a} g_{12} + g_{11}\\
%\sqrt{a} g_{11} - a g_{12} & \sqrt{a} g_{12} - g_{11}\\
%\ea\right)$$
%Analogously to the classical Galois theory, there exists subgroups $H$ in $\trm{DGal}(\currfield(x_1,x_2)/\currfield)$ such that $\currfield(x_1,x_2)^H := \{x \in K : \sigma(x) = x \forall \sigma \in H\}$ is a Picard-Vessiot field over $\currfield$. Conversely, the $\trm{DGal}(\currfield(x_1,x_2)/M)$ is a (closed) subgroup of $\trm{DGal}(\currfield(x_1,x_2)/\currfield)$ for any intermediate Picard-Vessiot field $\currfield \subset M \subset \currfield(x_1,x_2)$. %Since we already have two Picard-Vessiot subfields $M_i = \currfield(y_i)$ of $K$, we only need to compute the Galois groups of both differential fields.
\section{General theory}
The idea, pioneered by Picard and Vessiot, was for a given differential equation $L \in k[\partial]$ with differential extension $K/k$ and PV ring $R$ we get a functor
$$\trm{DGal}(K/k) : \trm{CAlg}_{k^\partial} \longrightarrow \trm{Grp}$$
which assigns to each commutative $k^\partial$ algebra $A$ the group of elements $\varphi \in \trm{Aut}_{k^\partial}(R \otimes A)$ such that
$$\xymatrix{
R \otimes_{k^\partial} A \ar[r]^{\partial_R \otimes id_A} \ar[d]_\varphi &R\otimes_{k^\partial} A\ar[d]^\varphi\\
R \otimes_{k^\partial} A \ar[r]_{\partial_R \otimes id_A} & R \otimes_{k^\partial} A\\
}$$
commutes. As all $A$ points (i.e. functor evaluated at $A \in \trm{CAlg}_{k^ \partial}$) are subgroups of $\trm{Gl}_l(k^\partial)$ for some $l \leq \trm{deg} L$, we see that it is an affine group scheme over $k^\partial$. Umemura extended the definition the functor via:
$$\trm{Inf-Gal} := \trm{Ume} : \trm{CAlg_k} \longrightarrow \trm{Grp}$$
\index{Symbol}{$\trm{Inf-Gal}$}
\index{Symbol}{$\trm{Ume}$}
which assigns to each commutative differential algebra $A$ over a differential field $k$ a group object $G$ or more generally: a formal group law/formal group scheme. In \cite{Heid13}, this functor is introduced as the \textit{Umemura-functor}. However, Heiderich uses a bialgebraic approach - that is - let $D$ be a bialgebra and $A$ a $D$-module algebra. The question tackled by the bialgebraic approach is:
\bd
\item[How] to extend the action of a derivation on tensor products of $D$-modules $A^{\otimes n}$ for all $n \geq 0$?\\
\item[Answer] use its coalgebraic structure $\Delta, \eps$ (or equivalently the $D$-module algebraic structure).
\ed
More importantly, the question for formalization and generalizations of the bialgebraic approach of the PV theory builds an extended framework to deal with the above question in terms of iterative derivation and difference equations in positive characteristic or arbitrary characteristic, respectively. However, this is well beyond the scope of this paper.\\
\indent To explain this approach, we have to introduce some additional constructs. Although, most of the definition in the linear case are still used (as for instance Picard-Vessiot rings/fields). However, we usually do not assume $K/k$ to be field extension, rather some ring (i.e. an associative $k$-algebra).
\subsection{Basics}
As in the last section, $(k,\partial)$ is differential field and $K$ is a differential extension such that $k(x) := k(x_1,\ldots,x_n)$ is a differential field/ring and $[K:k(x)] = [K:k(x)]_{\trm{sep}} < \infty$, as field/ring extension in the classical sense. Note, that $(K, \{\partial, \partial_i = [x_j \longmapsto \delta_{i,j}] : 1 \leq i \leq n\})$ is also a differential algebra. In addition, let $\trm{char} k = 0$.
\subsubsection{Universal Taylor homomorphism and iterative derivations}
Let $K/k$ be as above and $K[[t]]$ the ring of formal power series over $K$.
\begin{defi}[Universal Taylor]
The map
$$\iota : K \longrightarrow K[[t]],\ a \longmapsto \sum_{n \geq 0} \frac{\partial(a)}{n!} t^n$$
is called the universal Taylor-morphism.
\index{Symbol}{$\iota$}
\index{Index}{homomorphism!universal Taylor}
\end{defi}
Before exploring why this is called universal we need some additional definitions.
\begin{defi}[$n$-variate iterative derivations]
Let $K/k$ be as above, %$K[[w]]$ be the ring of formal power series in multiple variables $w = (w_i)_{i=1}^n$, 
with $k$-derivations $\partial_{x_i}$, where $x = (x_i)_{i=1}^n$. A family of $C_k$-module homomorphisms $(\theta^{(\alpha)})_{\alpha \in \nz_0^n} \subset \trm{Hom}_{C_k}(K,K)$ defines an $n$-variate iterative derivation, if
\bn
\item $\theta^{(0)} = id_K$,
\item $\theta^{(\alpha)}(a + b) = \theta^{(\alpha)}(a) + \theta^{(\alpha)}(b)$,
\item $\theta^{(\alpha)} (a b) = \sum_{\alpha_1 + \alpha_2 = \alpha} \theta^{(\alpha_1)} (a) \theta^{(\alpha_2)} (b)$ and
\item $\theta^{(\alpha_1)} \circ \theta^{(\alpha_2)}(a) = \left(\bao{c}
\alpha_1 + \alpha_2\\
\alpha_1\\
\ea\right) \theta^{(\alpha_1 + \alpha_2)} (a)$,
\en
for all $a, b \in K$ and $\alpha, \alpha_i \in \nz_0^n$. Moreover, for an iterative derivation $\theta$ we call the subring
$$k^\theta := \left\{x \in k : \theta^{(\alpha)}(x) = 0,\ \forall \alpha \in \nz_0^n\bsl \{0\}\right\} = \bigcap_{\alpha \in \nz_0^n \bsl \{0\}} \ker \theta^{(\alpha)}$$
its subring of iterative constants (or simply constants).
\index{Symbol}{$\theta$}
\index{Index}{derivation!iterative}
\end{defi}
\bsp
\bn
\item The family of $C_k$-homomorphisms $\{\iota^{\alpha}\}$, given in the definition of the universal Taylor-morphism, clearly defines a mono-variate iterative derivation. We will show this shortly.
\item Let $\trm{char} k = p \neq 0$, and $K = k(x)$, then
$$\theta^{(m)} = \left[x^n \longmapsto \left(\bao{c}
n\\
k\\
\ea\right) x^{n - m}\right],\ m \geq 0$$
is an example in positive characteristic. Indeed, its ring of iterative constants is $k$.
\en
\bmk The $n$-variate iterative derivations may be applied to positive characteristic. However, in case of the Taylor-morphism, this is only applicable to characteristic zero.\\
\indent We call a ring $(R, \theta)$ an iterative differential ring, in particular for $R = k[x_1,\ldots,x_n]$ it is $\trm{Der}_{\trm{ID^n}}$. The transcendence degree $n$ can be omitted to get a more general definition of $\trm{Der}_{\trm{ID}}$, the set of iterative derivations.
\begin{defi}
Let $K/k$ be as above (with $\trm{char} k = 0$) and let $\theta^{(\alpha)} := \left[a \longmapsto \frac{\partial_x^\alpha(a)}{\alpha!}\right]$ define the iterative derivation (wrt. $x$):
$$\theta_x := \sum_{\alpha \in \nz_0^n} \theta^{(\alpha)} w^\alpha : K \longmapsto K[[w]],\ a \longmapsto \sum_{\alpha \in \nz_0^n} \theta^{(\alpha)}(a) w^\alpha.$$
\end{defi}
Both maps play a prominent role in the definition of the so called Umemura functor. Returning to the universal Taylor-morphism, we have
\begin{lemm}
Let $K/k$ be as above, $\partial_t$ be the $K$-derivation $t \longmapsto 1$ on $K[[t]]$ and $\iota$ denote the Taylor-morphism. Then the following diagrams commute:
$$\bao{cc}
\xymatrix{
K \ar[rd]_{id_K}\ar[r]^\iota &K[[t]]\ar[d]^{\pi_t}\\
&K
}&
\xymatrix{
K \ar[r]^{\iota_u}\ar[d]_{\iota} &K[[u]]\ar[d]^{\iota[[u]]}\\
K[[t]] \ar[r]_{t\mapsto t+u}&K[[t]][[u]]\\
}
\ea,$$
where $\iota_u$ represents an equivalent Taylor morphism $K \longrightarrow K[[u]]$ (replacing $t$ with $u$ in $\iota : K \longrightarrow K[[t]]$), $\pi_t = [x \longmapsto x \mod t]$ and $\iota[[u]] : K[[u]] \longrightarrow K[[t]][[u]], \sum_{n \geq 0} a_n u^n \longmapsto \sum_{n \geq 0} \iota(a_n) u^n$.\\
In addition, the following diagram also commutes for all $i \geq 0$:
$$\xymatrix{
K \ar[r]^\iota\ar[d]_{\partial_K^i}&K[[t]]\ar[d]^{\partial_t^i}\\
K&K[[t]]\ar[l]^{\pi_t},\\
}$$
where $\partial_K$ is the extension of $\partial \in \trm{Der}_C(k)$ on $K$.
\end{lemm}
\bws The first diagram and the third are equivalent if $i = 0$. Also, the first diagram is an immediate consequence of the definition of $\iota$ as it is defined by iterative derivations $\iota^{(k)} : K \longrightarrow K$.
\bn
\item Pick some $a \in K$, then the upper part of the diagram yields:
$$\iota[[u]] \circ \iota_u(a) = \iota[[u]]\left(\sum_{n\geq 0} \frac{\partial^n(a)}{n!} u^n\right) = \sum_{n + m\geq0} \frac{\partial^{n+m}(a)}{n!m!} t^m u^n$$
Following the lower part:
$$f(a) = \sum_{n\geq0} \frac{\partial^n(a)}{n!}(t + u)^n = \sum_{n\geq0} \sum_{m \leq n}\left(\bao{c}
n\\
m\\
\ea\right) \frac{\partial^{n'}(a)}{n'!} t^{m'} u^{n'-m'} = \sum_{m'+n'\geq0} \frac{\partial^{n'+m'}(a)}{m'! n'!} t^{m'} u^{n'},$$
Setting $m = m'$ and $n' = n + m$ in both parts, we see by comparison of coefficients that the claim holds.
\item Pick $a \in K$, then:
$$\pi_t \circ \partial_t^i\circ\iota(a) = \pi_t \circ \partial_t^i\left(\sum_{n\geq0} \frac{\partial^n(a)}{n!} t^n\right) = \pi_t\left(\sum_{n \geq i} \frac{ \partial^n(a)}{(n - i)!} t^{n-i}\right) = \partial^i(a),$$
completing the prove.
\en
\bmk The second part of the prove, in essence, shows that $\iota$ is indeed an iterative derivation. Moreover, $K[[t]]$ is a $D = k[\partial]$-module algebra and $\iota$ is a homomorphism of $D$-module algebras.
\begin{defi}
The following algebras are differential sub-algebras of $K[[t]]$:
\bn
\item $\mathcal{K} := K\{\iota(K)\}_{\partial_x}$, i.e. is generated by $K$ and the image of $K$ under $\iota$ - closed under the $k$-derivations $\partial_x$.
\item $\kappa := K\{\iota(k)\}_{\partial_x}$, i.e. is generated by $K$ and the image of $k$ under $\iota$ - closed under the $k$-derivations $\partial_x$.
\en
\end{defi}
\bmk The partial differential subalgebras of $K[[t]]$ can be expressed as
\bn
\item $\kappa = \left<\iota(a), b : a \in k, b \in K\right>_{k-\trm{alg}}$ and
\item $\mathcal{K} = \left<\partial_x^{\alpha}(\iota(a)), b : a, b \in K, \alpha \in \nz_0^n\right>_{k-\trm{alg}}$.
\en
\subsubsection{The Umemura functor}
Let $K/k$ be as above and let $A$ be a commutative (ass.) $K$-algebra (i.e. a ring containing $K$). We will consider the following tensor product:
$$K[[t]] \otimes A[[w]] := K[[t]] \otimes_K A[[w]],$$
with the algebra structure induced by the composition of $\theta_x : K \longrightarrow K[[w]]$ and the image of $K[[w]]$ in $A[[w]]$ via the $K$-linear unit-homomorphisms:
$$\bao{rrclcrcl}                
\eta_{K[[t]]}': &K &\longrightarrow& K[[t]],&& a &\longmapsto& a\cdot t^0\\
&&&&&&&\\
\eta_A : &K &\longrightarrow& A,&& a &\longmapsto& a\cdot 1_A\\
\ea$$
and
$$\theta_x[[t]] : K[[t]] \longrightarrow K[[t]] \otimes K[[w]],\ \sum_{i\geq0} a_i t^i \longmapsto \sum_{\substack{i\geq0\\k \in \nz_0^n}} \frac{\partial_x^k(a_i)}{k!} t^i \otimes w^k,$$
where the tensor product is defined over $K$, making the following diagram commutative
$$\xymatrix{
K \ar[rr]^{\eta_{K[[t]]}} \ar[d]_{\eta_{K[[t]]} \otimes \eta_{A[[w]]}} && K[[t]]\ar[d]^{\theta_x[[w]]}\\
K[[t]] \otimes A[[w]] && K[[t]] \otimes K[[w]]\ar[ll]_{id \otimes \eta_{A}[[w]]}.\\
}$$
This defines a partial differential algebra structure on $K[[t]] \otimes A[[w]]$:
$$\bao{rrcl}                                    
\partial_t :& K[[t]]\otimes A[[w]] &\longrightarrow& K[[t]]\otimes A[[w]]\\
& \sum_{(i,k) \in \nz_0^{n+1}} a_{i,k} t^i\otimes w^k&\longmapsto& \sum_{(i+1,k) \in \nz_0^{n+1}} i a_{i,k} t^{i-1} \otimes w^k\\
&&&\\
\partial_{x_i} :& K[[t]]\otimes A[[w]] &\longrightarrow& K[[t]]\otimes A[[w]]\\
& \sum_{(i,k) \in \nz_0^{n+1}} a_{i,k} t^i\otimes w^k&\longmapsto& \sum_{(i,k) \in \nz_0^{n+1}} \partial_{x_i}(a_{i,k}) t^{i-1} \otimes w^k\\
&&&\\
\partial_{w_i} :& K[[t]]\otimes A[[w]] &\longrightarrow& K[[t]]\otimes A[[w]]\\
& \sum_{(i,k) \in \nz_0^{n+1}} a_{i,k} t^i\otimes w^k&\longmapsto& \sum_{(i,k+e_i) \in \nz_0^{n+1}} k_i a_{i,k} t^{i-1} \otimes w^{k-e_i},\\
\ea$$
where $e_i$ denotes the canonical base vector in $\zz^n$. We note that while $\partial_t, \partial_{w_i} \in \trm{Der}_K(K[[t]]\otimes A[[w]])$ the derivation $\partial_{x_i}$ is in $\trm{Der}_k(K[[t]] \otimes A[[w]])$. To specify,
$$\bao{rcl}
\partial_t &\in& \trm{Der}_{A[[w]]}(K[[t]] \otimes A[[w]]),\\
&&\\
 \partial_{w_i} &\in& \trm{Der}_{K[[t]] \otimes A[[w \bsl\{w_i\}]]}(K[[t]]\otimes A[[w]]),\\
&&\\
 \partial_{x_i} &\in& \trm{Der}_{k[[t]] \otimes k[[w]]} (K[[t]]\otimes A[[w]]),\\\ea$$
where $\partial_\zeta \in \trm{Der}_A(B)$ implies $B^{\partial_\zeta} \supseteq A$.
%\begin{defi}[Linear topological rings]
%A topological ring $R$ is called linear if there is a fundamental neighborhood basis of $0 \in R$.
%%Ein topologischer Ring $R$ hei\ss{}t linear, falls es eine fundamentale Umgebungsbasis von $0 \in R$ gibt.
%\end{defi}
%For further discussion see appendix. If $R$ is a linear topological ring with fundamental neighborhood basis $\beta(0)$ then every open neighborhood of zero contains at least one ideal, trivially the zero ideal as ideals are stable under intersection. Let $\{I_i \in \beta(0) : i \in \mathcal{I}\}$ be a system of ideals then the union is a subset of $R$ containing ideals of the form%  Ist nun $R$ ein linear topologischer Ring, mit fundamentaler Umgebungsbasis $\beta(0)$, dann enth\"alt jede offene Umgebung der Null mindestens ein Ideal $I \in \beta(0)$, trivialerweise mindestens $(0)$, da der Schnitt beliebiger Ideale wieder ein Ideal ergibt. Sei $\{I_i \in \beta(0) : i \in \mathcal{I}\}$ ein System von Idealen, dann ist deren Vereinigung eine Teilmenge von $R$, die alle Ideale der Form
%$$\bigcap_{i \in \mathcal{I}'} I_i, \forall \mathcal{I}' \subset \mathcal{I}\ \trm{and}\ |\mathcal{I}'| < \infty$$
%defining an open neighborhood of zero. On the other hand, the intersections are also neighborhoods of zero. Hence, all elements are clopen in $R$.%enth\"alt und definiert damit eine offene Umgebung der Null. Andererseits sind die Schnitte der Ideale auch Umgebungen der Null, d.h. damit sind alle Ideale \textit{clopen} in $R$.
%\begin{defi}[Complete topological rings]
%Let $R$ be a linear topological ring with fundamental neighborhood basis $\beta(0)$. $\hat{R}$ is called complete if%Sei $R$ ein linear topologischer Ring mit Fundamental-UB $\beta(0)$. Ein Ring $\hat{R}$ hei\ss{}t Vervollst\"andigung von $R$, falls
%$$\hat{R} \simeq \lim_{\substack{\longleftarrow\\I \in \beta(0)}} R/I$$
%i.e. the pro-finite limit of $R/I)$ for all $I \in \beta(0)$ and ring morphisms
%$R/I \longrightarrow R/J$, for all $I \subset J$ - ordered by inclusion.%d.h. der pro-endliche (oder inverse) Limes von  $(R/I)_{I \in \beta(0)}$, mit Ringmorphismen $R/I \longrightarrow R/J$ f\"ur alle $I \subset J$ und $\beta(0)$ angeordnet bzgl. Inklusion.
%\end{defi}
%Now, we are equipped with the appropriate tools to continue.
For the definition of linear topological rings, linear topological rings with fundamental basis and their completion consult the appendix.
\begin{defi}
Let $K/k$ and $K[[t]] \otimes A[[w]]$ be as above. The $K$-algebra $K[[t]] \hat{\otimes} A[[w]]$ is called the completion of $K[[t]] \otimes A[[w]]$ wrt the $\left<w\right>$-adic topology. To specify, if the neighborhood basis is defined by
$$\beta(0) = \left\{\left<1\otimes w\right>^i : i \geq 1\right\} = \left\{\left<1 \otimes w^i : \ w^i = w_1^{i_1} \ldots w_n^{i_n}, \sum i_j = i\right> : i \geq 1\right\},$$
then the completion is simply the pro-finite limit
$$\lim_{\substack{\longleftarrow\\I \in \beta(0)}} K[[t]] \otimes A[[w]]/I.$$
\end{defi}
Lastly, we note that the two algebras $\kappa \hat{\otimes} A[[w]]$ and $\mathcal{K} \hat{\otimes} A[[w]]$ are differential subalgebras of $K[[t]] \hat{\otimes} A[[w]]$. 
\begin{defi}[Umemura functor]
Let $K[[t]] \otimes A[[w]]$ and $K[[t]] \hat{\otimes} A[[w]]$ be as above. Let $\trm{CAlg}_K$ and $\trm{Grp}$ denote the categories of commutative $K$-algebras and groups, respectively. The Umemura functor is the functor
$$\trm{Ume}(K/k) : \trm{CAlg}_K \longrightarrow \trm{Grp}$$
assigning to every $A \in \trm{CAlg}_K$ the group of automorphisms $\varphi$ of $\mathcal{K} \hat{\otimes} A[[w]]$ wrt to derivations $\partial_t, \partial_x, \partial_w$ leaving $\kappa \hat{\otimes} A[[w]]$ fixed and making
$$\xymatrix{
\mathcal{K} \hat{\otimes} A[[w]] \ar[d]_{\varphi}\ar[rd]^{\psi}&\\
\mathcal{K} \hat{\otimes} A[[w]] \ar[r]_{\psi}&\mathcal{K} \hat{\otimes} \left(A/N(A)\right)[[w]]\\
}$$
commutative, where $\psi := id_\mathcal{K} \otimes \pi[[w]]$. Moreover, if $\lambda : A \longrightarrow B$ is a morphism in $\trm{CAlg}_K$, then one defines
$$\trm{Ume}(K/k)(\lambda) : \trm{Ume}(K/k)(A) \longrightarrow \trm{Ume}(K/k)(B),\ \varphi \longmapsto \varphi \otimes id_{B[[w]]}.$$
\index{Index}{functor!Umemura}
\end{defi}
\bmk Some additional statements are in place:
\bn
\item for any $A \in \trm{CAlg}_K$ the set $N(A) \subset \trm{Ann}(A)$ is the nilradical of $A$ and $\pi$ is its canonical projection $A \longrightarrow A/N(A)$.\index{Index}{nilradical}
\item for $A, B \in \trm{CAlg}_K$ and $\lambda \in \trm{Hom}_{K-\trm{alg}}(A,B)$ we regard $B[[w]]$ as a $A[[w]]$-algebra via
$$\lambda[[w]] : A[[w]] \longrightarrow B[[w]], \sum_{\alpha} a_\alpha w^\alpha \longmapsto \sum_\alpha \lambda(a_\alpha) w^\alpha.$$
\item Umemura introduced the so called Lie-Ritt functors and shows that $\trm{Ume}(K/k)$ is such a functor \cite{Ume96,Ume96b}. Heiderich gives a more general definition in \cite{Heid10} which we are going to repeat for clarity.
\en
\subsection{The Lie-Ritt functor}
As previously, we work in the same setting: $K/k$ a differential extension, $A \in \trm{CAlg}_K$ and all its algebras as above. Furthermore, let $n$ denote the transcendence degree of $K/k$.
\subsubsection{The infinitesimal coordinate transformation group}
Firstly, we define a special set of evaluation maps, which will be referred to as the set of infinitesimal coordinate transformation. It is shown that this set is indeed a group wrt. to composition. For $A[[w]]$ we define the differential algebra $A[[w]]\{\{Y\}\}$ to be the algebra of differential formal power series with coefficients in $A[[w]]$ (conforming to the definition of the ring of differential polynomials, \ref{RingOfDiffPolys} on pg. \pageref{RingOfDiffPolys}), i.e. a transcendental extension of $A[[w]]$, with variables $\left\{Y_i^{(j)} : 1 \leq i \leq n, j \in \nz_0^n\right\}$, where the super script index indicates $\partial_{Y_j}\left(Y_i^{(k)}\right) = Y_i^{(k+e_j)}$ (i.e. $A[[w]]$-derivations with $e_j \in \nz_0^n$ the canonical base vector).
\begin{defi}
We define
\bn
\item the set $$\Gamma(A,n) := \left\{\Phi = (\phi_1,\ldots,\phi_n) \in A[[w]]^n : \phi_i \equiv w_i \mod N(A)\right\},$$
with group structure via composition: $\Psi \cdot \Phi := \left(\psi_1(\Phi),\ldots,\psi_n(\Phi)\right)$ - the infinitesimal coordinate transformation group.
\item an $n$-variate iterative derivation $\theta$ wrt. $w$, such that
$$\theta^{(l)}\left(Y_i^{(k)}\right) := \left(\bao{c}
k + l\\
k\\
\ea\right) Y_i^{(k+l)}\ \forall l, k \in \nz_0^n,$$
and its restriction to $K$ coincides with $\theta_x$ as defined above.
\item the iterative differential subalgebra: 
$$A[[w]]\{A[[Y]]\}_\theta := A[[w]]\left[\left[\theta^{(l)}\left(Y_i^{(0)}\right) : l \in \nz_0^n, 1 \le i \leq n\right]\right] \subset A[[w]]\{\{Y\}\},$$
\item for $F \in A[[w]]\{A[[Y]]\}_\theta$ and $\Phi \in \Gamma(A,n)$ $F\mid_{Y = \Phi} = \sigma(F,\Phi)$,  where
$$\bao{rrcl}
\sigma : &A[[w]]\{A[[Y]]\}_\theta \times \Gamma(A,n) &\longrightarrow &A[[w]]\\
&&&\\
&\left(Y_i^{(k)},\Phi\right)&\longmapsto&\theta^{(k)}(\phi_i)\\
\ea$$
\en
\index{Index}{group!infinitesimal transformation}
\end{defi}
\bmk We shall note the set $\Gamma(A,n)$ can be considered as substitution homomorphism 
$$\hat{\phi}_i = \left[w_i \longmapsto
\phi_i\right],\ \hat{\phi}_i\mid_{A[[w]]/\left<w_i\right>} = id_{A[[w]]/\left<w_i\right>}$$
on $A[[w]]$. To see that this set is indeed a group, note that
\bn
\item associativity and closedness follows immediately from the last statement,
\item the unit element is simply $id_{A[[w]]}$ and
\item first, we note that if $u \in A^\times$ then $x := u + a \in A^\times$ for all $a \in N(A)$, as
$$x - u \in N(A) \LRA \exists m \in \nz,\ \trm{such}\ \trm{that}\ (x - u)^m = 0 = \sum_{0\leq l\leq m}\left(\bao{c}m\\l\\
\ea\right) x^l (-u)^{m-l}$$
$$\LRA (-u)^m = x \sum_{1 \leq l\leq m}\left(\bao{c}m\\l\\\ea\right) x^{l-1} (-u)^{m-l} \in A^\times.$$
Now, pick $\phi_i = a_i + w_i$ and $\psi_i = (1 + b_i) w_i$ then the inverse is $\phi_i^{-1} = w_i - a_i$ and $\psi_i^{-1} = (1 + b_i)^{-1} w_i$ for all $a_i, b_i \in N(A)$.
\en
Furthermore, let $X$ denote $\trm{map}(\{1,\ldots,n\} \times \nz_0^n,\nz_0) = \nz_0^{\{1,\ldots,n\} \times \nz_0^n}$ - is an element $F \in A[[w]]\{A[[Y]]\}_\Psi$ defined as
$$F = \sum_{\substack{\alpha \in \nz_0^n\\k \in X}} a_{\alpha,k} w^\alpha \prod_{(i,\beta) \in \{1,\ldots,n\} \times \nz_0^n} \left(Y_i^{(\beta)}\right)^{k(i,\beta)},$$
then the image $F\mid_{Y=\Phi}$ for a given $\Phi \in \Gamma(A,n)$ is
$$\sigma(F,\Phi) = F\mid_{Y=\Phi} = \sum_{\substack{\alpha \in \nz_0^n\\k \in X}} a_{\alpha,k} w^\alpha \prod_{(i,\beta) \in \{1,\ldots,n\} \times \nz_0^n} \theta^{(\beta)}\left(\phi_i\right)^{k(i,\beta)}.$$
With these definitions in place we can proceed with
\begin{defi}[Lie-Ritt functor]
A Lie-Ritt functor over $K$ is a group functor $G$ on $\trm{CAlg}_K$ such that there exits an $n \in \nz$ and an ideal $I \subset K[[w]]\{K[[Y]]\}_\theta$ such that $G(A) \simeq Z(I)(A)$, where
$$Z(I)(A) := \left\{\Phi \in \Gamma(A,n) : F\mid_{Y = \Phi} = 0\ \forall F \in I\right\}.$$
\index{Index}{functor!Lie-Ritt}
\end{defi}
\bmk If $\Phi \in \Gamma(A,n)$ is fixed we denote by $\sigma_\Phi(F)$ simply $\sigma(F,\Phi)$. Umemura defines the Lie-Ritt functors over $K$ via ideals in $K[[w]]\{\{Y\}\}$. However, Heiderich remarks that, in general, $\sigma_\Phi(F)$ is not well defined for arbitrary $F \in K[[w]]\{\{Y\}\}$.
\begin{prop}\label{LieRittFunctor}
Every Lie-Ritt functor over some commutative ring $K$ is isomorphic to a formal group scheme over $K$.
\end{prop}
\bws See \cite{Heid10}, proof of prop. 2.11.
\subsubsection{Umemura functor as Lie-Ritt functor}
We repeat and extend some of our above definitions. We set $k^\partial =: C$ (a commutative ring).
\begin{defi}
Let $G$ be a monoid, $D^1$ an irreducible pointed cocommutative $C$-Hopf-algebra of Birkhoff-Witt type and $D$ be the smashed product $D^1\#C[G]$ ($D^1$ as a $C[G]$-module algebra). If $A$ is a $D$-module algebra with structure map $\Psi$ %we denote by $_C\mathcal{M}(D,A)$ the $C$-module of $D$-module algebra homomorphisms, i.e. the subset of $f \in \trm{Hom}_C(D,A)$ such that the following diagram commutes:
%$$\xymatrix{
%D \otimes D\ar[r]^{id_D \otimes f}\ar[d]_{\Psi_D}&D \otimes A\ar[d]^{\Psi_A}\\
%D \ar[r]_f&A\\
%}$$
%where $\Psi_D$ denotes the $D$-module algebra structure on $D$ itself.
we define the map:
$$\rho : A \longrightarrow \trm{Hom}_C(D,A),\ a \longmapsto \Psi(\_ \otimes a) = [d \longmapsto \Psi(d \otimes a)].$$
This map is called the module algebra homomorphism.
\end{defi}
\bmk Let us discuss some immediate consequences for any $D$-module algebra $A$. Firstly, for the definition of $\rho$ we use the isomorphism:
$$\trm{Hom}_C(D \otimes A, A) \simeq \trm{Hom}_C(A, \trm{Hom}_C(D,A)).$$
Secondly, we get two other homomorphisms, induced by $\Psi_0 : D \otimes A \longrightarrow A$ and $\Psi_{\trm{int}} : D \otimes \trm{Hom}_C(D,A) \longrightarrow \trm{Hom}_C(D,A)$:
$$\bao{rrcl}
\rho_0 : & A & \longrightarrow & \trm{Hom}_C(D,A)\\
& a & \longmapsto & a \eps_D\\
&&&\\
\rho_{\trm{int}} : & \trm{Hom}_C(D,A) & \longrightarrow &\trm{Hom}_C(D \otimes D,A) \simeq \trm{Hom}_C(D,\trm{Hom}_C(D,A))\\
& f & \longmapsto &\Psi_{\trm{int}}(\_ \otimes f) := [d \otimes d' \longmapsto f \circ \mu_D(d \otimes d')].\\
\ea$$
$\Psi_0$ and $\Psi_{\trm{int}}$ are the trivial and internal module algebra homomorphism, respectively.
%\bn
%\item $_C\mathcal{M}(D,A)$ is a $C$-algebra via convolution:
%$$f \otimes g \longmapsto \mu_A \circ\left(f \otimes g\right) \circ \Delta_D,$$
%\item we define
%$$\rho = \left[a \longmapsto \Psi_A(\_ \otimes a) := \left[d \longmapsto \Psi_A(d\otimes a)\right]\right]\in\:_C\mathcal{M}(A,\!_C\mathcal{M}(D,A))$$
%via the isomorphism $_C\mathcal{M}(D\otimes A,A) \stackrel{\sim}{\longrightarrow} _C\mathcal{M}(A,\!_C\mathcal{M}(D,A))$,
%\item for every $C$ bialgebra/Hopf-algebra $D$ and $D$-module algebra $A$
%$$\rho_0 := [a \longmapsto \eps a := [d \longmapsto \eps(d) a]]$$
%defines the trivial $D$-module algebra structure on $A$.
%\en

\begin{lemm}
$\Psi$ is a morphism of $D$-module algebra structure on $A$ if and only if $\rho$ is morphism of $C$-algebras such that the following diagrams commute:
$$\bao{cc}
\xymatrix{
A \ar[rr]^{\rho}\ar[d]_\rho&&\trm{Hom}_C(D,A)\ar[d]^{\trm{Hom}_C(D,\rho)}\\
\trm{Hom}_C(D,A) \ar[rr]_{\trm{Hom}_C(\mu_D,A)}&&\trm{Hom}_C(D,\!\trm{Hom}_C(D,A))\\
}
&\xymatrix{
A \ar[r]^{\rho}\ar[rd]_{id_A}&\trm{Hom}_C(D,A)\ar[d]^{ev_{1_D}}\\
&A\\
}
\ea,$$
identifying $\trm{Hom}_C(D \otimes D,A)$ and $\trm{Hom}_C(D,\!\trm{Hom}_C(D,A))$.
\end{lemm}
\bmk A short proof is given in \cite{Heid10}, pg. 35. Nevertheless, we shall remark on some aspects of the notation:
\bn
\item the morphism $\trm{Hom}_C(\mu_D,A)$ is equivalent to the just defined $\rho_{\trm{int}}$.
\item the morphism $\trm{Hom}_C(D,\rho)$ denotes:
$$\bao{rcl}
D^* \otimes A &\longrightarrow& D^* \otimes \trm{Hom}(D,A)\\
&&\\
\delta \otimes a &\longmapsto& \delta \otimes \Psi(\_\otimes a) = \delta \otimes \rho(a)\\
\ea$$
\en
%In both cases, we are restricting to the submodule of $D$-module algebra morphisms in $\trm{Hom}(D,A)$ and $\trm{Hom}(D,\trm{Hom}(D,A))$, respectively.
In \cite{Heid13} it is shown that if $D \simeq D_{der}$ and $\qz \subset A$, then $\trm{Hom}_C(D,A)$ is isomorphic to $A[[t]]$ and $\rho$ is given by the universal Taylor homomorphism.
\begin{prop}\label{GroupLaw}
Let $F$ be an $n$-dimensional group law over some commutative ring $C$. The associated group functor $\mathfrak{F}$ is isomorphic to the Lie-Ritt functor $Z(I) \subset \Gamma(C,n)$ with $n$-variate higher differential ideal
$$I := \left<\theta^{(\alpha)}(F(w,\Psi(Y))) : \alpha \in \nz_0^n\bsl\{0\}\right>_{C[[w]]\{C[[Y]]\}},$$
where $\Psi \in C[[y]]^n$ such that $\Psi(0) = 0, F(\Psi(u),u) = 0$ for all $u \in C[[y]]^n$.
\end{prop}
\bmk Notion of formal group laws and formal groups is given in the appendix. A prove as well as the proposition can be found in \cite{Heid10}, pg. 97.
\bsp \label{example_Heid_add_mul_grp_law}%Let $n \in \nz$ and $\mathcal{F}$ be a family of differential polynomials over some differential field $(k,\partial)$, in particular explicit differential equations. Our Picard-Vessiot extension $k(x)$ is of the $\partial x_i = p_i(x_1,\ldots,x_n)$. Fix $K = k(x)$ and $A = K[\eps] \simeq K[X]/\left<X^2\right>$ (i.e. the dual numbers of $K$). We know that $\Gamma(A,n) = \{\Phi \in A[[w]]^n : \Phi \equiv w \mod N(A)^n\}$. Hence
%$$\Gamma(A,n) := \left\{\left(\sum_{\alpha \in \nz_0^n} a_{i,\alpha} w^\alpha\right)_{i=1}^n : a_{i,e_j} \equiv 1 \mod N(A) \wedge a_{i,\alpha} \equiv 0 \mod N(A) \forall \alpha \neq e_j,\ 1\leq i, j \leq n\right\}.$$
Heiderich shows in case of the $\zz$-algebra $\zz[[w]]\{\zz[[Y]]\}_\theta$ and $n = 1$ that the functor induced by the subset $\{a + w: a \in N(A)\}$ of $\Gamma(A,1)$ is isomorphic to the additive group scheme $\mathbb{G}_a$ for every $\zz$-algebra $A$. The associated ideal in $\zz[[w]]\{\zz[[Y]]\}$ is generated by $Y^{(1)} - 1$ and $Y^{(j)}$ for all $j \geq 2$. The subset $\{(1 + a) w : a \in N(A)\}$ is isomorphic to the multiplicative group scheme $\mathbb{G}_m$. The associated ideal is generated by $w Y^{(1)} - Y$ and $Y^{(j)}$ for all $j \geq 2$.% Extending his approach we construct the following sets:
%$$\bao{rcl}
%G_1(A) &:=& \left\{w + a_i e_i \in A[[w]]^n: a_i \in N(A), 1 \leq i \leq n\right\}\\
%&&\\
%G_2(A) &:=& \left\{w + b_i w_i e_i \in A[[w]]^n : b_i \in N(A), 1 \leq i \leq n\right\}\\
%\ea$$
%Here we use $w = \sum_{i=1}^n w_i e_i \in A[[w]]$.% Next we have to compute the ideal $I$ in $K[[w]]\{K[[Y]]\}$ such that $F\mid_{Y=\Phi} = 0$ for all $F \in I$ and $\Phi \in G_i(A)$ with $i = 1,2$.
\begin{satz}
The Umemura functor $\trm{Ume}$ is a Lie-Ritt functor.
\end{satz}
\bws This is a consequence of theorem 2.14 (summarized in corollary 2.15) in \cite{Heid10} and \cite{Heid11}.
\begin{koro}
$\trm{Ume}(K/k)$ is a formal group scheme.
\end{koro}
\subsection{PV-theory of Artinian simple module algebras}
Here, $(k,\partial)$ is again a differential field (more general a simple artinian $D$-module algebra) - with $\trm{char} k = 0$ and let $R$ be the PV-ring over $k$. For clarity, we are going to repeat some of the previous constructs, though we will adhere to the notation introduced in \cite{Heid10}. For $\partial : k \longrightarrow k$ we define
$$D := k[\partial]\ \trm{and}\ \Psi : D \otimes A \longrightarrow A, d \otimes a \longmapsto d(a)$$
for all $A \in \trm{CAlg}_k$, as derivation bialgebra over $k$ and $\Psi$ the $D$-module algebra structure morphism. Recall there is a unique morphism $\rho \in \trm{Hom}_C(A,\:\trm{Hom}_C(D,A))$, with
$$\rho = \left[a \longmapsto \left[d \longmapsto \Psi(d \otimes a)\right]\right].$$
In addition, the differential subalgebra $A^\rho$ is defined as $A^\Psi$ (i.e. the constant differential subalgebra).
\begin{defi}
Let $(K, \partial_K)/(k,\partial)$ be a differential extension. We call $K/k$ a PV extension if the following statements hold:
\bn
\item $K^{\rho_K} = k^{\rho}$,
\item there is a differential subalgebra $k \subset R \subset K$, with $R^{\rho_R} = k^\rho$ such that $Q(R) = K$ and a $k^\rho$-subalgebra:
$$H := (R \otimes_k R)^{\rho_R \otimes \rho_R},$$
and $H$ generates $R\otimes_k R$ as a left/right $R$-algebra.
\en
\end{defi}
\bmk \label{HeidRemk} In \cite{Heid10} it is shown that $R$ is unique and the map $R \otimes_{k^\rho} H \longrightarrow R \otimes_k R$ is an isomorphism of $D$-module algebras. Since we only restrict to derivation module algebras (Heidereich uses a general bialgebra) we want to elaborate on some of the constructs before proceeding.
\bn
\item instead of $R = k[x_{i,j},1/\det X]$, where $X \in  \trm{Gl}_n(R)$ is the fundamental solution, we use $k[X,X^{-1}]$. But clearly, both $k$-algebras define isomorphic rings (as the inverse matrix $X^{-1}$ is composed of entries in $k[x_{ij}]$ and has the inverse of $\det X$ as factor).
\item The subalgebra $H$ is called the Hopf-algebra of $K/k$ and $R$ is called prinicpal $D$-module algebra of $K/k$.
\item The Galois group $\trm{DGal}(K/k) := \trm{Spec}(H)$.
\item In addition we have $H \simeq k^\rho[(X\otimes1)(1 \otimes X^{-1}),(1\otimes X)(X^{-1} \otimes 1)]$. Nevertheless, we will not use this.
\en
\begin{prop}\label{prop_hopf_struct}
The differential subalgebra $H \subset R\otimes_k R$ carries an $R$-coalgebra structure given by the coalgebra structure on $R\otimes R$:
\bn
\item $\Delta_{R\otimes R} : R\otimes_k R \longrightarrow (R\otimes_k R) \otimes_R (R\otimes_k R)$, $a \otimes b \longmapsto a \otimes 1 \otimes 1 \otimes b$,
\item $\eps : R\otimes_k R \longrightarrow R$, $a \otimes b \longmapsto a b$ and lastly
\item $S: R \otimes_k R \longrightarrow R \otimes_k R$, $a \otimes b \longmapsto b \otimes a$ an antipode
\en
making $R\otimes R$ and its subalgebra $H$ a Hopf-algebra.
\end{prop}
\subsubsection{Comparing general theory with PV theory}
We assume as in \cite{Heid13} $(K/k, R, H)$ to be an finitely generated PV extension of an artinian $D$-module algebras with $D = D^1 \# k.G$ for some pointed irreducible cocommutative bialgebra of Birkhoff-Witt type (cofree), $R$ the principle $D$-module algebra and $H$ its associate Hopf algebra. Furthermore, let $X \in \trm{Gl}_n(R)$ be the fundamental matrix, i.e. $R \simeq k[X,X^{-1}]$, for each (minimal) prime ideal $\mathfrak{p}) \subset K$ the field $K/\mathfrak{p}$ be finitely generated and separable over $k/(k \cap \mathfrak{p}$ and the transcendence degree $n$ for $K/k$ agree for all $\mathfrak{p} \in \trm{Spec}(K)$. We have a unique $n$-variate iterative derivation
$$\theta_x : K \longrightarrow K[[w]],\ x_i \longmapsto x_i + w_i$$
and two $D$-module algebra homomrphisms:
$$\rho = [a \longmapsto ev_a = [d \longmapsto \Psi(d \otimes a)]] \in \trm{Hom}(K, \trm{Hom}(D, K))$$
$$\rho_0 = [a \longmapsto a \cdot \eps_D = [d \longmapsto \eps_D(d) a]] \in \trm{Hom}(K, \trm{Hom}(D,K)).$$
\begin{defi}
We denote with $D_{\trm{der}}$ the derivation bialgebra $k[\partial] \subset \trm{End}_{k^\partial}(k)$, with
$D_{\trm{ID}}$ the iterative derivation bialgebra $k[\theta]$ for some iterative derivation $\theta : k \longrightarrow k[[t]]$ and with $D_{\trm{ID}^n}$ the $n$-variate iterative derivation bialgebra.
\end{defi}
In \cite{Heid13} it is noted that $D_{\trm{der}} \simeq D_{\trm{ID}}$, $D_{\trm{ID}}^{\otimes n} \simeq D_{\trm{ID}^n}$ and $\trm{Hom}(D_{\trm{der}},A) \simeq A[[t]]$ and $\trm{Hom}(D_{ID}^{\otimes n}, A) \simeq A[[w]]$ with $w = (w_1,\ldots,w_n)$ for all commutative algebras $A$ and $\trm{char}k = 0$. Therefore, $\trm{Hom}(D_{\trm{der}},K)$ is clearly closed with respect to $\theta_x = \sum_{\alpha} \frac{1}{\alpha!}\partial_x^\alpha \otimes w^\alpha$:
$$f = [d \longmapsto f(d)] \longmapsto \theta_x(f) = \left[d \longmapsto \theta_x(f(d)) = \sum_\alpha \frac{1}{\alpha!} \partial_x^\alpha(f(d)) \otimes w^\alpha\right].$$
We recall that $[\partial_x,\partial_K] = 0$, i.e. $K$ is a partial different algebra wrt. $\{\partial_K = \partial, \partial_x\}$. It takes a little more to show closedness for $\rho(K)$:
$$\bao{rcl}
f = ev_a &=& [d \longmapsto \Psi_K(d \otimes a)]\\
&&\\
&\longmapsto& \theta_x(ev_a)\\
&&\\
&=& \left[d \longmapsto \sum_\alpha \theta_x^{(\alpha)} (ev_a(d)) \otimes w^\alpha = \sum_\alpha d\left(\theta^{(\alpha)}_x(a)\right) \otimes w^\alpha\right]\\
&&\\
&=& \sum_\alpha ev_{\theta^{(\alpha)}_x(a)} \otimes w^\alpha\\
\ea$$
For $w \stackrel{\pi_w}{\mapsto} 0$ we have identity and $\partial_{w_i} = [w_j \longmapsto \delta_{i,j}]$, $\partial_w^\beta = \partial_{w_1}^{\beta_1} \circ \ldots \circ \partial_{w_l}^{\beta_l}$ for all $\beta \in \nz_0^l$ we get:
$$\pi(\partial_w^\beta(\theta_x(f))) = \partial_x\beta(f).$$
\begin{defi}
For some field $k$ we call a $k$-algebra $K$ \'{e}tal if $K \otimes_k \ov{k} \simeq \ov{k}^n$ as a vector space over the algebraic closure $\ov{k}$ of $k$ and $n \geq n$ an integer.
\end{defi}
\bmk An algebra $K$ over $k$ is \'{e}tal if and only if
$$K \simeq \prod_{i=1}^n k[x]/\left<f_i\right>,\ f_i \in k[x] \trm{separable}.$$
\subsection{Example} Revisiting our example on \pageref{twoD} with $\left(k \subseteq \ov{\qz}, \partial = 0_{\ov{\qz}}\right)$ and the $k$-linear differential operator $L = \partial^2 - a \cdot id_k \in k[\partial] =: D$, $a \in k^\times$. We are going to use the notation already introducted in \ref{twoD}, pg. \pageref{twoD}. Again, we are discussing two cases:
\bd
\item[reducible] The polynomial $X^2 - a \in k[X]$ decomposes into two linear factors $X - \sqrt{a}, X + \sqrt{a} \in k[X]$. In this case, we denote the PV ring with $R_1$.
\item[irreducible] The polynomial $X^2 - a \in k[X]$ is irreducible - i.e. $k[X]/\left<X^2 - a\right>$ is a field extension over $k$. We denote the PV ring with $R_2$.
\ed
We remark that due to the "constness" of $a \in k$, $R_2(\sqrt{a)}) \simeq k(\sqrt{a}) \otimes_k R_2$ gets a $D$ module algebra via
$$\bao{rrcl}
\rho_{R(\sqrt{a})} : & k(\sqrt{a}) \otimes_k R_2 &\longrightarrow &\trm{Hom}_k(D, k(\sqrt{a}) \otimes R_2)\\
&&&\\
&\alpha \otimes r &\longmapsto & \left[d \otimes \alpha \otimes r \longmapsto \alpha \otimes d(r)\right].\\\ea$$
Furthermore, our two PV rings are isomorphic via the isomorphis defined on pg. \pageref{PVisomorph}, $R_1 \simeq R_2(\sqrt{a})$. As above, over Hopf algebra $D$ is $k[\partial]$ and $\Psi_k$ is trivial (i.e. subalgebra of $R^{\Psi_R}$). Next, we want to describe
\paragraph{The prinicple $D$ module algebra}
which in our case is simply $R_i$, $i = 1, 2$.
\subsubsection{The Hopf-algebra and its module algebra}
First, we note that $D$ is a cocommutative Hopf-algebra over $k = \currfield$, being a field, is simple (as a ring) and artinian since every descending chain of ideals stabilizes after finitely many steps ($(1)$ and $(0)$ are the only ideals). Next, we recall that $K = \currfield(y_1)$ with $\currfield$-derivation $\partial = [y_1 \longmapsto \sqrt{a} y_1, y_{-1} \longmapsto - \sqrt{a} y_{-1}]$. Now, we want to show the $D$-module algebra structure on $K$, or $R = \currfield[y_1,y_{-1}]$. Let $\Psi_K : D \otimes K \longrightarrow K,  
d \otimes x = \sum_i d_i \partial^i \otimes x \longrightarrow \sum_i d_i \partial^i(x) =: d(x)$. We need to show $\Psi_K(d_1 \otimes \Psi_K(d_2 \otimes x)) = \Psi_K(\mu_D \otimes id_K(d_1 \otimes d_2 \otimes x))$, i.e. $K$ is a $D$-left module, which is immediately clear as the LHS simply says $d_1(d_2(x))$ and the RHS says $\mu_D(d_1 \otimes d_2)(x) = (d_1 \circ d_2)(x)$ being equal. Next, we want to introduce the $D$-left comodule structure on $K$. An obvious choice is $\rho := \eta \otimes id_K : K \simeq k \otimes K \longrightarrow k[\partial] \otimes K, x \longmapsto 1_D \otimes x$ providing the desired commutativity of the diagrams:
$$\bao{cc}
\xymatrix{
K \ar[r]^\rho \ar[d]_\rho & D \otimes K\ar[d]^{id_D \otimes \rho}\\
D \otimes K \ar[r]_{\Delta_D \otimes id_K} & D \otimes D \otimes K\\
} &
\xymatrix{
K \ar[r]^\rho \ar[rd]_\sim & D \otimes K\ar[d]^{\eps \otimes id_K}\\
&K,}\\
\ea$$
in particular, we get $\Psi_K(\rho(x)) = id_K(x) = x$ (i.e. $\Psi_K$ is the left inverse of $\rho$). To conclude, we have shown that both
$$\bao{cc}
\xymatrix{
D \otimes K^{\otimes2} \ar[d]_{\Delta_D \otimes id_K \otimes id_K} \ar[rr]^{id_D \otimes \mu_K}& & D \otimes K \ar[r]^{\Psi_K} & K\\
D^{\otimes2} \otimes K^{\otimes2} \ar[d]_{id_D \otimes \tau \otimes id_K}&&&\\
(D \otimes K)^{\otimes2} \ar[rrr]_{\Psi_K \otimes \Psi_K} & & &K \otimes K \ar[uu]_{\mu_K}\\
} &
\xymatrix{
D \ar[rr]^{id_D \otimes \eta_K} \ar[rrd]_{\eps \otimes id_K}&& D \otimes R.1_K\ar[d]^{\Psi_K}\\
&&K\\
}\\
\ea$$
commute. 
\subsubsection{The Hopf-algebra of constants}
More precisely, $H$ is the kernel of $\Delta_D(\partial) : R \otimes_k R \longrightarrow R \otimes_k R$. We are going to show this in a short instance. Reformulating the definition of $H$ more generally (i.e. $D = k[\partial]$, $k^{\Psi_k} = k^\partial = \currfield$ in our case):
%$$\bao{rclcl}
%\partial_R(X X^{-1}) &=& \partial_R(1_R) &=& \partial_R(X) X^{-1} + X \partial_R(X^{-1})\\
%&&&&\\
%&=& 0&&\\
%&&\LRA&&\\
%X\partial_R(X^{-1}) &=& - \partial(X) X^{-1} &=& -A X X^{-1}\\
%&&\LRA&&\\
%\partial_R(X^{-1}) &=& -X^{-1} A&&\\
%\ea$$
%We get the same result for $\partial(X^{-1} X)$. On the other hand, $X^{-1} = \det X^{-1} \left(\bao{cc}a x_1 & -x_2\\
%-x_2 & x_1\\
%\ea\right)$, hence $\partial(X^{-1}) = \partial(\det X^{-1}) \left(\bao{cc}a x_1 & -x_2\\
%-x_2 & x_1\\
%\ea\right) + \det X^{-1} \left(\bao{cc}a x_2 & -a x_1\\
%-a x_1 & x_2\\
%\ea\right) \stackrel{!}{=} X^{-1} A$ implying $\partial(\det X^{-1}) = -\frac{\partial(\det X)}{\det X^2} = 0$. Direct computation confirms this. Hence we see that $\det X^{i} \otimes \det X^{j} \in H$ for $i, j \in \{0, \pm1\}$. Now let us consider the two factor decompositions of $\det X = a x_1^2 - x_2^2 = (\pm\sqrt{a} x_1 + x_2)(\pm\sqrt{a} x_1 - x_2)$ (where the roots of $a$ are always having the same sign).
%$$\bao{rcl}
%\Delta(\partial)\left([\sqrt{a} x_1 + x_2] \otimes [\sqrt{a} x_1 - x_2]\right) &=&
%(1\otimes \partial + \partial\otimes 1)\left([\sqrt{a} x_1 + x_2] \otimes [\sqrt{a} x_1 - x_2]\right)\\
%&&\\
%&=& (\sqrt{a} x_1 + x_2) \otimes \partial(\sqrt{a} x_1 - x_2)\\
%&& + \partial(\sqrt{a} x_1 + x_2) \otimes (\sqrt{a} x_1 - x_2)\\
%&&\\
%&=& (\sqrt{a} x_1 + x_2) \otimes (\sqrt{a} x_2 - a x_1)\\
%&& + (\sqrt{a} x_2 + a x_1) \otimes (\sqrt{a} x_1 - x_2)\\
%&&\\
%&=& 0\\
%\ea$$
%By symmetry, this holds for $(\sqrt{a} x_1 - x_2) \otimes (\sqrt{a} x_1 + x_2)$ and by Leibniz-rule for
%$(\det X^{-1} \otimes \det X^{-1}) (\sqrt{a} x_1 \pm x_2) \otimes (\sqrt{a} x_1 \mp x_2)$. On the other hand, $\Delta(1) = 1 \otimes 1$ and clearly all elements fulfill
$$H:= \left\{r_1 \otimes r_2 : \Psi_{R\otimes R}(d \otimes (r_1 \otimes r_2)) = \eps_D(d) (r_1 \otimes r_2)\right\}.$$
With $\eps_D(\partial^i) = \delta_{0,i}$ and image of $1_D$ under comultiplication being $1_D\otimes 1_D$, we only need to compute $\ker \Delta_D(\partial)$
$$\bao{rcl}
H &=& (\partial_R \circ \mu_R)^{-1}(0)\\
&&\\
&=& \{r_1 \otimes r_2 \in R\otimes_k R : \partial_R \circ \mu_R (r_1 \otimes r_2) = 0\}\\
&&\\
&=& \{r_1 \otimes r_2 : (1 \otimes \partial + \partial \otimes 1)(r_1 \otimes r_2) = 0\}\\
&&\\
&=& (\Delta (\partial))^{-1}(0) = \ker \Delta(\partial)\\
\ea.$$
%As we just saw, the elements $\alpha (\pm \sqrt{a} x_1 \pm x_2) \otimes (\pm\sqrt{a} x_1 \mp x_2) \in H$ for $\alpha \in \{1 \otimes 1, \det X^{-1} \otimes \det X^{-1}\}$. On the other hand, we get that $r_1 \otimes r_2 \in H\bsl\{0, 1\otimes 1\}$ if and only if $r_1\otimes r_2 \in \mu_R^{-1}(\det X)$. This is obviously the case for the above defined elements. Additionally, $a x_1 \otimes x_1 - x_2 \otimes x_2$ is an element in $H$ which ca be verified either by direct computation or by our reformulated definition of $H$.\\
As $\partial(y_{\pm 1}) = \pm \sqrt{a} y_{\pm 1}$ we get:
$$\bao{rclcl}
\Delta_D(\partial)(y_1 \otimes y_{-1}) &=& \sqrt{a} y_1 \otimes y_{-1} - \sqrt{a} y_{1} \otimes y_{-1} &=& 0\\
&&&&\\
\Delta_D(\partial)(y_{-1} \otimes y_{1}) &=& -\sqrt{a} y_{-1} \otimes y_{1} + \sqrt{a} y_{-1} \otimes y_{1} &=& 0\\
\ea$$
$$H \supset \currfield\left[y_1\otimes y_{-1},y_{-1} \otimes y_1\right].$$
Following our notation from example \ref{twoD} on page \pageref{twoD}, since $a x_1 \otimes x_1 + x_2 \otimes x_2, \sqrt{a} (x_1 \otimes x_2 - x_2 \otimes x_1) \in \left(S(L_+)\oplus (L_-)\right)^{\otimes 2}$ are the only other generating elements already contained in $\currfield\left[y_1\otimes y_{-1},y_{-1} \otimes y_1\right]$, we get
$$H \subset \currfield\left[y_1\otimes y_{-1},y_{-1} \otimes y_1\right].$$
Next, we want to introduce the comultiplication and counit for the elements defined above as described in \ref{prop_hopf_struct}.% Since $H$ is generated by units in $R$ (or more explicitly its tenors in $R\otimes_k R$) we see that all generators form a group-like sub Hopf algebra in $H$, i.e. $\Delta_H(x) = x \otimes x, \eps_H(x) = 1, S(x) = x^{-1}$ for some generator $x \in H$. Hence, if $x, y \in H$ are generators of $H$ we get
Hence, $\Delta_H = [y_{\pm 1} \otimes y_{\mp 1} \longmapsto y_{\pm 1} \otimes 1 \otimes 1 \otimes y_{\mp 1}], \eps = [y_{\pm 1} \otimes y_{\mp 1} \longmapsto y_{\pm 1} y_{\mp 1}]$ and $S = [y_{\pm 1} \otimes y_{\mp 1} \longmapsto y_{\mp 1} \otimes y_{\pm 1}]$.
$$\eps_H(a \otimes b) = a b = \frac{1}{2}\eps_H(a \otimes b + b \otimes a),\ \eps_H(a \otimes b - b \otimes a) = 0,\ \Delta(x - y) = x\otimes x - y \otimes y.$$
Expanding the coproduct:
$$\bao{rcl}
x - y &=& \underbrace{(\sqrt{a} x_1 + x_2)}_{y_1} \otimes \underbrace{(\sqrt{a} x_1 - x_2)}_{y_{-1}} - (\sqrt{a} x_1 - x_2) \otimes (\sqrt{a} x_1 + x_2)\\
&&\\
&=& a x_1 \otimes x_1 - \sqrt{a} x_1 \otimes x_2 + \sqrt{a} x_2 \otimes x_1 + x_2 \otimes x_2 \\
&&\\
&& - a x_1 \otimes x_1 - \sqrt{a} x_1 \otimes x_2 + \sqrt{a} x_2 \otimes x_1 - x_2 \otimes x_2\\
&&\\
&=& 2 \sqrt{a} (x_2 \otimes x_1 - x_1 \otimes x_2)\\
\ea$$
We remark that the coproduct is defined via $R^{\otimes 2} \otimes_R R^{\otimes 2}$. Hence, $R$-scalars in the inner positions cancel. Its coproduct is:
$$\bao{rcl}
\Delta_H(x - y) &=& x \otimes x - y \otimes y\\
&&\\
&=& (\sqrt{a} x_1 + x_2) \otimes (\sqrt{a} x_1 - x_2) \otimes (\sqrt{a} x_1 + x_2) \otimes (\sqrt{a} x_1 - x_2)\\
&&\\
&& - (\sqrt{a} x_1 - x_2) \otimes (\sqrt{a} x_1 + x_2) \otimes (\sqrt{a} x_1 - x_2) \otimes (\sqrt{a} x_1 + x_2)\\
&&\\
&=& (\sqrt{a} x_1 + x_2) \otimes 1_H \otimes 1_H \otimes (\sqrt{a} x_1 - x_2)\\
&&\\
&& - (\sqrt{a} x_1 - x_2) \otimes 1_H \otimes 1_H \otimes (\sqrt{a} x_1 + x_2)\\
&&\\
&=& 2 \sqrt{a} (x_2 \otimes 1 \otimes 1 \otimes x_1 - x_1 \otimes 1 \otimes 1 \otimes x_2)\\
%&=& 2 a \sqrt{a} x_1 \otimes x_1 \otimes (x_2 \otimes x_1 - x_1 \otimes x_2) + 2 a \sqrt{a} (x_2 \otimes x_1 - x_1 \otimes x_2) \otimes x_1 \otimes x_1\\
%&&\\
%&& + 2 \sqrt{a} x_2 \otimes x_2 \otimes (x_2 \otimes x_1 - x_1 \otimes x_2) + 2 \sqrt{a} (x_2 \otimes x_1 - x_1 \otimes x_2) \otimes x_2 \otimes x_2\\
%&&\\
%&=& 2 \sqrt{a} (x_2 \otimes x_1 - x_1 \otimes x_2) \otimes (a x_1 \otimes x_1 - x_2 \otimes x_2)\\
%&&\\
%&& + 2 \sqrt{a} (a x_1 \otimes x_1 - x_2 \otimes x_2) \otimes (x_2 \otimes x_1 - x_1 \otimes x_2)\\
%&&\\
%&=& 2 \sqrt{a} (x - y) \otimes (a x_1 \otimes x_1 - x_2 \otimes x_2) + 2 \sqrt{a} (a x_1 \otimes x_1 - x_2 \otimes x_2) \otimes (x - y)\\
%&&\\
%&=& 2 \sqrt{a} [(a x_1 \otimes x_1 - x_2 \otimes x_2),x - y]_{R\otimes R},\\
&&\\
&=& y_1 \otimes 1 \otimes 1 \otimes y_{-1} - y_{-1} \otimes 1 \otimes 1 \otimes y_1\\
\ea$$
%where $[.,.]_{R\otimes R}$ denotes the Lie-bracket of $R\otimes R$ wrt to the tensor product (not the intrinsic Lie-bracket). This is a direct proof of cocommutativity (for the sub Hopf algebra $k[x - y]$). But clearly, if all generators are cocommutative then so are their linear combinations. However, the other generators differ only in the sign of $\sqrt{a}$ and/or in carrying a factor $\det X^i \otimes \det X^j$, $i, j = 0, -1$. But this is also a group-like element implying all coproducts are of the above form (modulo sign of root and factor). 
In particular, following prop. \ref{GroupLikeHopfIdeal} we know the set of differences of group-like elements generates a bi-ideal in $H$. Since $H$ itself is generated by group-like elements, we get $I(\mathcal{G}(H)) := \left<g - h : g, h \in \mathcal{G}(H)\right>$ is a proper bi-ideal in $H$. It is enough to show that $I(\mathcal{G}(H))$ is stable under antipode action:
$$S : H \otimes H \longrightarrow H,\ y_{\pm 1}^i \otimes y_{\mp 1}^j \longmapsto y_{\pm}^{-i} \otimes y_{\mp}^{-j}, i, j \in \zz.$$
But $S$ maps the generators of $I(\mathcal{G}(H))$ to its generators:
$$y_1 \otimes y_{-1} \longmapsto y_{-1} \otimes y_1,\ y_{-1} \otimes y_1 \longmapsto y_1 \otimes y_{-1},$$
implying
$$S(g) \in I(\mathcal{G}(H)), \forall g \in \mathcal{G}(H).$$
Summarizing, we get:
%This shows that the coalgebra $I$ generated by $x - y$, where $x = y_1 \otimes y_{-1}, y = \tau_{R\otimes R}(x) \in H$, is a sub coalgebra of $\ker \eps$. Next, we have to show %indeed a (two-sided) coideal $I$ in $H$.
%$$\Delta(I) \subset H \otimes I + I \otimes H\ \wedge\ I \subset \ker \eps.$$
%But clearly:
%$$\bao{rcl}
%\Delta_H(x-y) &=& \frac{1}{2} \underbrace{(y_1 \otimes y_{-1} - y_{-1} \otimes y_1)}_{\in I} \otimes_R \underbrace{(y_1 \otimes y_{-1} + y_{-1} \otimes y_1)}_{\in H}\\
%&&\\
%&& + \frac{1}{2} \underbrace{(y_1 \otimes y_{-1} + y_{-1} \otimes y_1)}_{\in H} \otimes_R \underbrace{(y_1 \otimes y_{-1} - y_{-1} \otimes y_1)}_{\in I},\\
%\ea$$
%showing skew-primitivity and subsequently, our claim. The ideal property simply follows from the fact that $\eps$ is an algebra homomorphism. To summarize:
$$\bao{ccc}
H &=& \currfield[y_1 \otimes y_{-1},y_{-1} \otimes y_1]\\
&&\\
I &=& H.(y_1 \otimes y_{-1} - y_{-1} \otimes y_1) \ \trm{Hopf-ideal}\\
\ea$$
We have already shown, that our differential equation $L(x) = 0$ decomposes into two factors. This was used in our last computations. However, in case $L$ does not decompose the primary computations (PV-ring is $R = k[x_1,x_2,1/(a x_1^2 - x_2^2)]$, etc.) are still valid. Only our Hopf-algebra $H$ is generated by different elements.
%
%We remark that both $R = \currfield[y_1,y_{-1}]$ and $H = \currfield[y_1\otimes y_{-1},y_{-1}\otimes y_1]$ do not have any $D = \currfield[\partial]$-stable ideals:
%\bd
%\item[Case $R$] Let $I \subset R$ be an ideal and we assume differential closedness - i.e. $\partial(I) \subset I$.
%%As a noetherian $R$-submodule of a noetherian module $R$ (generated by $y_{\pm 1}$ over $\currfield$), $I$ is finitely generated. Hence, let $S:= \{s\} \subset I$ be one generating set. By differential closedness, we get for any $s \in S$:
%%$$\partial(s) = \partial\left(\sum_{i=-m}^n s_i y_1^i\right) = \sum_{i=-m}^n s_i \partial(y_1^i) = \sum_{i=-m}^n i \sqrt{a} s_i y_1^i \in I$$
%%$$\LRA \partial(s) - s = \sum_{i=-m}^n (i \sqrt{a} - 1) s_i y_1^i \in I$$
%%But both, $s, \partial(s) - s$ are of degree $n$, or $m$ wrt. $y_{\pm 1}$ and $y_1^m (\partial(s) - s) \in \currfield[y_1]$.
 %There is an $I' \subset \currfield[y_1]$, such that $S_{y_1}^{-1}(I') \subset I$. By definition of $I$, we get
%$$y_1^m t \in I' \RA \partial(y_1^m t) = \underbrace{m \sqrt{a} y_1^m t}_{\in I'} + \underbrace{y_1^m \partial(t)}_{\in \partial(I')},$$
%but identifying $I' := I \cap \frac{\currfield[y_1]}{1}$ we get $\partial(I') \subset I'$. Being a PID, $\currfield[y_1]$ all $I'$ are of the form $\left<s\right>$. On the other hand, $\partial$ operates on all weight spaces $\currfield.y_1^i$, $i \geq 1$, invariantly:
%$$\bao{rrcl}
%\partial_i := \partial\mid_{\currfield.y_1^i} : &\currfield.y_1^i &\longrightarrow& \currfield.y_1^i\\
%&&&\\
%&y_1^i &\longmapsto&i \sqrt{a} y_1^i\\
%\ea$$
%Hence, the degree of all polynomials in $\partial(I')$ and the preimages in $I'$ do agree. Each derivative of the generators $s$ agree in degree but also reduce to zero modulo $\left<s\right>$ contradicting our claim $\partial(s) \in \left<s\right>$. Thus, all $D$-stable ideals in $R$ are indeed trivial.
%\item[Case $H$] Exactly as the case above, since $H \simeq \currfield[y_1,y_{-1}]$.
%\ed
%
%i.e. both are simple $D$-module algebras over $\currfield$.
\paragraph{Isomorphism}
We claim, that $(H, \Psi_H) \simeq (R, \Psi_0)$ as $D$-module algebras and $\currfield$-Hopf algebras,
 where
$$\Psi_0 = [d \otimes r \longmapsto \eps_D(d) r].$$
\bws Consider the map 
$$\bao{rrcl}
\varphi : &R &\longrightarrow &H\\
&y_1^i&\longmapsto&y_1^i \otimes y_{-1}^i\\
&y_{-1}^i&\longmapsto&y_{-1}^i \otimes y_1^i,\\
\ea$$
defining an $\currfield$-algebra homomorphism. We remark that $R$ has group-like generators $1, y_1, y_{-1}$ which are antipode-stable. Consequently, $\varphi$ does commute:
$$(\varphi \otimes \varphi) \Delta_R = \Delta_H \varphi,\ S_H \varphi = \varphi S_R,\ \eps_H = \eps_R \varphi.$$
Thus, it is enough to show that $\varphi$ is a bijection. As $1_R \longmapsto 1_R \otimes 1_R$ $\varphi$ is a monomorphism. And clearly, $\sum_{i=-m}^n \lambda_i y_1^i \in \varphi^{-1}\left(\sum_{i=-m}^n \lambda_i y_{1}^i \otimes y_{-1}^i\right)$, making $\varphi$ surjective. Therefore, $\trm{Spec}(R) \simeq \trm{Spec}(H)$. Consequently, we have that
$$\trm{Spec}(H) \simeq \trm{Spec}(\currfield[y_1]) \bsl \left\{\left<y_1\right>\right\} = \left\{\left<y_1 - a\right> : a \in \currfield^\times\right\} \cup \{0\}.$$
Recalling the definition of $H = \ker \Delta_D(\partial)$ clearly shows the first part. The set of maximal ideals $\max(R)$ forms indeed a group:
$$X := \max(R) = \trm{Spec}(R) \bsl \{0\} = \left\{\left<y_1 - a\right> : a \in \currfield^\times\right\} \simeq \currfield^\times,$$
as claimed in remark  \ref{HeidRemk} if we consider the following map:
$$\bao{rrcl}
m : &X \times X &\longrightarrow& X,\\
&&&\\
& \left(\left<y_1 - a\right>,\left<y_1 - b\right>\right) &\longmapsto& \left<y_1 - a b\right>.\\
\ea$$
Lastly, we recall the example \ref{example_Heid_add_mul_grp_law} on pg. \pageref{example_Heid_add_mul_grp_law}, second part. Analogously to the example of Heiderich, we have for any $A \in \trm{CAlg}_{\currfield(y_1)}$,
$$\mathbb{G}_\cdot := \{ \varphi = [\lambda_0 + \lambda_1 w \longmapsto \lambda_0 + \lambda_1 (1 + a) w] : a \in N(A)\}$$
defines the group functor - assigning to each $A$ the subgroup of all automorphisms its group of infinitesimal transformation group $\Gamma(1,A)$.% We fix one $n \geq 2$ and set $A = \currfield(y_1)[\eps_n] \simeq \currfield(y_1)[X]/\left<X^n\right>$. Therefore,
%$$N(A) = \bigoplus_{i=1}^{n-1} \currfield(y_1).\eps_n^i$$
%and $\Gamma(1,\currfield(y_1)[\eps_n]) = \{w \mapsto w (1 + a) : a \in N(\currfield(y_1)[\eps_n])\}$.
%Next, we shall compute the algebras $\kappa$ and $\mathcal{K}$, or $\kappa \otimes A[[w]]$ and $\mathcal{K} \otimes A[[w]]$ respectively, for $A = K[\eps] \simeq K[X]/\left<X^2\right>$ and the univeral Taylor-morphism $\iota : K \longrightarrow K[[t]]$ and Umemura morphism $\theta_x : K \longrightarrow K[[w]]$
%$$\bao{rcl}
%\trm{im} \iota &=& \left\{\sum_{i \geq 0} \frac{1}{i!} \partial^i (a) t^i : a \in K\right\}\\
%&&\\
%&=& \left\{\sum_{i \geq 0} \frac{1}{i!} \partial^i \left(\frac{f}{g}\right) t^i : f, g \in \ov{\qz}[x_1,x_2], g \neq 0\right\}\\
%&&\\
%\trm{im} \iota\mid_k &:=& \left\{\sum_{i \geq 0} \frac{\partial^i(a)}{i!} t^i : a \in \ov{\qz}\right\}\\
%&&\\
%&=& \ov{\qz}\\
%&&\\
%\trm{im} \theta_x &=& \left\{\sum_{\alpha \in \nz_0^2} \frac{1}{\alpha!} \partial_x^\alpha(a) w^\alpha : a \in \ov{\qz}(x_1,x_2)\right\}\\
%&&\\
%\ea$$
%Hence, $\kappa = K$ and $\mathcal{K} = \left<\partial_x^\alpha(\iota(a)), b : a, b \in \ov{\qz}(x_1,x_2)\right>$. The extension $\theta_x[[t]] : K[[t]] \longrightarrow K[[t]] \otimes_K K[[w]]$ yields
%$$\bao{rcl}
%\trm{im} \theta_x[[t]]\mid_\kappa &=& \left\{\sum_{\alpha \in \nz_0^2} \frac{1}{\alpha!} \partial_x^\alpha(a) w^\alpha : a \in \ov{\qz}(x_1,x_2)\right\}\\
%&&\\
%\trm{im} \theta_x[[t]]\mid_{\mathcal{K}} &=& \left\{\sum_{\alpha \in \nz_0^2} \frac{1}{\alpha!} \partial_x^\alpha(a) w^\alpha : a \in \mathcal{K}\right\}\\
%&&\\
%&=& \left\{\sum_{(i,\alpha) \in \nz_0^3} \frac{1}{\alpha! i!} \partial_x^\alpha(\partial^i(a)) t^i \otimes w^\alpha : a \in \ov{\qz}(x_1,x_2)\right\}\\
%\ea$$
%Furthermore, let $\Phi = (\phi_i)_{i=1}^2 \in A[[w]]^2$ with $\phi_i \equiv w_i \mod N(A)[[w]] \simeq \left<1_{K[[t]]} \otimes \eps \right>$, i.e. $\Phi \in \Gamma(K[\eps],2)$. That means $\phi_i = \sum_\alpha a_{i,\alpha} w^\alpha \in \Gamma(K[\eps],2) \LRA a_{i,\alpha} \equiv 0 \mod N(A)$ for all $\alpha \neq e_i$ and $a_{i,e_i} \equiv 1 \mod N(A)$ and $1 \leq i \leq 2$:
%\commt{This is some stupid shit...}
\paragraph{General constructs from Umemura and Heiderich}
Next, we want to construt the differential subalgebras $\kappa$ and $\mathcal{K}$ in $K[[t]]$. As $k = \currfield$ we get that
$$\kappa := \currfield(y)\{\iota(\currfield)\}_{\partial_y} = \currfield(y).$$
This is obvious, as $\partial^i(\currfield) = \{0\}$ for all $i \geq 1$. We want to show that our PV-ring $R = \currfield[y,y^{-1}]$ is a differential subalgebra in the differential ring $(\currfield[[t]], \partial_t := \frac{d}{d t})$.
$$\bao{rcl}
y &=& \sum_{i \geq 0} y_i t^i,\ y_i \in \currfield\\
\partial(y) &=& \partial_t(y)\\
&=&\sum_{i \geq 0} (i + 1) y_{i + 1} t^{i} = \sqrt{a} y = \sqrt{a} \sum_{i\geq 0} y_i t^i\\
&\LRA&\\
y_{i} &=& \sqrt{a}\frac{y_{i-1}}{i}, \forall i \geq 1\\
&=& \sqrt{a}^i \frac{y_0}{i!}\\
\ea$$
Hence, we get that $y = \sum_{i \geq 0} \frac{(\sqrt{a} y_0 t)^i}{i!}$. Since $\partial_t(y) = \sqrt{a} y_0 y$, we have $y_0 = 1$ or in short:
$$y = \exp(\sqrt{a} t),$$
with $\exp$ as defined in Analysis. The inverse $y^{-1}$ is easily compute in the same fashion. Computing the image of $y$ under $\iota$:
$$
\iota(y) = \sum_{i \geq 0} \frac{\partial^i(y)}{i!} t^i = \sum_{i \geq 0} \sqrt{a}^i \frac{y}{i!} t^i
= y \exp(\sqrt{a} t),$$
and applying the iterative derivation 
$\theta_y(x) = \sum_{\alpha \in \nz_0^n} \frac{\partial_y^\alpha(x)}{\alpha!} w^\alpha$ to $\iota(y)$, we yield
$$\theta_y(\iota(y)) = \theta_y(y \exp(\sqrt{a}t )) = \sum_{\alpha \in \nz_0^1} \frac{\partial_y^\alpha(y \exp(\sqrt{a}t ))}{\alpha!} w^\alpha = y \exp(\sqrt{a}t ) + w \exp(\sqrt{a}t ) = (y + w)\exp(\sqrt{a}t ).$$
Umemura calls this the generalized solution of our differential equation in $\currfield(y)[[w]][[t]][t^{-1}]$ (\cite{Ume96b}, exp. 3.4.2). Now, let us pick $A = \currfield(y)[\eps] = \currfield(y)[X]/\left<X^2\right>$ then $$\trm{Ume}(\currfield(y)/\currfield)(A) = \{\phi \in \Gamma_{1 A} : \phi \equiv w \mod N(A)[[w]]\}$$
Therefore, $N(A)$ is $\currfield(y).\eps$ and we get either of the two possible (affine group) schemes $\mathbb{G}_a$ and $\mathbb{G}_m$ as $A$ point of $\trm{Ume}(\currfield(y)/\currfield)$ as described by Heiderich. Checking both:
$$\bao{rcl}
\varphi_a &=& [(y + w) \exp (\sqrt{a} t) \longmapsto (y + w + a) \exp (\sqrt{a} t)] \in \mathbb{G}_a,\ a \in \currfield(y).\eps\\
&&\\
\varphi_m &=& [(y + w) \exp (\sqrt{a} t) \longmapsto (y + w (1 + a)) \exp (\sqrt{a} t)] \in \mathbb{G}_m,\ a \in \currfield(y).\eps
\ea$$
However, as $w \longmapsto 0$ does not commute with the first type of $\currfield(y)$ automorphisms we get that clearly the multiplicative group scheme is the $A$ point.
\paragraph{Conclusion}
In stead of working in the algebraic closure $\currfield$, we could have just as easily worked in $\qz$ and $\qz(\sqrt{a})$. The only difference would be the prime ideals in the PV ring $R$ or $R(\sqrt{a}) \simeq \qz \otimes_\qz R$, respectively. Both cases would rely on the fact where the polynomial $X^2 - a \in k[X]$ is irreducible over $k$. In the first case $\sqrt{a} \notin k$, we get a two-dimensional $k$ solution space, in the latter a one-dimensional solutions space.
\section{Conclusion}
The aim of this paper was to introduce a Galois theory for general differential equations in characteristic zero. To accomplish this we had to introduce a large body of algebraic concepts.
\subsection{Summary}
We are giving a biref summary of the concepts introduced.
\subsubsection{Algebras and coalgebras}
The primer was clearly the introduction of algebras and a certain type of algebra extensions, the so called Ore-extensions. In summary, for a given algebra $A$ over a ring $R$, an algebra automorphism $\alpha \in \trm{Aut}_{R-\trm{alg}}(A)$ and an $\alpha$-derivation $\delta \in \trm{End}_R(A)$ which is
$$\delta(a b) = \alpha(a) \delta(b) + \delta(a) b,\ \forall a, b \in A$$
the Ore extension $A[X,\alpha,\delta]$ is an $A$-algebra, with
$$X a = \alpha(a) X - \delta(a)$$
for all $a \in A$. In addition, we provided the basic concepts of Lie algebras and their enveloping algebras. A Lie algebra is an $R$-module over a ring $R$ with a antisymmetric multiplication map $\mu$ fulfulling the Jacobi-identity. Their enveloping algebras are unital associative algebras in the above sense. Lie algebras are intimitely connected to derivations and differential rings.\\
Coalgebras a categorically dual to algebras, that is the dual module $C^*$ for any coalgebra $C$ is an algebra over the same ring. Furthermore, the module $\trm{Hom}_R(C,A)$ is an algebra as well, with convolution
$$\mu : \trm{Hom}_R(C,A) \otimes \trm{Hom}_R(C,A) \longrightarrow \trm{Hom}_R(C,A),\ f \otimes g \longmapsto \mu_A \circ (f \otimes g) \circ \Delta_C$$
as multiplication and $\eta_A \circ \eps_C$ as unit.
\subsubsection{Bialgebras, module algebras and Hopf algebras}
A bialgebra $(D,\mu_D,\eta_D,\Delta_D,\eps_D)$ has both, the structure of an algebra $(D,\mu_D,\eta_D)$ and that of a coalgebra $(D,\Delta_D,\eps_D)$, such that multiplication and unit map are homomorphisms of coalgebras and comultiplication and counit are homomorphisms of algebras.\\
For a bialgebra $D$, a $D$-module algebra $(A,\Psi_A)$ is an algebra $(A,\mu_A,\eta_A)$ such that
$$\Psi (a b) = \sum_{(d)} \mu_A(d_{(1)}(a) \otimes d_{(2)}(b)),\ \Psi_A(1_D \otimes a) = \eps_D(1_D) a$$
holds.\\
A Hopf algebra $D$ is a bialgebra, with a bialgebra antihomomorphism $S$:
$$S : D \longrightarrow D^{\trm{copop}}$$
such that $\eta \eps(d) = \mu(id_D \otimes S(\Delta(d))) = \mu(S \otimes id(\Delta(d)))$ holds for all $d \in D$.
\subsubsection{Differential modules and their constructs}
We introduced the concept of derivations over arbitrary modules over differential rings. Next, we introduced the ring of differential operators and showed that this ring is an Ore extension of the differential ring. This is followed by the defintion of the ring of differential polynmials. We showed that this ring is a module algebra over the ring of differential operators.\\
Furthermore, we discussed the basic Picard-Vessiot theory for linear differential equations in characteristic zero. We showed that any simple differential ring containing all solutions for suchs an equations is isomorphic to the PV ring. Moreover, algebraic elements over the field of constants are constant over the PV ring and any constant algebraic element over the PV ring is already algebraic over the field of constants. We defined the Galois group for this type of equations to be all $k^\partial$-linear bijections commuting with the derivation $\partial$. Concluding this section, we showed to simple examples over a non-trivial and a trivial differential ring:
$$L_1 = \partial - a, \ k = \currfield(z),\ L_2 = \partial^2 - a, \ k = \currfield,\ a \in \currfield^\times$$
in both cases. We computed the PV rings $R_1 = \currfield(z)[y,y^{-1}], \partial(y) = a y$ and $R_2 = \currfield[y,y^{-1}], \partial(y) = \sqrt{a} y$ and with Galois group $\currfield^\times$.
\subsubsection{General theory by Heiderich}
Next, we introduced the concept of iterative derivations and the universal Taylor homomorphism for differential ring extensions in characteristic zero. This was followed by the definition of the Umemura functor which assigns to each commutative algebra over a differential ring extension the group of algebra automorphisms leaving the image of the differential ring under the Taylor homomorphism fixed and making the following diagram commutative:
%diagram!
This led us the the definition of the Lie-Ritt functor assigning to each non-reduced algebra over $K$ the group of infinitesimal coordinate transformation. Heiderich proved that the Umemura functor is such a Lie-Ritt functor. Lastly, we compared the general theory with the PV theory and revisited our previous example $L_2$. There, we computed the Hopf algebra $H = R \otimes R^{\Psi_R \otimes \Psi_R}$ and saw that its set of prime ideals is indeed isomorphic to the spectrum of the ring of Laurent polynomials over $\currfield$.
\subsection{Outlook}
As we only computed a linear example over a trivial differential ring, it would be most interesting to extend this to non-trivial differential rings. Furthermore, the theory is general enough to deal with non-linear differential equations. Rather simple examples as $\partial(x) - x^2$ could be a starting point to further our understanding of this intriguing theory.\\
Secondly, the theory developed by Umemura and Heiderich does not involve non-commutative cases as the definition of the Lie-Ritt functors heavily depends on commutativity of the underlying algebras. An inverstigation  into this would seem promissing.
%We broadly introduced the concepts of module algebras and their Galois theory according to \cite{Heid10,Heid13}. This concept got applied to a rather simple example. In addition, we compared the classical Galois theory in the sense of Picard-Vessiot with the expanded theory.\\
%\indent To conclude, this theory provides alternative means not only to linear differential equations but rather to a wide range of similar problems, as iterative derivatives, difference equations and, of course, non-linear (ordinary or partial) differential equations.
\newpage
\section*{Acknowledgements}
First of all, I would like to thank professor Gro\ss{}e-Kl\"onne for offering me this challenging topic and providing invaluable hints, as well as doctor R\"uhling. Furthermore, I am grateful to Greta, Uwe, Anne and Anne W. for supporting me almost endlessly. In addition, Verena and Alex are not to be forgotten, last but not least Christian and Otis.
\newpage
\section{Appendix}\label{appendix}
%\subsection{Basic category theory}
%To generalize or abstract certain statements it is convenient to introduce the notion of categories. Here, we give a brief introduction for the sake of clarity following loosely \cite{Awo} and \cite{Borc}. % In short, a category is a class of sets $\trm{Obj}$ sharing a common mathematical structure. Maps for two objects in $\trm{Obj}$ are called morphism on $\trm{Obj}$ if they preserve the structure. More formally, 
% A category $\mathcal{C}$ conists of the class of objects $\trm{Obj}(\mathcal{C})$ and morphisms for two objects $A, B$ in $\trm{Obj}(\mathcal{C})$ which is denoted by $\mathcal{C}(A,B)$, $\mathcal{M_C}$ or $\trm{Morph}_{\mathcal{C}}(A,B)$. Moreover, if $A, B, C$ are three (not necessarily distinct) objects in $\mathcal{C}$, the composition of two morphisms $f: A \longrightarrow B$ and $g : B \longrightarrow C$ is denoted by $g \circ f : A \longrightarrow C$ and is a morphism in $\mathcal{C}(A,C)$. Lastly, the identity map $id : A \longrightarrow A$ is a morphism for all objects $A$ in $\trm{Obj}(\mathcal{C})$.%if $X$ and $Y$ are two objects in $\trm{Obj}$ %(with structure maps $\phi$, $\psi$ or tuples of structure maps), then a map $f : X \longrightarrow Y$ is called a morphism in the category of $\trm{Obj}$ if and only there is a map $\wt{f} : \im \phi \longrightarrow \im \psi$ such that the diagram commutes\index{Index}{category}\index{Index}{morphism}
%%$$\xymatrix{
%%X \ar[r]^f \ar[d]_\phi & Y \ar[d]^\psi\\
%%\trm{im} \phi \ar[r]_{\wt{f}} & \im \psi.\\
%%}$$
%%The class of 
%\bsp Some prominent examples
%\bn
%\item the category of sets with morphisms simply all maps between two sets.
%\index{Index}{category!of sets}
%\item the category of pointed spaces $\trm{PSpc}$, which can be considered as all non-empty sets with a designated element - the basepoint - and its morphisms are maps preserving the basepoints (the mandatory element in each object).
%\item the category of abelian groups $\trm{Abel}$, where morphisms are group homomorphisms,
%\index{Index}{category!of groups!abelian}
%\item the category of non-abelian groups $\trm{NAbel}$, where morphisms are also group homomorphisms, in case we do not know if a given group belongs to either of the two categories we simply assign it to $\trm{Grp}$,
%\index{Index}{category!of groups}
%\index{Index}{category!of groups!non-abelian}
%\item the category of rings $\trm{Rng}$, with ring homomorphisms as morphisms (note this is a proper subcategory of $\trm{Abel}$), with its prominent subcategory $\trm{CRng}$, the category of commutative rings and unital rings $\trm{URng}$.
%\index{Index}{category!of rings}
%\index{Index}{category!of rings!commutative}
%\index{Index}{category!of rings!unital}
%\item the category of differential manifolds, with smooth maps as morphisms,
%\index{Index}{category!of differential manifolds}
%\item the category of topological spaces $\trm{Top}$, with continuous maps as morphisms,
%\index{Index}{category!of topological spaces}
%\en
%We note, that in general there is no concept of union, products etc. - that is the union or product of to objects in the same category does not necessarily constitute an other object in same category, respectively. Therefore, we avoid talking about sets of certain mathematical objects (rather classes).
%\begin{defi}[Covariant and contravariant functors]
%Let $\mathcal{C}$ and $\mathcal{D}$ be two categories with morphisms $\mathcal{M_C}$ and $\mathcal{M_D}$, respectively. A (co/contravariant) functor $F$ is a pairing of $\mathcal{C}$ and $\mathcal{D}$ such that
%\bd
%\item[covariant]
%\bn
%\item for all $X \in \mathcal{C}$ the image $F(X)$ is an object in $\mathcal{D}$,
%\item for all morphisms $f : X \longrightarrow Y$, $X, Y \in \mathcal{C}$, the map $F(f) : F(X) \longrightarrow F(Y)$ is a morphism of $\mathcal{D}$
%\item $F(id_X) = id_{F(X)}$.
%\en
%\item[contravariant] as in covariant except:
%\bn
%\item for all morphisms $f : X \longrightarrow Y$, $X, Y \in \mathcal{C}$, the map $F(f) : F(Y) \longrightarrow F(X)$ is a morphism of $\mathcal{D}$, as well as
%\item $F(id_{F(X)}) = id_X$.
%\en
%\ed
%\index{Index}{functor!covariant}
%\index{Index}{functor!contravariant}
%\end{defi}
%\bsp Classical examples are
%\bn
%\item from main theorem of (classical Galois theory): let $L/K$ be a field extension
%$$\trm{Fix} : \trm{Grp} \longrightarrow \trm{CRng},\ H \longmapsto L^H := \{x \in L : g x = x\ \forall g \in H\}$$
%$$\trm{Gal} : \trm{CRng} \longrightarrow \trm{Grp},\ M \longmapsto \{\varphi \in \trm{Aut}_K(M) : \varphi\mid_K = id_K\},$$
%both define functors. Although, these functors are only defined on subcategories. In particular, the second functor is only well defined for normal separable fields $K \subset M \subset L$.
%\item in algebraic geometry, the (pre-) sheave defines a functor from the category of topolocial subspaces (of a variety - affine, projective, ...) to the category of associative algebras over a given ground field,
%\item furthermore, the so called forgetful functor: let $\mathcal{C}$ be a subcategory of $\mathcal{D}$, then $F : \mathcal{C} \longrightarrow \mathcal{D}$ is called the forgetful functor (informally, we simply omit some of the structure maps from $\mathcal{C}$), e.g
%$$F : \trm{Grp} \longrightarrow \trm{Set},$$
%\item the set of all automorphisms $\trm{Aut}$ on a given algebraic category (e.g. $\trm{Vec}$, $\trm{Grp}$, $\trm{Rng}$,...) is also a functor - assigning to an object $X$ the set of all structure preserving bijections $(X,\phi) \longrightarrow (X,\phi)$, i.e.
%$$\trm{Aut} : \trm{AlgCat} \longrightarrow \trm{Grp}.$$
%More generally, any structure preserving bijection (i.e. each map having a two-sided inverse which still preserves $\phi$) defines the $\trm{Aut}$ functor on their respective category.
%\en
%\begin{defi}
%For a given category $\mathcal{C}$ we construct a new category $\mathcal{C}^{\trm{op}}$ by simply reversing arrows and composition order of morphisms:
%$$f, g \in \mathcal{M_C},\ g \circ f \in \mathcal{M_C} \RA f^{\trm{op}}, g^{\trm{op}} \in \mathcal{M}_{\mathcal{C}^{\trm{op}}},\  (g \circ f)^{\trm{op}} := f^{\trm{op}} \circ g^{\trm{op}} \in \mathcal{M}_{\mathcal{C}^{\trm{op}}}.$$
%We call $\mathcal{C}^{\trm{op}}$ the opposite or dual category of $\mathcal{C}$.
%\index{Index}{category!dual}
%\end{defi}
%We already encountered the opposite category in case of $C$-coalgebras and $C$-algebras (i.e. coalgebras are dual to the algebras).
%\begin{defi}
%We call a category small if it defines an actual set. Otherwise it is called a large category. In addition, we call a category $\mathcal{C}$ a category with initial object, if there is an object $I$ in $\mathcal{C}$ such that for all $X$ in $\mathcal{C}$ there is exactly one morphism $\iota : I \longrightarrow X$. The dual notion is the category with terminal object: if there is an object $T$ in $\mathcal{C}$ such that for every object $X$ in $\mathcal{C}$ there is one morphism $\tau : X \longrightarrow T$.
%\index{Index}{category!small}
%\index{Index}{category!large}
%\index{Index}{category!with initial object}
%\index{Index}{category!with terminal object}
%\end{defi}
%\bsp To illustrate the last definitions:
%\bn
%\item $\trm{Set}$ is a large category.
%\item Usually the class of morphisms for a given category $\mathcal{C}$ is also large. However, some counter examples are for instance module homomorphisms.
%\item The category of unital rings is a category with initial object $(\zz,+,\cdot,1)$.
%\item Dually, the category of schemes over unital rings is a category with terminal object (by duality: $\trm{Spec}(\zz)$).
%\en
%\begin{defi}
%A natural transformation $\zeta$ for two given categories $\mathcal{C}$, $\mathcal{D}$ and two functors $F, G$ between the two categories is a family of morphisms such that
%\bn
%\item it assigns to each object $X \in \mathcal{C}$ a morphism $\zeta_X : F(X) \longrightarrow G(X)$ and
%\item for each morphism $f : X \longrightarrow Y$ in $\mathcal{C}$ we have
%$$\bao{cc}
%\xymatrix{
%F(X) \ar[r]^{F(f)} \ar[d]_{\zeta_X} & F(Y)\ar[d]^{\zeta_Y}\\
%G(X) \ar[r]_{G(f)} & G(Y)\\
%} & 
%\xymatrix{
%F(Y) \ar[r]^{F(f)} \ar[d]_{\zeta_Y} & F(X)\ar[d]^{\zeta_X}\\
%G(Y) \ar[r]_{G(f)} & G(X)\\
%}\\
%\trm{covariant} & \trm{contravariant},\\
%\ea$$
%where covariant stands for covariant functors $F, G$ and contravariant stands for the other case.
%\index{Index}{natural transformation}
%\en
%\end{defi}
%Let $\mathcal{C}$ be a category. For two objects $A, B$ in $\mathcal{C}$ we denote the class of morphisms $\mathcal{M}_{\mathcal{C}}(A,B)$ simply by $\trm{Hom}_{\mathcal{C}}(A,B)$ or $\trm{Hom}(A,B)$ if there is no ambiguity. Every object $A \in \mathcal{C}$ defines a functor $F_A := \trm{Hom}(A,\_) : \mathcal{C} \longrightarrow \trm{Set}, B \longmapsto \trm{Hom}(A,B)$ such that for all maps $\varphi : B \longrightarrow B'$:
%$$F_A(\varphi)(f) := \varphi \circ f,\ \forall f \in F_A(B).$$
%Each morphism $\varphi : A' \longrightarrow A$ in $\mathcal{C}$ defines a map $f \longmapsto f \circ \varphi : F_A(B) \longrightarrow F_{A'}(B)$ being natural wrt $B$, i.e. this map is a natural transformation. In particular, the pairing $A \longmapsto F_A$ is a contravariant functor.
%\begin{defi}
%A functor $F : \mathcal{C} \longrightarrow \trm{Set}$ is called representable if it is isomorphic to $F_A$ for some $A \in \mathcal{C}$.
%\index{Index}{functor!representable}
%\end{defi}
%\subsubsection{Direct products and coproducts}
% %From now on, let $\mathcal{C}$ is a category with finite products.
%Given a category $\mathcal{C}$ and an index set $I$ we define
%\begin{defi}[direct product]
%for a family of objects $\{X_i : i \in I\}$ in $\mathcal{C}$ the direct product $X$ to be the object in $\mathcal{C}$ such that for each canonical projection $\pi_i : X \longrightarrow X_i$, $i \in I$ and an indexed family of morphisms $f_i : Y \longrightarrow X_i$ for all $i \in I$ and $Y \in \mathcal{C}$, there is a unique morphism $f : Y \longrightarrow X$ making
%$$\xymatrix{
%Y \ar[rd]_f\ar[r]^{f_i}&X_i\\
%&X\ar[u]_{\pi_i}\\
%}$$
%commutative. Sometimes the direct product is denoted by $\prod_{i \in I} X_i$.
%\index{Index}{product!direct}
%\end{defi}
%The coproduct is simply:
%\begin{defi}[coproduct]
%For a family of objects $\{X_i : i \in I\}$ in $\mathcal{C}$ the coproduct $X$ to be the object in $\mathcal{C}$ such that for each (not necessarily injective) inclusion $\iota_i : X_i \longrightarrow X$ and a family of morphisms $f_i : X_i \longrightarrow Y$ for all $i \in I$ and $Y \in \mathcal{C}$ there exists a unique $f : X \longrightarrow Y$ making
%$$\xymatrix{
%Y &X_i\ar[l]_{f_i}\ar[d]^{\iota_i}\\
%&X\ar[lu]^f\\
%}$$
%commutative. The coproduct $X$ is sometimes denoted by $\coprod_{i \in I} X_i$ or $\bigoplus_{i \in I} X_i$.
%\index{Index}{coproduct}
%\end{defi}
%\bmk For a finite index set $I$ both notions are equivalent. Consider for instance for any ring $R$ the left module $\prod_{i \leq n} R$ and $\bigoplus_{i \leq n} R$. However, if $I$ is not finite, then the coproduct is a strict subset of the direct product. Furthermore, the products and coproducts are only unique up to isomorphism (within their respective category).
%\begin{defi}
%$\mathcal{C}$ is called a category with finite products, if for any finite subcategory of $\mathcal{C}$ its coproduct is in $\mathcal{C}$ and there exists a finite object in $\mathcal{C}$, denoted by $*$ - called the empty product - and an isomorphism:
%$$S \times * \simeq S \simeq * \times S$$
%for all $S \in \mathcal{C}$.
%\index{Index}{category!with finite product}
%\index{Index}{category!empty object}
%\end{defi}
%$\trm{PSpc}$, the category of pointed spaces is an example of a category with finite products with one element sets as empty products.
%\subsubsection{Co-/limits, formal schemes and group schemes}
%
%To complete our defintions we need:
%\begin{defi} Let $\mathcal{C}$ be a category.
%\bn
%\item A diagram of type $\mathcal{J}$ is a functor $F : \mathcal{J} \longrightarrow \mathcal{C}$, where $\mathcal{J}$ is an index category and $F$ indexes objects and morphisms in $\mathcal{C}$. A diagram $F$ of type $\mathcal{J}$ is called small or finite if $\mathcal{J}$ is a small or finite category.
%\item Let $F : \mathcal{J} \longrightarrow \mathcal{C}$ be a diagram of type $\mathcal{J}$. A cone to $F$ is an object $N$ in $\mathcal{C}$ and a family of morphisms $\psi_X : N \longrightarrow F(X)$ indexed by $X$ in $\mathcal{J}$ such that for all morphisms $f : X \longrightarrow Y$ in $\mathcal{J}$ we get
%$$F(f) \circ \psi_X = \psi_Y.$$
%A cone is denoted by $(N,\psi)$.
%\item A co-cone is dual to cone: co-cone of a diagram $F$ is an object $N$ in $\mathcal{C}$ and family of morphisms $\psi_X : F(X) \longrightarrow N$ for every $X$ in $\mathcal{J}$ such that for any morphism $f : X \longrightarrow Y$ in $\mathcal{J}$ we have: $\psi_X \circ F(f) = \psi_Y$.
%The pair $(N,\phi)$ denotes the co-cone.
%\item A limit of a diagram of type $\mathcal{J}$ is a cone $(L,\phi)$ of $F$ such that for any other cone $(N,\psi)$ of $F$ there exists a unique morphism $u : N \longrightarrow L$ such that following diagram commutes:
%$$\xymatrix{
%& N \ar[ldd]_{\psi_X} \ar[d]^u \ar[rdd]^{\psi_Y}&\\
%&L\ar[ld]^{\phi_X} \ar[rd]_{\phi_Y}&\\
%F(X) \ar[rr]_{F(f)} & & F(Y)\\
%}$$
%\item A colimit of a diagram $F$ is a co-cone $(L,\phi)$ of $F$ such that for any other co-cone $(N,\psi)$ of $F$ there is a unique morphism $u : N \longrightarrow L$ such that the following diagram commutes:
%$$\xymatrix{
%F(X) \ar[rd]^{\phi_X}\ar[rdd]_{\psi_X}\ar[rr]^{F(f)} && F(Y)\ar[ld]_{\phi_Y}\ar[ldd]^{\psi_Y}\\
%&L\ar[d]_u&\\
%&N&\\
%}$$
%\en
%\index{Index}{diagram}
%\index{Index}{cone}
%\index{Index}{cocone}
%\index{Index}{limit}
%\index{Index}{colimit}
%\end{defi}
%\bmk A cone $(N,\psi)$ of a diagram $F : \mathcal{J} \longmapsto \mathcal{C}$ is characterized by the following commutative diagram:
%$$\xymatrix{
% & F(X) \ar[dd]^{F(f)} & X \ar[l]_F \ar[dd]^f\\ 
%N \ar[ru]^{\psi_X} \ar[rd]_{\psi_Y}& &\\
%& F(Y) & Y\ar[l]^F.\\
%}$$
%A co-cone $(N,\phi)$ of a diagram $F : \mathcal{J} \longrightarrow \mathcal{C}$ is characterized by the following commutative diagram:
%$$\xymatrix{
% & F(X) \ar[ld]_{\phi_X}\ar[dd]^{F(f)} & X \ar[l]_F \ar[dd]^f\\ 
%N & &\\
%& F(Y) \ar[lu]^{\phi_Y}& Y\ar[l]^F.\\
%}$$
%\subsubsection{Monoidal and group category}
%Let $\mathcal{C}$ is a category with finite products.
%\begin{defi}
%A monoid $G$ in $\mathcal{C}$ is a triple $(G, m, e)$, with operation morphism $m : G \times G \longrightarrow G$ and unit morphism $e : * \longrightarrow G$ satisfying the following commutative diagrams:
%\bd
%\item[associativity] $$\xymatrix{
%G \times G \times G \ar[rr]^{id_G \times m}\ar[d]_{m \times id_G} && G\times G\ar[d]^m\\
%G \times G \ar[rr]_m && G,\\
%}$$
%\item[unit] $$\xymatrix{
% & G \ar[ld]_{\simeq} \ar[rd]^{\simeq} & \\
% \ast \times G\ar[d]_{e \times id_G} & & G \times \ast \ar[d]^{id_G \times e}\\
% G \times G \ar[rd]_{m} & & G \times G \ar[ld]^m\\
% & G &\\
%}$$
%\ed
%We denote by $\trm{Mon}$ the monoidal category.
%\index{Index}{category!monoidal}
%\end{defi}
%\bmk Compare the two diagrams with the definition of associative unital algebras over some ring $R$ (replacing direct products with tensor products, then here the empty product is indeed $* = R$ and unit $\eta = e$).\\
%For each monoid $G$ we define the opposite monoid $G^{\trm{op}}$ being the same set with operation $(g, h) \longmapsto h g$.
%\begin{defi}
%A group $G$ in $\mathcal{C}$ is a quadruple $(G, m, e, S)$ being a monoid in $\mathcal{C}$ and $S : G \longrightarrow G$ defines a commuting diagram:
%$$\xymatrix{
%&G\ar[ld]_\Delta\ar[rd]^\Delta&\\
%G \times G \ar[rd]^{S \times id}\ar[dd]_{\pi \times \pi} && G \times G\ar[ld]_{id \times S}\ar[dd]^{\pi \times \pi}\\
%& G \times G \ar[d]_m &\\
%\ast \ar[r]_e & G & \ast\ar[l]^e,\\
%}$$
%where $\Delta : G \longrightarrow G \times G$ is the diagonal map, $f_1 \times f_2 = \left[(g,h) \longmapsto \left(f_1(g),f_2(h)\right)\right]$ and $\pi : G \longrightarrow \ast \simeq G/G$ is the trivial projection, such that
%$$S : G \longrightarrow G^{\trm{op}}$$
%is a group homomorphism.
%\index{Index}{category!of groups}
%\end{defi}
%\subsubsection{Schemes}
%Now, we introduce some important notations in the field of algebraic geometry. Let $\trm{UCRng}$ denote the category of unital commutative rings, $\trm{Mod}_R$ the category of $R$-modules over $R$ in $\trm{UCRng}$ and $\trm{Mod}_R \cap R$ the subcategory of $R$-submodules in $R$ (i.e. ideals). We have
%\begin{defi}[Spectrum of ring]
%The functor
%$$\trm{Spec} : \trm{UCRng} \longrightarrow \trm{Set},\ R \longmapsto \{\mathfrak{p} \in \trm{Mod}_R \cap R :  \trm{Ann}(R/\mathfrak{p}) = 0\}$$
%assigns to each commutative unital ring $R$ its spectrum, i.e. the set of its prime ideals.
%\index{Index}{spectrum}
%\end{defi}
%\bsp We have:
%\bn
%\item $\spec \zz = \left\{(p) : p \trm{~prime~number}\right\} \cup \{0\}$,
%\item for $n \geq 2$, $\spec \zz_n = \left\{(\ov{p}) : p \mid n \wedge p \trm{~prime}\right\} \cup \{\ov{0}\}$, in particular the first subset may be empty if $n$ is prime,
%\item $\spec \zz_2[X]$:
%$$\left\{\left<f = \sum_{i \leq n} f_i X^i\right> :\exists m \in \nz,\ X^m \mid (f + \ov{1}) \wedge |\{f_i : f_i \neq \ov{0}\}| \in 2 \nz + 1\right\} \cup \left\{\left<\ov{0}\right>, \left<X\right>, \left<X + \ov{1}\right>\right\},$$ in words:
%all polynomials $f$ with odd non-zero coefficients and $f \equiv 1 \mod X^m$ for some $m \geq 1$, zero and the two polynomials of degree one.
%\item in general, for any field $k$ we have
%$$\spec k[X] = \{(f) : f \trm{~irreducible}\} \cup \{0\},$$
%in case of algebraic closeness, $\spec k[X] = \{\left<X - a\right> : a \in k\} \cup \{\left<0\right>\}$.
%\item $R$ integral domain, iff $\{0\} \in \spec R$,
%\item $R$ a field, iff $\{0\} = \spec R$.
%\en
%Note, that the non-unitary ring $R := \left(\bao{cc} 0 & k\\0 & 0\\\ea\right)$ for some field $k$ has no prime ideal (but a maximal ideal (!)). Thus, unitality ensures the existence of prime ideals. % However, let us recall some further definitions.
%%\begin{defi}
%%Let $R = k$ be a algebraically closed field and identify the affine space $A_k^n$ with the coordinate space.
%%\bn
%%\item For some $S \subset k[X]$, the set $Z(S) = \{x \in k^n : f(x) = 0 \forall f \in S\}$ is called the zero set or algebraic set of $S$.
%%\item For each $Z \subset k^n$ an algebraic set, we have the ideal $I(Z) := \{f \in k[X] : f(x) = 0 \forall x \in Z\}$ and call $A(Z) := k[X]/I(Z)$ the coordinate ring of $Z$.
%%\item If $A(Z)$ is an integral domain, we call $Z$ an affine variety.
%%\item For $k^{n+1}$ we define projective space $\prjn_k$ the over $k$ as the quotient
%%$$\prjn_k := \left(k^{n+1} \bsl \{0\}\right)^2 /\sim,\ \sim := \left\{(x,y) \in \left(k^{n+1} \bsl \{0\}\right)^2 : \exists \lambda \in k^\times, y = \lambda x\right\}.$$
%%\item A polynomial is called homogeneous of degree $n$, if we have
%%$$f = \sum_{|\alpha| = n} f_\alpha X_\alpha, f_\alpha \in k, X_\alpha = \prod_{i \leq k} X^{s_i}_{\alpha_i}\ \trm{and}\ \sum s_i = n.$$
%%An ideal is called homogeneous, if it is generated by homogeneous polynomials and a projective algebraic set is simply the zero set of a homogeneous polynomial $f$, i.e. $Z(f) := \{x \in \prjn_k : f(x) = 0\}$. A projective variety is simply the zero set of homogeneous prime ideal.
%%\item Let $Z \subset k^n$ be an affine variety and $f \in k[X]$ some polynomial such that $Z(f) \nsubset Z$. We call $U_f := Z\bsl Z(f)$ the principal open sets of $Z$.
%%\item For some affine variety $Z$, the structure presheaf is defined as the functor $\trm{Top} \longrightarrow \trm{Alg}$ with
%%$$\bao{rrcl}
%%\mathcal{O}_Z :& \{U : U \subset Z\trm{~open}\} &\longrightarrow &\mathcal{O}_Z(U) := S^{-1}A(Z),\\
%%&&&\\
%%&S &:=& \{f \in A(Z) : f(x) \neq 0 \forall x \in U\}\\
%%\ea$$
%%assigning to every open set $U$ the ring of regular functions on $U$ with the following condition:
%%\bn
%%\item $\mathcal{O}_Z(\emptyset) = \{0\}$,
%%\item $\mathcal{O}_Z\mid_{V}(U) = \mathcal{O}_Z(V)$ for all $V \subset U$ open,
%%\item $\mathcal{O}_Z\mid_W \circ \mathcal{O}_Z\mid_V = \mathcal{O}_W$ for all $W \subset V \subset U$ open.
%%\en
%%\item We call a (structure) presheaf of an affine variety $X$ a sheaf, if there is the following glueing property. Let $\mathcal{U}$ be an open cover of some open $U \subset X$. For all $V, W \in \mathcal{U}$ and $f_V \in \mathcal{O}_X(V), f_W \in \mathcal{O}_X(W)$ such that $f_V\mid_{V \cap W} = f_W\mid_{V \cap W}$ then there exists a unique $f \in \mathcal{O}_X(U)$ such that $f\mid_V = f_V$ and $f\mid_W = f_W$.
%%\item A ringed space for some topological space $X$ consists of the pair $(X, \mathcal{O}_X)$, where $\mathcal{O}_X$ is the structure sheaf of $X$.
%%\en
%%\end{defi}
%%First, we note that our definition of algebraic sets induces a topology on $k^n$, the so called Zariski-topology by simply putting our algebraic sets as closed subsets of $k^n$. In particular, any affine variety defines an irreducible topological space. The definition of projective sheafs is omitted but the interested reader may consult \cite{Hart}\\
%%Furthermore, we may reformulate affine varieties as follows:
%We are omitting the definitions of presheafs, sheafs and ringed spaces and refer the interested reader to \cite{Hart}.
%%\begin{defi}
%%Let $(X, \mathcal{O}_X)$ be a ringed space. We call $X$ a scheme, if for any open cover $\mathcal{U}$ of $X$ and $U \in \mathcal{U}$ the ringed space $(U, \mathcal{O}_X\mid_{U})$ is isomorphic to some affine scheme.
%%\end{defi}
%\begin{defi}
%For some ideal $I \subset R$, the affine scheme $X_I$ is the set of all prime ideals $\mathfrak{p} \supset I$. 
%%$$X_I :=  \{\mathfrak{p} \in \tmr{Spec}(R) : a \notin \mthfrak{p}\}$
%A principal open set $U(a)$, for some $a \in R\bsl R^\times$, is $\{\mathfrak{p} \in \trm{Spec} R : a \notin \mathfrak{p}\}$. A scheme $X$ is simply some ringed space $(X,\mathcal{O}_X)$ such that for every open cover $\mathcal{U}$ of $X$ the restrictions $\mathcal{O}_X\mid_U$ as ringed spaces $(U, \mathcal{O}_X\mid_U)$ define affine schemes for all $U \in \mathcal{U}$.
%\index{Index}{scheme}
%\index{Index}{scheme!affine}
%\index{Index}{Set!principal open}
%\end{defi}
%In short, a scheme is a (topological) space with structure sheaf $\mathcal{O}_X$ that is locally isomorphic to some affine variety.
%\bsp For any ring $R$ in $\trm{UCRng}$:
%$$S^{-1}_{\mathfrak{p}}(R),\ S_{\mathfrak{p}} = R \bsl \mathfrak{p},$$
%the pair $(\trm{Spec}(R), S^{-1})$ is a scheme.% Here in obuse of notation, we use the multiplative system $S_{\mathfrak{p}}$ as a functor $\trm{Set} \longmapsto \trm{Mon}$.
%\begin{lemm}
%The following statements are equivalent:
%\bn
%\item the ringed space $(X, \mathcal{O}_X)$ is an affine variety,
%\item and:
%\bn
%\item $X$ is an irreducible topological space,
%\item $\mathcal{O}_X$ is a structure sheaf,
%\item $X$ is isomorphic to an affine variety.
%\en
%\en
%\end{lemm}
%A proof can be found in \cite{Hart}.
%\begin{defi}
%A formal scheme is a functor $X : \trm{CRng} \longrightarrow \trm{Set}$, that is a small filtered colimit of affine schemes. Its category is denoted by $\trm{FSch}$ - its morphisms are natural transformations. Given a scheme $S$, we define the formal schemes over $S$ as follows, all objects are morphisms $X \longrightarrow S$ of formal schemes and as morphisms between $X \longrightarrow S$ and $Y \longrightarrow S$ all morphisms $X \longrightarrow Y$, such that
%$$\xymatrix{
%X \ar[r] \ar[rd]& Y\ar[d]\\
%&S\\
%}$$
%commutes. We denote the formal schemes over $S$ with $\trm{FSch}_S$.
%\index{Index}{scheme!formal scheme}
%\end{defi}
%\bmk According to \cite{Strickl} a formal scheme to $\mathcal{X}$ is as follows: given a small filtered category $\mathcal{J}$ and a functor $i \longmapsto X_{i}$ from $\mathcal{J}$ to $\mathcal{X} = \{\trm{CRng}, \trm{Set}\}$ such that
%$$X = \lim_{\substack{\longrightarrow\\i}} X_{i} \in \mathcal{X}$$
%or equivalently $X(R) = \lim_{\substack{\longrightarrow\\i}} X_{i}(R)$ for all $R$.
%\bsp Two examples to illustrate the definition of formal schemes (following \cite{Strickl}):
%\bn
%\item Let $R$ be an object in $\trm{CRng}$ with unit and $N(R)$ denote its nilradical. The functors
%$$\hat{\mathbb{A}}_n = \left[R \longmapsto N(R)^n\right]$$
%are a prominent example.
%\item Let $X$ be some scheme and $Y = V(I)$ a closed subscheme then
%$$X_{\hat{Y}} := \lim_{\substack{\longrightarrow\\N}} V(I^N)$$
%defines a formal scheme.
%\en
%%An scheme is simply a ringed space $(Z,\mathcal{O}_Z)$ such that for ... open cover $\mathcal{U}$ the restriction $\mathcal{O}_Z\mid_U$ is an affine scheme for all $U \in \mathcal{U}$.
%\begin{defi}
%A group scheme is a scheme $X = X(G)$ which has a group structure $G$ as well, i.e. a functor $m : X \times X \longrightarrow X$. A principal homogeneous space - or torsor - for a given group (group scheme) $G$ is a pair $(G,X)$, with $X$ some set, such that the map
%$$\alpha : (X, G) \longrightarrow (X, X),\ (x, g) \longmapsto (x, g x)$$
%is a bijection.
%\index{Index}{group scheme}
%\index{Index}{space!principal homogeneous}
%\index{Index}{torsor}
%\end{defi}
%\begin{koro}
%The following statements are equivalent.
%\bn
%\item $(G, X)$ is a principle homogeneous space.
%\item The action $\alpha : G \times X \longrightarrow X$ is transitive on $X$ and its stabilizer $\trm{Stab}_G(X)$ is trivial.
%\en
%\end{koro} \bws The proof is simply the application of the above definitions.
%
%%Again, our notation, not incidentally, resembles the notation of Hopf-algebras. Now an important classification of affine groups:
%\begin{defi}
%An affine group is a representable functor $G : \trm{Alg} \longrightarrow \trm{Set}$ with natural transformation $\mu : G \times G \longrightarrow G$ such that $(G(A),\mu(A),e(A))$ is a group for all $A \in \trm{Alg}$. $G$ is called affine algebraic group if $G$ is represented by a finite presented algebra $A$.
%\index{Index}{affine group}
%\index{Index}{affine algebraic group}
%\end{defi}
%Formal group laws will be discussed in a later section.
%\newpage
\subsection{Topological basics}
We take some notions from topology as given.
\subsubsection{Basis and neighborhood basis}
\begin{defi}
Let $(X, \tau)$ be a topological space.
\bd
%\item[Filter] A subset $F \subset \tau$ is called a filter if:
%\bn
%\item for all $A, B \in F$, we have $A \cap B \in F$,
%\item the empty set $\emptyset$ is not in $F$,
%\item if $A \in F$ and $A \subset B$, then $B \in F$ for all $B \subset X$.
%\en
\item[Basis] A subset $\beta \subset \tau$ is called an open neighborhood basis for $x \in X$ if:% ein topologischer Raum. Eine Teilmenge $\beta \subset \tau$ hei\ss{}t Basis von $x \in X$, oder auch Umgebungsbasis von $x$, falls
\bn
\item for every open $V \subset X$ with $x \in V$ there is an $U \in \beta$ open in $X$ such that $x \in U \subset V$.%f\"ur alle offene $V \subset X$ offen, mit $x \in V$ gibt es ein $U \in \beta$ mit $x \in U \subset V$ offen,
\item for all $U, V \in \beta$ we have $x \in U \cap V$.%f\"ur $U_1, U_2 \in \beta$ ist $x \in U_1 \cap U_2$.
\en
A subset $\beta \subset \tau$ is called (open) basis of $X$ if
%Eine Teilmenge $\beta \subset \tau$ hei\ss{}t Basis von $X$, falls
\bn
\item all $U \in \beta$ are open in $X$,
\item every open subset of $X$ is a union of elements in $\beta$.%jede offene Teilmenge $V \in \tau$ eine Vereinigung von $U \in \beta$ ist.
\en
\ed
A neighborhood basis $\beta$ of $x \in X$ is called a fundamental basis of $x \in X$ if every neighborhood $x \in U$ is a finite intersection of elements in $\beta$.%Eine Umgebungsbasis $\beta$ von $x \in X$ eines topologischen Raums $(X, \tau)$ hei\ss{}t  Fundamentalbasis, falls jede offene Umgebung $V$ von $x$ einen endlichen Durchschnitt von $U_i \in \beta$ enth\"alt.
\index{Index}{filter}
\index{Index}{basis}
\index{Index}{basis!neighborhood}
\index{Index}{basis!fundamental}
\end{defi}
\subsubsection{Linear topological rings}
Topological rings $R$ are topological spaces $(R,\tau_R)$ such that addition and multiplication are continuous wrt. the product topology:
$$+, \cdot \in C(R\times R,R).$$%Topologische Ringe sind Ringe $R$, mit einer Topologie $\tau$, sodass
%$$+ : R \times R \longrightarrow R,\ \cdot : R \times R \longrightarrow R$$
%stetige Abbildungen in der Produkttopologie $\tau_{R \times R}$ sind. Die Menge der Ideale in $R$ definiert eine Umgebungsbasis der $0 \in R$. %Im Allgemeinen ist diese Basis abgeschlossen in $R$, 
\index{Index}{topological ring}
The set of ideals in $R$ is neighborhood basis of $0 \in R$. In general, the elements generated by union are not ideals but are contained in larger ideals (in general $I \cup J$ is not an ideal but is contained in its sum $I + J$).
\begin{defi}[Linear topological rings]
A topological ring $R$ is called linear if there is a fundamental neighborhood basis $\beta$ of $0 \in R$.
%Ein topologischer Ring $R$ hei\ss{}t linear, falls es eine fundamentale Umgebungsbasis von $0 \in R$ gibt.
\index{Index}{topological ring!linear}
\end{defi}
Is $R$ a linear topological ring with fundamental neigborhood basis $\beta(0)$ then any open neighborhood of zero contains at least one ideal, trivially $(0)$, since the intersection of ideals is again an ideal. Let $\{I_i \in \eta(0) : i \in \mathcal{I}\}$ be a system of ideals then the union:
$$\bigcup_{i \in \mathcal{I}'} I_i,\ \forall \mathcal{I}' \subset \mathcal{I},\ |\mathcal{I}'| < \infty$$
is a subset of $R$ containing each $I_i$. Consequently, $\beta(0)$ is an open neighborhood basis for zero. In addition, elements in $\beta(0)$ are also intersection stable. Hence, we get a clopen basis.
%Ist nun $R$ ein linear topologischer Ring, mit fundamentaler Umgebungsbasis $\beta(0)$, dann enth\"alt jede offene Umgebung der Null mindestens ein Ideal $I \in \beta(0)$, trivialerweise mindestens $(0)$, da der Schnitt beliebiger Ideale wieder ein Ideal ergibt. Sei $\{I_i \in \beta(0) : i \in \mathcal{I}\}$ ein System von Idealen, dann ist deren Vereinigung eine Teilmenge von $R$, die alle Ideale der Form
%$$\bigcap_{i \in \mathcal{I}'} I_i, \forall \mathcal{I}' \subset \mathcal{I}\ \trm{und}\ |\mathcal{I}'| < \infty$$
%enth\"alt und definiert damit eine offene Umgebung der Null. Andererseits sind die Schnitte der Ideale auch Umgebungen der Null, d.h. damit sind alle Ideale \textit{clopen} in $R$.
\begin{defi}
Let $R$ be a linear topological ring with fundamental neighborhood basis of zero: $\beta(0)$. $\hat{R}$ is called the completion of $R$, if
%Sei $R$ ein linear topologischer Ring mit Fundamental-UB $\beta(0)$. Ein Ring $\hat{R}$ hei\ss{}t Vervollst\"andigung von $R$, falls
$$\hat{R} \simeq \lim_{\substack{\longleftarrow\\I \in \beta(0)}} R/I,$$
i.e. the profinite limit of $(R/I)_{I \in \beta(0)}$, with ring homomorphisms $R/I \longrightarrow R/J$ for all $I \subset J$ and $\beta(0)$ ordered by wrt. inclusion.
%ist, d.h. der pro-endliche (oder inverse) Limes von  $(R/I)_{I \in \beta(0)}$, mit Ringmorphismen $R/I \longrightarrow R/J$ f\"ur alle $I \subset J$ und $\beta(0)$ angeordnet bzgl. Inklusion.
\index{Index}{topological ring!completion of}
\end{defi}
\bmk In terms of co/limits, the completion of a linear topological ring $R$ with neighborhood basis of zero is simply the limit.
\bsp Consider $(0) \neq \idealp = (p) \subset \zz$ and $S := \prod_{n \geq 1} \zz/\idealp^n$ with component-wise ring operations and inclusion map:
$$\zz \longrightarrow S, z \longmapsto (z \mod p^n)_{n\geq1}.$$
Furthermore, there are projections $\pi_{ij} : \zz/\idealp^i \longrightarrow \zz/\idealp^j, x \mod p^i \longmapsto x \mod p^j$ for all $i \geq j$. Clearly, $\beta(0) := \{\idealp^n : n \geq 1\}$ defines a neighborhood system of zero (all intersections contain the zero ideal) and via inclusion a partial order on $\beta(0)$ (in this cases total).
%\subsection{Commutative algebra}
%
%\subsubsection{Invariant and equivariant rings}
%Given a finite family of polynomials $\mathcal{F} \subset k[x]$, its invariant group is defined to be
%the set of all $k$-linear maps $\varphi : k[x] \longrightarrow k[x]$ such that $f \circ \varphi = f$ for all $f \in \mathcal{F}$. Conversely:
%\begin{defi}
%the invariant ring of a given group $G$ is the set of all polynomials in $k[x]$ such that $f g = f$ for all $g \in G$. This ring is denoted by
%$$k[x]^G.$$
%\end{defi}
%Since the invariant group only acts on the monomials of degree $\geq 1$ we have $g\mid_{k.1_{k[x]}} = id_{k.1_{k[x]}}$. For $n \in \nz$ the most prominent example is the ring of symmetric polynomials:
%$$k[x]^{S_n} = k[s_1,\ldots,s_n],$$
%with $s_i = \sum_{\substack{\alpha \in \nz_0^n\\|\alpha| = i}} x^\alpha$ the symmetric polynomials. They are of utmost importance in classical Galois theory.
%\begin{defi}
%For a given finite family of polynomials $\mathcal{F}$ we call the set of all $k$-linear maps $\varphi$ commuting with all $f \in \mathcal{F}$ the equivariant group of $\mathcal{F}$. Conversely, for a given group $G$ the set of all polynomials in $k[x]$ commuting with all $g \in G$ is called the equivariant ring of $G$ and is denoted by
%$$k[x]^G_G.$$
%\end{defi}
%\bsp Consider $\varphi = [x^i \longmapsto (-x)^i] \in \trm{End}(k[x])$, with $n = 1$ for all $i \geq 0$. The invariant ring is obviously $k[x^2]$.

%Although rather complicated, we simply recall our definition of linear topological rings $(R,+,\cdot,1,\tau)$, where the completion, in the above sense, is a colimit, indexed via some neighborhood basis $\beta(0)$ of zero (also filtered, as all ideals contain zero ideal):
%\bd
%\item[Index category] all neighborhood basis of zero:
%$$\mathcal{J} \subset \trm{Mod}_R(R) \subset \trm{Mod}_R$$
%clearly form a sub-category of all $R$-submodules. 
%\item[Diagram to $F$]  
%\item[Co-cone to $F$] 
%\ed
%\newpage
%\subsection{Formal group laws}
%Here we follow \cite{Strickl}.
%\begin{defi}
%Let $C$ be a commutative ring and $n \in \nz$. An $n$-dimensional formal group law $F$ is a formal power series in $C[[x_1,\ldots,x_n,y_1,\ldots,y_n]]^n := C[[x,y]]^n$ such that
%\bn
%\item $F(0,x) = x \in C[[x]]^n$,
%\item $F(x,y) = F(y,x) \in C[[x,y]]^n$,
%\item $F(F(x,y),z) = F(x,F(y,z)) \in C[[x,y,z]]^n$ and
%\item there is an map $m$ on $C[[x]]$ such that $F(m(x),x) = 0$.
%\en
%\index{Index}{formal group law}
%\end{defi}
%\bsp Some examples taken from Strickland 2011:
%\bn
%\item the map $F(x,y) = x + y \in C[[x,y]]^n$ is called the $n$-dimensional formal additive group law, with $m = [x \longmapsto -x]$,
%\item let $c \in C$, the map $F(x,y) = x + y + c x y \in C[[x,y]]$ is a 1-dimensional formal group law, with $m = [x \longmapsto -x/(1 + c x)]$ (recall if the constant term of a power series is a unit, then the formal power series itself is a unit - hence $1 + c x$ is a unit),
%\item if $c \in C^\times$, then $F(x,y) = \frac{x + y}{1 + \frac{xy}{c^2}}$ is a formal group law, with
%$m$ as in the first example. It is well-known in relativistic geometry - the so called Lorenz-FGL.
%\en
%\begin{lemm}
%Let $F$ be an $n$-dimensional formal group law over some commutative ring $R$. There exists a $\Psi \in R[[x_1,\ldots,x_n]]^n$ such that $\Psi(0) = 0$ and
%$$F(u, \Psi(u)) = F(\Psi(u),u) = 0.$$
%\end{lemm}
%A proof is given in \cite{Serr}.
%\bmk For any $n$-dimensional formal group law $F \in R[[x,y]]^n$, we define a formal group scheme over $\trm{Spec}(R)$ via
%$$\mathbf{F} : \trm{CAlg}_R \longrightarrow \trm{Grp},\ A \longmapsto N(A)^n$$
%where the operation is given via $(u,v) \longmapsto F(u,v)$.
\bibliography{references}
\bibliographystyle{plain}
\addcontentsline{toc}{section}{References}
\pagestyle{plain}
\pagenumbering{Roman}
\printindex{Symbol}{Symbols}
\printindex{Index}{Index}
%\addcontentsline{toc}{section}{Symbols}
%\addcontentsline{toc}{section}{Index}
\include{decl}
\end{document}


