nun habe ich ein wenig in Ihrem Entwurf gelesen. Die grundsaetzliche
Konzeption und Themenauswahl gefaellt mir gut, indem Sie sich bemuehen,
einen in sich runden Zugang zur modernen Differential-Galoistheorie
vorzustellen und die Sache von Ihrer Warte aus selbstaendig zu betrachten.

Im einzelnen Detail stolpere ich aber doch immer wieder. So scheinen mir
z.B. diverse Definitionen, die Sie in Ihre eigenen Worte fassen, entweder
unpraezise oder schwer verstaendlich zu sein. Was aber Definitionen
anbelangt so wuerde ich sehr zu groesstmoeglicher Naehe zur vorgefundenen
Literatur raten.

Im Appendix wuerde ich dazu raten, die Einfuehrung in die Algebraische
Geometrie (Abschnitt 6.3.2.) ersatzlos zu streichen --- bis auf die
wichtige Definition 6.20, auch 6.21 sollte dabei bleiben. Ebenso sollte
die Kategorientherie bis section 6.1.2 inklusive weggelassen werden (also
mit dem Stoff des jetzigen Abschnitts 6.1.3 beginnen).

Der einfuehrende Teiltext in Abschnitt 1.3 duerfte gerne ausfuehrlicher
sein. Es ist nicht verboten, hier schon die eine oder andere der
Schluesseldefinitionen (zumindest informal, falls andernfalls ausartend)
vorzustellen, oder auch die wesentlichen Einsichten zu skizzieren, die die
Diskussion der Beispiele vermittelt. Ferner ist zu erwaegen, die
"Conclusion" (Abschnitt 5) --- die gerne ueber die doch recht allgemein
gehaltenen statements hinausgehen darf ! --- in die Einleitung einzubauen.
Unter dem Strich wird der Leser mit den Konstruktionen aus Abschnitt 4
doch weitgehend alleine gelassen ... Wie stellt sich uebrigens das zweite
Beispiel aus Abschnitt 3 im Licht der Theorie aus Abschnitt 4 dar ?

Es folgen einige Detailbemerkungen. (Ich betone aber, dass ich den Text
nicht Wort fuer Wort gelesen habe.)

--- Introduction, erster Satz: " ... one of the greatest achievements of
E. Galois" Sollte es nicht gleich heissen: "... of modern algebra" ?

--- Def. 2.1.4 (Seite 4) Ist dies die offizielle Def. einer noetherschen
Algebra ?

--- Seite 4: R unitary ... spaeter in Def. 2 heisst ist von "unital"
algebras die Rede --- das gleiche gemeint ?

--- Den erlaeuternden Satz zu Remark 2.1.1.1 verstehe ich nicht.

--- Abschnitt 2.1 allgemein: Anders als im Titel angedeutet scheinen Sie
eben nicht Begriffe der (bloss) kommutativen Algebra, sondern
nichtkommutativen Algebra abzuhandeln ?!

--- Def. 2.3.2: Die Def einer Ore Erweiterung koennte klarer
(ausfuehrlicher) sein

--- Remark 2.3.1: Was ist eine "id_A-derivation in the Ore sense" ?

--- Def. 2.5: 3-cyle (1,2,3) ?

--- Def. 2.6: Was ist der Definitionsbereich von ${\mathfrak g}(f)$ ?

--- Prop. 2.5: Geben Sie die praezise Bourbaki-Referenz.

--- Prop. 2.7: Beweisreferenz, (wo war das????)

--- Def. 2.18: Ist "counterpart" ein mathematischer Begriff ?

--- Seite 31, Example 2: Referenz !

--- Seite 31, Satz nach "set of differential operators generated by..."
??? Was bedeutet der nachfolgende Satz ("... except for the trivial") ???

--- Seite 32, Remark 3.1.1: Genauer !

--- Seite 35/36: erfordert die PV-Definition nicht auch "Erzeugtsein durch
Loesungen" ??

--- Seite 37, Prop. 3.2: Sie sprechen von "the" PV-ring und "the"
fundamental matrix: inwiefern sind die denn eindeutig ?

--- Seite 38: PV ring By Prop. ....... ???

--- Seite 38: Differential Galois Group: Welches +ist+ denn nun die
DGalgruppe abstrakt ? Z ?

--- Seite 40: Welches +ist+ denn nun die DGalgruppe abstrakt ?

--- Seite 47, erste Saetze von Abschnitt 4.1: Was ist $k(x)$ ? Was ist
+der+ PV-Koerper ? (dazu ist doch erst eine DGL zu waehlen !) Was ist
$\partial_x$ ? Warum ist $(K,\{\partial, \partial_x\})$ eine differential
algebra ?

--- DSeite 47, Def. 4.2: Was ist $C_k$ ?

--- Seite 49, Remark 4.1.4: Was sind "partial differential subalgebras" ?

Viele Gruesse,

Elmar Grosse-Kloenne


