\section{Introduction}
%The introduction of the then still unknown study of group theory is one of the greatest achievement of E. Galois. Instead of studying solutions explicitly, he studied the $k$-automorphism groups acting on the roots.\\
The general Galois theory was started, of course, by E. Galois himself. Its modern interpretation in field extension theory is, informally, instead of studying sets of solutions (i.e. the roots) for some polynomial equation over a given field one shall study the symmetry group acting on the roots.\\
%\indent Naturally, in field extension theory and algebraic geometry Galois theory is a prominent subfield - nonetheless in number theory. However, the actual power of this theory is that its application does not stop there.\\
%\indent The question we are going to study is can we extend Galois's approach to a new setting and what that setting would look like.%Speaking roughly in a categorical manner, Galois theory describes a functor assigning to each object $X$ in the category of sets an object in the category of groups, group schemes or formal group schemes such that the original object is a torsor of the group object, its so-called Galois group (or principle homogeneous space or torsors, see appendix, pg. \pageref{appendix}, for a brief introduction to category theory). Importantly, the group object is intrinsically connected to the structure of $X$.% Appropriately extending our notation we will arrive at Galois theory of artinian $D$-module algebras.
\subsection{Aim of Thesis}
\indent The question, we are going to study, is can we generalize Galois's approach and for which algebraic structures can we do that. The answer to this question is addressed in the paper 'Introduction to Galois theory for simple artinian $D$-module algebras' by F. Heiderich \cite{Heid13} which we are going to present and exemplify within the framework of differential Galois theory in characteristic zero. This requires a large body of algebraic structures and a little help from category theory.\\
\indent Firstly, we are going to introduce the theory of (unital, associative) algebras $(A,\mu,\eta)$ over some given ring and a distinct type of algebra extensions, the so called Ore-extensions. Next, we discuss briefly Lie algebras and their enveloping algebras since they will be used in the discussion of our examples. Furthermore, we need the theory of (counital, coassociative) coalgebras $(C, \Delta, \eps)$ over some given ring, which are categorically dual to an algebra. Moreover, we are going to merge these theories into the theory of bialgebras. A bialgebra $(D, \mu, \eta, \Delta, \eps)$ is both, an algebra and a coalgebra, such that the coproduct $\Delta$ and counit $\eps$ are homomorphisms of algebras and product $\mu$ and unit $\eta$ are homomorphisms of coalgebras. Completing our brief visit to general algebra, we will introduce $D$-module algebras, denoted by $(A,\Psi_A)$, which are $D$-modules such that the representation homomorphism $\Psi_A : D \longrightarrow \trm{End}(A)$ commutes with the algebra structure on $A$. Finally, we are going to present the theory of Hopf-algebras $(H, \mu, \eta, \Delta, \eps, S)$, which is a bialgebra with a bialgebra anti-homomorphism - its antipode $S : H \longrightarrow H^{\trm{copop}}$, where $H^{\trm{copop}}$ is the opposite, co-opposite bialgebra. This will comprise the whole section 2\medskip\\
\indent Equipped with these theories and enhanced with the introductory theory of differential modules and rings, we will introduce an important example of a bialgebra and its module algebra: the ring of differential operators $(k[\partial], \mu, \eta)$ over a given differential ring $(k, \partial)$, and the ring of differential polynomials $(k\{u\}, \Psi_{k\{u\}})$, which both date back to Ritt and Kolchin (see \cite{Ritt} and \cite{Kol}). We are going to prove that the ring of differential operators is an Ore-extension of the differential ring $k$ and the ring of differential polynomials $k\{u\}$ is a $k[\partial]$-module algebra. Augmented with the theory of linear differential equations over algebraically closed fields of characteristic zero - called Picard-Vessiot theory, we are going to study two examples of homogeneous linear differential field extensions over $\currfield$ and show that their Galois group is a linear algebraic group over the field of constants. We will be discussing the abovementioned in section 3.\\
\indent This will lay the foundation of the generalized Galois theory of $D$-module algebras for a given bialgebra $D$, where $D \simeq D^1 \# k.G$, i.e. the smashed product of a group algebra and a pointed irreducible cocommutative cofree (= Birkhoff-Witt type) bialgebra for a given group $G$ and a Lie algebra $\mathfrak{g}$. In the case we will discuss $G$ equals the trivial group and $D^1$ the ring of differential operators $k[\partial]$. This, in turn, will clarify for which algebraic structures can we define a Galois-object. In a categorical manner, for any bialgebra $D$ we get a functor
$$\trm{Gal} : \trm{CAlgMod}_D \longrightarrow \trm{Grp},\ \trm{GrpSch}\ \trm{or}\ \trm{GrpLaw}$$
assigning to each $D$-module algebra $A$ a group, (formal) group scheme or group law. The application to general differential equations of fields of characteristic zero heavily depends on the definition of the universal Taylor homomorphism
$$\iota: (R, \partial) \longrightarrow (R[[t]],\partial), a \longmapsto \sum_i \frac{\partial^i(a)}{i!} t^i$$
as defined in \cite{Ume96}. This can be thought of as a generalized power series ansatz, as discussed in introductory Analysis. Heiderich subsequently defines a functor $\trm{Ume}$, dedicated to Umemura, assigning to each commutative differential algebra over the field of solutions its Galois group as follows: given a differential extension $K/k$ with transcendence degree $n$ such that $[K : k(x_1,\ldots,x_n)]_{\trm{sep}} < \infty$ and any commutative associative algebra $A$ over $K$, the elements of the Galois group are all differential algebra automorphisms $\varphi$ % \in \trm{Aut} (\mathcal{K} \\otimes_K A[[w]] \longrightarrow K[[t]] \otimes_K A[[w]]$ 
 acting on the topological completion of $\mathcal{K} \otimes A[[w]]$ with respect to the $\left<w\right>$-adic topology, denoted by $\mathcal{K} \hat{\otimes}_K A[[w]]$, that makes
$$\xymatrix{
\mathcal{K} \hat{\otimes}_K A[[w]] \ar[r]^\varphi \ar[rd]_{id_{\mathcal{K}} \otimes \pi[[w]]~~~~} & \mathcal{K} \hat{\otimes}_K A[[w]]\ar[d]^{id_{\mathcal{K}} \otimes \pi[[w]]}\\
&\mathcal{K} \hat{\otimes}_K A/N(A)[[w]]\\
}$$
commutative and acting on $\kappa \hat{\otimes}_K A[[w]]$ as identity. Here, $N(A)$ is the nilradical in $A$, $\pi : A \longrightarrow A/N(A)$ the canonical projection and $\pi[[w]]$ its extension to the ring of formal power series, $\kappa := K\{\iota(k)\}_{\partial_x}$ and $\mathcal{K} := K\{\iota(K)\}_{\partial_x}$ differential subalgebras of $K[[t]]$, closed under the $K$-derivations $\partial_{x_i}(x_j) = \delta_{ij}$.  This is a special version of a Lie-Ritt functor of the type:
$$\Gamma_{n,R} : \trm{CAlg} \longrightarrow \trm{Grp}, A \longmapsto \left\{\phi = (\phi_i)_{i=1}^n\in A[[w]]^n : \phi_i \equiv w_i \mod N(A)[[w]]\right\},$$
e.g. group of infinitesimal coordinate transformation congruent to identity module nilradical, introduced by Umemura in \cite{Ume96}. In the context of Picard-Vessiot theory, Heiderich shows that this functor is a formal group scheme such that for a PV field $K$ there exists a finite \'{e}tal field extension $K'$ such that for every commutative module algebra $A$ over $K'$
$$\bao{c}
\ker\left[\trm{Gal}(K/k)(A) \longrightarrow \trm{Gal}(K/k)(A/N(A))\right] \simeq \trm{Ume}(K/k)(A),\ \trm{and}\\ \trm{Ume}(K/k)(K'[\eps]) \simeq \trm{Lie}(K/k) \otimes K',\\\ea$$
where $K'[\eps] \simeq K'[X]/\left<X^2\right>$. Lastly, we are presenting the findings of Heiderich, that the Hopf-algebra $H := (R \otimes R)^{\Psi_R}$ encodes the PV-ring $R$ and its spectrum is the affine group scheme $\trm{Gal}(R/k)$ whose $A$ points are isomorphic to $\trm{Aut}_D(k \otimes A)$ with module algebra homomorphism $\rho \otimes \rho_0 \in \trm{Hom}(k \otimes A, \trm{Hom}(D,k \otimes A))$ with
$$\rho \otimes \rho_0 = \rho \otimes \rho_D = \left[a \otimes a' \longmapsto \Psi(\_\otimes a) \ast a' \eps_D := \left[d \longmapsto \sum_{(d)} \Psi(d_{(1)} \otimes a) \otimes \eps_D(d_{(2)}) a'\right]\right]$$
leaving $k \otimes A$ fixed and apply this to one of our examples discussed in the previous section. The bialgebra $D$ in this case is simply $\currfield[\partial]$ - i.e. the ring of differential operators over $\currfield$. This summarizes the fourth section.\\
% and apply the theory to some (more or less classical) examples. He also choses a bialgebraic approach. The term '$D$ measures to $k$' as in Takeuchi et al. is now termed '$k$ is a $D$-module algebra', for some bialgebra $D$ over ring/field of constants $C_k =: k^\Psi$. The homomorphism $\Psi : D \otimes A \longrightarrow A$ satisfies certain compatibility conditions wrt. the algebra structure $(A, \mu, \eta)$ of any unital associative $k$-algebra $A$ and some bialgebra $D$. The application to general differential equations of fields of characteristic zero heavily depends on the definition of the universal Taylor homomorphism
%$$\iota: (R, \partial) \longrightarrow (R[[t]],\partial), a \longmapsto \sum_i \frac{\partial^i(a)}{i!} t^i$$
%as defined in \cite{Ume96}. He subsequently defines a functor $\trm{Ume}$, dedicated to Umemura, assigning to each commutative differential algebra over the field of solutions its Galois group.\\
\indent The beauty of this theory lies in its generality - it covers classical linear and non-linear differential equations over fields of characteristic zero, iterative derivations for positive characteristic as well as so called difference equations in arbitrary characteristic. Furthermore, the prerequisit of algebraic closedness of the field of constants, as in the classical Picard Vessiot theory, is not required. In brief, Galois theory of simple artinian module algebras has applications in representation theory of groups and solution theory of differential equations.% The clue is to treat our solution algebras as module algebras over the ring of differential operators. Although beyond the scope of this thesis, there is an intriguing connection between both fields, differential and difference equations.
% Expressed in the language of modern algebra, we are going to describe the unifying theory to study $D$-module algebras and certain subalgebras - stable under $D$-action and their respective automorphism groups.
%\subsubsection{Outline of Thesis}
%This paper contains three parts loosely following the evolution of the theory. The first part contains a broad introduction to the main concepts used in this paper (though, some parts are placed in the appendix for brevity). These include Ore extensions, Lie-, co-, bi- and Hopf-algebras. Ore extensions generalize classic commutative element adjunctions: for some $R$-algebra $A$ and certain endomorphisms $\alpha, \delta \in \trm{End}_R(A)$, such that $A[\delta]$ has a unique $R$-bialgebra structure (to specify, $\delta$ is a $(\alpha,id_A)$-skew primitive element with coproduct $\alpha \otimes \delta + \delta\otimes id_A$). Coalgebras are $R$-modules that have an $R$-algebra as dual whilst bialgebras are $R$-co/algebras with a distinct compatibility condition (i.e. coproduct and counit are algebra homomorphisms, product and unit are coalgebra homomorphisms).\\
%\indent The second part includes the definition of the ring of differential operators over some differential ring of characteristic zero and its ring of differential polynomials. We are going to show, that the first is indeed a Hopf-algebra as well as an Ore-extension, the latter a module algebra for the first. Furthermore, we introduce the prominent PV-theory and conclude with two linear examples over $\ov{\qz}$.\\
%\indent The third part introduces the Umemura-functor and the general Lie-Ritt functors, as well as iterative derivations. This is augmented with infinite Galois group defined over any differential/ iterative differential/difference algebra. We show how Heiderich connects classic PV theory with the general approach and conclude with extension of the second example from the previous chapter.%(co-) module algebras over co-, bi- and Hopf-algebras as means for Galois theory. Some important theorems are included and finally, exemplifying the theory explicitly.
%\subsection{Aim of this Paper}
\subsection{Background}
We are going to briefly address why we are needing these constructs. In \cite{vdPS01} it is said that differential modules over some differential ring do not have tensor products, in general. That is why we need the coalgebra structure - as it encodes the action of differential operators on the tensor products and therefore the action of differential operators on products of the module algebra. Furthermore, Umemura explains that a generalized approach to differential equations in characteristic zero is far from obvious (\cite{Ume96b}, problem 1 and 2). Given a differential polynomial, we can construct a linear partial differential equation equivalent to our polynomial. However, it is not clear what the base field/ring looks like and if the object containing all symmetries acting on the solutions is in the category of groups.
\subsection{Short Historic Review}
The develpoment of this theory was everything but straight forward. The problems involved were manifold - e.g. is the action of the Galois group in- or equivariant. Furthermore, how to study differential equations in positive characteristic.
%Galois theory in the context of field theory is simply the study of field extensions and the actions of all automorphisms over the ground field leaving it fixed.
%As just mentioned, the Galois theory was introduced by E. Galois to study field extensions via the action of the $k$-automorphisms on roots of separable polynomial equations over some field $k$ leaving $k$ fixed. This led to the famous theorems as the correspondence theorem of normal subgroups and the normal separable extensions over $k$, or the solvability theorem - stating, the Galois group is solvable iff its polyomial solvable by radicals in its Galois extension.
%To establish a general Galois theory required an extensive study of concepts developed rather recently. We summarize the historic steps that let to the theory we will present shortly.
\subsubsection{19th Century to mid-20th century}
The first studies in the differential case were done by S. Lie in the mid-19th century which resulted in the well known theory of Lie groups and their algebras. This is a group with a set of derivations, compatible with the group left action, forming a Lie algebra.\\
\indent The first steps for a general Galois theory for linear differential equations over fields of characteristic zero were done by Picard and Vessiot in the late 19th century to early-mid 20th century. Hence, the naming in the linear case:
\bd
\item[Picard-Vessiot ring] a differential ring extension $(R,\partial)/(k/\partial)$ of a differential field $(k,\partial)$ is called a Picard-Vessiot ring (PV-ring) for a scalar differential equation $L \in k[\partial]$, if both the subring of constants agree $R^\partial = k^\partial$ and the ring extension is simple as a differential ring.
\item[Picard-Vessiot field] is the quotient field of the PV ring.
\ed
The Galois group, or more precisely - differential Galois group, is isomorphic to some linear algebraic subgroup, transforming a set of solutions into a new set of solutions, over their field of constants $k^\partial$. In \cite{vdPS01}, it is shown there is a correspondence between scalar differential equtions and matrix differential equtions. A correspondence theoreme equivalent to the main theoreme of Galois theory in field extensions reads as:
\bd
\item[Galois Correspondence] For a given (matrix/scalar) differential equation over some differential field with algebraically closed field of constants and characterisc zero, there is a one-to-one correspondence between the closed algebraic subgroups of the differential Galois group and the intermediate PV-subfields 
%Let $A \in \trm{Gl}_n(k)$ define some linear differential equation over the differential field $(k,\partial)$. There is a one-to-one correspondence between the closed algebraic subgroups $H_i$ of $G \leq \trm{Gl}_n(C_k)$ and the PV-subfields $M_i$ of $K = S_{\left<0\right>}^{-1}(R)$:
%$$\trm{DGal}(K/M_i) = H_i,\ \trm{Fix}^{H_i} = M_i = \trm{Stab}_{H_i}(K),$$
%with $\partial_K$ the extension of $\partial$ on $K \supset M_i \supset k$ and $R$ the PV-ring
(see \cite{vdPS01}, pg. 25).
\ed
Although being well established, this theory only applies to linear differential fields containing either the field of complex numbers or the algebraic closure of the field of rational numbers (i.e. characteristic zero).
\subsubsection{Mid-20th Century to today}
Next, there was Ritt how coined the term 'differential algebra' and ring of differential polynomials. Simply put, the ring of differential polynomials over some differential field/ring is a non-noetherian ring with unique differential relations. This allowed a more general approach in terms of its quotient rings wrt. differential ideals \cite{Ritt}. This was enhanced by Kolchin who described certain classes of non-linear differential field extensions for characteristic zero, the so-called strongly normal differential extensions. Furthermore, he was the first to state that algebraic geometry is simply a differential equation without any derivatives occurring - hence, just a subfield of differential algebra \cite{Kol}.\\
\indent However, it took until the late 20th century to introduce a differential Galois theory for positive characteristic. The problem arose from the fact that any differential extension enlarges the subring of constants for classical derivations over fields of positive characteristic. %To exemplify, $\left(k(t),\partial = \frac{d}{dt}\right)$ differential field with $\trm{char}(k) = p$ - if $\partial$ denotes the restrictions to the fields in question and $C_X$ denotes the field of constants for $X$, we get:
%$$(k(t),\partial) \supsetneq (C_{k(t)},\partial) = (C_k(t^p),\partial) \supsetneq (C_k, \partial),\ \trm{where}\ C_X = \{a \in X : \partial(a) = 0\},\ X = k, k(t)$$
%as a proper subfield with all its elements being constants. 
This problem was circumvented by the introduction of iterative differential equations in positive characteristic (see \cite{HassSchm} for iterative derivations, \cite{Matz} for iterative differential equations and their Galois theory).\\
\indent Takeuchi introducted a generalizing concept, the so called $C$-ferential fields $k$, where $C$ is a coalgebra with a unique homomorphism $\psi : C \otimes k \longrightarrow k$, said to measure $C$ to $k$. We remark that Takeuchi introduced the term of Birkhoff-Witt type coalgebras which simply means cofree. This made possible a Galois theory for iterative differential equation unifying PV theory in characteristic zero and in positive prime characteristic \cite{Take}. In context of the theory of coalgebras, $k$ is a $C$-comodule algebra (i.e. a comodule such that multiplication and unit map are coalgebra homomorphisms).\\
\indent Amano \cite{AM} as well as Amano and Masuoka \cite{Amo} came up with a Hopf-algebraic approach, where $C$, as in Takeuchi, was a cocommutative pointed irreducible Hopf-algebra of Birkhoff-Witt type (see section ... for an explanation). They showed that for any simple differential ring extension $R$ the Hopf-algebra $H = R \otimes_k R^{\Psi_R \otimes \Psi_R}$ is a principle homogeneous space of $\trm{Gal}(R/k)$ and there is a correspondence between Hopf-ideals in $H$ and differential subfields $k \subset M \subset K = Q(R)$. They show that instead of linear algebraic groups, group schemes are more suitable.