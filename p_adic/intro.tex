\section{Introduction}
We are going to analyse the ring of $p$-adic integers $\hat{\zz}_p$ for a given prime number $p \geq 2$. To establish this algebraic ring, there are several possibilities.
\subsection{Basic ring theory}
For clarity, we are going to repeat some important definitions of ring theory. We make usage of categorical notations as found in (Strickl. 2000, Formal groups and formal group laws).
\subsubsection{Rings and prime ideals}
We call a set $R$ a ring if it is an abelian group $(R,m_+,0,S_-)$ with structure maps
$$\bao{rrcl}
+ :& R \times R&\longrightarrow& R\\
&(r,s) &\longmapsto&r + s\\
&&&\\
0 :& \ast & \longrightarrow&R\\
&x&\longmapsto&0\\
&&&\\
- :& R &\longrightarrow&R\\
&r &\longmapsto&-r\\
\ea$$
and $(R,m_\cdot)$ is a semigroup such that the following diagram commutes:
$$\bao{cc}\xymatrix{
R^4 \ar[rrr]^{id_R \times \tau_{R^2} \times id_R} &&& R^4 \ar[d]^{m_\cdot \times m_\cdot}&\\
R^3 \ar[d]_{id_R \times m_+} \ar[u]^{\Delta \times id_{R^2}} &&&R^2\ar[d]^{m_+}\\
R^2\ar[rrr]^{m_\cdot}&&&R\\
} & \bao{c}
\trm{where}\bao{rrcl}
\Delta : & R & \longrightarrow& R^2\\
&r&\longmapsto&(r,r)\\
&&&\\
\tau_{R^2} :& R^2 &\longrightarrow&R^2\\
&(r,s)&\longmapsto&(s,r).\\
\ea\ea
\ea$$
This is just a rephrasing of distributivity. Clearly, $m_+$ and $m_\cdot$ are addition and multiplication, $S$ is the additive inversion and $0$ is the zero map. If no ambiguity can occur, we simply use the standard symbols for those operation maps.  Let $(R,+_R, 0_R, -_R,\cdot_S)$ and $(S,+_S,0_S,-_S,\cdot_S)$ be two rings. We call a map $f : R \longrightarrow S$ a ring homomorphism if
$$\xymatrix{
R^2 \ar[r]^{f \times f} \ar[d]_m & R^2\ar[d]^m\\
R \ar[r]_{f}&R\\
}$$
commutes for both operation maps $+, \cdot : R^2 \longrightarrow R$. The class of all rings forms a category with morphisms called ring homomorphisms. We call a ring unital, if there is a map $1 : \ast \longrightarrow R, x \longmapsto 1$ such that $(R,\cdot,1)$ is a monoid. Again, if no ambiguity can arise we simply denote a ring by its set. For a unital ring $R$, we denote by $R^\times$ the group of units:
$$R^\times := \{x \in R : \exists y \in R, x y = 1\} = \pi_1(\mathcal{M}_1),$$
$$\ \pi_1 : R \times R \longrightarrow R, (r,s) \longmapsto r\ \trm{and}\ \mathcal{M}_1 := \{(x,y) \in R^2 : x y = 1\}.$$
\subsubsection{Modules and ideals}
\indent For a ring $R$, we call an abelian group $(M,+_M,0_M,-_M)$ an left $R$-module if there is a map 
$m_\cdot : R \times M \longrightarrow M$ such that both diagrams
$$\bao{c}
\xymatrix{
R^4 \times M^2 \ar[rrr]^{id_{R^2} \times \tau_{R^2 \times M} \times id_{M^2}}& && (R^2 \times M)^2 \ar[d]^{+_R \times id_M \times +_R \times id_M}\\
R^2 \times M^2 \ar[d]_{+_R \times +_M} \ar[u]^{\Delta_{R^2} \times id_{M^2}} &&&(R \times M)^2\ar[d]^{m_+}\\
R\times M\ar[rrr]^{m_\cdot}&&&M\\
}\\
\xymatrix{
R^2 \times M \ar[r]^{id_R \times m_\cdot} \ar[d]_{\cdot_R \times id_M} & R \times M\ar[d]^{m_\cdot}\\
R \times M \ar[r]_{m_\cdot} &M\\
}
\ea$$
commute. We remark that the first diagram represents distributivity in both, $R$ and $M$. The latter represents the successive scalar multiplication commutes with multiplication of $R$ (i.e. $m_\cdot(r,m_\cdot(s,m)) = m_\cdot(rs, m)$). The map $m_\cdot$ is called the outer left product or scalar left multiplication. Similarily, we define a right $R$-module for $m_\cdot : M \times R \longrightarrow M$. An $R$-submodule $N$ of $M$ is a subset that is itself a module (i.e. the restrictions of the structure maps have images in product sets of $N$). The class of $R$-modules forms a category, with $R$-linear maps as morphisms.
\paragraph{Ideal}
An ideal $I \subset R$ is simply an $R$-submodule of $R$. Alternatively, we could have defined an ideal $I$ to be an abelian subgroup of $R$, that is stable under $R$-action: $R.I \subset I$. We call $R$ simple if $\left<0\right>$ and $R$ (if unital) are the only ideals in $R$. We call an ideal $I$ proper if $I \neq R = \left<1\right>$. A proper ideal $I$ is called prime if for all $a, b \in R$ holds $a b \in I$ implies $a \in I$ or $b \in I$. We call a proper ideal $I$ maximal if for any ideal $J \supset I$  we have either $J = I$ or $J = R$. Two ideals $I, J \subset R$ are called coprime if
$$I + J = \left<1\right>.$$
\begin{prop}
In a unital commutative ring, there always exists a maximal ideal and every maximal ideal is prime.
\end{prop}
\bws Proving the first statement we are considering the set $R^0 := R\backslash R^\times$. Then we construct the set of ideals $\mathcal{I}$ in $R^0$. Clearly, $\{0\} \in \mathcal{I}$. Then any element $x \in R^0 \backslash \{0\}$ generates an ideal in $R^0$. Thus
$$\mathcal{I} \supset \mathcal{I}_0 := \{R.x : x \in R^0\}.$$
Now we pick $I = R.x$ and $J = R.x'$ being not coprime then its sum is a proper ideal $I + J \subset R^0$. Therefore, we may attach
$$\mathcal{I}_0 \cup \left\{\sum_{x \in R^0} R.x : \sum R.x \cap R^\times = \emptyset\right\} =: \mathcal{I}.$$
It is not obvious that this is already the family of all ideals in $R^0$. However, since each $R$-combination of elements in $R^0$ is either a unit or a non-unit we have that
$$\forall I \in \mathcal{I} \wedge \forall y \in I \Rightarrow y \in R^0.$$
%We may repeat the this procedure in the following way:
%$$R^i = \bigcup_{I \in \mathcal{I}_i} I,\ \mathcal{I}_i = \mathcal{I}_{i-1} \cup \left\{\sum_{x \in R^{i-1}} R.x : \sum R.x \cap R^\times = \emptyset\right\}$$
%until no more proper ideals are left to be added.
Now we get a partially ordered set
$$(\mathcal{I}, \leq),\ \leq = \left\{(I,J) \in \mathcal{I}^2 : I \supset J\right\}.$$
Each chain of ideals $\mathcal{I}_I := \{I_i : \ldots \supset I_i \supset I_{i+1} \supset \ldots\}$ is a totally ordered set. Thus, by Zorns lemma we have that each $\mathcal{I}_I \subset \mathcal{I}$ has a lower bound. By definition it is the largest $R$-submodule in $R^0$ or equivalently a maximal ideal.\\
\indent To see that a maximal ideal is prime we simply mention that the factor ring of a maximal ideal is an algebraic field, hence an integral domain.
\paragraph{Spectrum}
%For a ring $R$ we can call an ideal $I$ prime if and only if $R/I$ is an integral domain, i.e. zero is the only $R$-torsion. Furthermore, an ideal $I$ is maximal if and only if $R/I$ is an algebraic field. 
Prime ideals are usually denoted by $\mathfrak{p}$, $\mathfrak{q}$ - maximal ideals by $\mathfrak{m}$. Let $\mathcal{M}_R := \mathcal{I} \cup \{\left<1\right>\}$ denote all ideals in $R$. We call the set of prime ideals
$$\trm{Spec}(R) := \{\mathfrak{p} \in \mathcal{M}_R : \trm{Ann}_R(R/\mathfrak{p}) = \left<0\right>\}$$
the spectrum of $R$. We remark that the set of maximal ideals is a subset of $\trm{Spec} R$ in all unital ring, as each field is obviously an integral domain. Lastly, we have that
$$R \simeq \prod_{\mathfrak{p} \in \trm{Spec} R} R/\mathfrak{p}.$$