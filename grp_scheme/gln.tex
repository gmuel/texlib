%gln
\section{General linear group}

As the most prominent example of a functor, we will now discuss the following construction. Let $R$, again, denote a unital commutative ring.
\begin{defi}
Let $\mathfrak{F}$ be a finite category and $\mathrm{UCRng}$ the category of unital commutative rings. The bifunctor
$$\bao{rrcl}
\mathrm{Gl} : &\mathfrak{F} \times \mathrm{UCRng} &\longrightarrow &\mathrm{Grp},\\
& (I,R) &\longmapsto& Gl(I,R) := Gl_{|I|}(R),\\
\ea$$
is a pairing of a finite set $I$, a ring $R$ and its group of $R$ module automorphism on the generic module $R^I$.
\end{defi}
\begin{coro}
We  immediately get
$$\bao{rcl}
Gl_0(R) &=& Gl(\emptyset,R) = \{1_R\} \ \mathrm{and}\\
Gl_1(R) &=& Gl(\mathcal{P}(\emptyset),R)\\ &=& R^\times = U(R)\\ &:=& \{r \in R : \exists r' \in R,\ r r' = 1\}.\\
\ea\\$$
\end{coro}
Both statements are clearly a consequence of our above definition.
\begin{theo}
Let $I \in \mathfrak{F}$ be fixed with $|I| = i$ and let $R \in \mathrm{UCRng}$ be an ID then
$$\mathrm{Gl}_i \simeq \mathrm{Hom}_{\mathrm{UCRng}}(\mathbb{Z}[X,Y]/\left<\det X \cdot Y - 1\right>,\ \cdot\ )$$
as functors.
\end{theo}
Before we proceed in proofing the theorem, let us dwell in the above formular for a moment. Firstly, $X$ stands for $\{X_{kl} : 1 \leq k, l \leq i\}$ - a set of $i^2$ variables and
$$\mathbb{Z}[X,Y]/\left<\det X \cdot Y - 1\right> = \mathbb{Z}[X_{kl},Y : 1 \leq k, l \leq i]/\left<\det \left(\bao{ccc}
X_{11} & \ldots & X_{1i}\\
\vdots&&\vdots\\
X_{i1}&\ldots&X_{ii}\ea\right) \cdot Y - 1\right>$$
for the ring of the polynomial ring over $\mathbb{Z}$ in $i^2 + 1$ indeterminants, factored out the ideal generated by $\det X \cdot Y - 1$. Thus, the right hand side claims that the class of all (unital commutative) ring homomorphisms into some ring $R$ are functorially isomorphic to $i$-th general linear group $\mathrm{Gl}_i(R)$. %In classical algebraic geometry we would get the zero set:

%$$Z(\det X \cdot Y - 1) = \left\{x \in \mathbb{A}^{i^2}_R \simeq M_i(R) : \det x \in \mathrm{Gl}_1(R)\right\}$$
%and its spectral interpretation:
%$$\mathfrak{Z}(\det X \cdot Y - 1) = \left\{\prm \in \mathrm{Spec}(\mathbb{Z}[X,Y]/\left<\det X Y - 1\right>) : \det X Y - 1 \in \prm\right\}$$

Given some ring homomorphism $\varphi : A := \mathbb{Z}[X,Y]/\left<\det X Y - 1\right> \longrightarrow R$, we get
$$\mathrm{Spec}(\varphi) = \left[\prm \in \mathrm{Spec}(R) \longmapsto \varphi^{-1}(\prm) := \{f \in A : \varphi(f) \in \prm\}\right],$$
in particular
$$\varphi^{-1}(\prm) \in \mathrm{Spec}(A).$$
Clearly, $\mrm{Spec}(A)$ contains the elements
$$\prm_\mu :=  \left\{\sum_{kl} \lambda_{kl} (X_{kl} - \mu_{kl}) : \mu = \left(\bao{ccc}\mu_{11}&\ldots&\mu_{1i}\\\vdots&&\vdots\\\mu_{i1}&\ldots&\mu_{ii}\\\ea\right), \lambda_{kl} \in A,\ \lambda_\mu (X - \mu) \in \ker eval_{\mu}\right\}$$
mapping the monomial $X$ to $\mu$. As $A$ is a UFD and $(X_{kl} - \mu_{kl})$ cannot have any other divisors we conclude $$\ker eval_\mu \in \spec{A},$$
as $\det \mu \cdot Y - 1 \in R[Y]$ has roots in $R$ (i.e.: for all $\det \mu \in \mrm{Gl}_1(R)$) and more importantly, $(0) \in \spec{R}$.
Here, we are using the following diagram:
$$\xymatrix{
A \ar[rr]^{eval_\mu}\ar[rrd]_f&&R[Y]\ar[d]^{eval_{\det \mu^{-1}}}\\
&&R.\\
}$$
Thus, the pairing
$$\bao{rrcl}
V :& \mrm{Gl}_i(R) \subset \affr{i^2+1} &\longrightarrow& \spec{\zz[X,Y]},\\
&\left(\mu_{k,l},\frac{1}{\det \mu}\right)_{1\leq k,l \leq i} &\longmapsto& \left<X_{kl} - \mu_{kl}, Y - \det \mu^{-1}\right>
\ea$$
lets us consider all points in $\mrm{Gl}_i(R)$ as an algebraic set also as (families of) prime ideals in $\zz[X,Y]$.