\section{Introduction}
We will begin with a brief introduction of the basic terminology of algebraic geometry. This script will follow standard notation as in Bourbaki cfg. \cite{bou} A integral domain (ID) is a unital commutative ring, with no zero divisors. For this, we define a specialised functor
$$\bao{rrcl}
\mrm{Ann}_R(\cdot) &:\modr& \longrightarrow& \modr(R)\\
&&&\\
&M &\longmapsto&\{r \in R : r m = 0\; \forall m \in M\}\\
\ea$$
In particular, $\mrm{Ann}_R(R)  = \{r \in R : r' \in R \backslash\{0\},\; r r' = 0\}$.
\subsection{Affine and projective schemes}
For a unital commutative ring $R$ we define a prime ideal to be an ideal $\prm \subsetneq R$ such that for all $r, s \in R$ we have
$$r s \in \prm \ \Rightarrow r \in \prm \vee s \in \prm.$$
Furthermore, for $R$ as above we define
$$\specr := \{\prm \in \mathrm{Mod}_R : \prm \ \mathrm{prime}\}$$
to be the spectrum of $R$. Thereby, $\mathrm{Mod}_R$ is the category of $R$-modules. We call an ideal $\mx \in \mathrm{Mod}_R$ maximal if for all ideals $\mathfrak{n} \in \mathrm{Mod}_R$
$$\mx \subset \mathfrak{n} \Leftrightarrow \mx = \mathfrak{n} \ \mathrm{or}\ R = \mathfrak{n}.$$
We denote the class of all maximal ideals with $\mathrm{Max}(R)$. If $R$ is unital as assumed we get that $\mathrm{Max}(R) \subset \specr$. Let $R$ be as above and $\prm \in \specr$ the localisation, $S_\prm^{-1} R$, of $R$ wrt. $\prm$ is defined to be
$$R_\prm := S_{\prm}^{-1}R = S_\prm \times R/\sim, \ \mathrm{with}\ \sim =\left\{\left((s,r),(s',r')\right) : \exists t \in S_\prm, \;\mrm{s.t.}\; t(s' r - s r') \in \prm\right\},$$
where $S_\prm = \{r \in R : r \notin \prm\} = R \backslash \prm$ and with addition and multiplication:
$$\begin{array}{rcl}
m_+ &=& \left[\left([s,r],[s',r']\right) \longmapsto [s s', s' r + s r']\right],\\
&&\\
m_\cdot &=& \left[\left([s,r],[s',r']\right) \longmapsto [s s', r r']\right],\\
\end{array}$$
respectively. We remark that $S_\prm^{-1}R$ is a local ring i.e. it contains only one maximal ideal. Given a non-zero divisor and non-unit $r \in R$, we define
$$S_r := R \backslash \bigcup_{r \notin \prm} \prm.$$
This is set is in general not an ideal. However, it is a multiplicative monoid: if $a$ is not in $\prm$ and $b$ is not in $\prm$ then neither is $a b$. This holds for all $\prm$ not containing $r$. Now, given a topological space $(X, \tau)$ and a ring $R$ we get the 
$$S := \mathrm{map}(X,R) = R^X$$
the ring of $R$-valued functions.
\subsubsection{Affine case}
\newcommand{\calf}{\mathcal{F}}
Given a finite set of elements $F = \{f \in R\}$, we define its ideal to be
$$I(F) := \left\{\sum_{f \in F} p_f f : p_f \in R,\ \forall f \in F\right\} = \sum_{f \in F} R. f.$$
Conversely, for any ideal $I \in \modr$, the set of prime ideals containing $I$ is defined to be
$$V(I) := \{\prm \in \specr : I \subset \prm\},$$
the so-called zero set (or more specifically: common set of zeros). For two ideals $I, J \in \modr$, the sums and products are:
$$\bao{rcl}
V(I \cdot J) &=& \{\prm : I J \subset \prm\}\\
&&\\
&=& \{\prm : I \cdot J \subset I \cap J \subset I, J \subset \prm\}\\
&&\\
&=& \{\prm : I \cdot J \subset I \cap J \subset I \subset \prm \vee I \cdot J \subset I \cap J \subset J \subset \prm\}\\
&&\\
&=& \{\prm : I \cdot J \subset I \cap J \subset I, J \subset \prm\} \cup \{\prm:I \cdot J \subset I \cap J \subset J \subset \prm\}\\
&&\\
&=&V(I) \cup V(J)\\

\ea$$
The argument goes as follows: $I J \subset \prm$ implies $I \subset \prm \vee J \subset \prm$.
$$\bao{rcl}
V(I + J) &=& \{\prm : I + J \subset \prm\}\\
&&\\
&=& \{\prm : I + J \subset \prm \Leftrightarrow I \subset \prm \vee J \subset \prm\}\\
&&\\
&=& \{\prm : I \subset \prm\} \cap \{\prm : J \subset \prm\}\\
&&\\
&=& V(I) \cap V(J)\\
\ea$$
We remark that in general unital commutative rings $R$, there need not to be a unique (up to sequence and association) factorisation. Therefore, $I \cdot J \subsetneq I \cap J$. One promiment example:
\begin{defi} Some important definitions:
\bd
\item[Associatedness] Two elements $r, r' \in R$ are called associated, if there exists an $c \in R^\times$ s.t.
$$r = c r'.$$
\item[Irreducibility] An element $r \in R \backslash (R^\times \cup \{0\})$ is called irreducible iff each factorization is associated:
$$c r' = r \Rightarrow c \in R^\times,$$
in particular, $r'$ is its associated irreducible.1
\item[UFD] An ID $R$ is called a unique factorization domain (UFD) if for each $r \in R\backslash R^\times \cup \{0\}$, there exists irreducible elements $p_i$ s.t. the factorization
$$r = \prod_i p_i^{s_i}, \sum_i s_i \geq 1$$
is unique up to associatedness and permutation.
\item[PID] An UFD $R$ is called a principle ideal domain, if each ideal $I$ is generated by one element.
\item[EUCL] A PID $R$ is called euclidean, if there are a multiplicative map $\phi : R \backslash\{0\} \longrightarrow \zz_{>0}, $
\ed
\end{defi}
\ex Following rings are all examples of UFDs:
$$\zz,\ \qz,\ \fz{p},\ \mrm{where}\ p\ \mrm{prime}.$$
\ex The ring $R = \zz[\sqrt{-3}] = \zz \oplus \zz.\sqrt{-3}$: The element $2 \in R$ is irreducible as any divisor $d \in R$ is either a unit $d \in R^\times = \mrm{Gl}_1(R) = \{\pm 1\}$, or associated to $2$.\\
\indent Similarly, $1 + \sqrt{-3}$ is irreducible by the same argument and the consideration that:
$$R \simeq_{\mrm{Rng}} \zz.I_2 \oplus \zz.\left(\bao{cc}0&-3\\1&0\\\ea\right), \det_\zz : R \longrightarrow \zz, r \longmapsto \det \left(\bao{cc}r & -3 r'\\r'&r\\\ea\right) = r^2 + 3 r'^2 .$$
Therefore, $\det_\zz (1 + \sqrt{-3}) = 1 + 3 = 4 = \det_\zz 2$. We arrive immediately at
$$2 \mid 4 \wedge 1 + \sqrt{-3} \mid 4 \Leftrightarrow 4 \in \left<2\right> \wedge 4 \in \left<1 + \sqrt{-3}\right>$$
without any of the two to be associated. Moreover, one can show quickly that
$$\left<2\right> \cdot \left<1 + \sqrt{-3}\right> = \left<2 + 2 \sqrt{-3}\right> \subsetneq \left<2\right> \cap \left<1 + \sqrt{-3}\right> = \left\{\left(\bao{cc}r&-3 r'\\r'&r\\\ea\right) : r + r' \in 2 \zz\right\}.$$
An interesting result - the intersection is strictly greater than the product of two (principal) ideals. This is one of the prominent examples of a non-UFD. Moreover, $4$ has no unique decomposition wrt to irreducibility:
$$(1 + \sqrt{-3}) (1 - \sqrt{-3}) = 1 - (-3) = 4 = 2^2.$$
as well as
$$2 - (1 + \sqrt{-3}) = 1 - \sqrt{-3} = \frac{4}{1 + \sqrt{-3}}.$$
This example will be revisited in one of the following sections.
\begin{theo}
If $R$ is a UFD then so is its polynomial ring $R[X]$.
\end{theo}
This theorem has important implications - we can propose the existence of a decomposition of each finitely generate module and a given endomorphism:
\begin{coro}
Let $M \in \mrm{Mod}_R$ finitely generated, $\varphi \in \mrm{End}_R(M)$, $\mathbb{P}$ a representation of all irreducible polynomials in $R[X]$ and $p_\varphi \in R[X]$ its characteristic polynomial then there exists a unique monic polynomial $m_\varphi$ of smallest degree with decomposition:
$$m_\varphi = \prod_{\substack{p \in \mathbb{P}\\p \mid p_\varphi\\}} p^{s_p},\ s_p \geq 1 \forall p$$
such that $m_\varphi(\varphi) = 0 \in R[\varphi]$. Then there exists a sequence of integers $t_p \leq s_p$ and $u_{t_p}\geq 1$ such that
$$R[\varphi] \simeq \sum_{\substack{p \in \mathbb{P}\\p \mid p_\varphi}}\left(R[X]/\left<p^{t_p}\right>\right)^{u_{t_p}}$$
and $\sum_p u_{t_p} t_p = deg p_\varphi$.
\end{coro}
