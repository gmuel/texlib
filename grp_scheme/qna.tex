%qna
\section{Topological basisc}
We are going to repeat some basic concepts from topology for the sake for coherence. A topological space is the pair of an object in the category of sets $X \in \mrm{Set}$ and a family of subsets $\tau \subset \mathcal{P}(X)$, called topology, fulfilling
\bn
\item $\emptyset, X \in \tau$,
\item at least one of
\bd
\item[Open] arbitrary unions and any finite intersection are elements in $\tau$:
$$U_i \in \tau \forall i \in I \Rightarrow \bigcup_{i \in I} U_i \in \tau,\ F_i \in \tau\ \forall i \in I' \subset I \wedge |I'| < \infty \Rightarrow \bigcap_{i \in I'} U_i \in \tau\ $$
for any given indexing set $I$.
\item[Closed] arbitrary intersections and any finite union are elements in $\tau$:
$$F_i \in \tau \forall i \in I \Rightarrow \bigcap_{i \in I} F_i \in \tau,\ \bigcup_{i \in I'} F_i \in \tau\ \forall I' \subset I \wedge |I'| < \infty$$
for any given indexing set $I$.
\ed
\en
A topology $\tau$ whose elements are open and closed at the same time is called clopen.
\exmpl For any object $X \in \mrm{Set}$ we get
\bd
\item[Lump topology] $(X,\{\emptyset,X\})$ is a topological space, with $\tau$ called the lump topology.
\item[Discrete topology] $(X,\mathcal{P}(X))$ with the power set $\mathcal{P}(X) = \{A \subset X\}$.
\item[Standard topology] on the real number line:
$$U = \bigcap_{i \in I} \bigcup_{j \in J} U_{ij} \in \mathcal{O}(\tau_{\mrm{std}}),\ U_{ij} := (a_{ij},b_{ij})\ \forall (i,j) \in I \times J$$
$$F = \bigcap_{i \in I} \bigcup_{j \in J} F_{ij} \in \mathcal{F}(\tau_{\mrm{std}}),\ F_{ij} := [a_{ij},b_{ij}]\ \forall (i,j) \in I \times J$$
\ed
Note that both topologies are examples of clopen topologies - each element is closed and open. If $X = \emptyset$ then both topologies collide.
We denote
$$\mathcal{O}(\tau) := \{U \in \tau: U\ \mrm{open}\}$$
as the open (sub) topology on $\tau$ and
$$\mathcal{F}(\tau) := \{F \in \tau: F\ \mrm{closed}\}$$
\begin{defi}
Let $(X, \tau_X)$ and $(Y,\tau_Y)$ be two topological spaces. We call a map $f : X \longrightarrow Y$ a continous map if either
\bn
\item if $V \in \mathcal{O}(\tau_Y) \Rightarrow f^{-1}(V) \in \mathcal{O}(\tau_X)$,
\item if $G \in \mathcal{F}(\tau_Y) \Rightarrow f^{-1}(G) \in \mathcal{F}(\tau_X)$,
\en
holds.
We call $f$ open if the images of open sets in $\tau_X$ are open in $\tau_Y$:
$$U \in \mathcal{O}(\tau_X) \Rightarrow f(U) \in \mathcal{O}(\tau_Y).$$
\end{defi}
\begin{prop}
The class of all topological spaces forms a category with continuous maps as morphisms and open, continous bijections (= homeomorphis) as automorphisms.
\end{prop}
\subsection{Filters and basis}
\begin{defi}
Let $(X,\tau)$ be a topological space. We call a family $\mathcal{F} \subset \tau$ of elements $F \in \mathcal{F}$ a filter if
\bn
\item $F \neq \emptyset \forall F \in \mathcal{F}$,
\item $F \subset F' \vee F' \subset F, \forall F, F' \in \mathcal{F}$.
\en
We call a family of elements $U \in \tau$ a basis if 
\end{defi}
\section{Topological groups}
In parlance with category theory, a topological group is the group object of the category of topological spaces.

\subsection{Noetherian groups}
Repeating the definition for noetherian spaces:
\begin{defi}
A topological space $(X,\tau_X)$ is called noetherian if
\bn
\item each strictly ascending chain of open subspaces $\emptyset \subset \ldots \subset U \subset U' \subset \ldots \subset X$ is stationary after finitely many steps, or
\item each strictly descending chain of closed subspaces $X \supset \ldots \supset F \supset F' \supset \ldots \supset \emptyset$ is stationary after finitely many steps.
\en
\end{defi}
We get immediately
\begin{coro}
The following statements are equivalent:
\bn
\item a topological group $(G, m, e, S, \tau_G)$ is noetherian,
\item each strictly ascending chain of open subgroups is finite (stationary after finitely many steps),
\item each strictly descending chain of closed subgroups is finite (stationary after finitely many steps),
\item each set of generators of $G$ has a finite subset.
\en
\end{coro}
Note, that noetherianicity is indeed a topological property - it only relies upon open/closed-ness and the subgroup property.

\subsection{Modules and topological spaces}
Let $R$ be a ring and $M$ an $R$ module. Following our convention from category theory, the class of submodules $N$ contained in $M$ is denoted by
$$\mrm{Mod}_R(M).$$
Let $F$ be an arbitrary family of submodules $N$, we get
$$\bigcup_{\substack{N \in F'\\F' \subset F\\|F'| < \infty}} N \subseteq \sum_{\substack{N \in F'\\F' \subset F\\|F'| < \infty}} N,\ \bigcap_{\substack{N \in F'\\F' \subset F\\}} N \in \mrm{Mod}_R(M).$$
The first part says that any finite union of submodules is a subset of the finite sum of the same modules. However, the union is in general not a module. The arbitrary intersection of submodules is, nevertheless, a submodule of $M$ and therefore, we get:
$$\mathcal{F}(M) := \left\{F \in \mathcal{P}(M) : F := \bigcup_{\substack{\mathcal{N} \subset \mathcal{N'}\\|\{\mathcal{N}\}| < \infty\\}} \bigcap_{N \in \mathcal{N}} N,\ \mathcal{N'} \subset \mrm{Mod}_R(M)\right\}$$
a closed topology by adding the empty set.
\subsubsection{Filters and neighbourhoods}
Given a topological space $(X, \tau)$ and a non-empty subset $Y \in \tau$, we call a class of subspaces $F(Y) := \{X_I : X_I \in \tau,\ \forall I \in \mathcal{I}\}$
\begin{defi}{Filter} of $Y$ if all of the following conditions hold:
\bn
\item $$X' \in \mathcal{F}(Y) \Rightarrow X' \neq \emptyset,$$
\item $$\forall X', X'' \in \mathcal{F}(Y)\ \Rightarrow X' \subset X'' \vee X'' \subset X'.$$
\en
$$$$
\end{defi}
Note that only the empty set as a topological space has no filters. Having at least one element provides us with at least one filter
$$\exists x_0 \in X:\ X\backslash \{x_0\} = \emptyset \Rightarrow \mathcal{F}(x_0) = \{X\}.$$
Also note that each ascending chain of open subsets forms an open filter and equivalently, a descending chain of closed subsets forms a closed filfter.
\paragraph{Example} Let $(\rz, \tau_{\mrm{std}})$ where $\tau_{\mrm{std}}$ is the standard topology induced by the metric $d = \left[(x,y) \longmapsto |x - y|\right]$.
\section{Group objects}
Given a category $\catc$ with terminal object $\ast$ we call an object $G \in \catc$ a group object of $\catc$ if $\grpobj$ is a functor of type
$$\catc \longrightarrow \mathrm{Grp}.$$
To specify, we say $G$ is a group object if and only if\\
\begin{enumerate}
\item $m : \grpobj \times \grpobj \longrightarrow \grpobj$
\end{enumerate}