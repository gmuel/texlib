\documentclass[10pt,a4paper]{article}
\usepackage[utf8]{inputenc}
\usepackage{amsmath}
\usepackage{amsfonts}
\usepackage{amssymb}
\usepackage{makeidx}
\usepackage{graphicx}
\newcommand{\lie}{\mathfrak{g}}
\author{moi}
\title{Special Lie algebras and module algebras of their enveloping algebras}
\begin{document}
\section{Introduction}
Although seaming rather ominious the topic is rather straight forward. Given a unital commutative ring $R$, we want to study the module algebras of $U(\mathfrak{sl}_n(R))$ in the polynomial ring $R[x_1,\ldots,x_n]$ and describe the algebraic constraints imposed on the derivations
$$\partial^X = f_x \partial_x + f_y \partial_y,\ \partial^Y = g_x \partial_x + g_y \partial_y$$
for the special case $n = 2$.
\subsection{Background}
From now on, we always assume $R$ to be unital and commutative. The question of $R[x_1,\ldots,x_n]$ as an $U(\mathfrak{sl}_n(R))$-module algebra arose from the study of non-partial differental algebras (algebras whose derivations do not commute, definition by Ritt, 1951). We assume the reader to be familiar with the notion of Lie algebras and their enveloping algebras, coalgebras and bialgebras. In particular, we assume the readers familiarity with the fact:
$$\left\{\left<x : \Delta(x) = 1 \otimes x + x \otimes 1\right>_{R-\mathrm{bialgebras}}\right\} \Leftrightarrow \{U(\mathfrak{g}) : \mathfrak{g} \in \mathrm{Lie}_R\},$$
or in words: each universal enveloping algebra for a given Lie algebra $\mathfrak{g}$ is isomorphic to a pointed irreducible bialgebra, generated by its primitive elements and a unique group like, $1 \in B$.
\subsubsection{Categorical considerations}
Consider the family of Lie algebras represented by a functor from the finite category to the category of $R$ Lie algebras:
$$\mathcal{F} : \{N : |N| < \infty\} \longrightarrow \mathrm{Lie}_R, N \longmapsto \mathfrak{sl}_{|N|}(R).$$
This is simply the bifunctor
$$\mathfrak{sl} : \mathrm{IndexCat} \times \mathrm{CRng} \longrightarrow \mathrm{Lie}_R$$ where $\mathrm{IndexCat}$ is a (small) indexing category, $\mathrm{CRng}$ is the category of unital commutative rings and $\mathrm{Lie}_R$ is the category of $R$ Lie algebras, fixed at a given unital commutative ring $R$.
\newpage
\section{Canonical module structure}
In order to show that $R[x_1,\ldots,x_n] = R[x]$ has a non-trivial $\mathfrak{sl}_2(R)$-module structure we start as follows.
\subsection{Tensor algebras as module algebras}
Let $\lie$ be an $R$ Lie algebra, $A := U(\lie)$ its enveloping algebra and $M$ a $\lie$ module (or equivalently, $(M, \rho)$ a $\lie$ representation).
\paragraph{Claim} The tensor algebra $T(M)$ has a canonical $\lie$ module structure.
\paragraph{Proof} First, we need a $\lie$ representation $(M^{\otimes n}, \rho_n)$ for all $n \geq 0$.
Clearly, $\rho_0 = 0$ as $M^{\otimes 0} = R$ and $\rho_1 = \rho$.
\begin{description}
\item[$n$-reprs] If $(M, \rho)$ and $(M', \rho')$ is a repr of $\lie$ then so is $(M \otimes M', \rho_{M \otimes M'} = id_M \otimes \rho' + \rho \otimes id_{M'})$.\\
We simply have to show that $\rho_{M \otimes M'} : \lie \longrightarrow \mathfrak{gl}_R(M \otimes M')$ is a homomorphism of Lie algebras. Let $x, y \in \lie$ then
$$\begin{array}{rcl}
\rho_{M\otimes M'}([x,y]) &=& id_M \otimes \rho'([x,y]) + \rho([x,y]) \otimes id_{M'}\\&&\\ &= &\left[m \otimes m' \longmapsto m \otimes \rho'([x,y])(m') + \rho([x,y])(m) \otimes m'\right]\\
&&\\
\left[\rho_{M \otimes M'}(x),\rho_{M \otimes M'}(y)\right] &=& \rho_{M\otimes M'}(x) \circ \rho_{M\otimes M'}(y) - \rho_{M\otimes M'}(y) \circ \rho_{M\otimes M'}(x)\\&&\\
&=& (id \otimes \rho'(x) + \rho(x) \otimes id)(id \otimes \rho'(y) + \rho(y) \otimes id)\\&& -(id \otimes \rho'(y) + \rho(y) \otimes id)(id \otimes \rho'(x) + \rho(x) \otimes id)\\&&\\
&=& id \otimes \rho'(x) \rho'(y) + \rho(y) \otimes \rho'(x) + \rho(x) \otimes \rho'(y) + \rho(x) \rho(y) \otimes id\\&&-\left(id \otimes \rho'(y) \rho'(x) + \rho(x) \otimes \rho'(y) + \rho(y) \otimes \rho'(x) + \rho(y) \rho(x) \otimes id\right)\\&&\\
&=& id \otimes (\rho'(x) \rho'(y) - \rho'(y) \rho'(x)) + (\rho(x) \rho(y) - \rho(y) \rho(x)) \otimes id\\
&&\\&=& id \otimes [\rho'(x),\rho'(y)] + [\rho(x), \rho(y)] \otimes id\\
&&\\&=& id \otimes \rho'([x,y]) + \rho([x,y]) \otimes id \ = \ \rho_{M \otimes M'} ([x,y])\\
\end{array}$$
Thus, the recursively defined family of maps $\rho_n : \lie \longrightarrow \mathfrak{gl}_R(M^{\otimes n}), x \longmapsto id_M \otimes \rho_{n-1}(x) + \rho_1(x) \otimes id_{M^{\otimes n-1}}$ are a family of Lie algebra homomorphisms making $T(M) = \bigoplus_{n \geq 0} M^{\otimes n}$ a $\lie$ module via the homomorphism
$$\rho_{T(M)} = \sum_{n \geq 0} \rho_n.$$
Lastly, we need to show $\rho_{T(M)} : \lie \longrightarrow \mathfrak{gl}_R(T(M))$ itself is a homomorphism of Lie algebras. Extending trivial, i.e. either $\rho(x) = id_{M^{\otimes m}}$ for all $n \neq m$ or $\rho_n(x) = 0$. In both cases, this implies $\rho_n(x) \rho_m(y) = \rho_m(y) \rho_n(x)$ completing our proof.
\newcommand{\eps}{\varepsilon}
\item[Bimodules] Now, let $(D,\mu,\eta,\Delta,\eps)$ be a unital associative counital coassociative bialgebra over $R$ and $(M, \rho)$ be a two-sided $D$ bimodule.
$$\left(T(M), \Psi_{T(M)}\right)$$
is a $D$ module algebra. We set $\Psi_D = \sum_{n \geq 0} \Delta_n$
\end{description}
\end{document}