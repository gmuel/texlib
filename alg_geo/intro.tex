%intro

\section{Introduction}
We are presenting a brief summory of basic concepts of modern algebraic geometry. In particular, we will discuss standard constructions and expand our learnings to a more abstract scheme theoretic approach.
\subsection{Requirements}
The reader only needs a minimum understanding in linear and abstract algebra. However, although some definitions will be missed we try to include as many as possible.\\
\subsection{Basics}
\indent We denote by $R$ an arbitrary ring with one (unital), unless mentioned otherwise, always commutative. A field is denoted by $k$ and its algebraic closure with $\overline{k}$, special fields and rings are denoted following \cite{Bour}. The polynomial ring for a given ring $R$ is denoted by $R[X] = R[X_1\ldots X_n]$ and the affine module $\mathbb{A}^n_R = (R^n, P(R^n), \phi)$.
\subsubsection{Zero sets and the Zariski topology}
Given a field $k$ and its polynomial ring $k[X] = k[X_1\ldots X_n]$ and a nonempty set of polynomials $F \subset R[X]$, the zero set of $F$, $Z(F)$, is defined:
$$\left\{x \in \kaffn : f(x) = 0 \ \forall f \in F\right\}.$$
We may rewrite:
$$\begin{array}{rcl}
Z(F) &=& \{x : f(x) = 0 \forall f \in F\}\\&&\\
&=& \bigcap_{f \in F} \{x : f(x) = 0\}.\\
\end{array}$$
In other words, $Z(F)$ is simply the intersection of common zeros of  each polynomial in $F$. Therefore, we could have simply defined zero sets for polynomials and then show that zero sets of $F$ are the intersections of the zeros of its elements.\\
\indent We remark that $Z(F)$ may be empty if no common zero is present (e.g. $f = X, g = X - 1 \in k[X]$). The zero set of the constant polynomial $1 \in k[X]$ is therefore empty and the zero set of $0 \in k[X]$ is the whole affine space.\\
\paragraph{Zeros and their topology}
By the above definition, we get that the set of all zeros is closed under arbitrary intersection. Furthermore, the empty set and the whole affine space are present in $\cldk := \{X \subset \kaffn : X \ \mathrm{zero~set}\}$. Now, let us check for two polynomials $f, g \in k[X]$:
$$\begin{array}{rcl}
Z(f) \cup Z(g) &=& \{x : f(x) = 0\} \cup \{x : g(x) = 0\}\\&&\\
&=& \{x : f(x) = 0 \vee g(x) = 0\}\\&&\\
&=& \{x : f(x) \cdot g(x) = 0\}\\&&\\
&=& Z(f g)\\
\end{array}$$
This shows $\cldk$ is closed under finite union and making it a closed topology (i.e. generated by closed sets) over $\kaffn$.\\
\paragraph{Principal open sets and their topology}
\indent Now, for a given $f \in k[X]$ we call the set
$$D(f) := \{x \in \kaffn : f(x) \neq 0\}$$
a principal open set of $\kaffn$. It is obviously the complement of a zero set. Now, for any nonempty set $F$ of polynomials we define the open set
$$D(F) := \bigcup_{f \in F} D(f).$$
Again, we remark that $D(F)$ may be empty or the full affine space:
$$D(F) = \kaffn \Leftrightarrow F \subset k^\times,\ D(F) = \emptyset \Leftrightarrow F = \{0\}.$$
As above, we need to check for closeness under intersection. However, as we defined our subset system via the complements of a topology generated by closed sets we get a topology generated by open sets. Therefore:
$$\opnk := \left\{U \subset \kaffn : U^c := \kaffn\backslash U \in \cldk\right\}$$
is an open topology on $\kaffn$.
\begin{defi}
The topology $\tau \subset \mathcal{P}(\kaffn)$ generated by both $\cldk$ and $\opnk$ is called the Zariski topology on $\kaffn$.
\end{defi}
\subsubsection{Affine schemes and the Zariski topology}
Given a unital commutative $k$ algebra $R$ and the class of all $R$-submodules of $R$, $\mathrm{Mod}_R(R)$, (i.e. its ideals)
$$\specr := \{\prm \in \mathrm{Mod}_R(R) : \prm \ \mathrm{prime}\},$$
the set of all prime ideals, is called the spectrum of $R$.
\paragraph{Pairing of points and ideals}
Given a $P = (P_1,\ldots,P_n) \in \kaffn$, the ideal
$$\mxx_P = \left<X_1 - P_1,\ldots,X_n - P_n\right>$$
is maximal in every polynomial ring $k[X]$ over some field $k$. As every maximal ideal in a unital ring is prime we get a unique subset of $\specr$:
$$\max (R) = \{\mxx \in \specr : \mxx \ \mathrm{maximal}\},$$
i.e. the set of maximal ideals in $R$.
\begin{prop}
Let $k = \overline{k}$ be a field and $R = k[X]$. The pairing
$$\kaffn \longrightarrow \max(R),\ P \longmapsto \mxx_P$$
is a bijection.
\end{prop}
\bmk This proposition is a rephrasing of Hilberts Nullstellensatz. Thus, the interested reader will find a proof in the standard literature as \cite{Hart}. In analogy to the Zariski topology on $\kaffn$ we define for some family of polynomials $F$:
$$V(F) = \{\prm \in \speck : f \in \prm,\ \forall f \in F\} = \bigcap_{f \in F} \{\prm : f \in \prm\}.$$
Accordingly, we call for some polynomial $f \in k[X]$ the set $U(f) = \{\prm : f \notin \prm\} = V(f)^c$ a prinicipal open set in $\speck$. Again, we get for some family $F$ of polynomials:
$$
\begin{array}{rclcl}
U(F) &:=& V(F)^c &=& \speck \backslash V(F)\\&&&&\\ &=& \speck \backslash \left(\bigcap_{f \in F} V(f)\right) &=& \bigcup_{f \in F} \left(\speck \backslash V(f)\right)\\&&&&\\ &=& \bigcup_{f \in F} V(f)^c &=& \bigcup_{f \in F} U(f).\\
\end{array}
$$
With this equivalences in place, we get the following proposition
\begin{prop}
The sets $\kaffn$ and $\X := \speck$ are isomorphic in the category of topological spaces with corresponding topologies $\cldk$ and $\opnk$, as well as
$$\mathcal{F}(\X) = \{V(F) : F \subset k[X]\}\ \mathrm{and}\ \mathcal{O}(\X) = \{V(F)^c : F \subset k[X]\},$$
respectively. Again, $V(F)^c := \speck \backslash V(F)$.
\end{prop}
\bws We simply use our bijective pairing above:
$$\kaffn \ni P = (P_1,\ldots,P_n) \longmapsto \mxx_P = \left<X_1 - P_1,\ldots,X_n - P_n\right> \in X.$$
Furthermore, we expand
$$\begin{array}{rclcl}
\cldk \ni Z(F) &=& \bigcap_{f \in F} Z(f) \longmapsto V(F) &=& \bigcap_{f \in F} V(f) \in \mathcal{F}(\X)\\&&&&\\
\opnk \ni D(F) &=& \bigcup_{f \in F} D(f) \longmapsto U(F) &=& \bigcup_{f \in F} U(f) \in \mathcal{O}(\X)\\
\end{array}$$